% !TEX root = ./main.tex
\documentclass[
  sigconf,
  nonacm,
  % draft,
]{acmart}
\usepackage{main}

\begin{document}

\title{Geometric Constraints in Algorithmic Design}
\author{Rui Ventura}
\email{rui.ventura@tecnico.ulisboa.pt}
\affiliation{%
  \institution{Instituto Superior Técnico, Universidade de Lisboa}
  \streetaddress{Avenida Rovisco Pais 1}
  \city{Lisbon}
  \country{Portugal}
  \postcode{1049-001}
}
% \renewcommand{\shortauthors}{Rui Ventura}

\begin{abstract}
\input{abstract.txt}
\end{abstract}

\keywords{\input{keywords.txt}}

\maketitle

% !TEX root = ../main.tex
% #############################################################################
% Do not forget to reorder Alphabetically the ACRONYMS !!!
% If there are special cases for Plural form see example hereunder
\begin{acronym}[B-Rep]
  \acro{AD}   {Algorithmic Design}
  \acro{AEC}  {Architecture, Engineering, and Construction}
  \acro{API}  {Application Programming Interface}
  \acro{B-Rep}{Boundary Representation}
  \acro{BIM}  {Building Information Modeling}
  \acro{CAD}  {Computer-Aided Design}
  \acro{CGAL} {Computational Geometry Algorithms Library}
  \acro{CS}   {Computer Science}
  \acro{CSG}  {Constructive Solid Geometry}
  \acro{CSP}  {Constraint Satisfaction Problem}
  \acro{DSL}  {Domain-specific Language}
  \acro{EPS}  {Encapsulated PostScript}
  \acro{GC}   {Geometric Constraint}
  \acro{GCS}  {Geometric Constraint Solving}
  \acro{GUI}  {Graphical User Interface}
  \acro{IDE}  {Integrated Development Environment}
  \acro{LEDA} {Library for Efficient Data Types and Algorithms}
  \acro{PGF}  {Portable Graphics Format}
  \acro{SDK}  {Software Development Kit}
  \acro{TikZ} {TikZ ist \textit{kein} Zeichenprogramm}
  \acro{TPL}  {Textual Programming Language}
  \acro{VBA}  {Visual Basic for Applications}
  \acro{VPL}  {Visual Programming Language}
\end{acronym}


% !TEX root = ../main.tex
\section{Introduction}%
\label{sec:intro}

Modern \ac{CAD} tools include substantial support for parametric operations and
\ac{GCS}.  These mechanisms have been developed over the past few
decades~\cite{Bettig:2011:GCSPC} and are heavily and ubiquitously used across
the Architecture, Engineering, and Construction industry.

Parametric modeling is used to design with constraints.  Users express a set of
parameters and operations, establishing restrictions between geometric entities.
The resulting geometry can be controlled from input parameters using two
computational mechanisms:
\begin{enumerate*}[label= (\arabic*)]
  \item parametric operations, and
  \item \ac{GCS}.
\end{enumerate*}

However, traditional interactive methods for parametric modeling suffer from the
disadvantage that they do not scale properly when designing more complex ideas.
In recent years, a novel approach to design named \ac{AD} has emerged, allowing
the specification of sketches and models through
algorithms~\cite{McCormack:2004:GDPDR} using either a \ac{TPL}, a \acp{VPL}, or
even a mixture of both.

Dealing with \acp{GC} can still prove to be an arduous task.  Take as an example
the sketch of a chair seat's outer frame, as seen in \cref{fig:intro.chair},
from a multi-purpose chair generation tool~\cite{Garcia:2012:ChairDNA} where the
chair's overall shape is controllable by specifying the values for a set of
input parameters.

\begin{figure}[htb]
  \includegraphics[width=.7\linewidth]{fig/chair-seat-outer-frame}
  \begin{minipage}{\linewidth}
  \scriptsize Source: Project source code, publicly unavailable.
  \end{minipage}
  \caption[Sketch of a chair seat's outer frame]{\label{fig:intro.chair}
    Sketch of a chair seat's outer frame, defined by several input parameters.}
\end{figure}

The seat's corners are defined by circles whose respective front and rear radii
are obtained by computing distances from which the circles' centers can be
obtained.  The circles are then connected through outer tangent lines, forming
the outer frame of the chair's seat.  Some of these operations, such as point
distance and tangency, must be handled carefully due to numerical robustness
issues that may arise when performing fixed-precision arithmetic computations.
This means that, on top of the design process itself, the user must spend
additional time identifying and carefully researching how to robustly implement
solutions to \acp{GC}.

To overcome this problem, this report proposes the implementation of \ac{GC}
primitives with specialized efficient solutions for different combinations of
input objects.  We additionally focus our work around \acp{TPL}, further making
them more attractive, and easier to both adopt and use.

% !TEX root = ../../main.tex
\subsection{Parametric Operations in CAD}%
\label{sec:intro.parametric}

Ivan Sutherland introduced the world to
Sketchpad~\cite{Sutherland:1964:Sketchpad} in 1963, an interactive 2D \ac{CAD}
program.  Sutherland's Sketchpad was capable of establishing atomic constraints
between objects, being the first of its kind.  The earliest 3D
system~\cite{Requicha:1980:RRS:356827.356833} dates from the 1970s.  This
system's parametric nature rested on a construction history tree.  The user
could modify an operation's parameters' values, reapply the modified history,
and regenerate the model.  Nearly a decade later,
Pro/ENGINEER\footnote{\url{https://www.ptc.com/en/products/creo/pro-engineer}}~\cite{Jabi:2013:PDA}
surfaced, enabling the creation of relations between the objects' sizes and
positions such that a change in a dimension between objects would automatically
change affected objects accordingly.  \Ac{GCS} soon became standard in drawings
by the early 1990s~\cite{Owen:1991:ASGDC,Bouma:1995:GCS}.  Efforts to expand the
benefits of \ac{GCS} beyond simple sketches were made, some systems having
implemented constraint solving in 3D.  Improvements from then on focused mostly
on robustness and operation variety.

In recent decades, emphasis shifted towards making parametric \ac{CAD} software
more interactive and user-friendly.  The intent was to make it as simple as
dragging a face of an object to where it should be instead of locating and
modifying operations buried in a construction history.  This is a tedious and
error-prone process that can lead to undesired side effects.  Nonetheless,
parametric operations will still see continued usage for the foreseeable future.

% !TEX root = ../../main.tex
\subsection{Constraints in CAD}%
\label{sec:intro.constraints}

Parametric operations allow the user to create geometric objects that satisfy
certain constraints \emph{implicitly} imposed on the objects when the user
selects the operation they want.  \Acp{GC}, on the other hand, allow the
repositioning and scaling of objects so that they satisfy constraints the user
\emph{explicitly} imposed on them.

The abstract problem of \ac{GCS} consists of assigning coordinates to
constrained geometric objects such that the constraints they are subject to are
satisfied.  Otherwise, the solver should report no such assignment can be found.

One of the important features of a solver is its \emph{competence}, which is
related to the capability of reporting unsolvability: if no solution for the
problem exists and the solver is capable of reporting unsolvability, the solver
is deemed fully competent.  Since constraint solving is mostly an exponentially
complex problem~\cite{Rossi:2006:Handbook}, partial competence suffices as long
as decent solutions can be found in affordable time and space.

In the context of \ac{GCS}, it is also important that the \ac{GC} system does
not have too few or too many constraints.  Summarily, a system can either be 
\begin{enumerate*}[label= (\arabic*)]
  \item under-constrained, if the number of solutions is unbound due to lack of
  constraint coverage;
  \item over-constrained, if there are no solutions because of contradictions;
  or
  \item well-constrained, if the number of solutions is finite.
\end{enumerate*}

Some of the subjects approached here are briefed in~\cite{Hoffmann:2005:BCS}.
The following sections present and briefly discuss the most relevant approaches
to constraint solving.

\subsubsection{Graph-Based Approaches}%
\label{sec:intro.constraints.graph}

The problem is translated into a labeled \textit{constraint graph}, where
vertices are constrained geometric objects, and edges the constraints
themselves.  These became the dominant \ac{GCS} approaches.

\subsubsection{Logic-Based Approaches}%
\label{sec:intro.constraints.logic}

The constraint problem is translated into a set of geometric assertions and
axioms which is then transformed in such a way that specific solution steps are
made explicit by applying geometric reasoning.  The solver then takes a set of
construction steps and assigns coordinate values to the geometric entities.

\subsubsection{Algebraic Approaches}%
\label{sec:intro.constraints.algebraic}

The problem is translated into a system of equations, which is generally
nonlinear.  This approach's main advantage is its completeness and dimension
independence.  However, it is difficult to decompose the equation system into
subproblems, and a general, complete solution of algebraic equations is
inefficient.  Nonetheless, small algebraic systems tend to appear in the other
approaches and are routinely solved.

\subsubsection{Symbolic Methods}%
\label{sec:intro.constraints.symbolic}

Symbolic methods rely on general equation solvers that employ techniques to
triangularize equation systems~\cite{Chou:1988:IWMMTPG,Buchberger:1995:Grobner}
that emerge from employing an algebraic approach.  These methods can produce
generic solutions, but solvers are very slow and computation demands a lot of
space, usually requiring exponential running time~\cite{Durand:1998:SNTCS}.

\subsubsection{Numerical Methods}%
\label{sec:intro.constraints.numerical}

Among the oldest approaches to constraint solving, numerical methods solve large
systems of equations iteratively.  Methods like Newton iteration work properly
if a good approximation of the intended solution can be supplied and the system
is not ill-conditioned.  Alas, such methods may find only one solution, even in
cases where there are many, and may not allow the user to select the one they
are interested in.

\subsubsection{Theorem Proving}%
\label{sec:intro.constraints.proving}

\ac{GCS} can be seen as a subproblem of geometric theorem proving, but the
latter requires general techniques, therefore requiring much more complex
methods than those required by the former.

% !TEX root = ../../main.tex
\section{\acl{GC} Problem Examples}%
\label{sec:intro.examples}

This section presents two simple examples of geometric models that are defined
through the specification of \acp{GC}, and the respective solutions using
intuitive algebraic formulation, accompanied by programmatic solutions using
\acs{TikZ}~\cite{Tantau:2021:TikZ} and Eukleides~\cite{Obrecht:2010:EM}.
Depictions of the aforementioned models can be seen in \cref{fig:intro.example}.
The examples are limited to the two-dimensional Euclidean plane over real
numbers, $\mathbb{R}^2$.  Solutions for analogous problems in three-dimensional
Euclidean space, $\mathbb{R}^3$, exist as well.

\begin{figure}[htpb]
  \subcaptionbox{\label{fig:intro.example.parallel}%
    Line that goes through $C$, strictly parallel to $\overleftrightarrow{AB}$.}
    [.45\linewidth]{\resizebox{!}{.2\textheight}{%
      \begin{tikzpicture}[rotate=20]
  \tkzDefPoints{0/0/A,3/0/B,1/1/C}
  \tkzDefLine[parallel=through C](A,B) \tkzGetPoint{D}
  \tkzDrawLines[add=.1 and .1](A,B C,D)
  \tkzDrawPoints(A,B,C)
  \tkzLabelPoints(A,B,C)
\end{tikzpicture}
}}
  \hspace{\fill}
  \subcaptionbox{\label{fig:intro.example.circumcenter}%
    $\odot O_r$ circumscribed about $\triangle ABC$.}
    [.45\linewidth]{\resizebox{!}{.2\textheight}{%
      \begin{tikzpicture}
  \tkzDefPoints{0/0/A,4/0/B,1/3/C}
  \tkzCircumCenter(A,B,C) \tkzGetPoint{O}
  \tkzDefMidPoint(A,B) \tkzGetPoint{AB}
  \tkzDefMidPoint(A,C) \tkzGetPoint{AC}
  \tkzDefMidPoint(B,C) \tkzGetPoint{BC}
  \tkzDrawSegments[style=dashed](AB,O AC,O BC,O)
  \tkzMarkRightAngles(A,AB,O B,BC,O C,AC,O)
  \tkzDrawPolygon(A,B,C)
  \tkzDrawCircle(O,A)
  \tkzDrawSegment(O,A)
  \tkzDrawPoints(A,B,C,O)
  \tkzLabelLine[above](O,A){$r$}
  \tkzLabelPoints[below left](A)
  \tkzLabelPoints[below right](B)
  \tkzLabelPoints[above left](C)
  \tkzLabelPoints(O)
\end{tikzpicture}
}}
  \caption[Geometric models defined using GCs]{
    Geometric models defined using \ac{GC} relations:
    \subref{fig:intro.example.parallel} showcases line parallelism, and
    \subref{fig:intro.example.circumcenter} showcases a circle circumscription
    about a triangle.}\label{fig:intro.example}
\end{figure}

The first problem is that of a parallelism constraint: specifying a line that
goes through a given point while also being strictly parallel to another already
defined line.  The second problem is a circumscription constraint: defining a
circle that tightly wraps around a triangle, i.e., the circle's circumference
goes through three given non-collinear points.

\subsection{Parallel lines}%
\label{sec:intro.examples.parallel}

Let $A,\,B,\,C \in \mathbb{R}^2$ such that $C$ is a point in the line which is
strictly parallel to the line $\overleftrightarrow{AB}$ (see
\cref{fig:intro.example.parallel}).

A line in $\mathbb{R}^2$ can be described by the parametric equation
\begin{equation}\label{eq:line.parametric.2}
  P_Q = Q + \lambda\vec{u} \Rightarrow
  \begin{cases}
    x = x_Q + \lambda u_x \\
    y = y_Q + \lambda u_y
  \end{cases},\,\lambda \in \mathbb{R}
\end{equation}\equations{Parametric equation of a line in $\mathbb{R}^2$}
where $Q = (x_Q, y_Q)$ is a point on the line that goes through $P_Q = (x, y)$,
and $\vec{u} = (u_x, u_y)$ is the vector that drives the line.  To then describe
the line that goes through $C$ and is parallel to $\overleftrightarrow{AB}$, one
must compute the base point $Q$, trivially $C$, and the directional vector
$\vec{u}$, which can be obtained from $\overleftrightarrow{AB}$.  Let $Q = C$,
and $\vec{u} = B - A$, such that
\[ P_C = C + \lambda \vec{u},\,\lambda \in \mathbb{R}. \]

\Cref{lst:intro.example.parallel.tikz} shows the code used to produce the
example shown in \cref{fig:intro.example.parallel} using \acs{TikZ} with the
\texttt{tkz-euclide}\footnote{\url{https://ctan.org/pkg/tkz-euclide}} \LaTeX{}
package, using \texttt{tkzDefLine}, which takes two points, $A$ and $B$, with
the \texttt{parallel} transformation option.  This option takes the point $C$
the resulting line goes through.  The result is a point $D = C + \vec{u}$, which
can be obtained using \texttt{tkzGetPoint} to later draw the line. 

\begin{listing}[htb]
  \inputminted[highlightlines=3]{latex}{tikz/ex-parallel.tikz}
  \caption[Parallel lines example using \texttt{tkz-euclide}]{
    Parallel lines example from \cref{fig:intro.example.parallel} using
    \texttt{tkz-euclide}.  The highlighted line shows how to define the line
    $L_C$ parallel to $\overleftrightarrow{AB}$.}%
  \label{lst:intro.example.parallel.tikz}
\end{listing}

\Cref{lst:intro.example.parallel.euk} shows the code used to produce an
identical figure using Eukleides.  In Eukleides, the parallel line $L_C$ can be
obtained through the \texttt{parallel} function, which takes the line
$\overleftrightarrow{AB}$ it is parallel to and the point $C$ it goes through.

\begin{listing}[htb]
  \inputminted[highlightlines=3]{text}{euk/ex-parallel.euk}
  \caption[Parallel lines example using Eukleides]{
    Parallel lines example from \cref{fig:intro.example.parallel} using
    Eukleides.  The highlighted line shows how to define the line $L_C$
    parallel to $\overleftrightarrow{AB}$.}%
  \label{lst:intro.example.parallel.euk}
\end{listing}

\begin{comment}
We can determine if two lines are parallel by determining the angle $\theta$
between them, and verifying it is equal to $0$.
%
\begin{equation}\label{eq:angle.vectors.2}
  \theta = \arccos \frac{\vec{u} \cdot \vec{v}}%
                        {||\vec{u}|| \cdot ||\vec{v}||},~%
  \theta \in \mathbb{R}.
\end{equation}
%
Having $\vec{v} = \lambda \vec{u}, \lambda \in \mathbb{R}$, and knowing
%
\begin{equation}\label{eq:dot.vector.2.same}
  \vec{u} \cdot \vec{u} = ||\vec{u}||^2,
\end{equation}
%
then \eqref{eq:angle.vectors.2} becomes
%
\[
  \begin{split}
    \theta & = \arccos \frac{\vec{u} \cdot \lambda\vec{u}}%
                            {||\vec{u}|| \cdot ||\lambda\vec{u}||}\\%
           & = \arccos \frac{\lambda\vec{u} \cdot \vec{u}}%
                            {\lambda||\vec{u}||^2}\\%
           & = 0.
  \end{split}
\]
%
This means that we can compute a directional vector for $CS$ from the line $AB$,
where $\vec{u} = B - A$.  Finally, we can obtain the equation for the parallel
line $CS$
\end{comment}

\subsection{Circumcenter}%
\label{sec:intro.examples.circumcenter}

Let $A,\,B,\,C,\,O \in \mathbb{R}^2$ be points such that $O$ is the center point
of a circle of radius $r$, $\odot O_r$, that is circumscribed about the triangle
$\triangle ABC$ (see \cref{fig:intro.example.circumcenter}).

A precondition for this computation is that $\triangle ABC$ is not degenerate,
i.e., its vertices are non-collinear.  That can be verified by computing the
cross product of any two distinct vectors that drive $\triangle ABC$'s edges and
verifying it does not equate to zero.

\begin{comment}
The Laplace expansion for the determinant of a generic matrix $A \in
\mathbb{R}^{n \times n}$ is given by
%
\begin{align}
  \det(A) = \begin{vmatrix}
              a_{11} & \cdots & a_{1n} \\
              \vdots & \ddots & \vdots \\
              a_{n1} & \cdots & a_{nn}
            \end{vmatrix}
          &= a_{1j} C_{1j} + \cdots + a_{nj} C_{nj}%
            = a_{i1} C_{i1} + \cdots + a_{in} C_{in} \nonumber \\
          &= \sum_{i'=1}^n a_{i'j}C_{i'j}%
            \label{eq:matrix.det.laplace.nxn.col} \\
          &= \sum_{j'=1}^n a_{ij'}C_{ij'}%
            \label{eq:matrix.det.laplace.nxn.row},
\end{align}
%
where $i,j \in [1,n] \subset \mathbb{N}$,
%
\begin{equation}\label{eq:matrix.cofactor}
  C_{ij} = (-1)^{i+j} \det(A_{ij})
\end{equation}
%
is the $i,j$ cofactor of $A$, and
%
\begin{equation}\label{eq:matrix.minor}
  A_{ij} = \begin{bmatrix}
    a_{11}     & \cdots & a_{1j-1}     & a_{1j+1}     & \cdots & a_{1n}     \\%
    \vdots     & \ddots & \vdots       & \vdots       & \ddots & \vdots     \\%
    a_{i-1\,1} & \cdots & a_{i-1\,j-1} & a_{i-1\,j+1} & \cdots & a_{i-1\,1} \\%
    a_{i+1\,1} & \cdots & a_{i+1\,j-1} & a_{i+1\,j+1} & \cdots & a_{i+1\,1} \\%
    \vdots     & \ddots & \vdots       & \vdots       & \ddots & \vdots     \\%
    a_{n1}     & \cdots & a_{nj-1}     & a_{nj+1}     & \cdots & a_{nn}     \\
  \end{bmatrix}_{(n - 1) \times (n - 1)}
\end{equation}
%
is the minor of matrix $A$ without the $i$-th row and $j$-th column.  This can
be further simplified for a $2\times 2$ matrix.  Let $B\in\mathbb{R}^{2\times
2}$ be the matrix whose columns are the vectors $\vec{AB} = (a, b)$ and
$\vec{AC} = (c, d)$, and $i' = 1$, for instance, such that,
from~\eqref{eq:matrix.det.laplace.nxn.col},
%
\begin{equation}\label{eq:matrix.det.laplace.2x2.col}
  \det(B) = \begin{vmatrix}
              a & c\\
              b & d
            \end{vmatrix}
          = \sum_{j=1}^2 a_{1j} C_{1j}
          = (-1)^{1+1}a \cdot d + (-1)^{1+2}c \cdot b
          = ad - cb
\end{equation}
\end{comment}

\begin{comment}
Let $A \in \mathbb{R}^{2 \times 2}$ be the matrix whose columns are the vectors
$\vec{AB} = (a, b)$ and $\vec{AC} = (c, d)$, for instance, such that
%
\begin{equation}\label{eq:matrix.det.2x2}
  \det(A) = \begin{vmatrix}
              a & c\\
              b & d
            \end{vmatrix}%
          = ad - cb
\end{equation}
% 
If the determinant is found to be $0$, then there is no possible solution.
Otherwise, one can proceed to draw $\odot O_r$.
\end{comment}

To draw $\odot O_r$, we must compute both its center and radius.  Its radius $r$
can be trivially defined as the distance of the center $O$ to any of the
$\triangle ABC$'s vertices, i.e., $r = \left\|O - A\right\| = \left\|O -
B\right\| = \left\|O - C\right\|$.  To determine $O$, one must compute the
intersection of the perpendicular bisectors of the triangle's edges.  Said
bisectors are the mediators between an edge's vertices, which can be described
by~\cref{eq:line.parametric.2}, where $P$ is the midpoint between the vertices,
and $\vec{u}$ is a vector normal to the edge.  The midpoint $M_{P_1P_2}$ of two
points $P_1,\,P_2 \in \mathbb{R}^2$ is given by
\begin{equation}\label{eq:midpoint.2}
  M_{P_1P_2} = \frac{P_1 + P_2}{2}%
             = \left(\frac{x_1 + x_2}{2}, \frac{y_1 + y_2}{2}\right).
\end{equation}\equations{Midpoint between two points in $\mathbb{R}^2$}
Further, the scalar product of two vectors $\vec{u}, \vec{v} \in \mathbb{R}^2$
is given by
\begin{equation}\label{eq:vector.scalar.2}
  \vec{u} \cdot \vec{v} = (u_x, u_y) \cdot (v_x, v_y) = u_x v_x + u_y v_y.
\end{equation}\equations{Scalar product of vectors in $\mathbb{R}^2$}
The normal vector $\vec{n}$ is such that, for some vector $\vec{u}$,
\[ \vec{u} \cdot \vec{n} = 0. \]
A vector $\vec{n} \in \mathbb{R}^2$ normal to another vector $\vec{u}$ can be
easily obtained by swapping the components of $\vec{u}$ while negating one of
them, a property easily verified by applying \cref{eq:vector.scalar.2}.

\begin{comment}
This comes as a direct result from applying a rotation transformation of 90
degrees, or $\pi/2$ radians, to $\vec{u}$, like so
%
\[
  \vec{n} = R(\pi/2)\vec{u}%
  = \begin{bmatrix}
      \cos(\pi/2) & -\sin(\pi/2) \\
      \sin(\pi/2) & \cos(\pi/2)
    \end{bmatrix}
    \begin{bmatrix}
      u_1 \\ u_2
    \end{bmatrix}
  = \begin{bmatrix}
      0 & -1 \\
      1 & 0
    \end{bmatrix}
    \begin{bmatrix}
      u_1 \\ u_2
    \end{bmatrix}
  = \begin{bmatrix}
      -u_2 \\ u_1
    \end{bmatrix}.
\]
%
Let $\vec{u},\,\vec{n} \in \mathbb{R}^2$, such that $\vec{u} = (u_x,
u_y),\,\vec{n} = (-u_y, u_x)$.  From~\eqref{eq:vector.dot.2}, we have
\[ \vec{u} \cdot \vec{n} = u_x u_y - u_y u_x = 0.  \]
\end{comment}

Computing the edges' midpoints and respective normal vectors, we can then
describe the mediators.  Let $M_{AB},\,M_{AC},\,M_{BC} \in \mathbb{R}^2$ be the
midpoints, by \cref{eq:midpoint.2}, of the respective edges, and
$\vec{u}_1,\,\vec{u}_2,\,\vec{u}_3 \in \mathbb{R}^2$ the edges' normal vectors,
such that
\[
  \begin{split}
    P_{M_{AB}} = M_{AB} + \lambda_1 \vec{u}_1 \\
    P_{M_{AC}} = M_{AC} + \lambda_2 \vec{u}_2 \\
    P_{M_{BC}} = M_{BC} + \lambda_3 \vec{u}_3 \\
  \end{split},\,\lambda_i \in \mathbb{R}.
\]
This problem can be further simplified by eliminating one of the redundant
bisectors.  Since the intersection of two lines already yields a single point,
we can eliminate one of the equations.  Say we discard the mediator of line
$\overleftrightarrow{BC}$.  We then require that
\[
  P_{M_{AB}} = P_{M_{AC}} \stackrel{\eqref{eq:line.parametric.2}}{\Rightarrow}
  \begin{cases}
    x_{M_{AB}} + \lambda_1 u_{1x} = x_{M_{AC}} + \lambda_2 u_{2x} \\
    y_{M_{AB}} + \lambda_1 u_{1y} = y_{M_{AC}} + \lambda_2 u_{2y} \\
  \end{cases}.
\]
Every variable is known except for $\lambda_1$ and $\lambda_2$, but the equation
system can be solved in order to assign values to both of them since we have
exactly two equations that relate them.  Finally, we can define $O$ using one of
the equations with the respectively found $\lambda$, i.e., using $L_{M_{AB}}$,
for instance, we have \[ O = M_{AB} + \lambda_1 \vec{u}_1. \]

\Cref{lst:intro.example.circumcenter.tikz} shows the code used to produce the
example in \Cref{fig:intro.example.circumcenter} using \acs{TikZ} with the
\texttt{tkz-euclide} \LaTeX{} package.  To compute the center point of $\odot
O_r$, one can use \texttt{tkzCircumCenter}, which takes three points, $A$, $B$,
and $C$, and generates the result $O$, obtainable using \texttt{tkzGetPoint}.

\begin{listing}[htb]
  \inputminted[highlightlines=3]{latex}{tikz/ex-circumcenter.tikz}
  \caption[Circumcenter example using TikZ]{
    Circumcenter example from \cref{fig:intro.example.circumcenter} using
    \acs{TikZ} alongside \texttt{tkz-euclide}.  The highlighted line shows how
    to obtain the center of $\odot O_r$ via the non-degenerate triangle
    $\triangle ABC$.}%
  \label{lst:intro.example.circumcenter.tikz}
\end{listing}

\Cref{lst:intro.example.circumcenter.euk} shows the code that produces an
identical figure using Eukleides.  In Eukleides, one can use the \texttt{circle}
function, which similarly takes three points, $A$, $B$, and $C$, and generates
the circle $\odot O_r$ circumscribed about $\triangle ABC$, while $O$ can be
obtained using the \texttt{center} function.

\begin{listing}[htb]
  \inputminted[highlightlines=2]{text}{euk/ex-circumcenter.euk}
  \caption[Circumcenter example using Eukleides]{
    Circumcenter example from \Cref{fig:intro.example.circumcenter} using
    Eukleides.  The highlighted line shows how to obtain the center of $\odot
    O_r$ via the non-degenerate triangle $\triangle ABC$.}%
  \label{lst:intro.example.circumcenter.euk}
\end{listing}

Both languages used to produce the examples' solutions provide a sensible set of
constraint primitives.  However, in the particular case of \texttt{tkz-euclide},
the syntax required for describing the models is outdated, rigid, and may cause
confusion.  For example, in
\cref{lst:intro.example.parallel.tikz,lst:intro.example.circumcenter.tikz},
command results can not be used directly as inputs to other commands and must
instead be obtained using another command to create a permanent symbol
associated with the resulting value.  By contrast, functions and expressions'
results in modern languages can be used directly as well as stored by using a
far friendlier assignment syntax.  Nonetheless, the underlying ideas can be
repurposed and adapted, implementing them in a modern and more expressive
language.

% !TEX root = ../../main.tex
\section{\acl{AD} Tools}
\label{sec:related.ad}

As discussed in \cref{sec:intro.ad}, \ac{AD} tools have been integrated into
several modern \ac{CAD} and \ac{BIM} applications; tools that use \acp{TPL},
\acp{VPL}, or even a mixture of both approaches.

Other tools, like OpenJSCAD and ImplicitCAD, are standalone \ac{CAD} software
hosted on the web.  Being cloud-based is advantageous in many fronts: it is
inherently portable, removes the additional typical installation steps required
for desktop applications.  Alas, being relatively new, they are lacking features
in comparison to the immense feature-set of applications such as AutoCAD.

\Cref{tab:related.ad.summary} succinctly summarizes a list of \ac{CAD} software
that includes the capability of designing resorting to the usage of a
programming language, as well as other \ac{AD} software and tools that live
detached from existing software.  From there, Dynamo and Grasshopper are further
comparatively discussed, being relatively similar tools, however integrated
within \ac{CAD}/\ac{BIM} software designed for performing different specific
tasks.  Moreover, both include \ac{TPL} and \ac{VPL} support in different forms.

\begin{table}[htbp]
  \begin{tabularx}{\textwidth}{|*{4}{c|}X|}
    \hline
    \textbf{Application} & \textbf{Tool} & \textbf{\acs{TPL}}
      & \textbf{\acs{VPL}} & \textbf{Note}\\
    \hline
    \hline
    \multirow{5}{*}{AutoCAD \cite{Autodesk:1982:AutoCAD}}
      & \multirow{2}{*}{.NET \acs{API}\label{acro:API}}
      & \multirow{2}{*}{\checkmark} & \multirow{2}{*}{\xmark}
      & \multirow{2}{*}{\parbox{\linewidth}{
        Powerful, but very verbose; C\# \& VB.NET}}\\
      &&&& \\ \cline{2-5}
      & \multirow{2}{*}{\parbox{7em}{\centering ActiveX Automation}}
        & \multirow{2}{*}{\checkmark} & \multirow{2}{*}{\xmark}
        & \multirow{2}{*}{\parbox{\linewidth}{
          Deprecated, bundled separately; \acs{VBA}\label{acro:VBA}}}\\
      &&&& \\ \cline{2-5}
      & Visual LISP & \checkmark & \xmark & \acs{IDE}\label{acro:IDE};
        AutoLISP extension\\
    \hline
    Dynamo Studio
      & \multirow{2}{*}{Dynamo \cite{Keough:2012:Dynamo}}
      & \multirow{2}{*}{\checkmark} & \multirow{2}{*}{\checkmark}
      & \multirow{2}{*}{\parbox{\linewidth}{%
        Data flow paradigm; Associative programming support through
        DesignScript}}\\\cline{1-1}
    Revit \cite{RevitTechCorp:2002:Revit} &&&&\\
    \hline
    ArchiCAD \cite{Graphisoft:2018:ArchiCAD}
      & \multirow{2}{*}{Grasshopper \cite{Rutten:2018:Grasshopper}}
      & \multirow{2}{*}{\checkmark} & \multirow{2}{*}{\checkmark}
      & \multirow{2}{*}{\parbox{\linewidth}{%
        Data flow paradigm; Rhino \acs{SDK} access, C\# \& VB.NET}}\\\cline{1-1}
    \multirow{4}{*}{Rhinoceros3D \cite{McNeel:2018:Rhinoceros3D}}
      &&&& \\ \cline{2-5}
      & \multirow{2}{*}{Python Scripting} & \multirow{2}{*}{\checkmark}
        & \multirow{2}{*}{\xmark}
        & \multirow{2}{*}{\parbox{\linewidth}{%
          Simple language; Create custom Grasshopper components}}\\
      &&&&\\\cline{2-5}
      & RhinoScript & \checkmark & \xmark & VBScript based\\
    \hline
    \multirow{5}{*}{\texttt{Standalone$^\dag$}}
      & ImplicitCAD \cite{Longtin:2018:ImplicitCAD}
        & \checkmark & \xmark & Web hosted; OpenSCAD inspired\\\cline{2-5}
      & OpenJSCAD \cite{Mueller:2019:OpenJSCAD}
        & \checkmark & \xmark & Web hosted; JavaScript\\\cline{2-5}
      & OpenSCAD \cite{Kintel:2019:OpenSCAD}
        & \checkmark & \xmark & Solid 3D models; Simple \acs{DSL}\label{acro:DSL}\\\cline{2-5}
      & \multirow{2}{*}{Rosetta \cite{Leitao:2011:PGDCAD}}
        & \multirow{2}{*}{\checkmark} & \multirow{2}{*}{\xmark}
        & \multirow{2}{*}{\parbox{\linewidth}{%
          Portable tool; Multiple front- and back-end support}}\\
      &&&&\\
    \hline
  \end{tabularx}
  \scriptsize
  $^\dag$These tools are standalone software, i.e., not directly integrated into
  any specific \ac{CAD} application.
  \caption[Table of programmatic \acs{CAD}/\acs{BIM} and \acs{AD} software]{%
    \ac{CAD}/\ac{BIM} software with programmatic capabilities and \ac{AD}
    software/tools.  Added notes per tool shortly outline deemed significant
    characteristics.}
  \label{tab:related.ad.summary}
\end{table}

\subsection{Dynamo}
\label{sec:related.ad.dynamo}

An open source \ac{AD} tool available as a plug-in for Revit or by itself within
Dynamo Studio, Dynamo extends \ac{BIM} with the data and logic environment of a
graphical algorithm editor \cite{Keough:2012:Dynamo}.  Dynamo can be used
through both a \ac{VPL} and a \ac{TPL}, showcased in
\cref{fig:related.ad.dynamo.node2code}.

\begin{figure}[htbp]
  \includegraphics[width=\textwidth]{fig/dynamo-node-to-code}
  {\scriptsize
  Source: \url{http://primer.dynamobim.org/en/07_Code-Block/7-2_Design-Script-syntax.html}
  (Jan 2019)}
  \caption[Dynamo's visual interface with node to code translation]{%
    Showcase of Dynamo's visual interface containing a workflow that produces
    the model on the top left.  The figure also shows Dynamo's capability of
    converting a the workflow to a single DesignScript code block.}
  \label{fig:related.ad.dynamo.node2code}
\end{figure}

In its visual form, Dynamo offers a wide variety of functions, called nodes,
most of them capable of generating an even wider variety of geometry through
node combination, wiring one's outputs to another's inputs, and resorting to
pre-defined mutable parameters which can serve as some of the nodes' initial
inputs.  The workflow itself is the final product: a visual program, usually
designed to execute a specific task.  Dynamo further allows extension through the
creation of custom nodes which can be shared as packages.

One of the nodes in Dynamo, aptly named code block, allows the usage of a
\ac{TPL}; a language called DesignScript.  Originally developed my Robert Aish
\cite{Aish:2011:DesignScript}, DesignScript is a multi-paradigm domain-specific
language and is the programming language at the core of Dynamo itself.  So much
so that entire workflows can be reduced to one code block (see
\Cref{fig:related.ad.dynamo.node2code}).

DesignScript is an associative language, which maintains a graph of dependencies
with variables.  Executing a script will effectively propagate the variables'
values accordingly.  By default, code blocks in Dynamo follow an associative
paradigm.  The user can, however, switch to an imperative paradigm approach
instead effortlessly if needed.

This \textit{change-propagation} mechanism in DesignScript, consequently present
in Dynamo, makes Dynamo a great tool for dealing with constraints.  However,
most users might not fully exercise DesignScript's associative capabilities and
instead approach the problem with the mindset of an imperative programming
paradigm given its overwhelming presence in and adoption by major well-known
\acp{TPL}.

\subsection{Grasshopper}
\label{sec:related.ad.grasshopper}

Grasshopper is a graphical algorithm editor tightly integrated with
Rhinoceros3D, destined for designers who are exploring generative algorithms
\cite{Rutten:2018:Grasshopper}.  In spite of tight integration with Rhino, a
\ac{CAD} application, it is possible to use Grasshopper along with ArchiCAD
\cite{Graphisoft:2018:ArchiCAD,Graphisoft:2018:RGACAD}, a \ac{BIM} tool.
\Cref{fig:related.ad.grasshopper.islamic-pattern} shows a simple example of a
Grasshopper workflow.

\todo[inline]{Replace example with Rythmic Gynmastics Center, Moscow, Russia}

\begin{figure}[htbp]
  \includegraphics[width=\textwidth]{fig/grasshopper-islamic-pattern}
  {\scriptsize
  Source: \url{https://www.grasshopper3d.com/photo/islamic-pattern-parakeet}
  (Jan 2019)}
  \caption[Islamic Pattern in Grasshopper using Parakeet]{
    Islamic Pattern, by Esmaeil Mottaghi.  On top is the Grasshopper workflow to
    produce the pattern below it, aided by Parakeet
    \cite{Esmaeil:2018:Parakeet}.}
  \label{fig:related.ad.grasshopper.islamic-pattern}
\end{figure}

It is a closed-source product, designed by David Rutten and developed by McNeel
and Associates, Rhino's developers.  Its \ac{VPL} is as simple to use as
Dynamo's, which is crucial for users who are not familiar with programming using
a \ac{TPL}.  Nonetheless, it offers a \ac{TPL} alternative by way of custom
programmatic components.  Using C\# or VB.NET, the user can create custom code
components with access to Rhino's \ac{SDK} and OpenNURBS
\cite{Lear:2018:openNURBS} within Rhino.  Alternatively, through GhPython
\cite{Giulio:2017:GhPython}, the user can also write Python code.  Unlike
DesignScript, Python and the .NET languages don't support an associative
programming model.

Functions in Grasshopper are called components and work just like Dynamo's
nodes; a wide variety of them exist, most of them capable of producing geometry,
and they are composable, generating a workflow destined to accomplish a specific
task.

Both Dynamo and Grasshopper's visual approach suffer from the unproportionate
scalability between the code and the respective model's complexity
\cite{Leitao:2014:PESLGD}.  Sophisticated modelling workflows tend to become
difficult to properly represent, and harder for a human to efficiently interpret
when compared to a textual approach.  This disadvantage, however, is mitigated
with their respective \ac{TPL} alternatives.


% !TEX root = ../main.tex
\fancychapter{Related Work}%
\label{chap:related}
\cleardoublepage{}

\noindent
In this chapter, we start by exposing and discussing numerical accuracy issues
that arise when performing computations with fixed-precision arithmetic in
\cref{sec:related.robustness}.  We then proceed to naming some precautions and
steps in order to obtain practical solutions, followed by a brief mention of
some software libraries dedicated to overcoming these issues.  To that end, said
libraries provide a series of exact algorithms and data structures.

Secondly, we comparatively analyze a set of \ac{GCS}-capable programming tools'
qualities, such as supported language paradigm, native \ac{GCS} capabilities, 2D
and 3D support; summarized in \cref{tab:related.constraints.summary}.  Of those
tools, Eukleides, GeoSolver, and \acs{TikZ} \& \acs{PGF} are extensively
discussed.

Finally, we similarly analyze \ac{AD} tools.  Some of them are integrated within
\ac{CAD} applications while a few of them are standalone applications.  These
tools and their capabilities are summarized in \cref{tab:related.ad.summary}.
Furthermore, Dynamo and Grasshopper are expanded upon.

% !TEX root = ../../main.tex
\section{Robustness}%
\label{sec:related.robustness}

The correctness proofs of nearly all geometric algorithms presented in
theoretical papers assumes exact computation with real
numbers~\cite{CGAL:4.13:23LGK}.  However, floating-point numbers are represented
with fixed precision in computers, making them inexact, which leads to
inaccurate representations of the conceptual real number counterparts.  For
example, the rational number one-tenth ($\frac{1}{10}$) cannot be accurately
represented as a floating-point number, nor is it guaranteed to be truly equal
to another seemingly identical number.  Such comparisons must be performed
relying on tolerances, i.e., if $a$ and $b$ are two floating-point numbers, they
are considered \textit{the same} if $|a - b| \le \epsilon$ for a given tolerance
$\epsilon$.

As an example, consider the problem of finding the closest of two points
to the origin.  The distance between two points $P,Q \in \mathbb{R}^2$ can be
expressed by
%
\begin{equation}\label{eq:distance.2}
  d(P, Q) = \sqrt{(x_Q - x_P)^2 + (y_Q - y_P)^2}.
\end{equation}\equations{Euclidean distance between two points in $\mathbb{R}^2$}
%
Let $A,B \in \mathbb{R}^2$ be two arbitrary points, and $O \in \mathbb{R}^2$ the
origin.  To determine which point, $A$ or $B$, is closest to the origin $O$, we
compare the former's distances to the latter's.  That is, if
%
\[ 
  d(A, O) < d(B, O) 
\]
%
holds, $A$ is the closest to the origin.  Otherwise, they are either equidistant
or $B$ is closer.  However, applying the square root operation in the distance
computation is a step that will introduce errors.  Given that we are only
interested in comparing distances, and not use their actual value, we can,
instead, compare the squared distances.  As such, we avoid the square root, thus
improving robustness, and speeding up the process because the square root is a
computationally heavy operation.  \Citet{Mei:2014:NRGC} further discuss the
issues with numerical robustness in geometric computation, namely how they
arise, and propose practical solutions.

When used without care, fixed-precision arithmetic almost always leads to
unwanted results due to marginal error accumulation caused by rounding
(\textit{roundoff}), propagated throughout a series of calculations.  As seen
above, careful observations must be made before proceeding with computations as
simple as distance calculation.  To help solve this problem, more robust
numerical constructs and concepts can be used.  In particular, exact numbers,
such as rational numbers or arbitrary precision numbers. The latter, also known
as \textit{bignums}, allow arbitrary-precision arithmetic, capable of
representing numbers with virtually infinite precision with the drawback that
arithmetic operations are slower, however mitigating precision issues, providing
more accurate constructs and improving code robustness.

Several libraries already strive to implement robust geometric computation.  One
such example is \acf{CGAL}~\cite{CGAL:2018}.  \Ac{CGAL} is a comprehensive
library that employs an exact computation paradigm~\cite{Yap:1995:ECP},
producing correct results despite roundoff errors and properly handling
\textit{degenerate} situations (e.g., 3D points on a 2D plane), relying on
numbers with arbitrary precision to do so.  Moreover, other libraries, such as
\acs{LEDA}\label{acro:LEDA}~\cite{LEDA:2017,Mehlhorn:1989:LEDA}, and
CORE~\cite{Karamcheti:1999:CLRNGC} and its successor~\cite{Yu:2010:CORE2}, also
deal with robustness problems in geometric computation, offering simpler
interfaces when compared to \ac{CGAL}.  However, \ac{CGAL} arguably remains the
\textit{de facto} library for robust exact geometric computation.

% !TEX root = ../../main.tex
\subsection{Constraints in CAD}%
\label{sec:intro.constraints}

Parametric operations allow the user to create geometric objects that satisfy
certain constraints \emph{implicitly} imposed on the objects when the user
selects the operation they want.  \Acp{GC}, on the other hand, allow the
repositioning and scaling of objects so that they satisfy constraints the user
\emph{explicitly} imposed on them.

The abstract problem of \ac{GCS} consists of assigning coordinates to
constrained geometric objects such that the constraints they are subject to are
satisfied.  Otherwise, the solver should report no such assignment can be found.

One of the important features of a solver is its \emph{competence}, which is
related to the capability of reporting unsolvability: if no solution for the
problem exists and the solver is capable of reporting unsolvability, the solver
is deemed fully competent.  Since constraint solving is mostly an exponentially
complex problem~\cite{Rossi:2006:Handbook}, partial competence suffices as long
as decent solutions can be found in affordable time and space.

In the context of \ac{GCS}, it is also important that the \ac{GC} system does
not have too few or too many constraints.  Summarily, a system can either be 
\begin{enumerate*}[label= (\arabic*)]
  \item under-constrained, if the number of solutions is unbound due to lack of
  constraint coverage;
  \item over-constrained, if there are no solutions because of contradictions;
  or
  \item well-constrained, if the number of solutions is finite.
\end{enumerate*}

Some of the subjects approached here are briefed in~\cite{Hoffmann:2005:BCS}.
The following sections present and briefly discuss the most relevant approaches
to constraint solving.

\subsubsection{Graph-Based Approaches}%
\label{sec:intro.constraints.graph}

The problem is translated into a labeled \textit{constraint graph}, where
vertices are constrained geometric objects, and edges the constraints
themselves.  These became the dominant \ac{GCS} approaches.

\subsubsection{Logic-Based Approaches}%
\label{sec:intro.constraints.logic}

The constraint problem is translated into a set of geometric assertions and
axioms which is then transformed in such a way that specific solution steps are
made explicit by applying geometric reasoning.  The solver then takes a set of
construction steps and assigns coordinate values to the geometric entities.

\subsubsection{Algebraic Approaches}%
\label{sec:intro.constraints.algebraic}

The problem is translated into a system of equations, which is generally
nonlinear.  This approach's main advantage is its completeness and dimension
independence.  However, it is difficult to decompose the equation system into
subproblems, and a general, complete solution of algebraic equations is
inefficient.  Nonetheless, small algebraic systems tend to appear in the other
approaches and are routinely solved.

\subsubsection{Symbolic Methods}%
\label{sec:intro.constraints.symbolic}

Symbolic methods rely on general equation solvers that employ techniques to
triangularize equation systems~\cite{Chou:1988:IWMMTPG,Buchberger:1995:Grobner}
that emerge from employing an algebraic approach.  These methods can produce
generic solutions, but solvers are very slow and computation demands a lot of
space, usually requiring exponential running time~\cite{Durand:1998:SNTCS}.

\subsubsection{Numerical Methods}%
\label{sec:intro.constraints.numerical}

Among the oldest approaches to constraint solving, numerical methods solve large
systems of equations iteratively.  Methods like Newton iteration work properly
if a good approximation of the intended solution can be supplied and the system
is not ill-conditioned.  Alas, such methods may find only one solution, even in
cases where there are many, and may not allow the user to select the one they
are interested in.

\subsubsection{Theorem Proving}%
\label{sec:intro.constraints.proving}

\ac{GCS} can be seen as a subproblem of geometric theorem proving, but the
latter requires general techniques, therefore requiring much more complex
methods than those required by the former.

% !TEX root = ../../main.tex
\section{\acl{AD} Tools}
\label{sec:related.ad}

As discussed in \cref{sec:intro.ad}, \ac{AD} tools have been integrated into
several modern \ac{CAD} and \ac{BIM} applications; tools that use \acp{TPL},
\acp{VPL}, or even a mixture of both approaches.

Other tools, like OpenJSCAD and ImplicitCAD, are standalone \ac{CAD} software
hosted on the web.  Being cloud-based is advantageous in many fronts: it is
inherently portable, removes the additional typical installation steps required
for desktop applications.  Alas, being relatively new, they are lacking features
in comparison to the immense feature-set of applications such as AutoCAD.

\Cref{tab:related.ad.summary} succinctly summarizes a list of \ac{CAD} software
that includes the capability of designing resorting to the usage of a
programming language, as well as other \ac{AD} software and tools that live
detached from existing software.  From there, Dynamo and Grasshopper are further
comparatively discussed, being relatively similar tools, however integrated
within \ac{CAD}/\ac{BIM} software designed for performing different specific
tasks.  Moreover, both include \ac{TPL} and \ac{VPL} support in different forms.

\begin{table}[htbp]
  \begin{tabularx}{\textwidth}{|*{4}{c|}X|}
    \hline
    \textbf{Application} & \textbf{Tool} & \textbf{\acs{TPL}}
      & \textbf{\acs{VPL}} & \textbf{Note}\\
    \hline
    \hline
    \multirow{5}{*}{AutoCAD \cite{Autodesk:1982:AutoCAD}}
      & \multirow{2}{*}{.NET \acs{API}\label{acro:API}}
      & \multirow{2}{*}{\checkmark} & \multirow{2}{*}{\xmark}
      & \multirow{2}{*}{\parbox{\linewidth}{
        Powerful, but very verbose; C\# \& VB.NET}}\\
      &&&& \\ \cline{2-5}
      & \multirow{2}{*}{\parbox{7em}{\centering ActiveX Automation}}
        & \multirow{2}{*}{\checkmark} & \multirow{2}{*}{\xmark}
        & \multirow{2}{*}{\parbox{\linewidth}{
          Deprecated, bundled separately; \acs{VBA}\label{acro:VBA}}}\\
      &&&& \\ \cline{2-5}
      & Visual LISP & \checkmark & \xmark & \acs{IDE}\label{acro:IDE};
        AutoLISP extension\\
    \hline
    Dynamo Studio
      & \multirow{2}{*}{Dynamo \cite{Keough:2012:Dynamo}}
      & \multirow{2}{*}{\checkmark} & \multirow{2}{*}{\checkmark}
      & \multirow{2}{*}{\parbox{\linewidth}{%
        Data flow paradigm; Associative programming support through
        DesignScript}}\\\cline{1-1}
    Revit \cite{RevitTechCorp:2002:Revit} &&&&\\
    \hline
    ArchiCAD \cite{Graphisoft:2018:ArchiCAD}
      & \multirow{2}{*}{Grasshopper \cite{Rutten:2018:Grasshopper}}
      & \multirow{2}{*}{\checkmark} & \multirow{2}{*}{\checkmark}
      & \multirow{2}{*}{\parbox{\linewidth}{%
        Data flow paradigm; Rhino \acs{SDK} access, C\# \& VB.NET}}\\\cline{1-1}
    \multirow{4}{*}{Rhinoceros3D \cite{McNeel:2018:Rhinoceros3D}}
      &&&& \\ \cline{2-5}
      & \multirow{2}{*}{Python Scripting} & \multirow{2}{*}{\checkmark}
        & \multirow{2}{*}{\xmark}
        & \multirow{2}{*}{\parbox{\linewidth}{%
          Simple language; Create custom Grasshopper components}}\\
      &&&&\\\cline{2-5}
      & RhinoScript & \checkmark & \xmark & VBScript based\\
    \hline
    \multirow{5}{*}{\texttt{Standalone$^\dag$}}
      & ImplicitCAD \cite{Longtin:2018:ImplicitCAD}
        & \checkmark & \xmark & Web hosted; OpenSCAD inspired\\\cline{2-5}
      & OpenJSCAD \cite{Mueller:2019:OpenJSCAD}
        & \checkmark & \xmark & Web hosted; JavaScript\\\cline{2-5}
      & OpenSCAD \cite{Kintel:2019:OpenSCAD}
        & \checkmark & \xmark & Solid 3D models; Simple \acs{DSL}\label{acro:DSL}\\\cline{2-5}
      & \multirow{2}{*}{Rosetta \cite{Leitao:2011:PGDCAD}}
        & \multirow{2}{*}{\checkmark} & \multirow{2}{*}{\xmark}
        & \multirow{2}{*}{\parbox{\linewidth}{%
          Portable tool; Multiple front- and back-end support}}\\
      &&&&\\
    \hline
  \end{tabularx}
  \scriptsize
  $^\dag$These tools are standalone software, i.e., not directly integrated into
  any specific \ac{CAD} application.
  \caption[Table of programmatic \acs{CAD}/\acs{BIM} and \acs{AD} software]{%
    \ac{CAD}/\ac{BIM} software with programmatic capabilities and \ac{AD}
    software/tools.  Added notes per tool shortly outline deemed significant
    characteristics.}
  \label{tab:related.ad.summary}
\end{table}

\subsection{Dynamo}
\label{sec:related.ad.dynamo}

An open source \ac{AD} tool available as a plug-in for Revit or by itself within
Dynamo Studio, Dynamo extends \ac{BIM} with the data and logic environment of a
graphical algorithm editor \cite{Keough:2012:Dynamo}.  Dynamo can be used
through both a \ac{VPL} and a \ac{TPL}, showcased in
\cref{fig:related.ad.dynamo.node2code}.

\begin{figure}[htbp]
  \includegraphics[width=\textwidth]{fig/dynamo-node-to-code}
  {\scriptsize
  Source: \url{http://primer.dynamobim.org/en/07_Code-Block/7-2_Design-Script-syntax.html}
  (Jan 2019)}
  \caption[Dynamo's visual interface with node to code translation]{%
    Showcase of Dynamo's visual interface containing a workflow that produces
    the model on the top left.  The figure also shows Dynamo's capability of
    converting a the workflow to a single DesignScript code block.}
  \label{fig:related.ad.dynamo.node2code}
\end{figure}

In its visual form, Dynamo offers a wide variety of functions, called nodes,
most of them capable of generating an even wider variety of geometry through
node combination, wiring one's outputs to another's inputs, and resorting to
pre-defined mutable parameters which can serve as some of the nodes' initial
inputs.  The workflow itself is the final product: a visual program, usually
designed to execute a specific task.  Dynamo further allows extension through the
creation of custom nodes which can be shared as packages.

One of the nodes in Dynamo, aptly named code block, allows the usage of a
\ac{TPL}; a language called DesignScript.  Originally developed my Robert Aish
\cite{Aish:2011:DesignScript}, DesignScript is a multi-paradigm domain-specific
language and is the programming language at the core of Dynamo itself.  So much
so that entire workflows can be reduced to one code block (see
\Cref{fig:related.ad.dynamo.node2code}).

DesignScript is an associative language, which maintains a graph of dependencies
with variables.  Executing a script will effectively propagate the variables'
values accordingly.  By default, code blocks in Dynamo follow an associative
paradigm.  The user can, however, switch to an imperative paradigm approach
instead effortlessly if needed.

This \textit{change-propagation} mechanism in DesignScript, consequently present
in Dynamo, makes Dynamo a great tool for dealing with constraints.  However,
most users might not fully exercise DesignScript's associative capabilities and
instead approach the problem with the mindset of an imperative programming
paradigm given its overwhelming presence in and adoption by major well-known
\acp{TPL}.

\subsection{Grasshopper}
\label{sec:related.ad.grasshopper}

Grasshopper is a graphical algorithm editor tightly integrated with
Rhinoceros3D, destined for designers who are exploring generative algorithms
\cite{Rutten:2018:Grasshopper}.  In spite of tight integration with Rhino, a
\ac{CAD} application, it is possible to use Grasshopper along with ArchiCAD
\cite{Graphisoft:2018:ArchiCAD,Graphisoft:2018:RGACAD}, a \ac{BIM} tool.
\Cref{fig:related.ad.grasshopper.islamic-pattern} shows a simple example of a
Grasshopper workflow.

\todo[inline]{Replace example with Rythmic Gynmastics Center, Moscow, Russia}

\begin{figure}[htbp]
  \includegraphics[width=\textwidth]{fig/grasshopper-islamic-pattern}
  {\scriptsize
  Source: \url{https://www.grasshopper3d.com/photo/islamic-pattern-parakeet}
  (Jan 2019)}
  \caption[Islamic Pattern in Grasshopper using Parakeet]{
    Islamic Pattern, by Esmaeil Mottaghi.  On top is the Grasshopper workflow to
    produce the pattern below it, aided by Parakeet
    \cite{Esmaeil:2018:Parakeet}.}
  \label{fig:related.ad.grasshopper.islamic-pattern}
\end{figure}

It is a closed-source product, designed by David Rutten and developed by McNeel
and Associates, Rhino's developers.  Its \ac{VPL} is as simple to use as
Dynamo's, which is crucial for users who are not familiar with programming using
a \ac{TPL}.  Nonetheless, it offers a \ac{TPL} alternative by way of custom
programmatic components.  Using C\# or VB.NET, the user can create custom code
components with access to Rhino's \ac{SDK} and OpenNURBS
\cite{Lear:2018:openNURBS} within Rhino.  Alternatively, through GhPython
\cite{Giulio:2017:GhPython}, the user can also write Python code.  Unlike
DesignScript, Python and the .NET languages don't support an associative
programming model.

Functions in Grasshopper are called components and work just like Dynamo's
nodes; a wide variety of them exist, most of them capable of producing geometry,
and they are composable, generating a workflow destined to accomplish a specific
task.

Both Dynamo and Grasshopper's visual approach suffer from the unproportionate
scalability between the code and the respective model's complexity
\cite{Leitao:2014:PESLGD}.  Sophisticated modelling workflows tend to become
difficult to properly represent, and harder for a human to efficiently interpret
when compared to a textual approach.  This disadvantage, however, is mitigated
with their respective \ac{TPL} alternatives.


% !TEX root = ../main.tex
\fancychapter{Solution}%
\label{chap:solution}
\cleardoublepage{}

\todo[inline]{{\bfseries TODO}: Vastly improve on this\ldots}

\noindent
Despite strides in enhancing performance and efficiency of geometric constraint
solving approaches, briefly discussed in \Cref{sec:intro.constraints}, the core
issue lies in the generality of geometric constraint solvers.  Although several
approaches employ efficient methods to find a solution, they resort to solving
potentially well-known problems generically when computationally lighter
solutions exist.  Instead of delegating the problem to a solver, a more
efficient approach would be to pinpoint the type of geometric constraint itself,
specializing a solution for several applicable entities.  Take the tangency
constraint as an example: positioning two circles tangent to each other or a
line tangent to an ellipse.  Depending on the inputs, these constraints might
have particularly efficient solutions for each case, in kind making the
computation more efficient.

Classical numerical methods constitute alluring alternatives to the predominant
graph-based approaches.  Having been studied for several decades, even if the
provided solution does not encompass all the possible values within the
problem's domain, they can be used to target specific problems efficiently.
Nonetheless, these suffer from robustness issues discussed in
\cref{sec:related.robustness}, effectively yielding inaccurate solutions if
precautions aren't taken.  A similar argument can be made about algebraic
methods.

This work aims to implement a series of geometric constraint primitives in an
already expressive \ac{TPL} to overcome the need for the specification of
unnecessary details when modeling geometrically constrained entities, promoting
an easier and more efficient usage.  Choosing to implement these in a \ac{TPL}
further avoids the poor scalability with increasing code complexity that arises
from what could be analogous specifications in a \ac{VPL}, a subject previously
discussed in \cref{sec:intro.ad}.

Moreover, by relying on an exact geometry computation library, one of the core
features of this solution lies in the capability of transparently dealing with
plenty of the previously addressed robustness issues.  The user can then resort
to these primitives, and, by composing them, elegantly specify the set of
geometric constraints necessary in order to produce the idealized model.  Since
the available primitives will implement specialized solutions for a finite set
of shapes the user can utilize in whichever combination possible during the
design process, the solution will be exempt of a generic solver component,
potentially boosting performance of design generation.

\Cref{fig:solution.arch} shows the typical \ac{AD} workflow and how the proposed
solution could be integrated with the \ac{AD} tool.  The encapsulated modules in
the figure represent the underlying computation library as an external
component, featuring the geometric constraint primitives library and the code
wrapping the computation library.

\begin{figure}[htbp]
  \includegraphics[width=\textwidth]{fig/solution-arch}
  \caption[Solution architecture within \acs{AD} workflow]{
    General overview of the solution's architecture encapsulated within the
    blue colored box beneath a depiction of the typical \ac{AD} workflow.}%
  \label{fig:solution.arch}
\end{figure}

The following sections go over the components in \cref{fig:solution.arch},
namely the \textit{Exact Computation Geometric Library}, the \textit{Wrapper
Code}, and the \textit{Geometric Constraint Primitives}.  Additionally, we
discuss a few trade-offs from tackling the problem in this fashion as opposed to
potential alternative routes, describing advantages and disadvantages of our
approach.

% !TEX root = ../../main.tex
\section{Implementation}%
\label{sec:solution.impl}

This section details implementation choices with regard to the chosen platforms
for realizing the initially proposed general solution architecture, previously
illustrated in \cref{fig:solution.arch}. Following a brief analysis, we expand
specifically on the concrete components corresponding to the ones within the
light blue rectangle.

Examining the \ac{AD} workflow portion of \cref{fig:solution.arch}, there are
depictions of \ac{CAD}, \ac{BIM}, and analysis tools, of which examples could be
Rhinoceros3D, Autodesk's Revit, and Radiance, respectively, with no particular
focus on any of them.  Digging a layer deeper, we find the \ac{AD} tool, which,
by means made available by the tools above it, produces models specific to those
tools from a description provided by the user.  The \ac{AD} tool we've chosen
was Khepri~\cite{Leitao:2019:GRUGEAV}, a text-based tool written in the Julia
programming language~\cite{Bezanson:2017:JAFANC}.  Khepri is the successor of
another text-based \ac{AD} tool named Rosetta~\cite{Leitao:2011:PGDCAD}, a tool
written in the Racket programming language~\cite{PLT:2010:Reference}.  It
follows that the \textit{Geometric Constraint Primitives} were implemented in
the Julia language as well, supported by an \textit{Exact Computation Geometry
Library}.  The library chosen for the effect was the
\acf{CGAL}~\cite{CGAL:2018}, a highly performant and robust geometric library
written in the C++ programming language~\cite{Stroustrup:2013:CPP}.

This language disparity between the \textit{Geometric Constraint Primitives}
module and the \textit{Exact Computation Geometry Library} requires a solution
for language interoperation.  In other words, we need to make \ac{CGAL}
available to the Julia language.  Fortunately, the Julia language already
possesses facilities that allow it to invoke functionality within libraries
written in the C~\cite{Kernighan:1988:C} or the
Fortran~\cite{Backus:1957:Fortran} programming languages.  This interfacing
mechanism is commonly known as \ac{FFI}.  It allows for the repurposing of
mature software libraries in foreign languages without the need for a complete
rewrite or adaptation.\footnote{The decision to include such a mechanism at the
language's core by the language designers makes it so the language can rapidly
evolve by avoiding reimplementing several facilities and software libraries in,
but not limited to, scientific and numerical computing areas.  Arguably, it may
be one of \textit{the} fundamental features that made the language as popular as
it is and kept it afloat, unlike other similar historical examples that might've
lacked such a mechanism.  \todo[inline]{Maybe expand on this with a concrete
example or so.  Could also be removed altogether}} This mechanism can also in
turn be leveraged and built upon to interface with other programming languages,
e.g., Java, Python, MATLAB, and, the one needed for our particular use-case,
C++.\footnote{There is an entire GitHub organization with projects dedicated to
foreign language interoperation at \url{https://github.com/JuliaInterop} (July
8, 2021)}

Overcoming the language interoperability hurdle, we can now start focusing on
the implementation of the \textit{Geometric Constraint Primitives}.
\todo[inline]{Elaborate a little bit more. Consider referring examples in
related work}

Adopting a bottom-up approach, we'll start by describing the underlying 
\textit{Exact Computation Geometry Library}, followed by the \textit{Wrapper
Code} layer, topped off by the \textit{Geometric Constraint Primitives} block.
The reason a wrapper code layer exists 

\section{Trade-offs}%
\label{sec:solution.tradeoffs}

\todo[inline]{Still not sure what to include here, but a section going over a
couple of issues circling the monstruous complexities of wrapping a gargantuan
library that CGAL proves to be a daunting task which, for simplicity's sake,
required hiding and pre-setting a lot of things on the C++ side of things.
Contrast this with the yet maturing Julia geometry ecosystem, which is proving
to be going somewhere, but it is still relatively young compared to things like
CGAL.  However, also illustrate that there are geometric Julia packages that
would be good candidates for replacing CGAL.

Additionally, explain why an approach using CxxWrap.jl was chosen, requiring an
explicit C++ wrapper library to hook into, which requires manual-ish compilation
and production, instead of using Cxx.jl, which can be used to inline C++ code
within Julia.  The former was chosen vs. the latter for what seemed like
stability reasons at the time.  The CxxWrap.jl approach seemed less complicated
despite the extra step of producing a C++ code shim that can then be fed into
CxxWrap.jl.}

Since virtually anything comes without trade-offs and compromises, it is
paramount we address our implementation's qualities, negative and positive.

Relying on a library such as \ac{CGAL} proves to be as great as it can be
daunting.  As mentioned in \cref{sec:solution.impl.cgal}, \ac{CGAL} is a very
comprehensive and mature software library, arguably even far exceeding our
solution's needs, yet fitting it perfectly.

It is, however, an external component, and with every such component, we do not
hold as much control over it if it was internal instead.  For example, in the
advent a bug is found within \ac{CGAL}, one cannot \textit{immediately} fix it
by altering its source code and use this fixed version.  Important emphasis
on bugs are not \textit{immediately} fixable lest we forget \ac{CGAL} is still
an Open Source project arguably anyone can contribute to.  Alas, said
contribution deployments are still out of our control.  In hindsight, however,
it can be considered just an inconvenience since it is a project that is
actively maintained by certainly more knowledgeable people in the computer
graphics and mathematics fields.

\Ac{CGAL} is also a highly generic library, making use and further abusing C++
templates.  Although its design makes usage an elegant experience (as elegant as
C++ can be), the same cannot be said with as much \texttt{gusto} when trying to
wrap its constructs to another language, especially a language with different
memory management paradigm which could lead to some nasty low-level ordeals.

\todo[inline]{Here come the trade-offs of how CGAL.jl was created.  We look at
CxxWrap specifically for aiding us in solving hurdles w.r.t. mapping C++
constructs, such as templates, memory management troubles, etc. However, we
opaquely map the geometric entities, hiding the kernel away, limiting it to an
inexact constructions kernel that may lead to not-as-robust results, a hassle
because our shoddy alternative at the time was supplying a different shared
library with a different kernel: an approach made virtually impossible to adopt
due to Julia's precompilation mechanisms which are also leveraged by CxxWrap.
Had we gone the Cxx.jl route, or even bare ccall's, things might've been
different and we might've been able to switch between kernel, however more
troublesome.  This can also be considered future work, i.e., to transparently
map kernels and number types CGAL additionally offers to Julia as well.
Nonetheless, using exact computation all the time can also gravely impact
program performance since computation using exact constructs is slower than
using inexact constructs for which some operations are even implemented in
hardware. Hence, the latter are oftentimes enough for plenty of cases, including
ours, as we've found.}

\todo[inline]{Lastly, go over some trade-offs in the implementation of our
constraint primitives which might involve loss of robustness as well when
dealing with problems that require us to apply operations such as square roots.
It is important to note, however, that these operations, if possible, are
delayed as much as possible since construction should be the last step in the
algorithm.  The issue is more noticeable as results from some functions are
composed with each other, circling back to the round-off errors that may arise
do to accumulated error propagation.  Again, this could potentially be solved by
mapping the kernels and numeric types, giving the user a choice, as well as
pre-emptively \texttt{warning} or educating the user that, in the presence of
\texttt{garbage} going in, there will be \texttt{garbage} coming back out.
Regardless, it is also important to note (I think) that many, and I do mean
\textit{many}, of the primitives came from CGAL alone, leaving us with a
platform to build upon that didn't require much building at all, another boon of
our approach.}


% !TEX root = ../main.tex
\fancychapter{Evaluation}%
\label{chap:eval}
\cleardoublepage{}

\section{Benchmarks}%
\label{sec:eval.perf}\todo{Working title}
\todo[inline]{Proper introduction + actual text}
Benchmarks were performed on a Lenovo ThinkPad E595 laptop computer running an
Arch Linux\footnote{\url{https://archlinux.org}} environment using the
Linux\textsuperscript{\textregistered} 5.12.15-zen1
kernel\footnote{\url{https://github.com/zen-kernel/zen-kernel/tree/v5.12.15-zen1}}
with the following hardware specifications:
%
\begin{itemize}
  \acused{AMD}\label{acro:AMD}
  \acused{CPU}\label{acro:CPU}
  \item \ac{AMD} Ryzen\textsuperscript{\texttrademark} 5 3500U \ac{CPU} @
  2.1GHz\footnote{Base clock frequency.  Can boost up to 3.7GHz.};
  \acused{SO-DIMM}\label{acro:SO-DIMM}
  \acused{DDR}\label{acro:DDR}
  \acused{RAM}\label{acro:RAM}
  \item 1×16GB \ac{SO-DIMM} of \ac{DDR}4 \ac{RAM} @ 2400MT/s\footnote{The
  disparity between \textbf{T}ransfers per \textbf{s}econd (T/s) and
  \textbf{H}ert\textbf{z} (Hz) depends on the memory's transfer rate.  \acf{DDR}
  memory can perform two operations per clock cycle, which means its transfer
  rate is double its clock frequency, i.e., 1Hz = 2T/s.  At 2400MT/s, the
  respective \ac{DDR} memory's clock frequency would be 1200MHz.}.
\end{itemize}
%
Additionally, the following software versions were configured, installed, and
used:
%
\begin{itemize}
  \item Julia 1.6.2;
  \item Maxima 5.45.1;
  \item Racket 8.2.
\end{itemize}

\subsection{ConstraintGM}%
\label{sec:eval.perf.cgm}
\todo[inline]{Introduce Fábio's work with ConstraintGM, point out the mishap of
feeling overconfident about maxima and how that prompted him, per suggestion, to
implement specific/contextual solutions for geometric constraint problems, i.e.,
what we did, instead of solely relying on a generic algebraic solver.  Recall
his benchmarks and compare them with identical benchmarks with our solutions.
Discuss the possibility that language differences might be the reason behind
performance differences alone.}

\subsection{VoronoiDelaunay.jl}%
\label{sec:eval.perf.vdjl}
\todo[inline]{Maybe leave the "how easy it is to leverage our approach to get
more, both in quantity and complexity, algorithsm from CGAL" discussion for this
part and evaluate the potential time it takes to use Voronoi Diagrams from CGAL
and either get more information on how long it took to implement
VoronoiDelaunay.jl and compare its "correctness" vs. CGAL's version of the
algorithm, assuming CGAL's results as a source of truth for correctness.
Testing revealed diagrams differed slightly when it came to some edges.  My
suspicion was the Julia algorithm "gobble" some very very very small triangles,
i.e., where the edges are very close together, but it only happens near the
outside of the diagram, and, again, a more drastic diagram can be manufactured
to showcase this.  Maybe read up more on the underlying algorithm that was
implemented in VoronoiDelaunay.jl}

\section{Case Studies}%
\label{sec:eval.studies}
\todo[inline]{Nexus article evaluation, mostly.  Maybe a bit more elaborate,
maybe a bit less.}

\subsection{Egg}%
\label{sec:eval.studies.egg}

\subsection{Rounded Trapezoid}%
\label{sec:eval.studies.rtrapezoid}

\subsection{Star with Semicircles}%
\label{sec:eval.studies.star}

\subsection{Voronoi Diagram}%
\label{sec:eval.studies.voronoi}

% !TEX root = ../main.tex
\fancychapter{Conclusion}%
\label{chap:conclusion}
\cleardoublepage{}

\noindent The generation of sophisticated designs is not viable through the
usage of interactive interfaces due to rigidity in the manipulation of existing
models in order to generate multiple variants.  This is where \ac{AD} comes in.
Even then, working with \acp{GC}, whether in a \ac{VPL} or a \ac{TPL}, imposes a
set of challenges, which can be overcome by resorting to \ac{GCS} approaches.
Alas, said approaches typically resort to generic \acs{GCS} algorithms, which
means \ac{GC} solvers, in general, have difficulties identifying specific
underlying subproblems with efficiently computable and robust solutions.

Nonetheless, the prior analysis of the set of \acp{GC} that must be
dealt with requires background knowledge on numerical robustness to mitigate
fixed-precision arithmetic issues, such as roundoff error accumulation.
Moreover, there is the added requirement of researching solutions to these
specific constraint problems and subsequent implementation of the corresponding
algorithms.  The user will end up spending more time and effort in this process
than in the design process itself.

Thus, in order to overcome these obstacles, an alternative approach is proposed
in the form of the implementation of \ac{GC} primitives supported by an exact
geometric computation library.  The latter provides a series of optimized
geometric algorithms and exact data structures that allow transparent handling
of robustness issues, lifting this concern from the user's shoulders, thus
facilitating the design process.

By supporting our solution on top of \ac{CGAL}, we transitively ensure its
correctness since \ac{CGAL} has decades of research, experience, and maintenance
behind it.  Consequently, development time is vastly reduced, choosing to
repurpose existing geometric constructs and functionality instead of
re-implementing them from scratch, leaving no room for error there.
Additionally, our solution's performance proves to be superior compared to
projects that implement some of the same functionality we managed to repurpose.
As a side effect, we opened the door for extending more functionality from
\ac{CGAL}.  These reasons suffice to validate our approach.

Finally, we proved the approach employed by our solution is one that creates
understandable programs that can be reproduced by hand, using a straightedge and
a compass.  By adopting a constructive approach to geometry specification, we
externalize and clarify the steps required to build geometric objects.  This is
contrasted with the more natural analytical approach programming languages
usually beg for.  Following the latter approach is not only more cumbersome due
to the solution derivation process, but it also produces less tangible programs,
hiding the concrete geometry behind formulas.  The former is preferred by
\ac{AD} practitioners, as well as industry professionals in general, but it
proves alluring to novice users all the same.  Said novice users might be
starting to learn and adopt \ac{AD}.  With our work, we aim to bolster this
adoption rate, driving more and more people to novel design paradigms.

\pdfbookmark[1]{Future Work}{future-work}
\section*{Future Work}

Our solution certainly has some drawbacks and misfeatures that could be
improved.  Some were already discussed in \cref{sec:solution.tradeoffs}.  To
briefly reiterate a few, our wrapper around the underlying library is less
transparent than desired.  Constructs and functionality should be mapped as
transparently as possible, fully parameterized, as to provide the user with more
control and choice over the constructs they are using.  There is potential for
this as \texttt{CxxWrap.jl} allows the mapping of parametric types.  We have
already leveraged these mechanisms and expanded on our mappings of the 2D
Triangulations and Voronoi Diagram Adaptor packages from \ac{CGAL} in
\texttt{CGAL.jl}.  More such work should follow for the aforementioned opaquely
mapped types.

Furthermore, the set of geometric primitives that were implemented could be
further expanded on.  This could initially be done by attempting to reach parity
with tools, like \texttt{tkz-euclide}, from which we can draw inspiration; or
even with more popular \ac{CAD} systems that industry professionals are familiar
with.  Hence, we would be providing a familiar interface to said professionals.
Once again, now with \ac{CGAL} in tow, we can explore the library further,
looking for additional algorithms that can complement this potential interface.
For example, there are packages within \ac{CGAL} that implement 2D and 3D
Boolean Operations, a set of operations all too familiar to \ac{CAD} system
users, that could be repurposed.

To divert focus from \ac{CGAL} for a moment, one should also look into other
geometry libraries as well.  We mentioned the Julia package ecosystem is still
catching up, but there is already a lot of work put into certain packages that
make their usage attractive since, for instance, we do not have to communicate
across languages, avoiding some issues we were met with, e.g., lack of
parametrization.

Finally, this work focused exclusively on constrained geometry bound to the
$\mathbb{R}^2$ Euclidean space, i.e., the 2D plane.  There is still much work to
be done researching problem solutions that encompass 3D space as well.
Solutions to problems once formulated in the 2D plane are not generally
applicable in 3D space, as the corresponding 3D problems may now be
under-constrained.  As an example, our solution for the circumcenter problem,
seen first in \cref{sec:intro.examples.circumcenter}, does not work in 3D space.
A line's bisector in 3D space is a plane, and noncoplanar nor parallel plane
intersection results in a line.  To obtain the actual circumcenter, one would
additionally, for example, have to intersect the resulting line with the plane
the circumscribed triangle sits on.  Among problems like this one, many more
exist, but much like we just illustrated, so do solutions.  To start, one could
yet again choose to explore \ac{CGAL} for additional 3D constructs and
functionality.

Going from the 2D plane into 3D space seems simple on paper.  But much like the
jump from sketching on paper to projecting skyscrapers, the dimensional gap is
wider than it initially seemed.


\begin{acks}
% !TEX root = ../main.tex
\begin{acknowledgments}
To anyone and everyone who supported and bore with me: thank you.
\end{acknowledgments}

\end{acks}

\bibliographystyle{ACM-Reference-Format}
\bibliography{main}

\appendix

\section{Appendix}

\begin{listing}[htb]
  \caption{\label{lst:solution.impl.jlcgal.jlcxx}
    C++ wrapper around the functionality required for recreating the example
    from \cref{lst:solution.impl.cgal.pas} in Julia.}
  \inputminted[breaklines]{cpp}{cpp/cgal_julia.cpp}
\end{listing}

\begin{listing}[htb]
  \caption{\label{lst:solution.impl.jlcgal.cgal}
    Example Julia module, wrapping the library produced from
    \cref{lst:solution.impl.jlcgal.jlcxx}.}
  \inputminted[breaklines]{julia}{jl/CGAL.jl} 
\end{listing}

\end{document}
\endinput

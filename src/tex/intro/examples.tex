% !TEX root = ../../main.tex
\subsection{\acl{GC} Problem Examples}%
\label{sec:intro.examples}

This section presents two simple examples of geometric models that are defined
through \acp{GC}, and the respective solutions using algebraic formulation,
accompanied by programmatic solutions using \acs{TikZ}~\cite{Tantau:2021:TikZ}.
Depictions of the models can be seen in \cref{fig:intro.example}.  The first
problem is that of a parallelism constraint, while the second problem is a
circumscription constraint.

\begin{figure}[htpb]
  \subcaptionbox{\label{fig:intro.example.parallel}%
    Line that goes through $C$, strictly parallel to $\overleftrightarrow{AB}$.}
    [.45\linewidth]{\resizebox{\linewidth}{!}{%
      \begin{tikzpicture}[rotate=20]
  \tkzDefPoints{0/0/A,3/0/B,1/1/C}
  \tkzDefLine[parallel=through C](A,B) \tkzGetPoint{D}
  \tkzDrawLines[add=.1 and .1](A,B C,D)
  \tkzDrawPoints(A,B,C)
  \tkzLabelPoints(A,B,C)
\end{tikzpicture}
}}
  \hspace{\fill}
  \subcaptionbox{\label{fig:intro.example.circumcenter}%
    $\odot O_r$ circumscribed about $\triangle ABC$.}
    [.45\linewidth]{\resizebox{\linewidth}{!}{%
      \begin{tikzpicture}
  \tkzDefPoints{0/0/A,4/0/B,1/3/C}
  \tkzCircumCenter(A,B,C) \tkzGetPoint{O}
  \tkzDefMidPoint(A,B) \tkzGetPoint{AB}
  \tkzDefMidPoint(A,C) \tkzGetPoint{AC}
  \tkzDefMidPoint(B,C) \tkzGetPoint{BC}
  \tkzDrawSegments[style=dashed](AB,O AC,O BC,O)
  \tkzMarkRightAngles(A,AB,O B,BC,O C,AC,O)
  \tkzDrawPolygon(A,B,C)
  \tkzDrawCircle(O,A)
  \tkzDrawSegment(O,A)
  \tkzDrawPoints(A,B,C,O)
  \tkzLabelLine[above](O,A){$r$}
  \tkzLabelPoints[below left](A)
  \tkzLabelPoints[below right](B)
  \tkzLabelPoints[above left](C)
  \tkzLabelPoints(O)
\end{tikzpicture}
}}
  \caption[Geometric models defined using GCs]{
    Geometric models defined using \ac{GC} relations:
    \subref{fig:intro.example.parallel} line parallelism, and
    \subref{fig:intro.example.circumcenter} circle circumscription.}%
  \label{fig:intro.example}
\end{figure}

\subsubsection{Parallel lines}%
\label{sec:intro.examples.parallel}

Let $A,\, B,\, C \in \mathbb{R}^2$ such that $C$ is a point in the line which is
strictly parallel to the line $\overleftrightarrow{AB}$ (see
\cref{fig:intro.example.parallel}).

A line in $\mathbb{R}^2$ can be described by the parametric equation
\begin{equation}\label{eq:line.parametric.2}
  P_Q = Q + \lambda\vec{u} \Rightarrow
  \begin{cases}
    x = x_Q + \lambda u_x \\
    y = y_Q + \lambda u_y
  \end{cases},\,\lambda \in \mathbb{R}.
\end{equation}
To then describe the line that goes through $C$ and is parallel to
$\overleftrightarrow{AB}$, one must compute the base point $Q$, trivially $C$,
and the vector $\vec{u}$, which can be obtained from $\overleftrightarrow{AB}$.

\Cref{lst:intro.example.parallel.tikz} shows the code used to produce the
example shown in \cref{fig:intro.example.parallel} using \acs{TikZ} with the
\texttt{tkz-euclide}\footnote{\url{https://ctan.org/pkg/tkz-euclide}} \LaTeX{}
package.

\begin{listing}[htb]
  \inputminted[highlightlines=3]{latex}{tikz/ex-parallel.tikz}
  \caption[Parallel lines example using \texttt{tkz-euclide}]{
    Parallel lines example from \cref{fig:intro.example.parallel} using
    \texttt{tkz-euclide}.}%
  \label{lst:intro.example.parallel.tikz}
\end{listing}

\subsubsection{Circumcenter}%
\label{sec:intro.examples.circumcenter}

Let $A,\, B,\, C,\, O \in \mathbb{R}^2$ be points such that $O$ is the center
point of a circle of radius $r$, $\odot O_r$, that is circumscribed about the
triangle $\triangle ABC$ (see \cref{fig:intro.example.circumcenter}).

To draw $\odot O_r$, we need its center $O$, which results from intersecting
the perpendicular bisectors of the triangle's edges; and its radius $r$, which
is the distance from $O$ to any of $\triangle ABC$'s vertices.  Said bisectors
can be described by~\cref{eq:line.parametric.2}, where $P$ is the midpoint
between the vertices, and $\vec{u}$ is a vector normal to the edge.  The normal
vector $\vec{n}$ is such that, for some vector $\vec{u}$,
\[
  \vec{u} \cdot \vec{n} = (u_x, u_y) \cdot (v_x, v_y) = u_x v_x + u_y v_y = 0.
\]
A vector $\vec{n} \in \mathbb{R}^2$ normal to another vector $\vec{u}$ can be
easily obtained by swapping the components of $\vec{u}$ while negating one of
them.

Let $M_{AB},\, M_{AC},\, M_{BC} \in \mathbb{R}^2$ be the midpoints of the
respective edges, and $\vec{u_1},\,\vec{u_2},\,\vec{u_3}$ the edges' normal
vectors, such that
\[
  \begin{split}
    P_{M_{AB}} = M_{AB} + \lambda_1 \vec{u_1} \\
    P_{M_{AC}} = M_{AC} + \lambda_2 \vec{u_2} \\
    P_{M_{BC}} = M_{BC} + \lambda_3 \vec{u_3} \\
  \end{split},\,\lambda_i \in \mathbb{R}.
\]
This problem can be further simplified by eliminating one of the redundant
bisectors.  Say we discard the mediator of line $\overleftrightarrow{BC}$.  We
then require that
\[
  P_{M_{AB}} = P_{M_{AC}} \stackrel{\eqref{eq:line.parametric.2}}{\Rightarrow}
  \begin{cases}
    x_{M_{AB}} + \lambda_1 u_{1x} = x_{M_{AC}} + \lambda_2 u_{2x} \\
    y_{M_{AB}} + \lambda_1 u_{1y} = y_{M_{AC}} + \lambda_2 u_{2y} \\
  \end{cases}.
\]
Every variable is known except for $\lambda_1$ and $\lambda_2$, but the equation
system can be solved in order to determine their values.  Finally, we can define
$O$ using one of the mediator line equations.  Using $P_{M_{AB}}$, for instance,
we have
\[
  O = M_{AB} + \lambda_1 \vec{u_1}.
\]

\Cref{lst:intro.example.circumcenter.tikz} shows the code used to produce the
example in \Cref{fig:intro.example.circumcenter} using \acs{TikZ} with the
\texttt{tkz-euclide} \LaTeX{} package.

\begin{listing}[htb]
  \inputminted[highlightlines=3]{latex}{tikz/ex-circumcenter.tikz}
  \caption[Circumcenter example using TikZ]{
    Circumcenter example from \cref{fig:intro.example.circumcenter} using
    \acs{TikZ} alongside \texttt{tkz-euclide}.}%
  \label{lst:intro.example.circumcenter.tikz}
\end{listing}

The language used to produce the examples' solutions provides a sensible set of
constraint primitives.  However, the syntax required for describing the models
is outdated, rigid, and may cause confusion.  Nonetheless, the underlying ideas
can be repurposed and adapted, implementing them in a modern and more expressive
language.

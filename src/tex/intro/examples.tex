% !TEX root = ../../main.tex
\section{Geometric Constraint Problem Examples}%
\label{sec:intro.examples}

This section presents two simple examples of geometric models that are defined
through the specification of \acp{GC}, and the respective solutions using
intuitive algebraic formulation, accompanied by programmatic solutions using
\acs{TikZ}~\cite{Tantau:2015:tikz-manual} and Eukleides~\cite{Obrecht:2010:EM}.
Depictions of the aforementioned models can be seen in \Cref{fig:intro.example}.
The examples are limited to the two-dimensional Euclidean plane over real
numbers, $\mathbb{R}^2$.  Solutions for analogous problems in three-dimensional
Euclidean space, $\mathbb{R}^3$, exist as well.

The first problem is that of a parallelism constraint: specifying a line that
goes through a given point while also being strictly parallel to another already
defined line.  The second problem is a circumscription constraint: defining a
circle that tightly wraps around a triangle, i.e., the circle's circumference
goes through three given non-collinear points.

\begin{figure}[htbp]
  \centering
  \subfloat[Line that goes through $C$, strictly parallel to
      $\protect\overleftrightarrow{AB}$]{
    \label{fig:intro.example.parallel}
    \resizebox{.4\textwidth}{!}{\begin{tikzpicture}[rotate=20]
  \tkzDefPoints{0/0/A,3/0/B,1/1/C}
  \tkzDefLine[parallel=through C](A,B) \tkzGetPoint{D}
  \tkzDrawLines[add=.1 and .1](A,B C,D)
  \tkzDrawPoints(A,B,C)
  \tkzLabelPoints(A,B,C)
\end{tikzpicture}
}}
  \hspace{\fill}
  \subfloat[$\odot O_r$ circumscribed about $\triangle ABC$]{
    \label{fig:intro.example.circumcenter}
    \resizebox{.4\textwidth}{!}{\begin{tikzpicture}
  \tkzDefPoints{0/0/A,4/0/B,1/3/C}
  \tkzCircumCenter(A,B,C) \tkzGetPoint{O}
  \tkzDefMidPoint(A,B) \tkzGetPoint{AB}
  \tkzDefMidPoint(A,C) \tkzGetPoint{AC}
  \tkzDefMidPoint(B,C) \tkzGetPoint{BC}
  \tkzDrawSegments[style=dashed](AB,O AC,O BC,O)
  \tkzMarkRightAngles(A,AB,O B,BC,O C,AC,O)
  \tkzDrawPolygon(A,B,C)
  \tkzDrawCircle(O,A)
  \tkzDrawSegment(O,A)
  \tkzDrawPoints(A,B,C,O)
  \tkzLabelLine[above](O,A){$r$}
  \tkzLabelPoints[below left](A)
  \tkzLabelPoints[below right](B)
  \tkzLabelPoints[above left](C)
  \tkzLabelPoints(O)
\end{tikzpicture}
}}
  \caption[Geometric models defined using \acsp{GC}]{
    Geometric models defined using \ac{GC} relations:
    \subref{fig:intro.example.parallel} showcases line parallelism, and
    \subref{fig:intro.example.circumcenter} showcases a circle circumscription
    about a triangle.}%
  \label{fig:intro.example}
\end{figure}

\subsection{Parallel lines}%
\label{sec:intro.examples.parallel}

Let $A,\,B,\,C \in \mathbb{R}^2$ such that $C$ is a point in the line which is
strictly parallel to the line $\overleftrightarrow{AB}$ (see
\Cref{fig:intro.example.parallel}).

A line in $\mathbb{R}^2$ can be described by the parametric equation
%
\begin{equation}\label{eq:line.vector.2}
  P_Q = Q + \lambda \vec{u} \Rightarrow \begin{cases}
    x = x_Q + \lambda u_x\\
    y = y_Q + \lambda u_y
  \end{cases}{\hskip -2.5ex}
  ,\,\lambda \in \mathbb{R}
\end{equation}
%
where $Q = (x_Q, y_Q)$ is a point on the line that goes through $P_Q = (x, y)$,
and $\vec{u} = (u_x, u_y)$ is the vector that drives the line.  To then describe
the line that goes through $C$ and is parallel to $\overleftrightarrow{AB}$, one
must compute the base point $Q$, trivially $C$, and the directional vector
$\vec{u}$, which can be obtained from $\overleftrightarrow{AB}$.  Let $Q = C$,
and $\vec{u} = B - A$, such that
\[
  P_C = C + \lambda \vec{u},\,\lambda \in \mathbb{R}.
\]

\Cref{lst:intro.example.parallel.tikz} shows the code used to produce the
example shown in \Cref{fig:intro.example.parallel} using TikZ with the
\texttt{tkz-euclide} \LaTeX{} package, using \texttt{tkzDefLine}, which takes
two points $A,\,B$, with the \texttt{parallel} transformation option.  This
option takes the point $C$ the resulting line goes through.  The result is a
point $D = C + \vec{u}$, which can be obtained using \texttt{tkzGetPoint} to
later draw the line. 

\begin{listing}[htbp]
  \inputminted[highlightlines=3]{latex}{tikz/ex-parallel.tex}
  \caption[Parallel lines example from \Cref{fig:intro.example.parallel}
    using \texttt{tkz-euclide}]{
      Parallel lines example from \Cref{fig:intro.example.parallel} using
      \texttt{tkz-euclide}.  The highlighted line shows how to define the line
      $L_C$ parallel to $\overleftrightarrow{AB}$.}
  \label{lst:intro.example.parallel.tikz}
\end{listing}

\Cref{lst:intro.example.parallel.euk} shows the code used to produce an
identical figure using Eukleides.  In Eukleides, the parallel line $L_C$ can be
obtained through the \texttt{parallel} function, which takes the line
$\overleftrightarrow{AB}$ it is parallel to and the point $C$ it goes through.

\begin{listing}[htbp]
  \inputminted[highlightlines=3]{text}{euk/ex-parallel.euk}
  \caption[Parallel lines example from \Cref{fig:intro.example.parallel}
    using Eukleides]{
      Parallel lines example from \Cref{fig:intro.example.parallel} using
      Eukleides.  The highlighted line shows how to define the line $L_C$
      parallel to $\overleftrightarrow{AB}$.}
  \label{lst:intro.example.parallel.euk}
\end{listing}

\begin{comment}
We can determine if two lines are parallel by determining the angle $\theta$
between them, and verifying it is equal to $0$.
%
\begin{equation}\label{eq:angle.vectors.2}
  \theta = \arccos \frac{\vec{u} \cdot \vec{v}}%
                        {||\vec{u}|| \cdot ||\vec{v}||},~%
  \theta \in \mathbb{R}.
\end{equation}
%
Having $\vec{v} = \lambda \vec{u}, \lambda \in \mathbb{R}$, and knowing
%
\begin{equation}\label{eq:dot.vector.2.same}
  \vec{u} \cdot \vec{u} = ||\vec{u}||^2,
\end{equation}
%
then \eqref{eq:angle.vectors.2} becomes
%
\[
  \begin{split}
    \theta & = \arccos \frac{\vec{u} \cdot \lambda\vec{u}}%
                            {||\vec{u}|| \cdot ||\lambda\vec{u}||}\\%
           & = \arccos \frac{\lambda\vec{u} \cdot \vec{u}}%
                            {\lambda||\vec{u}||^2}\\%
           & = 0.
  \end{split}
\]
%
This means that we can compute a directional vector for $CS$ from the line $AB$,
where $\vec{u} = B - A$.  Finally, we can obtain the equation for the parallel
line $CS$
\end{comment}

\subsection{Circumcenter}%
\label{sec:intro.examples.circumcenter}

Let $A,\,B,\,C,\,O \in \mathbb{R}^2$ be points such that $O$ is the center point
of a circle of radius $r$, $\odot O_r$, that is circumscribed about the triangle
$\triangle ABC$ (see \Cref{fig:intro.example.circumcenter}).

A precondition for this computation is that $\triangle ABC$ is not degenerate,
i.e., its vertices are non-collinear.  That can be verified by computing the
cross product of any two distinct vectors that drive $\triangle ABC$'s edges and
verifying it does not equate to $0$, a computation which in turn can be done by
computing the determinant of the matrix whose columns are the aforementioned
vectors.
%
\begin{comment}
The Laplace expansion for the determinant of a generic matrix $A \in
\mathbb{R}^{n \times n}$ is given by
%
\begin{align}
  \det(A) = \begin{vmatrix}
              a_{11} & \cdots & a_{1n} \\
              \vdots & \ddots & \vdots \\
              a_{n1} & \cdots & a_{nn}
            \end{vmatrix}
          &= a_{1j} C_{1j} + \cdots + a_{nj} C_{nj}%
            = a_{i1} C_{i1} + \cdots + a_{in} C_{in} \nonumber \\
          &= \sum_{i'=1}^n a_{i'j}C_{i'j}%
            \label{eq:matrix.det.laplace.nxn.col} \\
          &= \sum_{j'=1}^n a_{ij'}C_{ij'}%
            \label{eq:matrix.det.laplace.nxn.row},
\end{align}
%
where $i,j \in [1,n] \subset \mathbb{N}$,
%
\begin{equation}\label{eq:matrix.cofactor}
  C_{ij} = (-1)^{i+j} \det(A_{ij})
\end{equation}
%
is the $i,j$ cofactor of $A$, and
%
\begin{equation}\label{eq:matrix.minor}
  A_{ij} = \begin{bmatrix}
    a_{11}     & \cdots & a_{1j-1}     & a_{1j+1}     & \cdots & a_{1n}     \\%
    \vdots     & \ddots & \vdots       & \vdots       & \ddots & \vdots     \\%
    a_{i-1\,1} & \cdots & a_{i-1\,j-1} & a_{i-1\,j+1} & \cdots & a_{i-1\,1} \\%
    a_{i+1\,1} & \cdots & a_{i+1\,j-1} & a_{i+1\,j+1} & \cdots & a_{i+1\,1} \\%
    \vdots     & \ddots & \vdots       & \vdots       & \ddots & \vdots     \\%
    a_{n1}     & \cdots & a_{nj-1}     & a_{nj+1}     & \cdots & a_{nn}     \\
  \end{bmatrix}_{(n - 1) \times (n - 1)}
\end{equation}
%
is the minor of matrix $A$ without the $i$-th row and $j$-th column.  This can
be further simplified for a $2\times 2$ matrix.  Let $B\in\mathbb{R}^{2\times
2}$ be the matrix whose columns are the vectors $\vec{AB} = (a, b)$ and
$\vec{AC} = (c, d)$, and $i' = 1$, for instance, such that,
from~\eqref{eq:matrix.det.laplace.nxn.col},
%
\begin{equation}\label{eq:matrix.det.laplace.2x2.col}
  \det(B) = \begin{vmatrix}
              a & c\\
              b & d
            \end{vmatrix}
          = \sum_{j=1}^2 a_{1j} C_{1j}
          = (-1)^{1+1}a \cdot d + (-1)^{1+2}c \cdot b
          = ad - cb
\end{equation}
\end{comment}
%
\begin{comment}
Let $A \in \mathbb{R}^{2 \times 2}$ be the matrix whose columns are the vectors
$\vec{AB} = (a, b)$ and $\vec{AC} = (c, d)$, for instance, such that
%
\begin{equation}\label{eq:matrix.det.2x2}
  \det(A) = \begin{vmatrix}
              a & c\\
              b & d
            \end{vmatrix}%
          = ad - cb
\end{equation}
% 
If the determinant is found to be $0$, then there is no possible solution.
Otherwise, one can proceed to draw $\odot O_r$.
\end{comment}
%

To draw $\odot O_r$, we must compute both its center and radius.  Its radius $r$
can be trivially defined as the distance of the center $O$ to any of the
$\triangle ABC$'s vertices, i.e., $r = \overline{OA} = \overline{OB} =
\overline{OC}$.  To determine $O$, one must compute the intersection of the
perpendicular bisectors of the triangle's edges.  Said bisectors are the
mediators between an edge's vertices, which can be described
by~\eqref{eq:line.vector.2}, where $P$ is the midpoint between the vertices, and
$\vec{u}$ is a vector normal to the edge.  The midpoint $M_{P_1P_2}$ of two
points $P_1, P_2 \in \mathbb{R}$ is given by
%
\begin{equation}\label{eq:midpoint.points.2}
  M_{P_1P_2} = \frac{P_1 + P_2}{2}%
             = (\frac{x_1 + x_2}{2}, \frac{y_1 + y_2}{2}).
\end{equation}
%
Further, the scalar product of two vectors $\vec{u}, \vec{v} \in \mathbb{R}^2$
is given by
%
\begin{equation}\label{eq:vector.dot.2}
  \vec{u} \cdot \vec{v} = (u_x, u_y) \cdot (v_x, v_y) = u_x v_x + u_y v_y.
\end{equation}
%
The normal vector $\vec{n}$ is such that, for some vector $\vec{u}$,
%
\[
  \vec{u} \cdot \vec{n} = 0.
\]
%
We can easily obtain a normal vector $\vec{n} \in \mathbb{R}^2$ by swapping its
components while negating one of them.
%
\begin{comment}
This comes as a direct result from applying a rotation transformation of 90
degrees, or $\pi/2$ radians, to $\vec{u}$, like so
%
\[
  \vec{n} = R(\pi/2)\vec{u}%
  = \begin{bmatrix}
      \cos(\pi/2) & -\sin(\pi/2) \\
      \sin(\pi/2) & \cos(\pi/2)
    \end{bmatrix}
    \begin{bmatrix}
      u_1 \\ u_2
    \end{bmatrix}
  = \begin{bmatrix}
      0 & -1 \\
      1 & 0
    \end{bmatrix}
    \begin{bmatrix}
      u_1 \\ u_2
    \end{bmatrix}
  = \begin{bmatrix}
      -u_2 \\ u_1
    \end{bmatrix}.
\]
\end{comment}
%
Let $\vec{u},\,\vec{n} \in \mathbb{R}^2$, such that $\vec{u} = (u_x,
u_y),\,\vec{n} = (-u_y, u_x)$.  From~\eqref{eq:vector.dot.2}, we have
\[ \vec{u} \cdot \vec{n} = u_x u_y - u_y u_x = 0.  \]
%
Computing the edges' midpoints and respective normal vectors, we can then
describe the mediators.  Let $M_{AB},\,M_{AC},\,M_{BC} \in \mathbb{R}^2$ be the
midpoints of the respective edges, and $\vec{u_1},\,\vec{u_2},\,\vec{u_3}$ the
edges' normal vectors, such that
\[
  \begin{split}
    P_{M_{AB}} = M_{AB} + \lambda_1 \vec{u_1}\\
    P_{M_{AC}} = M_{AC} + \lambda_2 \vec{u_2}\\
    P_{M_{BC}} = M_{BC} + \lambda_3 \vec{u_3}
  \end{split}
  ,\,\lambda_i \in \mathbb{R}.
\]
This problem can be further simplified by eliminating one of the redundant
bisectors.  Since the intersection of two lines already yields a single point,
we can eliminate one of the equations.  Say we discard the mediator of line
$\overleftrightarrow{BC}$.  We then require that
\[
  P_{M_{AB}} = P_{M_{AC}} \Rightarrow \begin{cases}
    x_{M_{AB}} + \lambda_1 u_{1x} = x_{M_{AC}} + \lambda_2 u_{2x} \\
    y_{M_{AB}} + \lambda_1 u_{1y} = y_{M_{AC}} + \lambda_2 u_{2y}
  \end{cases}{\hskip -2ex}.
\]
Every variable is known except for $\lambda_1$ and $\lambda_2$, but the equation
system can be solved in order to assign values to both of them since we have
exactly two equations that relate them.  Finally, we can define $O$ using one of
the equations with the respectively found $\lambda$, i.e., using $L_{M_{AB}}$,
for instance, we have
\[
  O = M_{AB} + \lambda_1 \vec{u}.
\]

\Cref{lst:intro.example.circumcenter.tikz} shows the code used to produce the
example in \Cref{fig:intro.example.circumcenter} using \acs{TikZ} with the
\texttt{tkz-euclide} \LaTeX{} package.  To compute the center point of $\odot
O_r$, one can use \texttt{tkzCircumCenter}, which takes three points
$A,\,B,\,C$, and generates the result $O$, obtainable using
\texttt{tkzGetPoint}.

\begin{listing}[htbp]
  \inputminted[highlightlines=3]{latex}{tikz/ex-circumcenter.tex}
  \caption[Circumcenter example from \Cref{fig:intro.example.circumcenter}
      using \acs{TikZ}]{
    Circumcenter example from \Cref{fig:intro.example.circumcenter} using
    \acs{TikZ} alongside \texttt{tkz-euclide}.  The highlighted line shows how
    to obtain the center of $\odot O_r$ via the non-degenerate triangle
    $\triangle ABC$.}
  \label{lst:intro.example.circumcenter.tikz}
\end{listing}

\Cref{lst:intro.example.circumcenter.euk} shows the code
that produces an identical figure using Eukleides.  In Eukleides, one can use
the \texttt{circle} function, which similarly takes three points $A,\,B,\,C$,
and generates the circle $\odot O_r$ circumscribed about $\triangle ABC$, while
$O$ can be obtained using the \texttt{center} function.

\begin{listing}[htbp]
  \inputminted[highlightlines=2]{text}{euk/ex-circumcenter.euk}
  \caption[Circumcenter example from \Cref{fig:intro.example.circumcenter}
      using Eukleides]{
    Circumcenter example from \Cref{fig:intro.example.circumcenter} using
    Eukleides.  The highlighted line shows how to obtain the center of $\odot
    O_r$ via the non-degenerate triangle $\triangle ABC$.}
  \label{lst:intro.example.circumcenter.euk}
\end{listing}

Both languages used to produce the examples' solutions provide a sensible set of
constraint primitives.  However, in the particular case of \texttt{tkz-euclide},
the syntax required for describing the models is outdated, rigid, and may cause
confusion.  For example, in
\cref{lst:intro.example.parallel.tikz,lst:intro.example.circumcenter.tikz},
command results can not be used directly in other commands as inputs and must
instead be obtained using another command to create a permanent symbol
associated with the resulting value.  By contrast, functions and expressions'
results in modern languages can be used directly as well as stored by using a
far friendlier assignment syntax.  Nonetheless, the underlying ideas can be
repurposed and adapted, implementing them in a modern and more expressive
language.

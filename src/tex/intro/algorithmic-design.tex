% !TEX root = ../../main.tex
\subsection{\acl{AD}}%
\label{sec:intro.ad}

In spite of the improved usability and pervasiveness of parametric features in
modern \ac{CAD} applications, approaches reliant on these tools tend to not
scale well with design complexity.  Applying changes to existing models becomes
cumbersome and users end up wasting time and effort tweaking parameter values,
which is an error-prone process.

\Ac{AD} consists in the generation of models through the specification of
algorithmic descriptions~\cite{McCormack:2004:GDPDR}.  The parametric nature of
algorithmic specifications already implicitly constrains the model since
dependencies within the description change if an ancestor parameter's value is
changed.

Such an approach also lead to the creation and integration of programming tools
into existing \ac{CAD} and \ac{BIM} software such as
Grasshopper\footnote{\url{https://www.grasshopper3d.com}} for
Rhinoceros\footnote{\url{https://www.rhino3d.com}} or
Dynamo\footnote{\url{https://dynamobim.org}} for
Revit\footnote{\url{https://autodesk.com/revit}}.  Some tools, like
Rosetta~\cite{Leitao:2011:PGDCAD}, offer a distinctly portable solution,
enabling the generation of several identical models for a variety of different
tools from a single specification~\cite{CasteloBranco:2017:IAD}.

Despite the benefits that come with the integration of \ac{AD} tools in \ac{CAD}
and \ac{BIM} software, it is key that these tools also provide an expressive
platform to boost user productivity.  This means these tools should provide
capabilities that make them easier to create complex models and
designs~\cite{Leitao:2014:IGDAGHOP}.  The more expressive the platform is, the
better it is with respect to usage, making it easier to learn.  This becomes all
the more important when generating a constrained geometric model.  Thus, the
inclusion of \ac{GC} concepts in such tools would make working with constraints
easier, in turn mitigating error propagation throughout the algorithm,
increasing the tool's expressive power.

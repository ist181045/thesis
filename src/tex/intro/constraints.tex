% !TEX root = ../../main.tex
\section{Constraints in CAD}%
\label{sec:intro.constraints}

We have briefly seen how parametric operations in \ac{CAD} software have
evolved.  These operations allow the user to create geometric objects that
satisfy certain constraints \textit{implicitly} imposed on the objects when the
user selects the operations they want to use along with the respective
operation's inputs.  Naturally, \ac{GCS} fits well in \ac{CAD} applications.
\Acp{GC} allow the repositioning and scaling of geometric objects so that they
satisfy constraints \textit{explicitly} imposed on them by the user.

\acp{CSP} are a well-known subject of research both in mathematics and in the
\ac{CS} field.  \ac{GCS} is a subclass of \acp{CSP}. More specifically, it is a
\ac{CSP} in a computational geometry setting.  The abstract problem of \ac{GCS}
is often described as follows~\cite[pp.~6]{Bettig:2011:GCSPC}:

\begin{quote}
  {\itshape\color{gray}
  Given a set of geometric objects, such as points, lines, and circles; a set of
  geometric and dimensional constraints, such as distance, tangency, and
  perpendicularity; and an ambient space, usually the Euclidean plane; assign
  coordinates to the geometric objects such that the constraints are satisfied,
  or report that no such assignment has been found.}
\end{quote}

One of the important features of a solver is its \textit{competence}, which is
related to the capability of reporting unsolvability: if in fact no solution for
the problem at hand exists and the solver is capable of reporting unsolvability
in that case, the solver is deemed fully competent.  Since constraint solving is
mostly an exponentially complex problem~\cite{Rossi:2006:Handbook}, partial
competence suffices as long as decent solutions can be found in affordable time
and space.

There are multiple approaches to constraint solving, but the most relevant ones
are graph-based, logic-based, algebraic, and theorem prover-based, of which the
first is the predominant one.  It is important for these approaches that the
\ac{GC} system does not have too few or too many constraints.  Summarily, a
system can either be 
\begin{enumerate*}[label= (\arabic*)]
  \item under-constrained if the number of solutions is unbound due to lack of
  constraint coverage over the entities involved,
  \item over-constrained if there are no solutions because of constraint
  contradictions, or
  \item well-constrained if the number of solutions is bound to a finite
  positive number.
\end{enumerate*}

Some of the subjects approached here are briefed in~\cite{Hoffmann:2005:BCS}.
The following sections present and briefly discuss the aforementioned approaches
to constraint solving.

\subsection{Graph-Based Approaches}%
\label{sec:intro.constraints.graph}

In graph-based approaches, the problem is translated into a labeled
\textit{constraint graph}, where vertices are constrained geometric objects, and
edges the constraints themselves.  This approach is split into three main
branches:

\begin{itemize}
  \item[] \textbf{Constructive Approaches} The graph is decomposed and
  recombined to extract basic construction steps that must be solved.  In a
  subsequent phase, this is elaborated upon by employing algebraic and/or
  numerical methods.  This has become the dominant approach to \ac{GCS}, also
  becoming the target of considerable research and development.

  \item[] \textbf{Degrees of Freedom Analysis} The graph's vertices are labeled
  with the degrees of freedom of the represented object.  Each edge is labeled
  by the degrees of freedom the constraint cancels out.  This graph is then
  analyzed for a solution strategy.

  A symbolic solution method is derived using rules with geometric meaning, a
  method proved to be correct in~\cite{Kramer:1990:SGCS}.  It is further
  extended by using it along with numerical methods as a fallback if geometric
  reasoning fails~\cite{Hsu:1997:HCSEIGC}.

  \Citet{Latham:1996:CA} decompose the graph into minimal connected components
  they call \textit{balanced sets} that are solved by a geometric construction,
  falling back to a numerical solution attempt.  This method can deal with
  symbolic constraints and identifies under- and overconstrained problems, where
  the latter kind is approached by prioritizing the given constraints.

  \item[] \textbf{Propagation Approaches} The graph's vertices represent
  variables and equations, and the edges are labeled with occurrences of the
  variables in equations.  The goal is to orient the graph such that all
  incident edges to an equation vertex but one are incoming edges.  If so, the
  equation system has been triangularized.  Orientation algorithms include
  degree-of-freedom propagation and propagation of known
  values~\cite{Freeman:1990:ICS,Veltkamp:1992:Geometric} which can fail in the
  presence of orientation loops, but such situations are
  addressed~\cite{Veltkamp:1992:Geometric} and they may resort to numerical
  solvers.
\end{itemize}

\subsection{Logic-Based Approaches}%
\label{sec:intro.constraints.logic}

Using logic-based approaches, the constraint problem is translated into a set of
geometric assertions and axioms which is then transformed in such a way that
specific solution steps are made explicit by applying geometric reasoning.  The
solver then takes a set of construction steps and assigns coordinate values to
the geometric entities.

A geometric locus\footnote{In mathematics, a locus is a set of points that
satisfy some condition.  In layman's terms, a location or place.} at which
constrained elements must be is obtained using first order logic to derive
geometric information, applying a set of axioms from Hilbert's
geometry~\cite{Aldefeld:1988:VGBGRM,Sohrt:1991:IC3DM,Bruderlin:1993:USGRRSGSS}.
Two different types of constraints are further
considered~\cite{Sunde:1987:CADSDSS,Verroust:1992:RMPCAD}:
\begin{enumerate*}[label= (\arabic*)]
  \item sets of points placed with respect to a local coordinate frame, and
  \item sets of straight line segments whose directions are fixed.
\end{enumerate*}
The reasoning is performed by applying a rewriting system on the sets of
constraints.  Once every geometric element is in a unique set, the problem is
solved.

\subsection{Algebraic Approaches}%
\label{sec:intro.constraints.algebraic}

In the case of an algebraic approach, the problem is translated into a system of
equations where the variables are coordinates of geometric elements and the
equations, which are generally nonlinear, express the constraints upon them.
This approach's main advantage is its completeness and dimension independence.
However, it is difficult to decompose the equation system into subproblems, and
a general, complete solution of algebraic equations is inefficient.
Nonetheless, small algebraic systems tend to appear in the other approaches and
are routinely solved.

\subsection{Symbolic Methods}%
\label{sec:intro.constraints.symbolic}

Symbolic methods rely on general equation solvers which employ techniques to
triangularize equation systems~\cite{Chou:1988:IWMMTPG,Buchberger:1995:Grobner}
that emerge from employing an algebraic approach.  A solver built on top of the
Buchberger's algorithm is described in~\cite{Buchanan:1993:CDS};
\Citet{Kondo:1992:AMMDRGM} further reports on a symbolic algebraic method.

These methods can produce generic solutions which can be evaluated for a
different set of constraint assignments, then producing parameterized solutions.
However, solvers are very slow and computation demands a lot of space, usually
requiring exponential running time~\cite{Durand:1998:SNTCS}.

\subsection{Numerical Methods}%
\label{sec:intro.constraints.numerical}

Among the oldest approaches to constraint solving, numerical methods solve large
systems of equations iteratively.  Methods like Newton iteration work properly
if a good approximation of the intended solution can be supplied and the system
is not ill-conditioned.  Take, for example, a sketch of a model the user drew.
If the starting point comes from said sketch, then it should follow that the
result be close to what is intended.  Alas, such methods may find only one
solution, even in cases where there are many, and may not allow the user to
select the one they are interested in.  Such methods are called local methods,
as opposed to global methods, exploring the problem space for every possible
solution.

Relaxation
methods~\cite{Sutherland:1964:Sketchpad,Hillyard:1978:CNSTDT,Borning:1989:PLATL}
can be employed in attempts to partly minimize global error by perturbing the
values assigned to the variables.  However, in general, convergence to a
solution is slow.

The Newton-Raphson iteration method, the most widely used one, is a local method
and converges much faster than relaxation, but does not apply to
over-constrained systems of equations unless expanded
upon~\cite{Dedieu:2000:Newton}.

Global and guaranteed convergence can be had resorting to the \textit{Homotopy
continuation} family of methods~\cite{Allgower:1993:CPF}.  Despite usage in
\ac{GCS}~\cite{Lamure:1996:SGCH,Durand:1998:SNTCS}, these are far less efficient
than the Newton-Raphson method due to the latter's exhaustive nature.

\subsection{Theorem Proving}%
\label{sec:intro.constraints.proving}

\ac{GCS} can be seen as a subproblem of geometric theorem proving, but the
latter requires general techniques, therefore requiring much more complex
methods than those required by the former.

Wen-Tsün Wu's method~\cite{Wu:1984:BPMCTPG,Wu:1994:MTPG} is an algebraic-based
method that can be used to automatically find necessary conditions to obtain
non-degenerated solutions.  It can be used to prove novel geometric
theorems~\cite{Chou:1988:IWMMTPG}.
\Citet{Chou:1996:AGRPGI:1,Chou:1996:AGRPGI:2} develop on automatic geometric
theorem proving, allowing the interpretation of the computed proof.

\subsection{Other Areas}%
\label{sec:intro.constraints.other}

The following are briefly described key advances made during the past two
decades that interface with other areas or that cannot be readily integrated
into graph-constructive solvers.  These techniques also constitute examples of
further attempts to broaden the scope of \ac{GCS}, proving that it is a strong
field of research with many applications beyond \ac{CAD}.

\begin{itemize}
  \item[] \textbf{Deformations} When restrictions are placed on the type of
  deformation, these problems can be seen as constraint solving.  For
  example, \citet{Ahn:2014:GCQBCUML,Bao:2010:BIVCMSE,Moll:2006:PPDLO} consider
  deformations that minimize strain energy; \Citet{Xu:2009:SDUAC} entail
  surface deformation under area constraints.  However, such techniques are
  rarely integrated with other \acp{GC} such as point distance or
  perpendicularity.
  \item[] \textbf{Dynamic Geometry} The addition of constraints to a given
  under-constrained system can make it well-constrained, and such constraints
  can be seen as parameters when they are dimensional.  Varying their values,
  different solutions arise, which can be wholly understood as a dynamic
  geometric configuration.  Systems akin to
  Cinderella~\cite{Richter:2012:Cinderella.2} can deal with these problems.
  Further literature exists on these problems from a constraint solving
  perspective~\cite{Freixas:2010:CDGS}.
  \item[] \textbf{Evolutionary Methods} Consist of re-interpreting the problem
  as an optimization problem, attacking it using genetic, particle-swarm or
  other evolutionary methods~\cite{Chunhong:2006:PDBOEA,Li:2012:HASPSOASGCP}.
\end{itemize}

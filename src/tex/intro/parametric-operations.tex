% !TEX root = ../../main.tex
\section{Parametric Operations in CAD}%
\label{sec:intro.parametric}

Ivan Sutherland introduced the world to
Sketchpad~\cite{Sutherland:1964:Sketchpad} in 1963, an interactive 2D \ac{CAD}
program. Despite never using the word \textit{parametric} in writing,
Sutherland's Sketchpad was capable of establishing atomic constraints between
objects which had all the essential properties of parametric equations, being
the first of its kind and the prime ancestor of modern \ac{CAD} programs.  The
earliest 3D system~\cite{Requicha:1980:RRS:356827.356833} dates from the 1970s.
It used a \ac{CSG}~\cite{Requicha:1977:CSG,Foley:1996:CGPP} binary tree, and
\ac{B-Rep}~\cite{Stroud:2006:BRMT} for representing solid objects.  This
system's parametric nature rested in the \ac{CSG} tree, which acted as a
rudimentary construction step history.  The user could make modifications to the
controlling parameters' values of a certain operation in the tree, reapply the
modified history, and generate the newly updated model.  Surfacing nearly a
decade later, Pro/ENGINEER~\cite{PTC:1980:ProENGINEER,Jabi:2013:PDA} was the
first system to be acknowledged as a parametric system.  It enabled the
establishment of relations between the objects' sizes and positions such that a
change in a dimension between objects would automatically change affected
objects accordingly.  Unlike Sketchpad, it supported 3D geometry and changes
would propagate over different drawings made by different users.  This amid a
sudden flux of activity and interest, \ac{GCS} soon became standard in drawings
by the early 1990s~\cite{Chung:1990:TEVPD,Owen:1991:ASGDC,Bouma:1995:GCS}.
Efforts to expand the benefits of constraint solving beyond simple sketches were
made, some systems having implemented constraint solving in 3D.  Improvements
from then on focused mostly on robustness and operation variety.

In recent decades, emphasis shifted to making parametric \ac{CAD} software more
interactive and user-friendly.  The intent was to make it as simple as dragging
a face of an object to where it should be instead of scrolling through a
construction history in attempts to locate a specific operation, and hopefully
changing the correct controlling parameter's value within that operation.  This
in itself is a tedious and error-prone process that can lead to undesired
side effects instead of producing the intended changes.  A variety of systems
have been developed to mitigate this
rigidity~\cite{Samuel:2006:CPPUP,Wu:2007:MSMSM,Clarke:2009:SM}, but not without
drawbacks, since direct-manipulation operations were just added to the
construction history as transformation operations, oblivious to parent
operations the new ones might depend on.  Further limitations are discussed
in~\cite{Bettig:2005:LPOSSD}, along with a proposal for future design software
exempt of parametric operations.  Nonetheless, parametric operations will still
see continued usage for the foreseeable future.

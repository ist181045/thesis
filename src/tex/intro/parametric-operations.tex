% !TEX root = ../../main.tex
\subsection{Parametric Operations in CAD}%
\label{sec:intro.parametric}

Ivan Sutherland introduced the world to
Sketchpad~\cite{Sutherland:1964:Sketchpad} in 1963, an interactive 2D \ac{CAD}
program.  Sutherland's Sketchpad was capable of establishing atomic constraints
between objects, being the first of its kind.  The earliest 3D
system~\cite{Requicha:1980:RRS:356827.356833} dates from the 1970s.  This
system's parametric nature rested on a construction history tree.  The user
could modify an operation's parameters' values, reapply the modified history,
and regenerate the model.  Nearly a decade later,
Pro/ENGINEER\footnote{\url{https://www.ptc.com/en/products/creo/pro-engineer}}~\cite{Jabi:2013:PDA}
surfaced, enabling the creation of relations between the objects' sizes and
positions such that a change in a dimension between objects would automatically
change affected objects accordingly.  \Ac{GCS} soon became standard in drawings
by the early 1990s~\cite{Owen:1991:ASGDC,Bouma:1995:GCS}.  Efforts to expand the
benefits of \ac{GCS} beyond simple sketches were made, some systems having
implemented constraint solving in 3D.  Improvements from then on focused mostly
on robustness and operation variety.

In recent decades, emphasis shifted towards making parametric \ac{CAD} software
more interactive and user-friendly.  The intent was to make it as simple as
dragging a face of an object to where it should be instead of locating and
modifying operations buried in a construction history.  This is a tedious and
error-prone process that can lead to undesired side effects.  Nonetheless,
parametric operations will still see continued usage for the foreseeable future.

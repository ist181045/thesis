% !TEX root = ../../main.tex
\section{Parametric Operations in \acs{CAD}}
\label{sec:intro.parametric}

Despite never using the word \textit{parametric} in writing, Ivan Sutherland
introduced the world to Sketchpad \cite{Sutherland:1964:Sketchpad} in 1963, an
interactive 2D \ac{CAD} program capable of establishing atomic constraints
between objects which had all the essential properties of a parametric equation,
being the first of its kind and the prime ancestor of modern \ac{CAD} programs.
The earliest 3D parametric system \cite{Requicha:1980:RRS:356827.356833} dates
from the 1970s.  It used a \ac{CSG} \cite{Foley:1996:CGPP,Requicha:1977:CSG}
binary tree, and \ac{B-Rep} \cite{Stroud:2006:BRMT} for representing solid
objects.  This system's parametric nature rested in the \ac{CSG} tree, which
acted as a rudimentary construction step history.  The user could make
modifications to the controlling parameters' values of a certain operation in
the tree, reapply the modified history, and generate the newly updated model.
Nearly a decade later, the first parametric system as it is understood today
\cite{Jabi:2013:PDA,PTC:1980:ProENGINEER} surfaced, enabling the establishment
of relations between the objects' sizes and positions such that a change in a
dimension between objects would automatically move or change affected objects
accordingly.  Unlike Sketchpad, it supported 3D geometry and changes would
propagate over different drawings made by different users.  This lead to the
appearance of dimensions and \acp{GC} in parametric operations, having \ac{GCS}
in drawings become standard by the early 1990s
\cite{Bouma:1995:GCS,Chung:1990:TEVPD,Owen:1991:ASGDC}.  Efforts to expand the
benefits of constraint solving beyond simple sketches were made, having the
majority of some systems implemented constraint solving in 3D.  Improvements
from then on focused mostly on robustness and operation variety.

In recent decades, emphasis shifted to making parametric \ac{CAD} software more
interactive and user friendly.  The intent was to make it as simple as dragging
a face of an object to where it should be instead of scrolling through a
construction history in attempts to locate a specific operation, and hopefully
changing the correct controlling parameter's value within that operation.  This
in itself is a tedious and error-prone process that can lead to undesired
side-effects instead of producing the intended changes.  A variety of systems
have been developed to mitigate this rigidity
\cite{Clarke:2009:SM,Samuel:2006:CPPUP,Wu:2007:MSMSM}, but not without
drawbacks, since direct-manipulation operations were just added to the
construction history as transformation operations, oblivious to parent
operations the new ones might depend on.  Further limitations are discussed in
\cite{Bettig:2005:LPOSSD}, along with a proposal for future design software
exempt of parametric operations.  Nonetheless, parametric operations will still
see continued usage for the foreseeable future.

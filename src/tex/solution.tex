% !TEX root = ../main.tex
\fancychapter{Solution}%
\label{chap:solution}
\cleardoublepage{}

\noindent Despite strides in enhancing the performance and efficiency of
\ac{GCS} approaches, briefly discussed in \cref{sec:intro.constraints}, the core
issue lies in the generality of \ac{GC} solvers.  Although several approaches
employ efficient methods to find a solution, they resort to solving potentially
well-known problems generically when computationally lighter solutions exist.
Instead of delegating the problem to a solver, a more efficient approach would
be to pinpoint the type of \ac{GC} itself, specializing a solution for several
applicable entities.  Take the tangency constraint as an example: positioning
two circles tangent to each other or a line tangent to an ellipse.  Depending on
the inputs, these constraints might have particularly efficient solutions for
each case, in kind making the computation more efficient.

Classical numerical methods constitute alluring alternatives to the predominant
graph-based approaches.  Having been studied for several decades, even if the
provided solution does not encompass all the possible values within the
problem's domain, they can be used to target specific problems efficiently.
Nonetheless, these suffer from the robustness issues discussed in
\cref{sec:related.robustness}, effectively yielding inaccurate solutions if
precautions are not taken.  A similar argument can be made about algebraic
methods.

This work aims to implement a series of \ac{GC} primitives in an already
expressive \ac{TPL} to overcome the need for the specification of unnecessary
details when modeling geometrically constrained entities, promoting an easier
and more efficient usage.  Choosing to implement these in a \ac{TPL} further
avoids the poor scalability with increasing code complexity that arises from
what could be analogous specifications in a \ac{VPL}, a subject previously
discussed in \cref{sec:intro.ad}.

In order to implement these primitives, we require basic geometric objects and
functions to operate in terms of points, lines, and circles, instead of plain
coordinates and algebraic formulations.  However, instead of implementing these
basic constructs, potentially introducing errors in the implementation, we
decided to support our primitives on an exact geometry computation library
already containing a wide gamut of data structures and functionality with
decades of research and active maintenance.

Moreover, by relying on an exact geometry computation library, one of the core
features of this solution lies in the capability of transparently dealing with
plenty of the previously addressed robustness issues.  The user can then resort
to the implemented primitives, and, by composing them, elegantly specify the set
of \acp{GC} necessary to produce the idealized model.  Since the available
primitives will implement specialized solutions for a finite set of shapes that
the user can utilize, in whichever combination possible, during the design
process, the solution will be exempt of a generic solver component, potentially
boosting performance of design generation.

\Cref{fig:solution.arch} shows the typical \ac{AD} workflow and how the proposed
solution could be integrated with the \ac{AD} tool.  The encapsulated modules in
the figure represent the underlying computation library as an external
component, featuring the \ac{GC} primitives library and the code wrapping the
computation library.

\begin{figure}[htb]
  \includegraphics[width=\textwidth]{fig/solution-arch}
  \caption[Solution architecture within the AD workflow]{
    General overview of the solution's architecture encapsulated within the
    blue colored box beneath a depiction of the typical \ac{AD} workflow.}%
  \label{fig:solution.arch}
\end{figure}

The following sections go over the components in \cref{fig:solution.arch},
namely the \geomlibrary{}, the \wrapper{}, and the \primitives{}.  Additionally,
we discuss a few trade-offs from tackling the problem in this fashion as opposed
to potential alternative routes, describing advantages and disadvantages of our
approach.

% !TEX root = ../../main.tex
\section{Implementation}%
\label{sec:solution.impl}

This section details implementation choices with regard to the chosen platforms
for realizing the initially proposed general solution architecture, previously
illustrated in \cref{fig:solution.arch}. Following a brief analysis, we expand
specifically on the concrete components corresponding to the ones within the
light blue rectangle.

Examining the \ac{AD} workflow portion of \cref{fig:solution.arch}, there are
depictions of \ac{CAD}, \ac{BIM}, and analysis tools, of which examples could be
Rhinoceros3D, Autodesk's Revit, and Radiance, respectively, with no particular
focus on any of them.  Digging a layer deeper, we find the \ac{AD} tool, which,
by means made available by the tools above it, produces models specific to those
tools from a description provided by the user.  The \ac{AD} tool we've chosen
was Khepri~\cite{Leitao:2018:Khepri.jl,Leitao:2019:GRUGEAV}, a text-based tool
written in the Julia programming language~\cite{Bezanson:2017:JAFANC}.  Khepri
is the successor of another text-based \ac{AD} tool named
Rosetta~\cite{Leitao:2011:PGDCAD}, a tool written in the Racket programming
language~\cite{PLT:2010:Reference}.  It follows that the \textit{Geometric
Constraint Primitives} were implemented in the Julia language as well, supported
by an \textit{Exact Computation Geometry Library}.  The library chosen for the
effect was the \acf{CGAL}~\cite{CGAL:2018}, a highly performant and robust
geometric library written in the C++ programming
language~\cite{Stroustrup:2013:CPP}.

This language disparity between the \textit{Geometric Constraint Primitives}
module and the \textit{Exact Computation Geometry Library} requires a solution
for language interoperation.  In other words, we need to make \ac{CGAL}
available to the Julia language.  Fortunately, the Julia language already
possesses facilities that allow it to invoke functionality within libraries
written in the C~\cite{Kernighan:1988:C} or the
Fortran~\cite{Backus:1957:Fortran} programming languages.  This interfacing
mechanism is commonly known as \ac{FFI}.  It allows for the repurposing of
mature software libraries in foreign languages without the need for a complete
rewrite or adaptation.\footnote{The decision to include such a mechanism at the
language's core by the language designers makes it so the language can rapidly
evolve by avoiding reimplementing several facilities and software libraries in,
but not limited to, scientific and numerical computing areas.  Arguably, it may
be one of \textit{the} fundamental features that made the language as popular as
it is and kept it afloat, unlike other similar historical examples that might've
lacked such a mechanism.} This mechanism can also in turn be leveraged and built
upon to interface with other programming languages, e.g., Java, Python, MATLAB,
and, the one needed for our particular use-case, C++.\footnote{There is an
entire GitHub organization with projects dedicated to foreign language
interoperation at \url{https://github.com/JuliaInterop} (July 8, 2021)}

Overcoming the language interoperability hurdle, we can now start focusing on
the implementation of the \textit{Geometric Constraint Primitives}.  These
primitives build on top of the functionality available in \ac{CGAL}, some of
which is directly inherited from it, substantially helping us in the process,
e.g., intersections.  We further enriched the pool with a few more functions,
illustrating a constructive approach to \ac{GCS}, similar and inspired by the
approach of \textit{tkz-euclide} mentioned in
\cref{sec:related.constraints.tikz}.  By providing this abstraction over more
primitive functionality, we aimed to provide an easy to understand and utilize
set of tools so users can avoid reimplementing it themselves, which is an
error-prone process.  as levelling the playing field by working at a
conceptual level which is more familiar to and understood by traditional
\ac{CAD} software users rather than falling back to the more analytic approach
programming languages naturally offer.

The following sections will elaborate further on the components emphasized in
the previous paragraphs, adopting a bottom-up-like approach.  We'll discuss
\ac{CGAL} and what constructs and functionality it can provide to aid our goal,
as well as some added benefits of building on top of a very mature and
comprehensive library.  That will be followed by a section detailing how it was
possible to map said functionality to the Julia language, of which the result
was a Julia package aptly named \texttt{CGAL.jl}\footnote{Packages in the Julia
ecosystem are conventionally terminated with a \texttt{.jl} suffix, the
extension used for Julia files.  This is reminiscent of a familiar convention
followed in the Java ecosystem where libraries and tools are usually prefixed
with the letter \texttt{J}, e.g., \texttt{JUnit}, \texttt{JMeter},
\texttt{JDeveloper}, among others.}~\cite{Ventura:2019:CGAL.jl}.  Finally, we
showcase how we leveraged \texttt{CGAL.jl} to build the aforementioned
\textit{Geometric Constraint Primitives}, a set of functionality that implements
specialized yet comprehensible constructive approach solutions to \ac{GC}
problems.

\subsection{Exact Computation Geometric Library}%
\label{sec:solution.impl.cgal}
\todo[inline]{Introduce CGAL to the masses.  Showcase example program using
basic CGAL constructs and functionality.  Not sure where to put notes about the
library's complexity, but maybe include explanation of why it made sense to
choose CGAL as our library of choice (mature library, tons of research resulting
from several PhDs, open source $\implies$ open to contributions, lots of
contributors, active development, \textit{et cetera}).}

\begin{listing}[htb]
  \inputminted{cpp}{cpp/points_and_segments.cpp}
  \caption[CGAL: Three points and one segment]{
    An example CGAL program illustrating how to construct some points and a line
    segment, and perform some basic operations on them.  It uses \texttt{double}
    precision floating point numbers for cartesian coordinates.
  }
  \label{lst:solution.impl.cgal.pas}
\end{listing}

\subsection{Wrapper Code}%
\label{sec:solution.impl.jlcgal}
\todo[inline]{Reiterate the language interoperability hurdle, maybe mention
briefly there are complications especially when memory management models differ.
Showcase Julia's native capabilities of invoking native C++ libraries coupled w/
an example using CGAL.  Introduce a library that helps with C++ wrapping and
showcase a slightly more complex example, maybe involving classes and the like.
Consider demonstrating the ease with which it is possible to wrap things
incrementally as, on demand, requests for new features may require algorithms
from the library that weren't wrapped.  Just like following a recipe, it's
as simple as (1) looking at the docs, (2) mapping necessary types and functions,
and (3) run Julia (Carefully not mapping \textit{everything}, thought that is
what I strived to do with CGAL.jl, but that's another discussion).}

\begin{listing}[htb]
  \inputminted{cpp}{cpp/sqdist.cpp}
  \caption[C wrapper for squared distance functionality]{
    Example C library code that wraps \ac{CGAL}'s \texttt{squared\_distance}
    global function.  The original function takes in instances of
    \texttt{Point\_3} classes so we instantiate them from our \texttt{double}
    coordinate inputs.}
  \label{lst:solution.impl.jlcgal.sqdist.cpp}
\end{listing}

\begin{listing}[htb]
  \inputminted{julia}{jl/sqdist.jl}
  \caption[Julia squared distance example program]{
    Example Julia program that invokes the functionality from the library whose
    source is listed in \cref{lst:solution.impl.jlcgal.sqdist.cpp}.  Julia's
    \texttt{ccall} construct converts the input arguments' types to the types
    specified in the native C function's parameter types.}
  \label{lst:solution.impl.jlcgal.sqdist.jl}
\end{listing}

\begin{listing}[htb]
  \inputminted{cpp}{cpp/circ.cpp}
  \caption[C wrapper for circumcenter functionality]{
    Example C shared library source code that wraps \ac{CGAL}'s circumcenter
    global function.  In this instance, we use an additional struct to wrap
    around \ac{CGAL}'s \texttt{Point\_3} class to facilitate data transfer.}
  \label{lst:solution.impl.jlcgal.circ.cpp}
\end{listing}

\begin{listing}[htb]
  \inputminted{julia}{jl/circ.jl}
  \caption[Julia circumcenter example program]{
    Example Julia program that invokes the functionality from the library listed
    in \cref{lst:solution.impl.jlcgal.circ.cpp}.  We use an additional Julia
    struct that's equivalent to the one specified in C to facilitate data
    transfer.}
  \label{lst:solution.impl.jlcgal.circ.jl}
\end{listing}

\begin{listing}[htb]
  \inputminted{julia}{jl/points_and_segments.jl}
  \caption[CGAL.jl: Three points and one segment]{
    The example program as seen in \cref{lst:solution.impl.cgal.pas} written in
    the Julia programming language using \texttt{CGAL.jl}.  The kernel
    instantiation is hidden away in the C++ layer of the wrapper code.}
  \label{lst:solution.impl.jlcgal.pas}
\end{listing}

\subsection{Geometric Constraint Primitives}%
\label{sec:solution.impl.gcps}
\todo[inline]{With functionality available on the Julia side of things, call
back to the previously formulated examples, potentially elaborating with yet
another slightly more complex example.  Showcase that some of the functionality
that can be implemented with very little effort relying only on functionality
already present in mature library alone like CGAL is.  I confess I do not know
what to discuss in this section, despite feeling like it is the most important
one in some regard\ldots}

\begin{listing}[htb]
  \inputminted[highlightlines={5,19}]{julia}{jl/ex-parallel.jl}
  \caption[Parallel lines example using our solution]{
    Implementation of the parallel lines example illustrated in
    \cref{fig:intro.example.parallel} using Khepri alongside our solution
    backed by \texttt{CGAL.jl}.}
  \label{lst:solution.impl.gcps.parallel}
\end{listing}

\begin{listing}[htb]
  \inputminted[highlightlines={2,19}]{julia}{jl/ex-circumcenter.jl}
  \caption[Circumcenter example using our solution]{
    Implementation of the circumcenter example illustrated in
    \cref{fig:intro.example.circumcenter} using Khepri alongside our solution
    backed by \texttt{CGAL.jl}.  In this particular case, we can leverage
    \ac{CGAL}'s facilities directly.}
  \label{lst:solution.impl.gcps.circumcenter}
\end{listing}

\section{Trade-offs}%
\label{sec:solution.tradeoffs}

\todo[inline]{Still not sure what to include here, but a section going over a
couple of issues circling the monstruous complexities of wrapping a gargantuan
library that CGAL proves to be a daunting task which, for simplicity's sake,
required hiding and pre-setting a lot of things on the C++ side of things.
Contrast this with the yet maturing Julia geometry ecosystem, which is proving
to be going somewhere, but it is still relatively young compared to things like
CGAL.  However, also illustrate that there are geometric Julia packages that
would be good candidates for replacing CGAL.

Additionally, explain why an approach using CxxWrap.jl was chosen, requiring an
explicit C++ wrapper library to hook into, which requires manual-ish compilation
and production, instead of using Cxx.jl, which can be used to inline C++ code
within Julia.  The former was chosen vs. the latter for what seemed like
stability reasons at the time.  The CxxWrap.jl approach seemed less complicated
despite the extra step of producing a C++ code shim that can then be fed into
CxxWrap.jl.}

Since virtually anything comes without trade-offs and compromises, it is
paramount we address our implementation's qualities, negative and positive.

Relying on a library such as \ac{CGAL} proves to be as great as it can be
daunting.  As mentioned in \cref{sec:solution.impl.cgal}, \ac{CGAL} is a very
comprehensive and mature software library, arguably even far exceeding our
solution's needs, yet fitting it perfectly.

It is, however, an external component, and with every such component, we do not
hold as much control over it if it was internal instead.  For example, in the
advent a bug is found within \ac{CGAL}, one cannot \textit{immediately} fix it
by altering its source code and use this fixed version.  Important emphasis
on bugs are not \textit{immediately} fixable lest we forget \ac{CGAL} is still
an Open Source project arguably anyone can contribute to.  Alas, said
contribution deployments are still out of our control.  In hindsight, however,
it can be considered just an inconvenience since it is a project that is
actively maintained by certainly more knowledgeable people in the computer
graphics and mathematics fields.

\Ac{CGAL} is also a highly generic library, making use and further abusing C++
templates.  Although its design makes usage an elegant experience (as elegant as
C++ can be), the same cannot be said with as much \texttt{gusto} when trying to
wrap its constructs to another language, especially a language with different
memory management paradigm which could lead to some nasty low-level ordeals.

\todo[inline]{Here come the trade-offs of how CGAL.jl was created.  We look at
CxxWrap specifically for aiding us in solving hurdles w.r.t. mapping C++
constructs, such as templates, memory management troubles, etc. However, we
opaquely map the geometric entities, hiding the kernel away, limiting it to an
inexact constructions kernel that may lead to not-as-robust results, a hassle
because our shoddy alternative at the time was supplying a different shared
library with a different kernel: an approach made virtually impossible to adopt
due to Julia's precompilation mechanisms which are also leveraged by CxxWrap.
Had we gone the Cxx.jl route, or even bare ccall's, things might've been
different and we might've been able to switch between kernel, however more
troublesome.  This can also be considered future work, i.e., to transparently
map kernels and number types CGAL additionally offers to Julia as well.
Nonetheless, using exact computation all the time can also gravely impact
program performance since computation using exact constructs is slower than
using inexact constructs for which some operations are even implemented in
hardware. Hence, the latter are oftentimes enough for plenty of cases, including
ours, as we've found.}

\todo[inline]{Lastly, go over some trade-offs in the implementation of our
constraint primitives which might involve loss of robustness as well when
dealing with problems that require us to apply operations such as square roots.
It is important to note, however, that these operations, if possible, are
delayed as much as possible since construction should be the last step in the
algorithm.  The issue is more noticeable as results from some functions are
composed with each other, circling back to the round-off errors that may arise
do to accumulated error propagation.  Again, this could potentially be solved by
mapping the kernels and numeric types, giving the user a choice, as well as
pre-emptively \texttt{warning} or educating the user that, in the presence of
\texttt{garbage} going in, there will be \texttt{garbage} coming back out.
Regardless, it is also important to note (I think) that many, and I do mean
\textit{many}, of the primitives came from CGAL alone, leaving us with a
platform to build upon that didn't require much building at all, another boon of
our approach.}


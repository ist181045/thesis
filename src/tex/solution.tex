% !TEX root = ../main.tex
\fancychapter{Solution}%
\label{chap:solution}
\cleardoublepage{}

\todo[inline]{{\bfseries TODO}: Vastly improve on this\ldots}

\noindent
Despite strides in enhancing performance and efficiency of geometric constraint
solving approaches, briefly discussed in \Cref{sec:intro.constraints}, the core
issue lies in the generality of geometric constraint solvers.  Although several
approaches employ efficient methods to find a solution, they resort to solving
potentially well-known problems generically when computationally lighter
solutions exist.  Instead of delegating the problem to a solver, a more
efficient approach would be to pinpoint the type of geometric constraint itself,
specializing a solution for several applicable entities.  Take the tangency
constraint as an example: positioning two circles tangent to each other or a
line tangent to an ellipse.  Depending on the inputs, these constraints might
have particularly efficient solutions for each case, in kind making the
computation more efficient.

Classical numerical methods constitute alluring alternatives to the predominant
graph-based approaches.  Having been studied for several decades, even if the
provided solution does not encompass all the possible values within the
problem's domain, they can be used to target specific problems efficiently.
Nonetheless, these suffer from robustness issues discussed in
\cref{sec:related.robustness}, effectively yielding inaccurate solutions if
precautions aren't taken.  A similar argument can be made about algebraic
methods.

This work aims to implement a series of geometric constraint primitives in an
already expressive \ac{TPL} to overcome the need for the specification of
unnecessary details when modeling geometrically constrained entities, promoting
an easier and more efficient usage.  Choosing to implement these in a \ac{TPL}
further avoids the poor scalability with increasing code complexity that arises
from what could be analogous specifications in a \ac{VPL}, a subject previously
discussed in \cref{sec:intro.ad}.

Moreover, by relying on an exact geometry computation library, one of the core
features of this solution lies in the capability of transparently dealing with
plenty of the previously addressed robustness issues.  The user can then resort
to these primitives, and, by composing them, elegantly specify the set of
geometric constraints necessary in order to produce the idealized model.  Since
the available primitives will implement specialized solutions for a finite set
of shapes the user can utilize in whichever combination possible during the
design process, the solution will be exempt of a generic solver component,
potentially boosting performance of design generation.

\Cref{fig:solution.arch} shows the typical \ac{AD} workflow and how the proposed
solution could be integrated with the \ac{AD} tool.  The encapsulated modules in
the figure represent the underlying computation library as an external
component, featuring the geometric constraint primitives library and the code
wrapping the computation library.

\begin{figure}[htbp]
  \includegraphics[width=\textwidth]{fig/solution-arch}
  \caption[Solution architecture within \acs{AD} workflow]{
    General overview of the solution's architecture encapsulated within the
    blue colored box beneath a depiction of the typical \ac{AD} workflow.}%
  \label{fig:solution.arch}
\end{figure}

The following sections go over the components in \cref{fig:solution.arch},
namely the \textit{Exact Computation Geometric Library}, the \textit{Wrapper
Code}, and the \textit{Geometric Constraint Primitives}.  Additionally, we
discuss a few trade-offs from tackling the problem in this fashion as opposed to
potential alternative routes, describing advantages and disadvantages of our
approach.

% !TEX root = ../../main.tex
\section{Implementation}%
\label{sec:solution.impl}

This section details implementation choices with regard to the chosen platforms
for realizing the initially proposed general solution architecture, previously
illustrated in \cref{fig:solution.arch}. Following a brief analysis, we expand
specifically on the concrete components corresponding to the ones within the
light blue rectangle.

Examining the \ac{AD} workflow portion of \cref{fig:solution.arch}, there are
depictions of \ac{CAD}, \ac{BIM}, and analysis tools, of which examples could be
Rhinoceros3D, Autodesk's Revit, and Radiance, respectively, with no particular
focus on any of them.  Digging a layer deeper, we find the \ac{AD} tool, which,
by means made available by the tools above it, produces models specific to those
tools from a description provided by the user.  The \ac{AD} tool we've chosen
was Khepri~\cite{Leitao:2019:GRUGEAV}, a text-based tool written in the Julia
programming language~\cite{Bezanson:2017:JAFANC}.  Khepri is the successor of
another text-based \ac{AD} tool named Rosetta~\cite{Leitao:2011:PGDCAD}, a tool
written in the Racket programming language~\cite{PLT:2010:Reference}.  It
follows that the \textit{Geometric Constraint Primitives} were implemented in
the Julia language as well, supported by an \textit{Exact Computation Geometry
Library}.  The library chosen for the effect was the
\acf{CGAL}~\cite{CGAL:2018}, a highly performant and robust geometric library
written in the C++ programming language~\cite{Stroustrup:2013:CPP}.

This language disparity between the \textit{Geometric Constraint Primitives}
module and the \textit{Exact Computation Geometry Library} requires a solution
for language interoperation.  In other words, we need to make \ac{CGAL}
available to the Julia language.  Fortunately, the Julia language already
possesses facilities that allow it to invoke functionality within libraries
written in the C~\cite{Kernighan:1988:C} or the
Fortran~\cite{Backus:1957:Fortran} programming languages.  This interfacing
mechanism is commonly known as \ac{FFI}.  It allows for the repurposing of
mature software libraries in foreign languages without the need for a complete
rewrite or adaptation.\footnote{The decision to include such a mechanism at the
language's core by the language designers makes it so the language can rapidly
evolve by avoiding reimplementing several facilities and software libraries in,
but not limited to, scientific and numerical computing areas.  Arguably, it may
be one of \textit{the} fundamental features that made the language as popular as
it is and kept it afloat, unlike other similar historical examples that might've
lacked such a mechanism.  \todo[inline]{Maybe expand on this with a concrete
example or so.  Could also be removed altogether}} This mechanism can also in
turn be leveraged and built upon to interface with other programming languages,
e.g., Java, Python, MATLAB, and, the one needed for our particular use-case,
C++.\footnote{There is an entire GitHub organization with projects dedicated to
foreign language interoperation at \url{https://github.com/JuliaInterop} (July
8, 2021)}

Overcoming the language interoperability hurdle, we can now start focusing on
the implementation of the \textit{Geometric Constraint Primitives}.
\todo[inline]{Elaborate a little bit more. Consider referring examples in
related work}

Adopting a bottom-up approach, we'll start by describing the underlying 
\textit{Exact Computation Geometry Library}, followed by the \textit{Wrapper
Code} layer, topped off by the \textit{Geometric Constraint Primitives} block.
The reason a wrapper code layer exists 

\section{Trade-offs}%
\label{sec:solution.tradeoffs}

\todo[inline]{Still not sure what to include here, but a section going over a
couple of issues circling the monstruous complexities of wrapping a gargantuan
library that CGAL proves to be a daunting task which, for simplicity's sake,
required hiding and pre-setting a lot of things on the C++ side of things.
Contrast this with the yet maturing Julia geometry ecosystem, which is proving
to be going somewhere, but it is still relatively young compared to things like
CGAL.  However, also illustrate that there are geometric Julia packages that
would be good candidates for replacing CGAL.

Additionally, explain why an approach using CxxWrap.jl was chosen, requiring an
explicit C++ wrapper library to hook into, which requires manual-ish compilation
and production, instead of using Cxx.jl, which can be used to inline C++ code
within Julia.  The former was chosen vs. the latter for what seemed like
stability reasons at the time.  The CxxWrap.jl approach seemed less complicated
despite the extra step of producing a C++ code shim that can then be fed into
CxxWrap.jl.}

Since virtually anything comes without trade-offs and compromises, it is
paramount we address our implementation's qualities, negative and positive.

Relying on a library such as \ac{CGAL} proves to be as great as it can be
daunting.  As mentioned in \cref{sec:solution.impl.cgal}, \ac{CGAL} is a very
comprehensive and mature software library, arguably even far exceeding our
solution's needs, yet fitting it perfectly.

It is, however, an external component, and with every such component, we do not
hold as much control over it if it was internal instead.  For example, in the
advent a bug is found within \ac{CGAL}, one cannot \textit{immediately} fix it
by altering its source code and use this fixed version.  Important emphasis
on bugs are not \textit{immediately} fixable lest we forget \ac{CGAL} is still
an Open Source project arguably anyone can contribute to.  Alas, said
contribution deployments are still out of our control.  In hindsight, however,
it can be considered just an inconvenience since it is a project that is
actively maintained by certainly more knowledgeable people in the computer
graphics and mathematics fields.

\Ac{CGAL} is also a highly generic library, making use and further abusing C++
templates.  Although its design makes usage an elegant experience (as elegant as
C++ can be), the same cannot be said with as much \texttt{gusto} when trying to
wrap its constructs to another language, especially a language with different
memory management paradigm which could lead to some nasty low-level ordeals.

\todo[inline]{Here come the trade-offs of how CGAL.jl was created.  We look at
CxxWrap specifically for aiding us in solving hurdles w.r.t. mapping C++
constructs, such as templates, memory management troubles, etc. However, we
opaquely map the geometric entities, hiding the kernel away, limiting it to an
inexact constructions kernel that may lead to not-as-robust results, a hassle
because our shoddy alternative at the time was supplying a different shared
library with a different kernel: an approach made virtually impossible to adopt
due to Julia's precompilation mechanisms which are also leveraged by CxxWrap.
Had we gone the Cxx.jl route, or even bare ccall's, things might've been
different and we might've been able to switch between kernel, however more
troublesome.  This can also be considered future work, i.e., to transparently
map kernels and number types CGAL additionally offers to Julia as well.
Nonetheless, using exact computation all the time can also gravely impact
program performance since computation using exact constructs is slower than
using inexact constructs for which some operations are even implemented in
hardware. Hence, the latter are oftentimes enough for plenty of cases, including
ours, as we've found.}

\todo[inline]{Lastly, go over some trade-offs in the implementation of our
constraint primitives which might involve loss of robustness as well when
dealing with problems that require us to apply operations such as square roots.
It is important to note, however, that these operations, if possible, are
delayed as much as possible since construction should be the last step in the
algorithm.  The issue is more noticeable as results from some functions are
composed with each other, circling back to the round-off errors that may arise
do to accumulated error propagation.  Again, this could potentially be solved by
mapping the kernels and numeric types, giving the user a choice, as well as
pre-emptively \texttt{warning} or educating the user that, in the presence of
\texttt{garbage} going in, there will be \texttt{garbage} coming back out.
Regardless, it is also important to note (I think) that many, and I do mean
\textit{many}, of the primitives came from CGAL alone, leaving us with a
platform to build upon that didn't require much building at all, another boon of
our approach.}


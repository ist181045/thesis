% !TEX root = ../main.tex
\section{Conclusion}%
\label{sec:conclusion}

Generating highly constrained complex designs is not viable using traditional
methods due to the rigidity in manipulating existing models to generate multiple
variants.  This is where \ac{AD} comes in.  However, working with \acp{GC} still
proves challenging.  \Ac{GCS} approaches can be employed to solve complex
constraint systems, but they mostly resort to generic algorithms that struggle
to identify specific problems for which efficient solutions exist.

Nonetheless, the prior analysis of the set of \acp{GC} that must be dealt with
requires certain background knowledge on numerical robustness to mitigate
fixed-precision arithmetic issues.  Moreover, there is the added requirement of
researching solutions to these specific constraint problems,  wasting the user's
time that could be spent in the design process itself.

Thus, an alternative approach is proposed in the form of the implementation of
\ac{GC} primitives in an expressive \ac{TPL} supported by an exact geometric
computation library.  The latter provides a series of optimized geometric
algorithms and exact constructs that can transparently handle robustness issues,
lifting this concern from the user's shoulders.

Finally, we showed our approach creates programs that are easy to understand and
reproduce.  By adopting a constructive approach, we clarify the steps required
to build geometric objects, as opposed to the analytical approach programming
usually begs for.  The latter is not only more cumbersome to follow, but it also
produces programs that hide geometry behind formulas.  The former is preferred
by \ac{AD} practitioners while also being alluring to new adopters, contributing
to increasing \ac{AD} adoption rate, driving more people away from the rigid
traditional methods.

\subsection*{Future Work}

Our solution certainly has some drawbacks, some of which were already discussed
in \cref{sec:solution.tradeoffs}.  To briefly reiterate a few, our wrapper
around \ac{CGAL} is still rather opaque.  Functionality should be transparently
mapped, providing the user with more control and choice over the constructs they
are using.  Furthermore, the set of implemented \ac{GC} primitives was quite
limited in size and could be further expanded on.  However, let the ones
implemented serve as an example for expansion.

This work focused exclusively on geometry in the 2D plane, leaving 3D space
vastly unexplored.  Elevating a dimension means the solutions to problems once
formulated in the 2D plane may no longer be applicable in 3D space for some
problems may now be under-constrained.  Moving from 2D to 3D will certainly
elevate our work, like going from sketching on paper to projecting buildings.

% !TEX root = ../main.tex
\section{Conclusion}%
\label{sec:conclusion}

Generating highly complex designs is not viable using traditional
methods due to the rigidity in manipulating existing models to generate multiple
variants.  This is where \ac{AD} comes in.  However, working with \acp{GC} still
proves challenging.  \Ac{GCS} approaches can be employed to overcome them, but
they mostly resort to generic algorithms that struggle to identify specific
problems for which efficient solutions exist.

Nonetheless, the prior analysis of the set of \acp{GC} that must be dealt with
requires background knowledge on numerical robustness to mitigate
fixed-precision arithmetic issues.  Moreover, there is the added requirement of
researching solutions to these specific constraint problems,  wasting the user's
time that could be spent in the design process itself.

Thus, an alternative approach is proposed in the form of the implementation of
\ac{GC} primitives in an expressive \ac{TPL} supported by an exact geometric
computation library.  The latter provides a series of optimized geometric
algorithms and exact constructs that can transparently handle robustness issues,
lifting this concern from the user's shoulders.

Our solution is correct, ensured by supporting it on \ac{CGAL}.  Consequently,
development time is vastly reduced by repurposing existing functionality instead
of re-implementing it from scratch, also reducing the potential for producing
erroneous code.  Additionally, our solution's performance proves to be superior
when compared to projects that implement some of the same constructs we managed
to repurpose.  As a side effect, we opened the door to extending more
functionality in \ac{CGAL}.  These reasons suffice to validate our approach.

Finally, we showed our approach creates programs that are easy to understand and
reproduce using a straightedge and a compass.  By adopting a constructive
approach, we clarify the steps required to build geometry, as opposed to the
analytical approach programming usually begs for.  The latter is not only more
cumbersome to follow, but it also produces less tangible programs.  The former
is preferred by \ac{AD} practitioners while also being alluring to new adopters,
contributing to the increasing adoption of \ac{AD}.

\subsection*{Future Work}

Our solution certainly has some drawbacks, some of which were already discussed
in \cref{sec:solution.tradeoffs}.  To briefly reiterate a few, our wrapper
around \ac{CGAL} is still rather opaque.  Functionality should be transparently
mapped, providing the user with more control and choice over the constructs they
are using.  \texttt{CxxWrap.jl} enables the mapping of parametric types,
something that was done in part for some packages in \texttt{CGAL.jl}.  Adopting
a similar approach for the opaquely mapped constructs should be considered.

Furthermore, the set of implemented \ac{GC} primitives was quite could be
further expanded on by, for instance, attempting to reach parity with tools,
like \texttt{tkz-euclide}, or familiar \ac{CAD} systems, like AutoCAD, providing
users with a more complete interface.  Alternatively, one could look further
into \ac{CGAL} for more functionality, like the 2D \& 3D Boolean Operations
packages.

This work focused exclusively on geometry in the 2D plane, leaving 3D space
vastly unexplored.  Elevating a dimension means the solutions to problems once
formulated in the 2D plane are not applicable in 3D space for some problems may
now be under-constrained.  We can once again look into \ac{CGAL}, or, deviating
from it, explore other libraries, as well as research solutions for the 3D
versions of the primitives in our solution. Moving from 2D to 3D will certainly
elevate our work, like going from sketching on paper to projecting buildings.

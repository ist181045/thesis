% !TEX root = ../main.tex
\fancychapter{Conclusion}%
\label{chap:conclusion}
\cleardoublepage{}

\noindent The generation of sophisticated designs is not viable through the
usage of interactive interfaces due to rigidity in the manipulation of existing
models in order to generate multiple variants.  This is where \ac{AD} comes in.
Even then, working with \acp{GC}, whether in a \ac{VPL} or a \ac{TPL}, imposes a
set of challenges, which can be overcome by resorting to \ac{GCS} approaches.
Alas, said approaches typically resort to generic \acs{GCS} algorithms, which
means \ac{GC} solvers, in general, have difficulties identifying specific
underlying subproblems with efficiently computable and robust solutions.

Nonetheless, the prior analysis of the set of \acp{GC} that must be
dealt with requires background knowledge on numerical robustness to mitigate
fixed-precision arithmetic issues, such as roundoff error accumulation.
Moreover, there is the added requirement of researching solutions to these
specific constraint problems and subsequent implementation of the corresponding
algorithms.  The user will end up spending more time and effort in this process
than in the design process itself.

Thus, in order to overcome these obstacles, an alternative approach is proposed
in the form of the implementation of \ac{GC} primitives supported by an exact
geometric computation library.  The latter provides a series of optimized
geometric algorithms and exact data structures that allow transparent handling
of robustness issues, lifting this concern from the user's shoulders, thus
facilitating the design process.

By supporting our solution on top of \ac{CGAL}, we transitively ensure its
correctness since \ac{CGAL} has decades of research, experience, and maintenance
behind it.  Consequently, development time is vastly reduced, choosing to
repurpose existing geometric constructs and functionality instead of
re-implementing them from scratch, leaving no room for error there.
Additionally, our solution's performance proves to be superior compared to
projects that implement some of the same functionality we managed to repurpose.
As a side effect, we opened the door for extending more functionality from
\ac{CGAL}.  These reasons suffice to validate our approach.

Finally, we proved the approach employed by our solution is one that creates
understandable programs that can be reproduced by hand, using a straightedge and
a compass.  By adopting a constructive approach to geometry specification, we
externalize and clarify the steps required to build geometric objects.  This is
contrasted with the more natural analytical approach programming languages
usually beg for.  Following the latter approach is not only more cumbersome due
to the solution derivation process, but it also produces less tangible programs,
hiding the concrete geometry behind formulas.  The former is preferred by
\ac{AD} practitioners, as well as industry professionals in general, but it
proves alluring to novice users all the same.  Said novice users might be
starting to learn and adopt \ac{AD}.  With our work, we aim to bolster this
adoption rate, driving more and more people to novel design paradigms.

\pdfbookmark[1]{Future Work}{future-work}
\section*{Future Work}

Our solution certainly has some drawbacks and misfeatures that could be
improved.  Some were already discussed in \cref{sec:solution.tradeoffs}.  To
briefly reiterate a few, our wrapper around the underlying library is less
transparent than desired.  Constructs and functionality should be mapped as
transparently as possible, fully parameterized, as to provide the user with more
control and choice over the constructs they are using.  There is potential for
this as \texttt{CxxWrap.jl} allows the mapping of parametric types.  We have
already leveraged these mechanisms and expanded on our mappings of the 2D
Triangulations and Voronoi Diagram Adaptor packages from \ac{CGAL} in
\texttt{CGAL.jl}.  More such work should follow for the aforementioned opaquely
mapped types.

Furthermore, the set of geometric primitives that were implemented could be
further expanded on.  This could initially be done by attempting to reach parity
with tools, like \texttt{tkz-euclide}, from which we can draw inspiration; or
even with more popular \ac{CAD} systems that industry professionals are familiar
with.  Hence, we would be providing a familiar interface to said professionals.
Once again, now with \ac{CGAL} in tow, we can explore the library further,
looking for additional algorithms that can complement this potential interface.
For example, there are packages within \ac{CGAL} that implement 2D and 3D
Boolean Operations, a set of operations all too familiar to \ac{CAD} system
users, that could be repurposed.

To divert focus from \ac{CGAL} for a moment, one should also look into other
geometry libraries as well.  We mentioned the Julia package ecosystem is still
catching up, but there is already a lot of work put into certain packages that
make their usage attractive since, for instance, we do not have to communicate
across languages, avoiding some issues we were met with, e.g., lack of
parametrization.

Finally, this work focused exclusively on constrained geometry bound to the
$\mathbb{R}^2$ Euclidean space, i.e., the 2D plane.  There is still much work to
be done researching problem solutions that encompass 3D space as well.
Solutions to problems once formulated in the 2D plane are not generally
applicable in 3D space, as the corresponding 3D problems may now be
under-constrained.  As an example, our solution for the circumcenter problem,
seen first in \cref{sec:intro.examples.circumcenter}, does not work in 3D space.
A line's bisector in 3D space is a plane, and noncoplanar nor parallel plane
intersection results in a line.  To obtain the actual circumcenter, one would
additionally, for example, have to intersect the resulting line with the plane
the circumscribed triangle sits on.  Among problems like this one, many more
exist, but much like we just illustrated, so do solutions.  To start, one could
yet again choose to explore \ac{CGAL} for additional 3D constructs and
functionality.

Going from the 2D plane into 3D space seems simple on paper.  But much like the
jump from sketching on paper to projecting skyscrapers, the dimensional gap is
wider than it initially seemed.

% !TEX root = ../main.tex
\fancychapter{Conclusion}%
\label{chap:conclusion}
\cleardoublepage{}

\noindent The generation of highly constrained sophisticated designs is not
viable through usage of interactive interfaces due to rigidity in the
manipulation of existing models in order to generate multiple variants.  
\Ac{AD} Even
then, \Acp{VPL} suffer from the disproportionate relation between the resulting
workflow and respective design complexity.  However, working with geometric
constraints in \acp{TPL} imposes a set of challenges, which can be overcome
through the usage of \ac{GCS} approaches to solve complex systems of
constraints.  To achieve that goal, several methods can be employed, but they
mostly resort to generic \acs{GCS} algorithms.  Alas, solvers, in general, have
difficulties identifying specific underlying subproblems for which efficiently
computable and robust solutions might be available.

The prior analysis of the set of geometric constraints that must be dealt with.
Nonetheless, it requires certain background knowledge on numerical robustness to
mitigate fixed-precision arithmetic issues, such as \textit{roundoff} error
accumulation throughout computation, as well as researching solutions to these
specific constraint problems.  The user will end up spending more time and
effort in this process than in the design process itself.

Thus, in order to overcome these obstacles, an alternative approach is proposed
in the form of the implementation of geometric constraint primitives in an
expressive \ac{TPL} supported by an exact geometric computation library.  The
latter provides a series of optimized geometric algorithms and exact data
structures that allow transparent handling of robustness issues, lifting this
concern from the user's shoulders with the goal of improving constrained
geometry specification efficiency as well as consequently facilitating the
design process.

However, the usage of exact data structures incurs a substantial performance hit
that does not justify its pervasive use, only fitting very few scenarios in
practice.  We leveraged faster, albeit inexact, constructs, while still
preserving exact predicate computation.  Additionally, the implemented
primitives still delay geometry construction as much as possible, remaining
robust and preserving exactness up until that point.  Misuse of resulting
geometry might still lead to surprising erroneous situations, though in
practical terms, these cases are few and far apart, meaning our solution still
holds value.

Finally, we proved the approach employed by our solution is one that creates
understandable programs that can be manually reproduced.  By adopting a
constructive approach to geometry specification, we externalize and clarify the
steps required to build geometric objects.  This is contrasted with the more
natural analytical approach programming languages usually beg for.  Following
the latter approach is not only more cumbersome due to the solution derivation
process, but it produces incomprehensible programs, hiding the concrete geometry
behind formulas.  The former is preferred by \ac{AD} practitioners, as well as
industry professionals in general, but it proves alluring to novice users all
the same.  Said novice users might be starting to learn and adopt \ac{AD}.  With
our work, we aim to bolster this adoption rate, driving more and more people
from traditional means to novel design paradigms.

\section*{Future Work}

Our solution certainly has some drawbacks and misfeatures that could be
improved.  Some were already discussed in \cref{sec:solution.tradeoffs}.  To
briefly reiterate a few, our wrapper around the underlying library is less
transparent than desired.  Constructs and functionality should be mapped as
transparently as possible, fully parameterized, as to provide the user with more
control and choice over the constructs they are using.  Furthermore, the set of
geometric primitives that were implemented was quite limited in size and could
be further expanded on.  However, let the ones showcased serve as an example for
expanding the set of primitives even further.

This work focuses is exclusively on constrained geometry bound to the
$\mathbb{R}^2$ Euclidean space, i.e., the 2D plane.  There is still much work to
be done researching problem solutions that encompass 3D space as well.
Elevating a dimension means the solutions to problems once formulated in the 2D
plane are no longer applicable in 3D space for some problems may now be
under-constrained.  As an example, our solution for the circumcenter problem,
exemplified in \cref{sec:intro.examples.circumcenter}, would no longer work in
3D space.  A line's bisector in 3D space is a plane, and noncoplanar nor
parallel plane intersection results in a line.  To obtain the actual
circumcenter, one would additionally, for example, have to intersect the
resulting line with the plane the circumscribed triangle sits on.

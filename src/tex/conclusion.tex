% !TEX root = ../main.tex
\fancychapter{Conclusion}%
\label{chap:conclusion}
\cleardoublepage{}

\noindent
The generation of highly constrained sophisticated designs is not viable through
usage of interactive interfaces due to rigidity in the manipulation of existing
models in order to generate multiple variants, or through \acp{VPL} because of
the disproportionate relation between the resulting workflow and respective
design complexity.  However, working with geometric constraints in \acp{TPL}
imposes a set of challenges, which can be overcome through the usage of \ac{GCS}
approaches to solve complex systems of constraints.  To achieve that goal,
several methods can be employed, but they mostly resort to generic \acs{GCS}
algorithms, but solvers, in general, have difficulty in identifying specific
underlying subproblems for which efficiently computable and robust solutions
might be available.

The prior analysis of the set of geometric constraints that must be dealt with,
nonetheless, requires certain background knowledge on numerical robustness to
mitigate fixed-precision arithmetic issues, such as \textit{roundoff} error
accumulation throughout calculation, as well as investigation on how to solve
these specific constraint problems.  The user will end up having to spend more
time and effort in this process than in the design process itself.

Thus, in order to overcome these obstacles, an alternative approach is proposed
in the form of the implementation of geometric constraint primitives in an
expressive \ac{TPL} supported by an exact geometric computation library.  The
latter provides a series of optimized geometrical algorithms and exact data
structures that allow transparent handling of robustness issues, lifting this
concern from the user's shoulders with the goal of improving constrained
geometry specification efficiency as well as consequently facilitating the
design process.

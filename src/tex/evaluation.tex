% !TEX root = ../main.tex
\fancychapter{Evaluation}%
\label{chap:eval}
\cleardoublepage{}

\noindent In this chapter, we evaluate our solution by measuring the qualities
of the approach we took to tackle \ac{GCS}.

Firstly, we aim to benchmark our solution's performance by comparing it to a
similar project called ConstraintGM~\cite{Pinheiro:2016:MGR}.  We were able to
reproduce the project's original benchmark tests and create an analogous test
suite for our solution so we can compare both projects.

Secondly, we explore our approach's potential regarding repurposing more complex
geometric algorithms with intrinsic \ac{GC} relations in contrast with
re-implementing a version of said algorithms from scratch.  Specifically, we set
out to repurpose \ac{CGAL}'s 2D Delaunay Triangulation and Voronoi Diagram
algorithms and further compare the resulting mapping's performance and
correctness when compared to a native Julia implementation of the algorithms as
provided by the package
\texttt{VoronoiDelaunay.jl}\footnote{\url{https://github.com/JuliaGeometry/VoronoiDelaunay.jl}},
that, in the past, was once benchmarked against
\ac{CGAL}\footnote{\url{https://gist.github.com/skariel/3d2018f9341a058e00fc}}.
Additionally, we set out to estimate the effort it took to develop
\texttt{VoronoiDelaunay.jl} and compare that to the effort it took to extract an
analogous algorithm from \ac{CGAL}.

Finally, we end our evaluation by showcasing four different case studies
inspired by existing designs.  This section of the evaluation focuses on
comparing different approaches to solving \ac{GC} problems present in each case
study by adopting two different approaches: 
\begin{enumerate*}[label= (\arabic*)]
  \item an analytic approach, one programming naturally begs for, and 
  \item a constructive approach, essentially adding a conceptual abstraction
  layer over the former.
\end{enumerate*}
We aim to show that, by following the latter approach, resulting programs become
both easier to understand and reproducible as the set of instructions can be
used to recreate the resulting geometry by hand, using a ruler and a compass.

Benchmarks were performed on a Lenovo{\textregistered}
ThinkPad{\textregistered} E595 laptop computer with the following
system specifications:

\begin{itemize}
  \item \acsu{AMD}\label{acro:AMD} Ryzen\textsuperscript{\texttrademark} 5 3500U
  \acsu{CPU}\label{acro:CPU} @ 2.1GHz\footnote{Base clock frequency.  Can boost
  up to 3.7GHz.};
  \item 1×16GB \acsu{SO-DIMM}\label{acro:SO-DIMM} of \acsu{DDR}\label{acro:DDR}4
  \acsu{RAM}\label{acro:RAM} @ 2400MT/s.
  \item Arch
  Linux\textsuperscript{\texttrademark}\footnote{\url{https://archlinux.org}}
  x86 64-bit, Linux{\textregistered} Kernel
  5.12.15-zen1\footnote{\url{https://github.com/zen-kernel/zen-kernel/tree/v5.12.15-zen1}}
\end{itemize}
\clearpage

% !TEX root = ../../main.tex
\section{ConstraintGM}%
\label{sec:eval.cgm}

ConstraintGM is a domain-specific language developed with the shared goal of
tackling \ac{GC} problem specification using a \ac{TPL} (Racket, to be precise).
This solution heavily and blindly relied on Maxima~\cite{Maxima:2021:Maxima}, a
generic \ac{CAS} to solve \ac{GC} problems.  Geometric entities were serialized
into their algebraic equation representation coupled with the problem's
constraints and sent to Maxima.  Maxima would then attempt to solve the system
of equations and return a result that was parsed and sent back to Racket.

This approach came at a severe performance cost, the reason being two-fold:
\begin{enumerate*}[label= (\arabic*)]
  \item the communication between ConstraintGM and Maxima was slow, and
  \item Maxima is a \emph{generic} solver and could not take advantage of the
  geometric characteristics of the problem at hand.
\end{enumerate*}
The considerable performance penalty of this approach are hard to justify in the
case of simple geometric problems for which there are well-known efficient
solutions. This lead to an \textit{impromptu} implementation of some \ac{GC}
problem solutions, creating what was called the \ac{GFL}. As expected, the
latter approach revealed that contextually specialized solutions had much better
performance than relying on a generic solver.

It is worth noting that relying on Maxima meant ConstraintGM was intrinsically
exact and robust since symbolic \ac{GCS} methods, used by Maxima, automatically
provide those features.  The \ac{GFL}, when compared to the Maxima-based
approach, is more fragile in that regard because it will depend on the
underlying constructs' number type.  If an exact arbitrary precision number type
is used, exactness and robustness will be preserved, but at a performance cost
since arbitrary precision arithmetic is computationally heavy.  Relying on
inexact number types can eventually lead to erroneous results if one is not
careful to avoid error-inducing computations, but, by contrast, offers better
performance.  That said, some computations are unavoidable, such as the
computation of the squared root, for example, when the effective distance is
necessary to operate with.  In the end, it is a matter of making a compromise
and evaluating which fits the case at end the best.

The project's benchmark involve three different \ac{GC} problems focused around
object intersection, namely
\begin{enumerate*}[label= (\arabic*)]
  \item line-line intersection,
  \item circle-line intersection, and
  \item circle-circle intersection.
\end{enumerate*}
These problems expanded into thirteen different scenarios that consisted of
rearranging the geometric entities' disposition in order to evaluate the
intersection operation's results and measure each individual scenario's
performance.

We measured real execution time instead of \ac{CPU} time both for ConstraintGM
and for our solution.  We ran ConstraintGM's benchmarks a total of nine times to
obtain a relatively decent sample of results, while our solution's benchmarks
were aided by \texttt{BenchmarkTools.jl}~\cite{Chen:2016:BenchmarkTools.jl}, a
package that considerably facilitates benchmarking Julia code tremendously.  The
source code used to benchmark both ConstraintGM and our solution is listed in
\cref{lst:appendix.cgm.bench.rkt,lst:appendix.cgm.bench.jl} respectively.  The
benchmark's results are gathered in \cref{tab:eval.cgm.perf}.  To facilitate
comprehension, these are also plotted in \cref{fig:eval.cgm.perf}.

\begin{table}[htb]
  \caption[ConstraintGM performance benchmarks]{\label{tab:eval.cgm.perf}%
    Performance comparison between ConstraintGM's solutions, both using Maxima
    and \ac{GFL}, and our solution.}
  \footnotesize\centering
  \begin{tabular*}{\linewidth}{r*{3}{l}l}
    \toprule
    \multirow{2}{*}{\textbf{Scenario}}
    & \multicolumn{3}{c}{\textbf{Execution time ($\mathrm{mean}\pm\sigma$)}}
    & \multirow{2}{*}{\textbf{GC problem}} \\
    & \multicolumn{1}{c}{\textbf{Maxima}}
    & \multicolumn{1}{c}{\textbf{GFL}}
    & \multicolumn{1}{c}{\textbf{Our solution}} & \\
    \midrule
     1 & \texttt{~2.265 s ± 142.344 ms}
       & \texttt{~8.778 ms ± ~~1.641 ms}
       & \texttt{362.191 μs ± ~4.841 ms}
       & \multirow{2}{*}{Line $\cap$ Line}\\
     2 & \texttt{~1.645 s ± 197.824 ms}
       & \texttt{~6.222 ms ± 440.959 μs}
       & \texttt{159.567 μs ± ~4.634 μs} &\\
    \midrule
     3 & \texttt{~3.958 s ± 295.673 ms}
       & \texttt{~9.444 ms ± ~~1.333 ms}
       & \texttt{~~1.871 ms ± ~4.892 ms} 
       & \multirow{3}{*}{Circle $\cap$ Line}\\
     4 & \texttt{~3.070 s ± 243.768 ms}
       & \texttt{~8.000 ms ± ~~0.000 ns}
       & \texttt{955.815 μs ± ~3.710 ms} &\\
     5 & \texttt{~1.839 s ± ~84.460 ms}
       & \texttt{~6.000 ms ± ~~0.000 ns}
       & \texttt{600.386 μs ± 19.807 μs} &\\
    \midrule
     6 & \texttt{~1.311 s ± ~59.139 ms}
       & \texttt{~5.111 ms ± 333.333 μs}
       & \texttt{241.531 μs ± 12.504 μs} 
       & \multirow{8}{*}{Circle $\cap$ Circle}\\
     7 & \texttt{~1.847 s ± ~50.466 ms}
       & \texttt{~5.333 ms ± 500.000 μs}
       & \texttt{242.780 μs ± ~5.692 μs} &\\
     8 & \texttt{~1.868 s ± ~37.650 ms}
       & \texttt{~5.333 ms ± 500.000 μs}
       & \texttt{245.425 μs ± 11.840 μs} &\\
     9 & \texttt{~4.277 s ± ~37.053 ms}
       & \texttt{~5.222 ms ± 440.959 μs}
       & \texttt{515.829 μs ± ~4.881 ms} &\\
    10 & \texttt{~2.506 s ± 238.696 ms}
       & \texttt{~7.222 ms ± 440.959 μs}
       & \texttt{629.341 μs ± ~5.360 ms} &\\
    11 & \texttt{~3.493 s ± 258.715 ms}
       & \texttt{~7.222 ms ± 440.959 μs}
       & \texttt{~~1.453 ms ± ~5.070 ms} &\\
    12 & \texttt{~3.830 s ± 150.402 ms}
       & \texttt{~9.444 ms ± 527.046 μs}
       & \texttt{~~1.455 ms ± ~5.088 ms} &\\
    13 & \texttt{11.111 s ± ~81.302 ms}
       & \texttt{10.222 ms ± ~~1.093 ms}
       & \texttt{~~1.463 ms ± ~5.055 ms} &\\
    \bottomrule
  \end{tabular*}
\end{table}

Observing the results, we can see the disparity between the approach reliant on
Maxima when compared to both the \ac{GFL} and our solution, which was to be
expected.  Even so, amidst relatively consistent results, scenario 13 made
Maxima slug more than usual.  That scenario consists of two circles intersecting
at two different points, illustrated in \cref{fig:eval.cgm.perf.13}.  It is not
the only scenario of the set that involves two circles that intersect.  However,
this is the one that produces relatively more complex results, which could
justify why Maxima took relatively longer to compute the solution than it did
for the scenarios that preceded this one.

\begin{figure}[htb]
  \begin{subfigure}[t]{.75\linewidth}
    \centering
    \begin{tikzpicture}
    \begin{semilogyaxis}[cgm,
      title={Maxima vs. GFL vs. Our solution},
      xlabel={Scenarios},
      ylabel={Average Time (ns, log)},
      width={\linewidth},
      height=5.5cm,
      ytick distance={1e1},
      bar width=.3\linewidth/14,
      enlarge y limits={.26,upper},
      legend columns=-1,
    ]
      \addplot+ table {data/cgm-maxima.csv};
      \addplot+ table {data/cgm-gfl.csv};
      \addplot+ [color=green,draw=green!60!black] table {data/jlcgal.csv};
      \legend{Maxima,GFL,Our solution}
    \end{semilogyaxis}
    \end{tikzpicture}
    \subcaption{Y-bar plot of every approaches' benchmark results.  N.B.: The
      Y-axis is logarithmic.}\label{fig:eval.cgm.perf.plot}
  \end{subfigure}
  \hfill
  \begin{subfigure}[t]{.22\linewidth}
    \centering
    \raisebox{1.25cm}{\resizebox{\linewidth}{!}{
    \begin{tikzpicture}
      \tkzDefPoints{0/0/A,1/1/B}
      \tkzDrawCircle[R](A, 1.5cm)\tkzGetLength{rA}
      \tkzDrawCircle[R](B, 1cm)  \tkzGetLength{rB}
      \tkzInterCC[R](A,\rA pt)(B,\rB pt) \tkzGetPoints{C}{D}
      \tkzDrawPoints[color=red](C,D)
    \end{tikzpicture}}}
    \subcaption{Scenario 13 from the benchmark suite.}%
    \label{fig:eval.cgm.perf.13}
  \end{subfigure}
  \caption[ConstraintGM benchmarks and Scenario 13]{\label{fig:eval.cgm.perf}%
    ConstraintGM benchmark results collected from \cref{tab:eval.cgm.perf} in a
    multi-series plot \subref{fig:eval.cgm.perf.plot} beside the depiction of
    scenario 13 \subref{fig:eval.cgm.perf.13} from the benchmark suite.}
\end{figure}

Analyzing the results further, we can see our solution also outdoes
ConstraintGM's \ac{GFL} by a significant margin.  This is most likely due to the
fact that we are relying on \ac{CGAL}, a library implemented in C++.  The latter
is notoriously known for being a high-performance language, considerably
outperforming Racket in a series of benchmarks.  Nevertheless, despite some
overhead in the case of Julia, the results are still positive, on average
beating the \ac{GFL} by an order of magnitude.

In conclusion, our solution proves capable and performant, having surpassed
ConstraintGM's \ac{GFL} by an entire order of magnitude on average.


\section{VoronoiDelaunay.jl}%
\label{sec:eval.vdjl}

% \todo[inline]{Maybe leave the "how easy it is to leverage our approach to get
% more, both in quantity and complexity, algorithsm from CGAL" discussion for this
% part and evaluate the potential time it takes to use Voronoi Diagrams from CGAL
% and either get more information on how long it took to implement
% VoronoiDelaunay.jl and compare its "correctness" vs. CGAL's version of the
% algorithm, assuming CGAL's results as a source of truth for correctness.
% Testing revealed diagrams differed slightly when it came to some edges.  My
% suspicion was the Julia algorithm "gobble" some very very very small triangles,
% i.e., where the edges are very close together, but it only happens near the
% outside of the diagram, and, again, a more drastic diagram can be manufactured
% to showcase this.  Maybe read up more on the underlying algorithm that was
% implemented in VoronoiDelaunay.jl}

% !TeX root = ../../main.tex
\subsection{Case Studies}%
\label{sec:eval.studies}

In this section, we aim to demonstrate our solution when applied to two
different case studies, each presenting a parametric geometric shape: a star
with semicircles, and a Voronoi diagram.  Each case, illustrated in
\cref{fig:eval.studies.designs}, was inspired by an existing design:
\begin{enumerate*}[label= (\arabic*)]
  \item Eero Aarnio's Egg chair,
  \item Thonet 214 chair seat,
  \item César Pelli's Petronas tower section, and
  \item PTW Architects' Beijing National Aquatics Center.
\end{enumerate*}
These problems were solved employing both an \textit{analytic} approach, an
approach \acp{TPL} naturally demand, and a \textit{constructive} approach, the
one made possible by relying on our solution.

\begin{figure}[htb]
  \subcaptionbox{Islamic towers.\label{fig:eval.studies.designs.petronas}}
    [.32\linewidth]{\includegraphics[height=3cm]{fig/case-study-petronas}}
  \hfill
  \subcaptionbox{Voronoi cage.\label{fig:eval.studies.designs.watercube}}
    [.65\linewidth]{%
      \includegraphics[width=\linewidth]{fig/case-study-water-cube}}
  \caption{\label{fig:eval.studies.designs}
    Case study designs inspired by César Pelli's Petronas Twin
    Towers~\subref{fig:eval.studies.designs.petronas}, and PTW Architects'
    Beijing National Aquatics
    Center~\subref{fig:eval.studies.designs.watercube}.}%
\end{figure}

\subsubsection{Star with Semicircles}%
\label{sec:eval.studies.star}

The first case study is a star shape with semicircles, inspired by César Pelli's
Petronas tower floor plan.  The contour of the Petronas tower floor plan is
formed by two overlapping congruent squares, forming an octagram, and by eight
circles each centered on one of the eight intersection points and tangent to the
bounding octagon.  This shape can be generalized to a parametric shape, shown in
(\cref{fig:eval.studies.star.prob.params}).  Variations are illustrated in
\cref{fig:eval.studies.star.prob}.

\begin{figure}[htb]
  \centering
  \subcaptionbox{\label{fig:eval.studies.star.prob.params}%
    Parametrization.}
    {\includegraphics[width=.45\linewidth]{fig/star-problem-params}}
  \hfill
  \subcaptionbox{\label{fig:eval.studies.star.prob.vars}%
    Shape variations.}
    {\includegraphics[width=.45\linewidth]{fig/star-problem-vars}}
  \caption{\label{fig:eval.studies.star.prob}
    Star with semicircles problem: \subref{fig:eval.studies.star.prob.params}
    shows our parametrization of the star which can be used to generate shape
    variations, some of them shown in
    \subref{fig:eval.studies.star.prob.vars}.}%
\end{figure}

Both analytic and constructive solutions are based on computing one side of the
star, composed of the line segment $\overline{V_1 I_1}$, the arc centered on
$O_1$ from $I_1$ to $I_2$, with radius $r_1$, and the line segment 
$\overline{I_2 V_2}$ (\cref{fig:eval.studies.star.sol}).

\begin{figure}[htb]
  \includegraphics[width=.7\linewidth]{fig/star-solution}
  \caption{\label{fig:eval.studies.star.sol}
    Analytic and constructive solutions to the star with semicircles problem.}%
\end{figure}

The analytic solution is described below.

\begin{enumerate}
  \item $r_1 = r\frac{\sin^2\frac{\pi}{n}}{\cos\frac{\pi}{n}}$
  \item $r_2 = r\cos\frac{\pi}{n} - r_1$
  \item $O_1 = O + \left(r_2, \angle\frac{\pi}{n}\right)$
  \item $I_1 = O_1 + \left(r_1, \angle\frac{2\pi}{n} - \frac{\pi}{2}\right)$
  \item $I_2 = O_1 + \left(r_1, \angle\frac{\pi}{2}\right)$
\end{enumerate}

The constructive solution is described below.  It uses two primitives from our
solution, namely \texttt{intersection} and \texttt{tangent\_circle}.

\begin{enumerate}
  \item $O_1 = \operatorname{intersection}\left(\overline{V_1 V_3},
  \overline{V_2 V_n}\right)$
  \item $C_1 = \operatorname{tangent_{circle}}\left(O_1, \overline{V_1
  V_2}\right)$
  \item $P,r_1 = C_1$
  \item $I_1 = \operatorname{intersection}\left(\overline{V_1 V_3}, C_1\right)$
  \item $I_2 = \operatorname{intersection}\left(\overline{V_2 V_n}, C_1\right)$
\end{enumerate}

Achieving the equations in the \textit{analytic solution} is not a
straightforward task. It is also unclear how those equations were derived. By
contrast, in the \textit{constructive solution}, all the steps are clearly
externalized, which makes it much easier to understand.

\subsubsection{Voronoi Diagram}%
\label{sec:eval.studies.voronoi}

Our second case study is that of Voronoi diagrams, which are used in a variety
of design fields.  For instance, several facade designs exhibit a Voronoi
appearance, such as PTW Architects' Beijing National Aquatics Center.

\Cref{fig:eval.studies.voronoi.prob} shows three Voronoi diagrams generated from
entirely randomly distributed points, from random points with one attractor
point, and from random points with one attractor line.

\begin{figure}[htb]
  \subcaptionbox{\label{fig:eval.studies.voronoi.prob.rand}}
    {\includegraphics[width=.3\linewidth]{fig/voronoi-problem-rand}}
  \hfill
  \subcaptionbox{\label{fig:eval.studies.voronoi.prob.1attr}}
    {\includegraphics[width=.3\linewidth]{fig/voronoi-problem-1attr}}
  \hfill
  \subcaptionbox{\label{fig:eval.studies.voronoi.prob.edge}}
    {\includegraphics[width=.3\linewidth]{fig/voronoi-problem-edge}}
  \caption[Voronoi diagram problem]{
    Voronoi diagram problem: \subref{fig:eval.studies.voronoi.prob.rand} is
    entirely random, \subref{fig:eval.studies.voronoi.prob.1attr} adds an
    attractor point, and \subref{fig:eval.studies.voronoi.prob.edge} adds an
    attractor line.}%
  \label{fig:eval.studies.voronoi.prob}
\end{figure}

Both the analytic and constructive methods focus on computation of a vertex
relies on the computation of the circumcenter of a triangle, for instance,
triangle $\triangle P_1 P_2 P_3$ (\cref{fig:eval.studies.voronoi.sol}).

\begin{figure}[htb]
  \centering
  \includegraphics[width=.7\linewidth]{fig/voronoi-solution}
  \caption{\label{fig:eval.studies.voronoi.sol}
    Analytic and constructive approaches to computing a Voronoi vertex.}%
\end{figure}

One possible analytic solution is based on the \textit{circumradius} formula.
The circumcenter $C$ can then be easily computed by a translation from $P_1$
following the angle $\alpha$.

\begin{enumerate}
  \item $a, b, c = \lVert P_2 - P_1 \rVert, \lVert P_3 - P_1 \rVert, \lVert P_3
  - P_2 \rVert$
  \item $s = \frac{a + b + c}{2}$
  \item $A = \sqrt{s(s - a)(s - b)(s - c)}$
  \item $r = \frac{abc}{4A}$
  \item $\alpha = \arccos\frac{a}{2r}$
  \item $C = P_1 + \left(r, \angle\alpha\right)$
\end{enumerate}

The constructive solution computes the circumcenter, directly provided by our
\primitives{}.

\begin{itemize}
  \item[] $C = \operatorname{circumcenter}\left(P_1, P_2, P_3\right)$
\end{itemize}

The circumcenter is only a sub-problem of the generation of a Voronoi diagram.
We first need to build a Delaunay triangulation.  Then, we can apply the
\texttt{circumcenter} to find the Voronoi vertices and draw the diagram's edges.

Implementing this functionality from scratch is a demanding and error-prone
task.  Fortunately, \ac{CGAL} already has an algorithm that produces Voronoi
diagrams.  This algorithm was made available in \texttt{CGAL.jl},
and, thus, it is also available in our solution.  The final section of the
evaluation goes over how we can repurpose this algorithm as a side effect of
integrating such a comprehensive library as \ac{CGAL}.


% !TEX root = ../main.tex
\section{Evaluation}%
\label{sec:eval}

In this section, we evaluate our solution by measuring the qualities of the
approach we took to tackle \ac{GCS}.

Firstly, we benchmark our solution's performance by comparing it to a
similar project called ConstraintGM~\cite{Pinheiro:2016:MGR}.

Secondly, we showcase two different case studies, focusing on solving \ac{GC}
problems by adopting two different approaches: 
\begin{enumerate*}[label= (\arabic*)]
  \item an analytic approach, one programming naturally begs for, and 
  \item a constructive approach, adding an abstraction over the former.
\end{enumerate*}
We aim to show the latter produces programs that are both easier to understand
and to reproduce.

Finally, we explore our approach's potential for obtaining more complex
geometric algorithms vs.\ re-implementing a version of said algorithms from
scratch.  Specifically, we repurpose \ac{CGAL}'s 2D Delaunay Triangulation and
Voronoi Diagram algorithms, comparing them to a Julia implementation of the
algorithms provided by the
\texttt{VoronoiDelaunay.jl}\footnote{\url{https://github.com/JuliaGeometry/VoronoiDelaunay.jl}},
package.  Additionally, we estimate the effort it took to extract the algorithms
from \ac{CGAL} and compare it to the effort it took to develop the ones in
\texttt{VoronoiDelaunay.jl}.

Benchmarks were performed on a Lenovo{\textregistered}
ThinkPad{\textregistered} E595 laptop computer with the following
system specifications:

\begin{itemize}
  \item \acsu{AMD}\label{acro:AMD} Ryzen\textsuperscript{\texttrademark} 5 3500U
  \acsu{CPU}\label{acro:CPU} @ 2.1GHz\footnote{Base clock frequency.  Can boost
  up to 3.7GHz.};
  \item 1×16GB \acsu{SO-DIMM}\label{acro:SO-DIMM} of \acsu{DDR}\label{acro:DDR}4
  \acsu{RAM}\label{acro:RAM} @ 2400MT/s.
  \item Arch
  Linux\textsuperscript{\texttrademark}\footnote{\url{https://archlinux.org}}
  x86 64-bit, Linux{\textregistered} Kernel
  5.12.15-zen1\footnote{\url{https://github.com/zen-kernel/zen-kernel/tree/v5.12.15-zen1}}
\end{itemize}

% !TEX root = ../../main.tex
\section{ConstraintGM}%
\label{sec:eval.cgm}

ConstraintGM is a domain-specific language developed with the shared goal of
tackling \ac{GC} problem specification using a \ac{TPL} (Racket, to be precise).
This solution heavily and blindly relied on Maxima~\cite{Maxima:2021:Maxima}, a
generic \ac{CAS} to solve \ac{GC} problems.  Geometric entities were serialized
into their algebraic equation representation coupled with the problem's
constraints and sent to Maxima.  Maxima would then attempt to solve the system
of equations and return a result that was parsed and sent back to Racket.

This approach came at a severe performance cost, the reason being two-fold:
\begin{enumerate*}[label= (\arabic*)]
  \item the communication between ConstraintGM and Maxima was slow, and
  \item Maxima is a \emph{generic} solver and could not take advantage of the
  geometric characteristics of the problem at hand.
\end{enumerate*}
The considerable performance penalty of this approach are hard to justify in the
case of simple geometric problems for which there are well-known efficient
solutions. This lead to an \textit{impromptu} implementation of some \ac{GC}
problem solutions, creating what was called the \ac{GFL}. As expected, the
latter approach revealed that contextually specialized solutions had much better
performance than relying on a generic solver.

It is worth noting that relying on Maxima meant ConstraintGM was intrinsically
exact and robust since symbolic \ac{GCS} methods, used by Maxima, automatically
provide those features.  The \ac{GFL}, when compared to the Maxima-based
approach, is more fragile in that regard because it will depend on the
underlying constructs' number type.  If an exact arbitrary precision number type
is used, exactness and robustness will be preserved, but at a performance cost
since arbitrary precision arithmetic is computationally heavy.  Relying on
inexact number types can eventually lead to erroneous results if one is not
careful to avoid error-inducing computations, but, by contrast, offers better
performance.  That said, some computations are unavoidable, such as the
computation of the squared root, for example, when the effective distance is
necessary to operate with.  In the end, it is a matter of making a compromise
and evaluating which fits the case at end the best.

The project's benchmark involve three different \ac{GC} problems focused around
object intersection, namely
\begin{enumerate*}[label= (\arabic*)]
  \item line-line intersection,
  \item circle-line intersection, and
  \item circle-circle intersection.
\end{enumerate*}
These problems expanded into thirteen different scenarios that consisted of
rearranging the geometric entities' disposition in order to evaluate the
intersection operation's results and measure each individual scenario's
performance.

We measured real execution time instead of \ac{CPU} time both for ConstraintGM
and for our solution.  We ran ConstraintGM's benchmarks a total of nine times to
obtain a relatively decent sample of results, while our solution's benchmarks
were aided by \texttt{BenchmarkTools.jl}~\cite{Chen:2016:BenchmarkTools.jl}, a
package that considerably facilitates benchmarking Julia code tremendously.  The
source code used to benchmark both ConstraintGM and our solution is listed in
\cref{lst:appendix.cgm.bench.rkt,lst:appendix.cgm.bench.jl} respectively.  The
benchmark's results are gathered in \cref{tab:eval.cgm.perf}.  To facilitate
comprehension, these are also plotted in \cref{fig:eval.cgm.perf}.

\begin{table}[htb]
  \caption[ConstraintGM performance benchmarks]{\label{tab:eval.cgm.perf}%
    Performance comparison between ConstraintGM's solutions, both using Maxima
    and \ac{GFL}, and our solution.}
  \footnotesize\centering
  \begin{tabular*}{\linewidth}{r*{3}{l}l}
    \toprule
    \multirow{2}{*}{\textbf{Scenario}}
    & \multicolumn{3}{c}{\textbf{Execution time ($\mathrm{mean}\pm\sigma$)}}
    & \multirow{2}{*}{\textbf{GC problem}} \\
    & \multicolumn{1}{c}{\textbf{Maxima}}
    & \multicolumn{1}{c}{\textbf{GFL}}
    & \multicolumn{1}{c}{\textbf{Our solution}} & \\
    \midrule
     1 & \texttt{~2.265 s ± 142.344 ms}
       & \texttt{~8.778 ms ± ~~1.641 ms}
       & \texttt{362.191 μs ± ~4.841 ms}
       & \multirow{2}{*}{Line $\cap$ Line}\\
     2 & \texttt{~1.645 s ± 197.824 ms}
       & \texttt{~6.222 ms ± 440.959 μs}
       & \texttt{159.567 μs ± ~4.634 μs} &\\
    \midrule
     3 & \texttt{~3.958 s ± 295.673 ms}
       & \texttt{~9.444 ms ± ~~1.333 ms}
       & \texttt{~~1.871 ms ± ~4.892 ms} 
       & \multirow{3}{*}{Circle $\cap$ Line}\\
     4 & \texttt{~3.070 s ± 243.768 ms}
       & \texttt{~8.000 ms ± ~~0.000 ns}
       & \texttt{955.815 μs ± ~3.710 ms} &\\
     5 & \texttt{~1.839 s ± ~84.460 ms}
       & \texttt{~6.000 ms ± ~~0.000 ns}
       & \texttt{600.386 μs ± 19.807 μs} &\\
    \midrule
     6 & \texttt{~1.311 s ± ~59.139 ms}
       & \texttt{~5.111 ms ± 333.333 μs}
       & \texttt{241.531 μs ± 12.504 μs} 
       & \multirow{8}{*}{Circle $\cap$ Circle}\\
     7 & \texttt{~1.847 s ± ~50.466 ms}
       & \texttt{~5.333 ms ± 500.000 μs}
       & \texttt{242.780 μs ± ~5.692 μs} &\\
     8 & \texttt{~1.868 s ± ~37.650 ms}
       & \texttt{~5.333 ms ± 500.000 μs}
       & \texttt{245.425 μs ± 11.840 μs} &\\
     9 & \texttt{~4.277 s ± ~37.053 ms}
       & \texttt{~5.222 ms ± 440.959 μs}
       & \texttt{515.829 μs ± ~4.881 ms} &\\
    10 & \texttt{~2.506 s ± 238.696 ms}
       & \texttt{~7.222 ms ± 440.959 μs}
       & \texttt{629.341 μs ± ~5.360 ms} &\\
    11 & \texttt{~3.493 s ± 258.715 ms}
       & \texttt{~7.222 ms ± 440.959 μs}
       & \texttt{~~1.453 ms ± ~5.070 ms} &\\
    12 & \texttt{~3.830 s ± 150.402 ms}
       & \texttt{~9.444 ms ± 527.046 μs}
       & \texttt{~~1.455 ms ± ~5.088 ms} &\\
    13 & \texttt{11.111 s ± ~81.302 ms}
       & \texttt{10.222 ms ± ~~1.093 ms}
       & \texttt{~~1.463 ms ± ~5.055 ms} &\\
    \bottomrule
  \end{tabular*}
\end{table}

Observing the results, we can see the disparity between the approach reliant on
Maxima when compared to both the \ac{GFL} and our solution, which was to be
expected.  Even so, amidst relatively consistent results, scenario 13 made
Maxima slug more than usual.  That scenario consists of two circles intersecting
at two different points, illustrated in \cref{fig:eval.cgm.perf.13}.  It is not
the only scenario of the set that involves two circles that intersect.  However,
this is the one that produces relatively more complex results, which could
justify why Maxima took relatively longer to compute the solution than it did
for the scenarios that preceded this one.

\begin{figure}[htb]
  \begin{subfigure}[t]{.75\linewidth}
    \centering
    \begin{tikzpicture}
    \begin{semilogyaxis}[cgm,
      title={Maxima vs. GFL vs. Our solution},
      xlabel={Scenarios},
      ylabel={Average Time (ns, log)},
      width={\linewidth},
      height=5.5cm,
      ytick distance={1e1},
      bar width=.3\linewidth/14,
      enlarge y limits={.26,upper},
      legend columns=-1,
    ]
      \addplot+ table {data/cgm-maxima.csv};
      \addplot+ table {data/cgm-gfl.csv};
      \addplot+ [color=green,draw=green!60!black] table {data/jlcgal.csv};
      \legend{Maxima,GFL,Our solution}
    \end{semilogyaxis}
    \end{tikzpicture}
    \subcaption{Y-bar plot of every approaches' benchmark results.  N.B.: The
      Y-axis is logarithmic.}\label{fig:eval.cgm.perf.plot}
  \end{subfigure}
  \hfill
  \begin{subfigure}[t]{.22\linewidth}
    \centering
    \raisebox{1.25cm}{\resizebox{\linewidth}{!}{
    \begin{tikzpicture}
      \tkzDefPoints{0/0/A,1/1/B}
      \tkzDrawCircle[R](A, 1.5cm)\tkzGetLength{rA}
      \tkzDrawCircle[R](B, 1cm)  \tkzGetLength{rB}
      \tkzInterCC[R](A,\rA pt)(B,\rB pt) \tkzGetPoints{C}{D}
      \tkzDrawPoints[color=red](C,D)
    \end{tikzpicture}}}
    \subcaption{Scenario 13 from the benchmark suite.}%
    \label{fig:eval.cgm.perf.13}
  \end{subfigure}
  \caption[ConstraintGM benchmarks and Scenario 13]{\label{fig:eval.cgm.perf}%
    ConstraintGM benchmark results collected from \cref{tab:eval.cgm.perf} in a
    multi-series plot \subref{fig:eval.cgm.perf.plot} beside the depiction of
    scenario 13 \subref{fig:eval.cgm.perf.13} from the benchmark suite.}
\end{figure}

Analyzing the results further, we can see our solution also outdoes
ConstraintGM's \ac{GFL} by a significant margin.  This is most likely due to the
fact that we are relying on \ac{CGAL}, a library implemented in C++.  The latter
is notoriously known for being a high-performance language, considerably
outperforming Racket in a series of benchmarks.  Nevertheless, despite some
overhead in the case of Julia, the results are still positive, on average
beating the \ac{GFL} by an order of magnitude.

In conclusion, our solution proves capable and performant, having surpassed
ConstraintGM's \ac{GFL} by an entire order of magnitude on average.

% !TeX root = ../../main.tex
\subsection{Case Studies}%
\label{sec:eval.studies}

In this section, we aim to demonstrate our solution when applied to two
different case studies, each presenting a parametric geometric shape: a star
with semicircles, and a Voronoi diagram.  Each case, illustrated in
\cref{fig:eval.studies.designs}, was inspired by an existing design:
\begin{enumerate*}[label= (\arabic*)]
  \item Eero Aarnio's Egg chair,
  \item Thonet 214 chair seat,
  \item César Pelli's Petronas tower section, and
  \item PTW Architects' Beijing National Aquatics Center.
\end{enumerate*}
These problems were solved employing both an \textit{analytic} approach, an
approach \acp{TPL} naturally demand, and a \textit{constructive} approach, the
one made possible by relying on our solution.

\begin{figure}[htb]
  \subcaptionbox{Islamic towers.\label{fig:eval.studies.designs.petronas}}
    [.32\linewidth]{\includegraphics[height=3cm]{fig/case-study-petronas}}
  \hfill
  \subcaptionbox{Voronoi cage.\label{fig:eval.studies.designs.watercube}}
    [.65\linewidth]{%
      \includegraphics[width=\linewidth]{fig/case-study-water-cube}}
  \caption{\label{fig:eval.studies.designs}
    Case study designs inspired by César Pelli's Petronas Twin
    Towers~\subref{fig:eval.studies.designs.petronas}, and PTW Architects'
    Beijing National Aquatics
    Center~\subref{fig:eval.studies.designs.watercube}.}%
\end{figure}

\subsubsection{Star with Semicircles}%
\label{sec:eval.studies.star}

The first case study is a star shape with semicircles, inspired by César Pelli's
Petronas tower floor plan.  The contour of the Petronas tower floor plan is
formed by two overlapping congruent squares, forming an octagram, and by eight
circles each centered on one of the eight intersection points and tangent to the
bounding octagon.  This shape can be generalized to a parametric shape, shown in
(\cref{fig:eval.studies.star.prob.params}).  Variations are illustrated in
\cref{fig:eval.studies.star.prob}.

\begin{figure}[htb]
  \centering
  \subcaptionbox{\label{fig:eval.studies.star.prob.params}%
    Parametrization.}
    {\includegraphics[width=.45\linewidth]{fig/star-problem-params}}
  \hfill
  \subcaptionbox{\label{fig:eval.studies.star.prob.vars}%
    Shape variations.}
    {\includegraphics[width=.45\linewidth]{fig/star-problem-vars}}
  \caption{\label{fig:eval.studies.star.prob}
    Star with semicircles problem: \subref{fig:eval.studies.star.prob.params}
    shows our parametrization of the star which can be used to generate shape
    variations, some of them shown in
    \subref{fig:eval.studies.star.prob.vars}.}%
\end{figure}

Both analytic and constructive solutions are based on computing one side of the
star, composed of the line segment $\overline{V_1 I_1}$, the arc centered on
$O_1$ from $I_1$ to $I_2$, with radius $r_1$, and the line segment 
$\overline{I_2 V_2}$ (\cref{fig:eval.studies.star.sol}).

\begin{figure}[htb]
  \includegraphics[width=.7\linewidth]{fig/star-solution}
  \caption{\label{fig:eval.studies.star.sol}
    Analytic and constructive solutions to the star with semicircles problem.}%
\end{figure}

The analytic solution is described below.

\begin{enumerate}
  \item $r_1 = r\frac{\sin^2\frac{\pi}{n}}{\cos\frac{\pi}{n}}$
  \item $r_2 = r\cos\frac{\pi}{n} - r_1$
  \item $O_1 = O + \left(r_2, \angle\frac{\pi}{n}\right)$
  \item $I_1 = O_1 + \left(r_1, \angle\frac{2\pi}{n} - \frac{\pi}{2}\right)$
  \item $I_2 = O_1 + \left(r_1, \angle\frac{\pi}{2}\right)$
\end{enumerate}

The constructive solution is described below.  It uses two primitives from our
solution, namely \texttt{intersection} and \texttt{tangent\_circle}.

\begin{enumerate}
  \item $O_1 = \operatorname{intersection}\left(\overline{V_1 V_3},
  \overline{V_2 V_n}\right)$
  \item $C_1 = \operatorname{tangent_{circle}}\left(O_1, \overline{V_1
  V_2}\right)$
  \item $P,r_1 = C_1$
  \item $I_1 = \operatorname{intersection}\left(\overline{V_1 V_3}, C_1\right)$
  \item $I_2 = \operatorname{intersection}\left(\overline{V_2 V_n}, C_1\right)$
\end{enumerate}

Achieving the equations in the \textit{analytic solution} is not a
straightforward task. It is also unclear how those equations were derived. By
contrast, in the \textit{constructive solution}, all the steps are clearly
externalized, which makes it much easier to understand.

\subsubsection{Voronoi Diagram}%
\label{sec:eval.studies.voronoi}

Our second case study is that of Voronoi diagrams, which are used in a variety
of design fields.  For instance, several facade designs exhibit a Voronoi
appearance, such as PTW Architects' Beijing National Aquatics Center.

\Cref{fig:eval.studies.voronoi.prob} shows three Voronoi diagrams generated from
entirely randomly distributed points, from random points with one attractor
point, and from random points with one attractor line.

\begin{figure}[htb]
  \subcaptionbox{\label{fig:eval.studies.voronoi.prob.rand}}
    {\includegraphics[width=.3\linewidth]{fig/voronoi-problem-rand}}
  \hfill
  \subcaptionbox{\label{fig:eval.studies.voronoi.prob.1attr}}
    {\includegraphics[width=.3\linewidth]{fig/voronoi-problem-1attr}}
  \hfill
  \subcaptionbox{\label{fig:eval.studies.voronoi.prob.edge}}
    {\includegraphics[width=.3\linewidth]{fig/voronoi-problem-edge}}
  \caption[Voronoi diagram problem]{
    Voronoi diagram problem: \subref{fig:eval.studies.voronoi.prob.rand} is
    entirely random, \subref{fig:eval.studies.voronoi.prob.1attr} adds an
    attractor point, and \subref{fig:eval.studies.voronoi.prob.edge} adds an
    attractor line.}%
  \label{fig:eval.studies.voronoi.prob}
\end{figure}

Both the analytic and constructive methods focus on computation of a vertex
relies on the computation of the circumcenter of a triangle, for instance,
triangle $\triangle P_1 P_2 P_3$ (\cref{fig:eval.studies.voronoi.sol}).

\begin{figure}[htb]
  \centering
  \includegraphics[width=.7\linewidth]{fig/voronoi-solution}
  \caption{\label{fig:eval.studies.voronoi.sol}
    Analytic and constructive approaches to computing a Voronoi vertex.}%
\end{figure}

One possible analytic solution is based on the \textit{circumradius} formula.
The circumcenter $C$ can then be easily computed by a translation from $P_1$
following the angle $\alpha$.

\begin{enumerate}
  \item $a, b, c = \lVert P_2 - P_1 \rVert, \lVert P_3 - P_1 \rVert, \lVert P_3
  - P_2 \rVert$
  \item $s = \frac{a + b + c}{2}$
  \item $A = \sqrt{s(s - a)(s - b)(s - c)}$
  \item $r = \frac{abc}{4A}$
  \item $\alpha = \arccos\frac{a}{2r}$
  \item $C = P_1 + \left(r, \angle\alpha\right)$
\end{enumerate}

The constructive solution computes the circumcenter, directly provided by our
\primitives{}.

\begin{itemize}
  \item[] $C = \operatorname{circumcenter}\left(P_1, P_2, P_3\right)$
\end{itemize}

The circumcenter is only a sub-problem of the generation of a Voronoi diagram.
We first need to build a Delaunay triangulation.  Then, we can apply the
\texttt{circumcenter} to find the Voronoi vertices and draw the diagram's edges.

Implementing this functionality from scratch is a demanding and error-prone
task.  Fortunately, \ac{CGAL} already has an algorithm that produces Voronoi
diagrams.  This algorithm was made available in \texttt{CGAL.jl},
and, thus, it is also available in our solution.  The final section of the
evaluation goes over how we can repurpose this algorithm as a side effect of
integrating such a comprehensive library as \ac{CGAL}.

% !TeX root = ../../main.tex
\subsection{Voronoi Diagrams Extended}%
\label{sec:eval.voronoi}

In the previous section, we left the problem of Voronoi Diagrams partially
unresolved.  This section expands on it by repurposing \ac{CGAL}'s version of
the Voronoi Diagram algorithm~\cite{CGAL:5.3:VDA2} and comparing it with a
native Julia implementation of the algorithm described
in~\cite{Springel:2010:GCHSMM}, provided by the
\texttt{VoronoiDelaunay.jl}\footnote{\url{https://github.com/JuliaGeometry/VoronoiDelaunay.jl}}
package.  We estimate the effort required to obtain either implementation,
measuring Delaunay Triangulation construction performance, and compare the
outputs of both algorithms.

Wrapping \ac{CGAL}'s Voronoi diagram algorithm follows a similar process to that
used by our solution's \wrapper{} component.  Requiring bare minimal C++
knowledge and following reference documentation as if it were a recipe book, it
may take no more than a full day to obtain the necessary functionality.

The algorithm present in \texttt{VoronoiDelaunay.jl} is the result of an immense
body of research~\cite{Springel:2010:GCHSMM}.  It is safe to say that it took
more than a full day to obtain a robust implementation, requiring interpretation
and understanding of the approach described in the~\cite{Springel:2010:GCHSMM}
since there is no explicit algorithm listed.

Regarding both algorithms' performance, \cref{fig:eval.voronoi.bench} shows
the results of building Delaunay Triangulations by batch inserting several
powers-of-ten sets of points.

\begin{figure}[htb]
  \centering
  \begin{tikzpicture}
  \begin{loglogaxis}[ybar=0pt,
    title={\texttt{CGAL.jl} vs.\ \texttt{VoronoiDelaunay.jl}},
    xlabel={Number of Points (log)},
    ylabel={Average Time (ns, log)},
    width={\linewidth},
    height=5cm,
    bar width=3.16227766,% 10^(1/n) where n = 2 (bars)
    enlarge y limits={.25,upper},
    enlarge x limits={abs=3.16227766},
    legend columns=-1,
    table/col sep=comma,
  ]
    \addplot+ table {data/voronoi-vdjl.csv};
    \addplot+ table {data/voronoi-cgal.csv};
    \legend{VoronoiDelaunay.jl,CGAL.jl}
  \end{loglogaxis}
  \end{tikzpicture}
  \caption{\label{fig:eval.voronoi.bench}
    Delaunay Triangulation benchmark results.}%
\end{figure}

Results are pretty identical.  However, we see \ac{CGAL}'s variant of the
algorithm beating \texttt{VoronoiDelaunay.jl}'s by a relatively small margin.

Finally, we take a look at the output meshes produced by both implementations,
illustrated in \cref{fig:eval.voronoi.output}.

\begin{figure}[htb]
  \resizebox{\linewidth}{!}{\begin{tikzpicture}[/tikz/background rectangle/.style={fill={rgb,1:red,1.0;green,1.0;blue,1.0}, draw opacity={1.0}}, show background rectangle]
\begin{axis}[point meta max={nan}, point meta min={nan}, title={Delaunay Triangulation}, title style={at={{(0.5,1)}}, anchor={south}, font={{\fontsize{10 pt}{13.0 pt}\selectfont}}, color={rgb,1:red,0.0;green,0.0;blue,0.0}, draw opacity={1.0}, rotate={0.0}}, legend style={color={rgb,1:red,0.0;green,0.0;blue,0.0}, draw opacity={1.0}, line width={1}, solid, fill={rgb,1:red,1.0;green,1.0;blue,1.0}, fill opacity={1.0}, text opacity={1.0}, font={{\fontsize{8 pt}{10.4 pt}\selectfont}}, text={rgb,1:red,0.0;green,0.0;blue,0.0}, cells={anchor={west}}, at={(0.98, 0.98)}, anchor={north east}}, axis background/.style={fill={rgb,1:red,1.0;green,1.0;blue,1.0}, opacity={1.0}}, anchor={north west}, xshift={1.0mm}, yshift={-1.0mm}, width={86.9mm}, height={86.9mm}, scaled x ticks={false}, xlabel={$x$}, x tick style={color={rgb,1:red,0.0;green,0.0;blue,0.0}, opacity={1.0}}, x tick label style={color={rgb,1:red,0.0;green,0.0;blue,0.0}, opacity={1.0}, rotate={0}}, xlabel style={at={(ticklabel cs:0.5)}, anchor=near ticklabel, font={{\fontsize{11 pt}{14.3 pt}\selectfont}}, color={rgb,1:red,0.0;green,0.0;blue,0.0}, draw opacity={1.0}, rotate={0.0}}, xmajorgrids={true}, xmin={0.9}, xmax={2.1}, xtick={{1.0,1.25,1.5,1.75,2.0}}, xticklabels={{$1.00$,$1.25$,$1.50$,$1.75$,$2.00$}}, xtick align={inside}, xticklabel style={font={{\fontsize{8 pt}{10.4 pt}\selectfont}}, color={rgb,1:red,0.0;green,0.0;blue,0.0}, draw opacity={1.0}, rotate={0.0}}, x grid style={color={rgb,1:red,0.0;green,0.0;blue,0.0}, draw opacity={0.1}, line width={0.5}, solid}, axis x line*={left}, x axis line style={color={rgb,1:red,0.0;green,0.0;blue,0.0}, draw opacity={1.0}, line width={1}, solid}, scaled y ticks={false}, ylabel={$y$}, y tick style={color={rgb,1:red,0.0;green,0.0;blue,0.0}, opacity={1.0}}, y tick label style={color={rgb,1:red,0.0;green,0.0;blue,0.0}, opacity={1.0}, rotate={0}}, ylabel style={at={(ticklabel cs:0.5)}, anchor=near ticklabel, font={{\fontsize{11 pt}{14.3 pt}\selectfont}}, color={rgb,1:red,0.0;green,0.0;blue,0.0}, draw opacity={1.0}, rotate={0.0}}, ymajorgrids={true}, ymin={0.9}, ymax={2.1}, ytick={{1.0,1.25,1.5,1.75,2.0}}, yticklabels={{$1.00$,$1.25$,$1.50$,$1.75$,$2.00$}}, ytick align={inside}, yticklabel style={font={{\fontsize{8 pt}{10.4 pt}\selectfont}}, color={rgb,1:red,0.0;green,0.0;blue,0.0}, draw opacity={1.0}, rotate={0.0}}, y grid style={color={rgb,1:red,0.0;green,0.0;blue,0.0}, draw opacity={0.1}, line width={0.5}, solid}, axis y line*={left}, y axis line style={color={rgb,1:red,0.0;green,0.0;blue,0.0}, draw opacity={1.0}, line width={1}, solid}, colorbar={false}, legend columns={-1}]
    \addplot[color={rgb,1:red,1.0;green,0.0;blue,0.0}, name path={d7364828-3d0d-41da-8576-92cd13e62e99}, draw opacity={1.0}, line width={1}, solid]
        table[row sep={\\}]
        {
            \\
            1.1748122130520642  1.9119394752196963  \\
            1.3983968361280859  1.8132919843774284  \\
        }
        ;
    \addlegendentry {CGAL.jl}
    \addplot[color={rgb,1:red,1.0;green,0.0;blue,0.0}, name path={d7364828-3d0d-41da-8576-92cd13e62e99}, draw opacity={1.0}, line width={1}, solid, forget plot]
        table[row sep={\\}]
        {
            \\
            1.3983968361280859  1.8132919843774284  \\
            1.3702157551888376  1.9199406439438458  \\
        }
        ;
    \addplot[color={rgb,1:red,1.0;green,0.0;blue,0.0}, name path={d7364828-3d0d-41da-8576-92cd13e62e99}, draw opacity={1.0}, line width={1}, solid, forget plot]
        table[row sep={\\}]
        {
            \\
            1.3702157551888376  1.9199406439438458  \\
            1.1748122130520642  1.9119394752196963  \\
        }
        ;
    \addplot[color={rgb,1:red,1.0;green,0.0;blue,0.0}, name path={d7364828-3d0d-41da-8576-92cd13e62e99}, draw opacity={1.0}, line width={1}, solid, forget plot]
        table[row sep={\\}]
        {
            \\
            1.3983968361280859  1.8132919843774284  \\
            1.411079611396417  1.603287132689729  \\
        }
        ;
    \addplot[color={rgb,1:red,1.0;green,0.0;blue,0.0}, name path={d7364828-3d0d-41da-8576-92cd13e62e99}, draw opacity={1.0}, line width={1}, solid, forget plot]
        table[row sep={\\}]
        {
            \\
            1.411079611396417  1.603287132689729  \\
            1.4094003408681601  1.8230532801244403  \\
        }
        ;
    \addplot[color={rgb,1:red,1.0;green,0.0;blue,0.0}, name path={d7364828-3d0d-41da-8576-92cd13e62e99}, draw opacity={1.0}, line width={1}, solid, forget plot]
        table[row sep={\\}]
        {
            \\
            1.4094003408681601  1.8230532801244403  \\
            1.3983968361280859  1.8132919843774284  \\
        }
        ;
    \addplot[color={rgb,1:red,1.0;green,0.0;blue,0.0}, name path={d7364828-3d0d-41da-8576-92cd13e62e99}, draw opacity={1.0}, line width={1}, solid, forget plot]
        table[row sep={\\}]
        {
            \\
            1.8946103360503908  1.4291503357235342  \\
            1.8488018650095905  1.693958807038143  \\
        }
        ;
    \addplot[color={rgb,1:red,1.0;green,0.0;blue,0.0}, name path={d7364828-3d0d-41da-8576-92cd13e62e99}, draw opacity={1.0}, line width={1}, solid, forget plot]
        table[row sep={\\}]
        {
            \\
            1.8488018650095905  1.693958807038143  \\
            1.8273643609136339  1.502810534555465  \\
        }
        ;
    \addplot[color={rgb,1:red,1.0;green,0.0;blue,0.0}, name path={d7364828-3d0d-41da-8576-92cd13e62e99}, draw opacity={1.0}, line width={1}, solid, forget plot]
        table[row sep={\\}]
        {
            \\
            1.8273643609136339  1.502810534555465  \\
            1.8946103360503908  1.4291503357235342  \\
        }
        ;
    \addplot[color={rgb,1:red,1.0;green,0.0;blue,0.0}, name path={d7364828-3d0d-41da-8576-92cd13e62e99}, draw opacity={1.0}, line width={1}, solid, forget plot]
        table[row sep={\\}]
        {
            \\
            1.670456608990207  1.4213281597476453  \\
            1.8273643609136339  1.502810534555465  \\
        }
        ;
    \addplot[color={rgb,1:red,1.0;green,0.0;blue,0.0}, name path={d7364828-3d0d-41da-8576-92cd13e62e99}, draw opacity={1.0}, line width={1}, solid, forget plot]
        table[row sep={\\}]
        {
            \\
            1.8273643609136339  1.502810534555465  \\
            1.7508308666292574  1.5433638719841838  \\
        }
        ;
    \addplot[color={rgb,1:red,1.0;green,0.0;blue,0.0}, name path={d7364828-3d0d-41da-8576-92cd13e62e99}, draw opacity={1.0}, line width={1}, solid, forget plot]
        table[row sep={\\}]
        {
            \\
            1.7508308666292574  1.5433638719841838  \\
            1.670456608990207  1.4213281597476453  \\
        }
        ;
    \addplot[color={rgb,1:red,1.0;green,0.0;blue,0.0}, name path={d7364828-3d0d-41da-8576-92cd13e62e99}, draw opacity={1.0}, line width={1}, solid, forget plot]
        table[row sep={\\}]
        {
            \\
            1.4094003408681601  1.8230532801244403  \\
            1.8488018650095905  1.693958807038143  \\
        }
        ;
    \addplot[color={rgb,1:red,1.0;green,0.0;blue,0.0}, name path={d7364828-3d0d-41da-8576-92cd13e62e99}, draw opacity={1.0}, line width={1}, solid, forget plot]
        table[row sep={\\}]
        {
            \\
            1.8488018650095905  1.693958807038143  \\
            1.3702157551888376  1.9199406439438458  \\
        }
        ;
    \addplot[color={rgb,1:red,1.0;green,0.0;blue,0.0}, name path={d7364828-3d0d-41da-8576-92cd13e62e99}, draw opacity={1.0}, line width={1}, solid, forget plot]
        table[row sep={\\}]
        {
            \\
            1.3702157551888376  1.9199406439438458  \\
            1.4094003408681601  1.8230532801244403  \\
        }
        ;
    \addplot[color={rgb,1:red,1.0;green,0.0;blue,0.0}, name path={d7364828-3d0d-41da-8576-92cd13e62e99}, draw opacity={1.0}, line width={1}, solid, forget plot]
        table[row sep={\\}]
        {
            \\
            1.7508308666292574  1.5433638719841838  \\
            1.8488018650095905  1.693958807038143  \\
        }
        ;
    \addplot[color={rgb,1:red,1.0;green,0.0;blue,0.0}, name path={d7364828-3d0d-41da-8576-92cd13e62e99}, draw opacity={1.0}, line width={1}, solid, forget plot]
        table[row sep={\\}]
        {
            \\
            1.4094003408681601  1.8230532801244403  \\
            1.7508308666292574  1.5433638719841838  \\
        }
        ;
    \addplot[color={rgb,1:red,1.0;green,0.0;blue,0.0}, name path={d7364828-3d0d-41da-8576-92cd13e62e99}, draw opacity={1.0}, line width={1}, solid, forget plot]
        table[row sep={\\}]
        {
            \\
            1.6398598169907925  1.1640638990793377  \\
            1.8701207209378476  1.1696756919845939  \\
        }
        ;
    \addplot[color={rgb,1:red,1.0;green,0.0;blue,0.0}, name path={d7364828-3d0d-41da-8576-92cd13e62e99}, draw opacity={1.0}, line width={1}, solid, forget plot]
        table[row sep={\\}]
        {
            \\
            1.8701207209378476  1.1696756919845939  \\
            1.670456608990207  1.4213281597476453  \\
        }
        ;
    \addplot[color={rgb,1:red,1.0;green,0.0;blue,0.0}, name path={d7364828-3d0d-41da-8576-92cd13e62e99}, draw opacity={1.0}, line width={1}, solid, forget plot]
        table[row sep={\\}]
        {
            \\
            1.670456608990207  1.4213281597476453  \\
            1.6398598169907925  1.1640638990793377  \\
        }
        ;
    \addplot[color={rgb,1:red,1.0;green,0.0;blue,0.0}, name path={d7364828-3d0d-41da-8576-92cd13e62e99}, draw opacity={1.0}, line width={1}, solid, forget plot]
        table[row sep={\\}]
        {
            \\
            1.1105229298118504  1.3180863440502435  \\
            1.5939507135190067  1.132305847713724  \\
        }
        ;
    \addplot[color={rgb,1:red,1.0;green,0.0;blue,0.0}, name path={d7364828-3d0d-41da-8576-92cd13e62e99}, draw opacity={1.0}, line width={1}, solid, forget plot]
        table[row sep={\\}]
        {
            \\
            1.5939507135190067  1.132305847713724  \\
            1.3071074625477195  1.3007149568293244  \\
        }
        ;
    \addplot[color={rgb,1:red,1.0;green,0.0;blue,0.0}, name path={d7364828-3d0d-41da-8576-92cd13e62e99}, draw opacity={1.0}, line width={1}, solid, forget plot]
        table[row sep={\\}]
        {
            \\
            1.3071074625477195  1.3007149568293244  \\
            1.1105229298118504  1.3180863440502435  \\
        }
        ;
    \addplot[color={rgb,1:red,1.0;green,0.0;blue,0.0}, name path={d7364828-3d0d-41da-8576-92cd13e62e99}, draw opacity={1.0}, line width={1}, solid, forget plot]
        table[row sep={\\}]
        {
            \\
            1.3230645160656422  1.6671039194334296  \\
            1.1387269033584746  1.3902763912919909  \\
        }
        ;
    \addplot[color={rgb,1:red,1.0;green,0.0;blue,0.0}, name path={d7364828-3d0d-41da-8576-92cd13e62e99}, draw opacity={1.0}, line width={1}, solid, forget plot]
        table[row sep={\\}]
        {
            \\
            1.1387269033584746  1.3902763912919909  \\
            1.4044450228102505  1.4121578056365252  \\
        }
        ;
    \addplot[color={rgb,1:red,1.0;green,0.0;blue,0.0}, name path={d7364828-3d0d-41da-8576-92cd13e62e99}, draw opacity={1.0}, line width={1}, solid, forget plot]
        table[row sep={\\}]
        {
            \\
            1.4044450228102505  1.4121578056365252  \\
            1.3230645160656422  1.6671039194334296  \\
        }
        ;
    \addplot[color={rgb,1:red,1.0;green,0.0;blue,0.0}, name path={d7364828-3d0d-41da-8576-92cd13e62e99}, draw opacity={1.0}, line width={1}, solid, forget plot]
        table[row sep={\\}]
        {
            \\
            1.5201537685934454  1.5121217011474073  \\
            1.4094003408681601  1.8230532801244403  \\
        }
        ;
    \addplot[color={rgb,1:red,1.0;green,0.0;blue,0.0}, name path={d7364828-3d0d-41da-8576-92cd13e62e99}, draw opacity={1.0}, line width={1}, solid, forget plot]
        table[row sep={\\}]
        {
            \\
            1.411079611396417  1.603287132689729  \\
            1.5201537685934454  1.5121217011474073  \\
        }
        ;
    \addplot[color={rgb,1:red,1.0;green,0.0;blue,0.0}, name path={d7364828-3d0d-41da-8576-92cd13e62e99}, draw opacity={1.0}, line width={1}, solid, forget plot]
        table[row sep={\\}]
        {
            \\
            1.5939507135190067  1.132305847713724  \\
            1.4044450228102505  1.4121578056365252  \\
        }
        ;
    \addplot[color={rgb,1:red,1.0;green,0.0;blue,0.0}, name path={d7364828-3d0d-41da-8576-92cd13e62e99}, draw opacity={1.0}, line width={1}, solid, forget plot]
        table[row sep={\\}]
        {
            \\
            1.4044450228102505  1.4121578056365252  \\
            1.3071074625477195  1.3007149568293244  \\
        }
        ;
    \addplot[color={rgb,1:red,1.0;green,0.0;blue,0.0}, name path={d7364828-3d0d-41da-8576-92cd13e62e99}, draw opacity={1.0}, line width={1}, solid, forget plot]
        table[row sep={\\}]
        {
            \\
            1.1387269033584746  1.3902763912919909  \\
            1.3071074625477195  1.3007149568293244  \\
        }
        ;
    \addplot[color={rgb,1:red,1.0;green,0.0;blue,0.0}, name path={d7364828-3d0d-41da-8576-92cd13e62e99}, draw opacity={1.0}, line width={1}, solid, forget plot]
        table[row sep={\\}]
        {
            \\
            1.1387269033584746  1.3902763912919909  \\
            1.125095868483186  1.7752906912937756  \\
        }
        ;
    \addplot[color={rgb,1:red,1.0;green,0.0;blue,0.0}, name path={d7364828-3d0d-41da-8576-92cd13e62e99}, draw opacity={1.0}, line width={1}, solid, forget plot]
        table[row sep={\\}]
        {
            \\
            1.125095868483186  1.7752906912937756  \\
            1.1105229298118504  1.3180863440502435  \\
        }
        ;
    \addplot[color={rgb,1:red,1.0;green,0.0;blue,0.0}, name path={d7364828-3d0d-41da-8576-92cd13e62e99}, draw opacity={1.0}, line width={1}, solid, forget plot]
        table[row sep={\\}]
        {
            \\
            1.1105229298118504  1.3180863440502435  \\
            1.1387269033584746  1.3902763912919909  \\
        }
        ;
    \addplot[color={rgb,1:red,1.0;green,0.0;blue,0.0}, name path={d7364828-3d0d-41da-8576-92cd13e62e99}, draw opacity={1.0}, line width={1}, solid, forget plot]
        table[row sep={\\}]
        {
            \\
            1.3230645160656422  1.6671039194334296  \\
            1.125095868483186  1.7752906912937756  \\
        }
        ;
    \addplot[color={rgb,1:red,1.0;green,0.0;blue,0.0}, name path={d7364828-3d0d-41da-8576-92cd13e62e99}, draw opacity={1.0}, line width={1}, solid, forget plot]
        table[row sep={\\}]
        {
            \\
            1.3983968361280859  1.8132919843774284  \\
            1.125095868483186  1.7752906912937756  \\
        }
        ;
    \addplot[color={rgb,1:red,1.0;green,0.0;blue,0.0}, name path={d7364828-3d0d-41da-8576-92cd13e62e99}, draw opacity={1.0}, line width={1}, solid, forget plot]
        table[row sep={\\}]
        {
            \\
            1.3230645160656422  1.6671039194334296  \\
            1.3983968361280859  1.8132919843774284  \\
        }
        ;
    \addplot[color={rgb,1:red,1.0;green,0.0;blue,0.0}, name path={d7364828-3d0d-41da-8576-92cd13e62e99}, draw opacity={1.0}, line width={1}, solid, forget plot]
        table[row sep={\\}]
        {
            \\
            1.4235251017380506  1.5806863505858928  \\
            1.5201537685934454  1.5121217011474073  \\
        }
        ;
    \addplot[color={rgb,1:red,1.0;green,0.0;blue,0.0}, name path={d7364828-3d0d-41da-8576-92cd13e62e99}, draw opacity={1.0}, line width={1}, solid, forget plot]
        table[row sep={\\}]
        {
            \\
            1.411079611396417  1.603287132689729  \\
            1.4235251017380506  1.5806863505858928  \\
        }
        ;
    \addplot[color={rgb,1:red,1.0;green,0.0;blue,0.0}, name path={d7364828-3d0d-41da-8576-92cd13e62e99}, draw opacity={1.0}, line width={1}, solid, forget plot]
        table[row sep={\\}]
        {
            \\
            1.125095868483186  1.7752906912937756  \\
            1.1748122130520642  1.9119394752196963  \\
        }
        ;
    \addplot[color={rgb,1:red,1.0;green,0.0;blue,0.0}, name path={d7364828-3d0d-41da-8576-92cd13e62e99}, draw opacity={1.0}, line width={1}, solid, forget plot]
        table[row sep={\\}]
        {
            \\
            1.4235251017380506  1.5806863505858928  \\
            1.3230645160656422  1.6671039194334296  \\
        }
        ;
    \addplot[color={rgb,1:red,1.0;green,0.0;blue,0.0}, name path={d7364828-3d0d-41da-8576-92cd13e62e99}, draw opacity={1.0}, line width={1}, solid, forget plot]
        table[row sep={\\}]
        {
            \\
            1.4044450228102505  1.4121578056365252  \\
            1.4235251017380506  1.5806863505858928  \\
        }
        ;
    \addplot[color={rgb,1:red,1.0;green,0.0;blue,0.0}, name path={d7364828-3d0d-41da-8576-92cd13e62e99}, draw opacity={1.0}, line width={1}, solid, forget plot]
        table[row sep={\\}]
        {
            \\
            1.5201537685934454  1.5121217011474073  \\
            1.7508308666292574  1.5433638719841838  \\
        }
        ;
    \addplot[color={rgb,1:red,1.0;green,0.0;blue,0.0}, name path={d7364828-3d0d-41da-8576-92cd13e62e99}, draw opacity={1.0}, line width={1}, solid, forget plot]
        table[row sep={\\}]
        {
            \\
            1.4044450228102505  1.4121578056365252  \\
            1.5201537685934454  1.5121217011474073  \\
        }
        ;
    \addplot[color={rgb,1:red,1.0;green,0.0;blue,0.0}, name path={d7364828-3d0d-41da-8576-92cd13e62e99}, draw opacity={1.0}, line width={1}, solid, forget plot]
        table[row sep={\\}]
        {
            \\
            1.670456608990207  1.4213281597476453  \\
            1.4044450228102505  1.4121578056365252  \\
        }
        ;
    \addplot[color={rgb,1:red,1.0;green,0.0;blue,0.0}, name path={d7364828-3d0d-41da-8576-92cd13e62e99}, draw opacity={1.0}, line width={1}, solid, forget plot]
        table[row sep={\\}]
        {
            \\
            1.4044450228102505  1.4121578056365252  \\
            1.6398598169907925  1.1640638990793377  \\
        }
        ;
    \addplot[color={rgb,1:red,1.0;green,0.0;blue,0.0}, name path={d7364828-3d0d-41da-8576-92cd13e62e99}, draw opacity={1.0}, line width={1}, solid, forget plot]
        table[row sep={\\}]
        {
            \\
            1.3230645160656422  1.6671039194334296  \\
            1.411079611396417  1.603287132689729  \\
        }
        ;
    \addplot[color={rgb,1:red,1.0;green,0.0;blue,0.0}, name path={d7364828-3d0d-41da-8576-92cd13e62e99}, draw opacity={1.0}, line width={1}, solid, forget plot]
        table[row sep={\\}]
        {
            \\
            1.670456608990207  1.4213281597476453  \\
            1.8946103360503908  1.4291503357235342  \\
        }
        ;
    \addplot[color={rgb,1:red,1.0;green,0.0;blue,0.0}, name path={d7364828-3d0d-41da-8576-92cd13e62e99}, draw opacity={1.0}, line width={1}, solid, forget plot]
        table[row sep={\\}]
        {
            \\
            1.8946103360503908  1.4291503357235342  \\
            1.8701207209378476  1.1696756919845939  \\
        }
        ;
    \addplot[color={rgb,1:red,1.0;green,0.0;blue,0.0}, name path={d7364828-3d0d-41da-8576-92cd13e62e99}, draw opacity={1.0}, line width={1}, solid, forget plot]
        table[row sep={\\}]
        {
            \\
            1.5201537685934454  1.5121217011474073  \\
            1.670456608990207  1.4213281597476453  \\
        }
        ;
    \addplot[color={rgb,1:red,1.0;green,0.0;blue,0.0}, name path={d7364828-3d0d-41da-8576-92cd13e62e99}, draw opacity={1.0}, line width={1}, solid, forget plot]
        table[row sep={\\}]
        {
            \\
            1.5939507135190067  1.132305847713724  \\
            1.8701207209378476  1.1696756919845939  \\
        }
        ;
    \addplot[color={rgb,1:red,1.0;green,0.0;blue,0.0}, name path={d7364828-3d0d-41da-8576-92cd13e62e99}, draw opacity={1.0}, line width={1}, solid, forget plot]
        table[row sep={\\}]
        {
            \\
            1.6398598169907925  1.1640638990793377  \\
            1.5939507135190067  1.132305847713724  \\
        }
        ;
    \addplot[color={rgb,1:red,0.0;green,1.0;blue,0.498}, name path={39a7f4b5-27cf-4c6f-86dc-50813783b1d9}, draw opacity={0.8}, line width={1}, solid]
        table[row sep={\\}]
        {
            \\
            1.411079611396417  1.603287132689729  \\
            1.3983968361280859  1.8132919843774284  \\
        }
        ;
    \addlegendentry {VoronoiDelaunay.jl}
    \addplot[color={rgb,1:red,0.0;green,1.0;blue,0.498}, name path={39a7f4b5-27cf-4c6f-86dc-50813783b1d9}, draw opacity={0.8}, line width={1}, solid, forget plot]
        table[row sep={\\}]
        {
            \\
            1.4094003408681601  1.8230532801244403  \\
            1.3983968361280859  1.8132919843774284  \\
        }
        ;
    \addplot[color={rgb,1:red,0.0;green,1.0;blue,0.498}, name path={39a7f4b5-27cf-4c6f-86dc-50813783b1d9}, draw opacity={0.8}, line width={1}, solid, forget plot]
        table[row sep={\\}]
        {
            \\
            1.4094003408681601  1.8230532801244403  \\
            1.411079611396417  1.603287132689729  \\
        }
        ;
    \addplot[color={rgb,1:red,0.0;green,1.0;blue,0.498}, name path={39a7f4b5-27cf-4c6f-86dc-50813783b1d9}, draw opacity={0.8}, line width={1}, solid, forget plot]
        table[row sep={\\}]
        {
            \\
            1.1387269033584746  1.3902763912919909  \\
            1.4044450228102505  1.4121578056365252  \\
        }
        ;
    \addplot[color={rgb,1:red,0.0;green,1.0;blue,0.498}, name path={39a7f4b5-27cf-4c6f-86dc-50813783b1d9}, draw opacity={0.8}, line width={1}, solid, forget plot]
        table[row sep={\\}]
        {
            \\
            1.3071074625477195  1.3007149568293244  \\
            1.4044450228102505  1.4121578056365252  \\
        }
        ;
    \addplot[color={rgb,1:red,0.0;green,1.0;blue,0.498}, name path={39a7f4b5-27cf-4c6f-86dc-50813783b1d9}, draw opacity={0.8}, line width={1}, solid, forget plot]
        table[row sep={\\}]
        {
            \\
            1.3071074625477195  1.3007149568293244  \\
            1.1387269033584746  1.3902763912919909  \\
        }
        ;
    \addplot[color={rgb,1:red,0.0;green,1.0;blue,0.498}, name path={39a7f4b5-27cf-4c6f-86dc-50813783b1d9}, draw opacity={0.8}, line width={1}, solid, forget plot]
        table[row sep={\\}]
        {
            \\
            1.5201537685934454  1.5121217011474073  \\
            1.4094003408681601  1.8230532801244403  \\
        }
        ;
    \addplot[color={rgb,1:red,0.0;green,1.0;blue,0.498}, name path={39a7f4b5-27cf-4c6f-86dc-50813783b1d9}, draw opacity={0.8}, line width={1}, solid, forget plot]
        table[row sep={\\}]
        {
            \\
            1.5201537685934454  1.5121217011474073  \\
            1.411079611396417  1.603287132689729  \\
        }
        ;
    \addplot[color={rgb,1:red,0.0;green,1.0;blue,0.498}, name path={39a7f4b5-27cf-4c6f-86dc-50813783b1d9}, draw opacity={0.8}, line width={1}, solid, forget plot]
        table[row sep={\\}]
        {
            \\
            1.670456608990207  1.4213281597476453  \\
            1.6398598169907925  1.1640638990793377  \\
        }
        ;
    \addplot[color={rgb,1:red,0.0;green,1.0;blue,0.498}, name path={39a7f4b5-27cf-4c6f-86dc-50813783b1d9}, draw opacity={0.8}, line width={1}, solid, forget plot]
        table[row sep={\\}]
        {
            \\
            1.4044450228102505  1.4121578056365252  \\
            1.6398598169907925  1.1640638990793377  \\
        }
        ;
    \addplot[color={rgb,1:red,0.0;green,1.0;blue,0.498}, name path={39a7f4b5-27cf-4c6f-86dc-50813783b1d9}, draw opacity={0.8}, line width={1}, solid, forget plot]
        table[row sep={\\}]
        {
            \\
            1.4044450228102505  1.4121578056365252  \\
            1.670456608990207  1.4213281597476453  \\
        }
        ;
    \addplot[color={rgb,1:red,0.0;green,1.0;blue,0.498}, name path={39a7f4b5-27cf-4c6f-86dc-50813783b1d9}, draw opacity={0.8}, line width={1}, solid, forget plot]
        table[row sep={\\}]
        {
            \\
            1.5201537685934454  1.5121217011474073  \\
            1.670456608990207  1.4213281597476453  \\
        }
        ;
    \addplot[color={rgb,1:red,0.0;green,1.0;blue,0.498}, name path={39a7f4b5-27cf-4c6f-86dc-50813783b1d9}, draw opacity={0.8}, line width={1}, solid, forget plot]
        table[row sep={\\}]
        {
            \\
            1.4044450228102505  1.4121578056365252  \\
            1.5201537685934454  1.5121217011474073  \\
        }
        ;
    \addplot[color={rgb,1:red,0.0;green,1.0;blue,0.498}, name path={39a7f4b5-27cf-4c6f-86dc-50813783b1d9}, draw opacity={0.8}, line width={1}, solid, forget plot]
        table[row sep={\\}]
        {
            \\
            1.125095868483186  1.7752906912937756  \\
            1.3230645160656422  1.6671039194334296  \\
        }
        ;
    \addplot[color={rgb,1:red,0.0;green,1.0;blue,0.498}, name path={39a7f4b5-27cf-4c6f-86dc-50813783b1d9}, draw opacity={0.8}, line width={1}, solid, forget plot]
        table[row sep={\\}]
        {
            \\
            1.1387269033584746  1.3902763912919909  \\
            1.3230645160656422  1.6671039194334296  \\
        }
        ;
    \addplot[color={rgb,1:red,0.0;green,1.0;blue,0.498}, name path={39a7f4b5-27cf-4c6f-86dc-50813783b1d9}, draw opacity={0.8}, line width={1}, solid, forget plot]
        table[row sep={\\}]
        {
            \\
            1.1387269033584746  1.3902763912919909  \\
            1.125095868483186  1.7752906912937756  \\
        }
        ;
    \addplot[color={rgb,1:red,0.0;green,1.0;blue,0.498}, name path={39a7f4b5-27cf-4c6f-86dc-50813783b1d9}, draw opacity={0.8}, line width={1}, solid, forget plot]
        table[row sep={\\}]
        {
            \\
            1.3702157551888376  1.9199406439438458  \\
            1.3983968361280859  1.8132919843774284  \\
        }
        ;
    \addplot[color={rgb,1:red,0.0;green,1.0;blue,0.498}, name path={39a7f4b5-27cf-4c6f-86dc-50813783b1d9}, draw opacity={0.8}, line width={1}, solid, forget plot]
        table[row sep={\\}]
        {
            \\
            1.1748122130520642  1.9119394752196963  \\
            1.3983968361280859  1.8132919843774284  \\
        }
        ;
    \addplot[color={rgb,1:red,0.0;green,1.0;blue,0.498}, name path={39a7f4b5-27cf-4c6f-86dc-50813783b1d9}, draw opacity={0.8}, line width={1}, solid, forget plot]
        table[row sep={\\}]
        {
            \\
            1.1748122130520642  1.9119394752196963  \\
            1.3702157551888376  1.9199406439438458  \\
        }
        ;
    \addplot[color={rgb,1:red,0.0;green,1.0;blue,0.498}, name path={39a7f4b5-27cf-4c6f-86dc-50813783b1d9}, draw opacity={0.8}, line width={1}, solid, forget plot]
        table[row sep={\\}]
        {
            \\
            1.3983968361280859  1.8132919843774284  \\
            1.125095868483186  1.7752906912937756  \\
        }
        ;
    \addplot[color={rgb,1:red,0.0;green,1.0;blue,0.498}, name path={39a7f4b5-27cf-4c6f-86dc-50813783b1d9}, draw opacity={0.8}, line width={1}, solid, forget plot]
        table[row sep={\\}]
        {
            \\
            1.3983968361280859  1.8132919843774284  \\
            1.3230645160656422  1.6671039194334296  \\
        }
        ;
    \addplot[color={rgb,1:red,0.0;green,1.0;blue,0.498}, name path={39a7f4b5-27cf-4c6f-86dc-50813783b1d9}, draw opacity={0.8}, line width={1}, solid, forget plot]
        table[row sep={\\}]
        {
            \\
            1.4044450228102505  1.4121578056365252  \\
            1.3230645160656422  1.6671039194334296  \\
        }
        ;
    \addplot[color={rgb,1:red,0.0;green,1.0;blue,0.498}, name path={39a7f4b5-27cf-4c6f-86dc-50813783b1d9}, draw opacity={0.8}, line width={1}, solid, forget plot]
        table[row sep={\\}]
        {
            \\
            1.4094003408681601  1.8230532801244403  \\
            1.7508308666292574  1.5433638719841838  \\
        }
        ;
    \addplot[color={rgb,1:red,0.0;green,1.0;blue,0.498}, name path={39a7f4b5-27cf-4c6f-86dc-50813783b1d9}, draw opacity={0.8}, line width={1}, solid, forget plot]
        table[row sep={\\}]
        {
            \\
            1.5201537685934454  1.5121217011474073  \\
            1.7508308666292574  1.5433638719841838  \\
        }
        ;
    \addplot[color={rgb,1:red,0.0;green,1.0;blue,0.498}, name path={39a7f4b5-27cf-4c6f-86dc-50813783b1d9}, draw opacity={0.8}, line width={1}, solid, forget plot]
        table[row sep={\\}]
        {
            \\
            1.6398598169907925  1.1640638990793377  \\
            1.5939507135190067  1.132305847713724  \\
        }
        ;
    \addplot[color={rgb,1:red,0.0;green,1.0;blue,0.498}, name path={39a7f4b5-27cf-4c6f-86dc-50813783b1d9}, draw opacity={0.8}, line width={1}, solid, forget plot]
        table[row sep={\\}]
        {
            \\
            1.4044450228102505  1.4121578056365252  \\
            1.5939507135190067  1.132305847713724  \\
        }
        ;
    \addplot[color={rgb,1:red,0.0;green,1.0;blue,0.498}, name path={39a7f4b5-27cf-4c6f-86dc-50813783b1d9}, draw opacity={0.8}, line width={1}, solid, forget plot]
        table[row sep={\\}]
        {
            \\
            1.670456608990207  1.4213281597476453  \\
            1.8701207209378476  1.1696756919845939  \\
        }
        ;
    \addplot[color={rgb,1:red,0.0;green,1.0;blue,0.498}, name path={39a7f4b5-27cf-4c6f-86dc-50813783b1d9}, draw opacity={0.8}, line width={1}, solid, forget plot]
        table[row sep={\\}]
        {
            \\
            1.6398598169907925  1.1640638990793377  \\
            1.8701207209378476  1.1696756919845939  \\
        }
        ;
    \addplot[color={rgb,1:red,0.0;green,1.0;blue,0.498}, name path={39a7f4b5-27cf-4c6f-86dc-50813783b1d9}, draw opacity={0.8}, line width={1}, solid, forget plot]
        table[row sep={\\}]
        {
            \\
            1.8488018650095905  1.693958807038143  \\
            1.8946103360503908  1.4291503357235342  \\
        }
        ;
    \addplot[color={rgb,1:red,0.0;green,1.0;blue,0.498}, name path={39a7f4b5-27cf-4c6f-86dc-50813783b1d9}, draw opacity={0.8}, line width={1}, solid, forget plot]
        table[row sep={\\}]
        {
            \\
            1.8273643609136339  1.502810534555465  \\
            1.8946103360503908  1.4291503357235342  \\
        }
        ;
    \addplot[color={rgb,1:red,0.0;green,1.0;blue,0.498}, name path={39a7f4b5-27cf-4c6f-86dc-50813783b1d9}, draw opacity={0.8}, line width={1}, solid, forget plot]
        table[row sep={\\}]
        {
            \\
            1.8273643609136339  1.502810534555465  \\
            1.8488018650095905  1.693958807038143  \\
        }
        ;
    \addplot[color={rgb,1:red,0.0;green,1.0;blue,0.498}, name path={39a7f4b5-27cf-4c6f-86dc-50813783b1d9}, draw opacity={0.8}, line width={1}, solid, forget plot]
        table[row sep={\\}]
        {
            \\
            1.5939507135190067  1.132305847713724  \\
            1.3071074625477195  1.3007149568293244  \\
        }
        ;
    \addplot[color={rgb,1:red,0.0;green,1.0;blue,0.498}, name path={39a7f4b5-27cf-4c6f-86dc-50813783b1d9}, draw opacity={0.8}, line width={1}, solid, forget plot]
        table[row sep={\\}]
        {
            \\
            1.670456608990207  1.4213281597476453  \\
            1.7508308666292574  1.5433638719841838  \\
        }
        ;
    \addplot[color={rgb,1:red,0.0;green,1.0;blue,0.498}, name path={39a7f4b5-27cf-4c6f-86dc-50813783b1d9}, draw opacity={0.8}, line width={1}, solid, forget plot]
        table[row sep={\\}]
        {
            \\
            1.4235251017380506  1.5806863505858928  \\
            1.3230645160656422  1.6671039194334296  \\
        }
        ;
    \addplot[color={rgb,1:red,0.0;green,1.0;blue,0.498}, name path={39a7f4b5-27cf-4c6f-86dc-50813783b1d9}, draw opacity={0.8}, line width={1}, solid, forget plot]
        table[row sep={\\}]
        {
            \\
            1.4235251017380506  1.5806863505858928  \\
            1.4044450228102505  1.4121578056365252  \\
        }
        ;
    \addplot[color={rgb,1:red,0.0;green,1.0;blue,0.498}, name path={39a7f4b5-27cf-4c6f-86dc-50813783b1d9}, draw opacity={0.8}, line width={1}, solid, forget plot]
        table[row sep={\\}]
        {
            \\
            1.4235251017380506  1.5806863505858928  \\
            1.5201537685934454  1.5121217011474073  \\
        }
        ;
    \addplot[color={rgb,1:red,0.0;green,1.0;blue,0.498}, name path={39a7f4b5-27cf-4c6f-86dc-50813783b1d9}, draw opacity={0.8}, line width={1}, solid, forget plot]
        table[row sep={\\}]
        {
            \\
            1.1748122130520642  1.9119394752196963  \\
            1.125095868483186  1.7752906912937756  \\
        }
        ;
    \addplot[color={rgb,1:red,0.0;green,1.0;blue,0.498}, name path={39a7f4b5-27cf-4c6f-86dc-50813783b1d9}, draw opacity={0.8}, line width={1}, solid, forget plot]
        table[row sep={\\}]
        {
            \\
            1.670456608990207  1.4213281597476453  \\
            1.8946103360503908  1.4291503357235342  \\
        }
        ;
    \addplot[color={rgb,1:red,0.0;green,1.0;blue,0.498}, name path={39a7f4b5-27cf-4c6f-86dc-50813783b1d9}, draw opacity={0.8}, line width={1}, solid, forget plot]
        table[row sep={\\}]
        {
            \\
            1.670456608990207  1.4213281597476453  \\
            1.8273643609136339  1.502810534555465  \\
        }
        ;
    \addplot[color={rgb,1:red,0.0;green,1.0;blue,0.498}, name path={39a7f4b5-27cf-4c6f-86dc-50813783b1d9}, draw opacity={0.8}, line width={1}, solid, forget plot]
        table[row sep={\\}]
        {
            \\
            1.8488018650095905  1.693958807038143  \\
            1.7508308666292574  1.5433638719841838  \\
        }
        ;
    \addplot[color={rgb,1:red,0.0;green,1.0;blue,0.498}, name path={39a7f4b5-27cf-4c6f-86dc-50813783b1d9}, draw opacity={0.8}, line width={1}, solid, forget plot]
        table[row sep={\\}]
        {
            \\
            1.4094003408681601  1.8230532801244403  \\
            1.8488018650095905  1.693958807038143  \\
        }
        ;
    \addplot[color={rgb,1:red,0.0;green,1.0;blue,0.498}, name path={39a7f4b5-27cf-4c6f-86dc-50813783b1d9}, draw opacity={0.8}, line width={1}, solid, forget plot]
        table[row sep={\\}]
        {
            \\
            1.8946103360503908  1.4291503357235342  \\
            1.8701207209378476  1.1696756919845939  \\
        }
        ;
    \addplot[color={rgb,1:red,0.0;green,1.0;blue,0.498}, name path={39a7f4b5-27cf-4c6f-86dc-50813783b1d9}, draw opacity={0.8}, line width={1}, solid, forget plot]
        table[row sep={\\}]
        {
            \\
            1.4094003408681601  1.8230532801244403  \\
            1.3702157551888376  1.9199406439438458  \\
        }
        ;
    \addplot[color={rgb,1:red,0.0;green,1.0;blue,0.498}, name path={39a7f4b5-27cf-4c6f-86dc-50813783b1d9}, draw opacity={0.8}, line width={1}, solid, forget plot]
        table[row sep={\\}]
        {
            \\
            1.411079611396417  1.603287132689729  \\
            1.3230645160656422  1.6671039194334296  \\
        }
        ;
    \addplot[color={rgb,1:red,0.0;green,1.0;blue,0.498}, name path={39a7f4b5-27cf-4c6f-86dc-50813783b1d9}, draw opacity={0.8}, line width={1}, solid, forget plot]
        table[row sep={\\}]
        {
            \\
            1.4235251017380506  1.5806863505858928  \\
            1.411079611396417  1.603287132689729  \\
        }
        ;
    \addplot[color={rgb,1:red,0.0;green,1.0;blue,0.498}, name path={39a7f4b5-27cf-4c6f-86dc-50813783b1d9}, draw opacity={0.8}, line width={1}, solid, forget plot]
        table[row sep={\\}]
        {
            \\
            1.1105229298118504  1.3180863440502435  \\
            1.3071074625477195  1.3007149568293244  \\
        }
        ;
    \addplot[color={rgb,1:red,0.0;green,1.0;blue,0.498}, name path={39a7f4b5-27cf-4c6f-86dc-50813783b1d9}, draw opacity={0.8}, line width={1}, solid, forget plot]
        table[row sep={\\}]
        {
            \\
            1.1105229298118504  1.3180863440502435  \\
            1.1387269033584746  1.3902763912919909  \\
        }
        ;
    \addplot[color={rgb,1:red,0.0;green,1.0;blue,0.498}, name path={39a7f4b5-27cf-4c6f-86dc-50813783b1d9}, draw opacity={0.8}, line width={1}, solid, forget plot]
        table[row sep={\\}]
        {
            \\
            1.1105229298118504  1.3180863440502435  \\
            1.125095868483186  1.7752906912937756  \\
        }
        ;
    \addplot[color={rgb,1:red,0.0;green,1.0;blue,0.498}, name path={39a7f4b5-27cf-4c6f-86dc-50813783b1d9}, draw opacity={0.8}, line width={1}, solid, forget plot]
        table[row sep={\\}]
        {
            \\
            1.7508308666292574  1.5433638719841838  \\
            1.8273643609136339  1.502810534555465  \\
        }
        ;
\end{axis}
\begin{axis}[point meta max={nan}, point meta min={nan}, title={Voronoi Diagram}, title style={at={{(0.5,1)}}, anchor={south}, font={{\fontsize{10 pt}{13.0 pt}\selectfont}}, color={rgb,1:red,0.0;green,0.0;blue,0.0}, draw opacity={1.0}, rotate={0.0}}, legend style={color={rgb,1:red,0.0;green,0.0;blue,0.0}, draw opacity={1.0}, line width={1}, solid, fill={rgb,1:red,1.0;green,1.0;blue,1.0}, fill opacity={1.0}, text opacity={1.0}, font={{\fontsize{8 pt}{10.4 pt}\selectfont}}, text={rgb,1:red,0.0;green,0.0;blue,0.0}, cells={anchor={west}}, at={(0.98, 0.98)}, anchor={north east}}, axis background/.style={fill={rgb,1:red,1.0;green,1.0;blue,1.0}, opacity={1.0}}, anchor={north west}, xshift={89.9mm}, yshift={-1.0mm}, width={86.9mm}, height={86.9mm}, scaled x ticks={false}, xlabel={$x$}, x tick style={color={rgb,1:red,0.0;green,0.0;blue,0.0}, opacity={1.0}}, x tick label style={color={rgb,1:red,0.0;green,0.0;blue,0.0}, opacity={1.0}, rotate={0}}, xlabel style={at={(ticklabel cs:0.5)}, anchor=near ticklabel, font={{\fontsize{11 pt}{14.3 pt}\selectfont}}, color={rgb,1:red,0.0;green,0.0;blue,0.0}, draw opacity={1.0}, rotate={0.0}}, xmajorgrids={true}, xmin={0.9}, xmax={2.1}, xtick={{1.0,1.25,1.5,1.75,2.0}}, xticklabels={{$1.00$,$1.25$,$1.50$,$1.75$,$2.00$}}, xtick align={inside}, xticklabel style={font={{\fontsize{8 pt}{10.4 pt}\selectfont}}, color={rgb,1:red,0.0;green,0.0;blue,0.0}, draw opacity={1.0}, rotate={0.0}}, x grid style={color={rgb,1:red,0.0;green,0.0;blue,0.0}, draw opacity={0.1}, line width={0.5}, solid}, axis x line*={left}, x axis line style={color={rgb,1:red,0.0;green,0.0;blue,0.0}, draw opacity={1.0}, line width={1}, solid}, scaled y ticks={false}, ylabel={$y$}, y tick style={color={rgb,1:red,0.0;green,0.0;blue,0.0}, opacity={1.0}}, y tick label style={color={rgb,1:red,0.0;green,0.0;blue,0.0}, opacity={1.0}, rotate={0}}, ylabel style={at={(ticklabel cs:0.5)}, anchor=near ticklabel, font={{\fontsize{11 pt}{14.3 pt}\selectfont}}, color={rgb,1:red,0.0;green,0.0;blue,0.0}, draw opacity={1.0}, rotate={0.0}}, ymajorgrids={true}, ymin={0.9}, ymax={2.1}, ytick={{1.0,1.25,1.5,1.75,2.0}}, yticklabels={{$1.00$,$1.25$,$1.50$,$1.75$,$2.00$}}, ytick align={inside}, yticklabel style={font={{\fontsize{8 pt}{10.4 pt}\selectfont}}, color={rgb,1:red,0.0;green,0.0;blue,0.0}, draw opacity={1.0}, rotate={0.0}}, y grid style={color={rgb,1:red,0.0;green,0.0;blue,0.0}, draw opacity={0.1}, line width={0.5}, solid}, axis y line*={left}, y axis line style={color={rgb,1:red,0.0;green,0.0;blue,0.0}, draw opacity={1.0}, line width={1}, solid}, colorbar={false}, legend columns={-1}]
    \addplot[color={rgb,1:red,1.0;green,0.0;blue,0.0}, name path={6628918f-3636-4ecd-a93d-61f6bf9c0272}, draw opacity={1.0}, line width={1}, solid]
        table[row sep={\\}]
        {
            \\
            1.2604794002114923  1.803403112861966  \\
            1.27570865405724  1.8379202298383621  \\
        }
        ;
    \addlegendentry {CGAL.jl}
    \addplot[color={rgb,1:red,1.0;green,0.0;blue,0.0}, name path={6628918f-3636-4ecd-a93d-61f6bf9c0272}, draw opacity={1.0}, line width={1}, solid, forget plot]
        table[row sep={\\}]
        {
            \\
            1.3653640860156322  1.8616109822110092  \\
            1.27570865405724  1.8379202298383621  \\
        }
        ;
    \addplot[color={rgb,1:red,1.0;green,0.0;blue,0.0}, name path={6628918f-3636-4ecd-a93d-61f6bf9c0272}, draw opacity={1.0}, line width={1}, solid, forget plot]
        table[row sep={\\}]
        {
            \\
            1.4207721973802245  1.7092578946573438  \\
            1.496462604009042  1.7138290481784513  \\
        }
        ;
    \addplot[color={rgb,1:red,1.0;green,0.0;blue,0.0}, name path={6628918f-3636-4ecd-a93d-61f6bf9c0272}, draw opacity={1.0}, line width={1}, solid, forget plot]
        table[row sep={\\}]
        {
            \\
            1.5967479273801457  1.7145953453990177  \\
            1.496462604009042  1.7138290481784513  \\
        }
        ;
    \addplot[color={rgb,1:red,1.0;green,0.0;blue,0.0}, name path={6628918f-3636-4ecd-a93d-61f6bf9c0272}, draw opacity={1.0}, line width={1}, solid, forget plot]
        table[row sep={\\}]
        {
            \\
            1.3653640860156322  1.8616109822110092  \\
            1.496462604009042  1.7138290481784513  \\
        }
        ;
    \addplot[color={rgb,1:red,1.0;green,0.0;blue,0.0}, name path={6628918f-3636-4ecd-a93d-61f6bf9c0272}, draw opacity={1.0}, line width={1}, solid, forget plot]
        table[row sep={\\}]
        {
            \\
            1.8295057313643917  1.599346634254463  \\
            1.9876471535093847  1.5816108873343284  \\
        }
        ;
    \addplot[color={rgb,1:red,1.0;green,0.0;blue,0.0}, name path={6628918f-3636-4ecd-a93d-61f6bf9c0272}, draw opacity={1.0}, line width={1}, solid, forget plot]
        table[row sep={\\}]
        {
            \\
            1.7835778408291005  1.3953116355771533  \\
            1.9876471535093847  1.5816108873343284  \\
        }
        ;
    \addplot[color={rgb,1:red,1.0;green,0.0;blue,0.0}, name path={6628918f-3636-4ecd-a93d-61f6bf9c0272}, draw opacity={1.0}, line width={1}, solid, forget plot]
        table[row sep={\\}]
        {
            \\
            1.7835778408291005  1.3953116355771533  \\
            1.7527984799071734  1.4545823716018462  \\
        }
        ;
    \addplot[color={rgb,1:red,1.0;green,0.0;blue,0.0}, name path={6628918f-3636-4ecd-a93d-61f6bf9c0272}, draw opacity={1.0}, line width={1}, solid, forget plot]
        table[row sep={\\}]
        {
            \\
            1.8295057313643917  1.599346634254463  \\
            1.7527984799071734  1.4545823716018462  \\
        }
        ;
    \addplot[color={rgb,1:red,1.0;green,0.0;blue,0.0}, name path={6628918f-3636-4ecd-a93d-61f6bf9c0272}, draw opacity={1.0}, line width={1}, solid, forget plot]
        table[row sep={\\}]
        {
            \\
            1.6348828068615442  1.5322431195203052  \\
            1.7527984799071734  1.4545823716018462  \\
        }
        ;
    \addplot[color={rgb,1:red,1.0;green,0.0;blue,0.0}, name path={6628918f-3636-4ecd-a93d-61f6bf9c0272}, draw opacity={1.0}, line width={1}, solid, forget plot]
        table[row sep={\\}]
        {
            \\
            1.6220013889866987  1.7343406005768556  \\
            1.6990408328985667  1.996561346861277  \\
        }
        ;
    \addplot[color={rgb,1:red,1.0;green,0.0;blue,0.0}, name path={6628918f-3636-4ecd-a93d-61f6bf9c0272}, draw opacity={1.0}, line width={1}, solid, forget plot]
        table[row sep={\\}]
        {
            \\
            1.3653640860156322  1.8616109822110092  \\
            1.6990408328985667  1.996561346861277  \\
        }
        ;
    \addplot[color={rgb,1:red,1.0;green,0.0;blue,0.0}, name path={6628918f-3636-4ecd-a93d-61f6bf9c0272}, draw opacity={1.0}, line width={1}, solid, forget plot]
        table[row sep={\\}]
        {
            \\
            1.8295057313643917  1.599346634254463  \\
            1.6220013889866987  1.7343406005768556  \\
        }
        ;
    \addplot[color={rgb,1:red,1.0;green,0.0;blue,0.0}, name path={6628918f-3636-4ecd-a93d-61f6bf9c0272}, draw opacity={1.0}, line width={1}, solid, forget plot]
        table[row sep={\\}]
        {
            \\
            1.6095671713405044  1.7191615435771748  \\
            1.6220013889866987  1.7343406005768556  \\
        }
        ;
    \addplot[color={rgb,1:red,1.0;green,0.0;blue,0.0}, name path={6628918f-3636-4ecd-a93d-61f6bf9c0272}, draw opacity={1.0}, line width={1}, solid, forget plot]
        table[row sep={\\}]
        {
            \\
            1.7605048201946907  0.9405988741695328  \\
            1.7522049938096065  1.2811541224295153  \\
        }
        ;
    \addplot[color={rgb,1:red,1.0;green,0.0;blue,0.0}, name path={6628918f-3636-4ecd-a93d-61f6bf9c0272}, draw opacity={1.0}, line width={1}, solid, forget plot]
        table[row sep={\\}]
        {
            \\
            1.786608982970989  1.3084506635857562  \\
            1.7522049938096065  1.2811541224295153  \\
        }
        ;
    \addplot[color={rgb,1:red,1.0;green,0.0;blue,0.0}, name path={6628918f-3636-4ecd-a93d-61f6bf9c0272}, draw opacity={1.0}, line width={1}, solid, forget plot]
        table[row sep={\\}]
        {
            \\
            1.5412601702997408  1.306242080125303  \\
            1.7522049938096065  1.2811541224295153  \\
        }
        ;
    \addplot[color={rgb,1:red,1.0;green,0.0;blue,0.0}, name path={6628918f-3636-4ecd-a93d-61f6bf9c0272}, draw opacity={1.0}, line width={1}, solid, forget plot]
        table[row sep={\\}]
        {
            \\
            1.472714966518481  1.25429855722196  \\
            1.156326549607483  0.7154091672536667  \\
        }
        ;
    \addplot[color={rgb,1:red,1.0;green,0.0;blue,0.0}, name path={6628918f-3636-4ecd-a93d-61f6bf9c0272}, draw opacity={1.0}, line width={1}, solid, forget plot]
        table[row sep={\\}]
        {
            \\
            1.2098306655723479  1.3208922816783164  \\
            1.156326549607483  0.7154091672536667  \\
        }
        ;
    \addplot[color={rgb,1:red,1.0;green,0.0;blue,0.0}, name path={6628918f-3636-4ecd-a93d-61f6bf9c0272}, draw opacity={1.0}, line width={1}, solid, forget plot]
        table[row sep={\\}]
        {
            \\
            1.1487653185833961  1.5833802374257493  \\
            1.2628404793646477  1.5074183453813343  \\
        }
        ;
    \addplot[color={rgb,1:red,1.0;green,0.0;blue,0.0}, name path={6628918f-3636-4ecd-a93d-61f6bf9c0272}, draw opacity={1.0}, line width={1}, solid, forget plot]
        table[row sep={\\}]
        {
            \\
            1.2690346489872688  1.4321991169634372  \\
            1.2628404793646477  1.5074183453813343  \\
        }
        ;
    \addplot[color={rgb,1:red,1.0;green,0.0;blue,0.0}, name path={6628918f-3636-4ecd-a93d-61f6bf9c0272}, draw opacity={1.0}, line width={1}, solid, forget plot]
        table[row sep={\\}]
        {
            \\
            1.276983303491567  1.5119328295486254  \\
            1.2628404793646477  1.5074183453813343  \\
        }
        ;
    \addplot[color={rgb,1:red,1.0;green,0.0;blue,0.0}, name path={6628918f-3636-4ecd-a93d-61f6bf9c0272}, draw opacity={1.0}, line width={1}, solid, forget plot]
        table[row sep={\\}]
        {
            \\
            1.6095671713405044  1.7191615435771748  \\
            1.5967479273801457  1.7145953453990177  \\
        }
        ;
    \addplot[color={rgb,1:red,1.0;green,0.0;blue,0.0}, name path={6628918f-3636-4ecd-a93d-61f6bf9c0272}, draw opacity={1.0}, line width={1}, solid, forget plot]
        table[row sep={\\}]
        {
            \\
            1.5599023713656444  1.67051177226448  \\
            1.5967479273801457  1.7145953453990177  \\
        }
        ;
    \addplot[color={rgb,1:red,1.0;green,0.0;blue,0.0}, name path={6628918f-3636-4ecd-a93d-61f6bf9c0272}, draw opacity={1.0}, line width={1}, solid, forget plot]
        table[row sep={\\}]
        {
            \\
            1.5209194746064107  1.2869409180911984  \\
            1.472714966518481  1.25429855722196  \\
        }
        ;
    \addplot[color={rgb,1:red,1.0;green,0.0;blue,0.0}, name path={6628918f-3636-4ecd-a93d-61f6bf9c0272}, draw opacity={1.0}, line width={1}, solid, forget plot]
        table[row sep={\\}]
        {
            \\
            1.2690346489872688  1.4321991169634372  \\
            1.472714966518481  1.25429855722196  \\
        }
        ;
    \addplot[color={rgb,1:red,1.0;green,0.0;blue,0.0}, name path={6628918f-3636-4ecd-a93d-61f6bf9c0272}, draw opacity={1.0}, line width={1}, solid, forget plot]
        table[row sep={\\}]
        {
            \\
            1.2098306655723479  1.3208922816783164  \\
            1.2690346489872688  1.4321991169634372  \\
        }
        ;
    \addplot[color={rgb,1:red,1.0;green,0.0;blue,0.0}, name path={6628918f-3636-4ecd-a93d-61f6bf9c0272}, draw opacity={1.0}, line width={1}, solid, forget plot]
        table[row sep={\\}]
        {
            \\
            1.1487653185833961  1.5833802374257493  \\
            0.5887246387327791  1.563552573781597  \\
        }
        ;
    \addplot[color={rgb,1:red,1.0;green,0.0;blue,0.0}, name path={6628918f-3636-4ecd-a93d-61f6bf9c0272}, draw opacity={1.0}, line width={1}, solid, forget plot]
        table[row sep={\\}]
        {
            \\
            1.2098306655723479  1.3208922816783164  \\
            0.5887246387327791  1.563552573781597  \\
        }
        ;
    \addplot[color={rgb,1:red,1.0;green,0.0;blue,0.0}, name path={6628918f-3636-4ecd-a93d-61f6bf9c0272}, draw opacity={1.0}, line width={1}, solid, forget plot]
        table[row sep={\\}]
        {
            \\
            1.2622085164571029  1.7909675059457593  \\
            1.1487653185833961  1.5833802374257493  \\
        }
        ;
    \addplot[color={rgb,1:red,1.0;green,0.0;blue,0.0}, name path={6628918f-3636-4ecd-a93d-61f6bf9c0272}, draw opacity={1.0}, line width={1}, solid, forget plot]
        table[row sep={\\}]
        {
            \\
            1.2604794002114923  1.803403112861966  \\
            1.2622085164571029  1.7909675059457593  \\
        }
        ;
    \addplot[color={rgb,1:red,1.0;green,0.0;blue,0.0}, name path={6628918f-3636-4ecd-a93d-61f6bf9c0272}, draw opacity={1.0}, line width={1}, solid, forget plot]
        table[row sep={\\}]
        {
            \\
            1.4207721973802245  1.7092578946573438  \\
            1.2622085164571029  1.7909675059457593  \\
        }
        ;
    \addplot[color={rgb,1:red,1.0;green,0.0;blue,0.0}, name path={6628918f-3636-4ecd-a93d-61f6bf9c0272}, draw opacity={1.0}, line width={1}, solid, forget plot]
        table[row sep={\\}]
        {
            \\
            1.434708982548336  1.4940758047564862  \\
            1.5599023713656444  1.67051177226448  \\
        }
        ;
    \addplot[color={rgb,1:red,1.0;green,0.0;blue,0.0}, name path={6628918f-3636-4ecd-a93d-61f6bf9c0272}, draw opacity={1.0}, line width={1}, solid, forget plot]
        table[row sep={\\}]
        {
            \\
            1.2815350666192726  1.5172242627114598  \\
            1.5599023713656444  1.67051177226448  \\
        }
        ;
    \addplot[color={rgb,1:red,1.0;green,0.0;blue,0.0}, name path={6628918f-3636-4ecd-a93d-61f6bf9c0272}, draw opacity={1.0}, line width={1}, solid, forget plot]
        table[row sep={\\}]
        {
            \\
            1.2815350666192726  1.5172242627114598  \\
            1.276983303491567  1.5119328295486254  \\
        }
        ;
    \addplot[color={rgb,1:red,1.0;green,0.0;blue,0.0}, name path={6628918f-3636-4ecd-a93d-61f6bf9c0272}, draw opacity={1.0}, line width={1}, solid, forget plot]
        table[row sep={\\}]
        {
            \\
            1.434708982548336  1.4940758047564862  \\
            1.276983303491567  1.5119328295486254  \\
        }
        ;
    \addplot[color={rgb,1:red,1.0;green,0.0;blue,0.0}, name path={6628918f-3636-4ecd-a93d-61f6bf9c0272}, draw opacity={1.0}, line width={1}, solid, forget plot]
        table[row sep={\\}]
        {
            \\
            1.6348828068615442  1.5322431195203052  \\
            1.6095671713405044  1.7191615435771748  \\
        }
        ;
    \addplot[color={rgb,1:red,1.0;green,0.0;blue,0.0}, name path={6628918f-3636-4ecd-a93d-61f6bf9c0272}, draw opacity={1.0}, line width={1}, solid, forget plot]
        table[row sep={\\}]
        {
            \\
            1.5389442089407321  1.3734229705441723  \\
            1.434708982548336  1.4940758047564862  \\
        }
        ;
    \addplot[color={rgb,1:red,1.0;green,0.0;blue,0.0}, name path={6628918f-3636-4ecd-a93d-61f6bf9c0272}, draw opacity={1.0}, line width={1}, solid, forget plot]
        table[row sep={\\}]
        {
            \\
            1.5389442089407321  1.3734229705441723  \\
            1.5412601702997408  1.306242080125303  \\
        }
        ;
    \addplot[color={rgb,1:red,1.0;green,0.0;blue,0.0}, name path={6628918f-3636-4ecd-a93d-61f6bf9c0272}, draw opacity={1.0}, line width={1}, solid, forget plot]
        table[row sep={\\}]
        {
            \\
            1.5209194746064107  1.2869409180911984  \\
            1.5412601702997408  1.306242080125303  \\
        }
        ;
    \addplot[color={rgb,1:red,1.0;green,0.0;blue,0.0}, name path={6628918f-3636-4ecd-a93d-61f6bf9c0272}, draw opacity={1.0}, line width={1}, solid, forget plot]
        table[row sep={\\}]
        {
            \\
            1.2815350666192726  1.5172242627114598  \\
            1.4207721973802245  1.7092578946573438  \\
        }
        ;
    \addplot[color={rgb,1:red,1.0;green,0.0;blue,0.0}, name path={6628918f-3636-4ecd-a93d-61f6bf9c0272}, draw opacity={1.0}, line width={1}, solid, forget plot]
        table[row sep={\\}]
        {
            \\
            1.786608982970989  1.3084506635857562  \\
            1.7835778408291005  1.3953116355771533  \\
        }
        ;
    \addplot[color={rgb,1:red,1.0;green,0.0;blue,0.0}, name path={6628918f-3636-4ecd-a93d-61f6bf9c0272}, draw opacity={1.0}, line width={1}, solid, forget plot]
        table[row sep={\\}]
        {
            \\
            1.5389442089407321  1.3734229705441723  \\
            1.6348828068615442  1.5322431195203052  \\
        }
        ;
    \addplot[color={rgb,1:red,1.0;green,0.0;blue,0.0}, name path={6628918f-3636-4ecd-a93d-61f6bf9c0272}, draw opacity={1.0}, line width={1}, solid, forget plot]
        table[row sep={\\}]
        {
            \\
            1.5209194746064107  1.2869409180911984  \\
            1.7605048201946907  0.9405988741695328  \\
        }
        ;
    \addplot[color={rgb,1:red,0.0;green,1.0;blue,0.498}, name path={285cc138-eec4-4d90-a209-683155680019}, draw opacity={0.8}, line width={1}, solid]
        table[row sep={\\}]
        {
            \\
            2.3386377425544103  1.6423277600009902  \\
            1.6892290828527665  1.9631648905951962  \\
        }
        ;
    \addlegendentry {VoronoiDelaunay.jl}
    \addplot[color={rgb,1:red,0.0;green,1.0;blue,0.498}, name path={285cc138-eec4-4d90-a209-683155680019}, draw opacity={0.8}, line width={1}, solid, forget plot]
        table[row sep={\\}]
        {
            \\
            2.3386377425544103  1.6423277600009902  \\
            3.109564899629648  1.4999999999999998  \\
        }
        ;
    \addplot[color={rgb,1:red,0.0;green,1.0;blue,0.498}, name path={285cc138-eec4-4d90-a209-683155680019}, draw opacity={0.8}, line width={1}, solid, forget plot]
        table[row sep={\\}]
        {
            \\
            2.3386377425544103  1.6423277600009902  \\
            1.9876471535093847  1.5816108873343284  \\
        }
        ;
    \addplot[color={rgb,1:red,0.0;green,1.0;blue,0.498}, name path={285cc138-eec4-4d90-a209-683155680019}, draw opacity={0.8}, line width={1}, solid, forget plot]
        table[row sep={\\}]
        {
            \\
            1.496462604009042  1.7138290481784513  \\
            1.4207721973802245  1.7092578946573438  \\
        }
        ;
    \addplot[color={rgb,1:red,0.0;green,1.0;blue,0.498}, name path={285cc138-eec4-4d90-a209-683155680019}, draw opacity={0.8}, line width={1}, solid, forget plot]
        table[row sep={\\}]
        {
            \\
            1.496462604009042  1.7138290481784513  \\
            1.3653640860156322  1.8616109822110092  \\
        }
        ;
    \addplot[color={rgb,1:red,0.0;green,1.0;blue,0.498}, name path={285cc138-eec4-4d90-a209-683155680019}, draw opacity={0.8}, line width={1}, solid, forget plot]
        table[row sep={\\}]
        {
            \\
            1.496462604009042  1.7138290481784513  \\
            1.5967479273801457  1.7145953453990177  \\
        }
        ;
    \addplot[color={rgb,1:red,0.0;green,1.0;blue,0.498}, name path={285cc138-eec4-4d90-a209-683155680019}, draw opacity={0.8}, line width={1}, solid, forget plot]
        table[row sep={\\}]
        {
            \\
            1.7574867860475272  0.9449617124415791  \\
            1.7602506305045384  0.9510286866762687  \\
        }
        ;
    \addplot[color={rgb,1:red,0.0;green,1.0;blue,0.498}, name path={285cc138-eec4-4d90-a209-683155680019}, draw opacity={0.8}, line width={1}, solid, forget plot]
        table[row sep={\\}]
        {
            \\
            1.7574867860475272  0.9449617124415791  \\
            1.5  0.1547296285271399  \\
        }
        ;
    \addplot[color={rgb,1:red,0.0;green,1.0;blue,0.498}, name path={285cc138-eec4-4d90-a209-683155680019}, draw opacity={0.8}, line width={1}, solid, forget plot]
        table[row sep={\\}]
        {
            \\
            1.7574867860475272  0.9449617124415791  \\
            1.5209194746064107  1.2869409180911981  \\
        }
        ;
    \addplot[color={rgb,1:red,0.0;green,1.0;blue,0.498}, name path={285cc138-eec4-4d90-a209-683155680019}, draw opacity={0.8}, line width={1}, solid, forget plot]
        table[row sep={\\}]
        {
            \\
            1.2690346489872688  1.4321991169634372  \\
            1.2628404793646477  1.5074183453813343  \\
        }
        ;
    \addplot[color={rgb,1:red,0.0;green,1.0;blue,0.498}, name path={285cc138-eec4-4d90-a209-683155680019}, draw opacity={0.8}, line width={1}, solid, forget plot]
        table[row sep={\\}]
        {
            \\
            1.2690346489872688  1.4321991169634372  \\
            1.472714966518481  1.25429855722196  \\
        }
        ;
    \addplot[color={rgb,1:red,0.0;green,1.0;blue,0.498}, name path={285cc138-eec4-4d90-a209-683155680019}, draw opacity={0.8}, line width={1}, solid, forget plot]
        table[row sep={\\}]
        {
            \\
            1.2690346489872688  1.4321991169634372  \\
            1.2098306655723479  1.3208922816783164  \\
        }
        ;
    \addplot[color={rgb,1:red,0.0;green,1.0;blue,0.498}, name path={285cc138-eec4-4d90-a209-683155680019}, draw opacity={0.8}, line width={1}, solid, forget plot]
        table[row sep={\\}]
        {
            \\
            2.1788293709650297  1.2714323013949198  \\
            3.109564899629648  1.4999999999999998  \\
        }
        ;
    \addplot[color={rgb,1:red,0.0;green,1.0;blue,0.498}, name path={285cc138-eec4-4d90-a209-683155680019}, draw opacity={0.8}, line width={1}, solid, forget plot]
        table[row sep={\\}]
        {
            \\
            2.1788293709650297  1.2714323013949198  \\
            1.7602506305045384  0.9510286866762687  \\
        }
        ;
    \addplot[color={rgb,1:red,0.0;green,1.0;blue,0.498}, name path={285cc138-eec4-4d90-a209-683155680019}, draw opacity={0.8}, line width={1}, solid, forget plot]
        table[row sep={\\}]
        {
            \\
            2.1788293709650297  1.2714323013949198  \\
            1.786608982970989  1.3084506635857562  \\
        }
        ;
    \addplot[color={rgb,1:red,0.0;green,1.0;blue,0.498}, name path={285cc138-eec4-4d90-a209-683155680019}, draw opacity={0.8}, line width={1}, solid, forget plot]
        table[row sep={\\}]
        {
            \\
            1.5967479273801457  1.7145953453990177  \\
            1.6095671713405044  1.7191615435771748  \\
        }
        ;
    \addplot[color={rgb,1:red,0.0;green,1.0;blue,0.498}, name path={285cc138-eec4-4d90-a209-683155680019}, draw opacity={0.8}, line width={1}, solid, forget plot]
        table[row sep={\\}]
        {
            \\
            1.5967479273801457  1.7145953453990177  \\
            1.5599023713656444  1.67051177226448  \\
        }
        ;
    \addplot[color={rgb,1:red,0.0;green,1.0;blue,0.498}, name path={285cc138-eec4-4d90-a209-683155680019}, draw opacity={0.8}, line width={1}, solid, forget plot]
        table[row sep={\\}]
        {
            \\
            1.3149299332578315  0.9855507909474527  \\
            1.472714966518481  1.25429855722196  \\
        }
        ;
    \addplot[color={rgb,1:red,0.0;green,1.0;blue,0.498}, name path={285cc138-eec4-4d90-a209-683155680019}, draw opacity={0.8}, line width={1}, solid, forget plot]
        table[row sep={\\}]
        {
            \\
            1.3149299332578315  0.9855507909474527  \\
            1.5  0.1547296285271399  \\
        }
        ;
    \addplot[color={rgb,1:red,0.0;green,1.0;blue,0.498}, name path={285cc138-eec4-4d90-a209-683155680019}, draw opacity={0.8}, line width={1}, solid, forget plot]
        table[row sep={\\}]
        {
            \\
            1.3149299332578315  0.9855507909474527  \\
            1.1913502614960443  1.1117574812212023  \\
        }
        ;
    \addplot[color={rgb,1:red,0.0;green,1.0;blue,0.498}, name path={285cc138-eec4-4d90-a209-683155680019}, draw opacity={0.8}, line width={1}, solid, forget plot]
        table[row sep={\\}]
        {
            \\
            1.5412601702997408  1.306242080125303  \\
            1.7522049938096065  1.2811541224295153  \\
        }
        ;
    \addplot[color={rgb,1:red,0.0;green,1.0;blue,0.498}, name path={285cc138-eec4-4d90-a209-683155680019}, draw opacity={0.8}, line width={1}, solid, forget plot]
        table[row sep={\\}]
        {
            \\
            1.5412601702997408  1.306242080125303  \\
            1.5209194746064107  1.2869409180911981  \\
        }
        ;
    \addplot[color={rgb,1:red,0.0;green,1.0;blue,0.498}, name path={285cc138-eec4-4d90-a209-683155680019}, draw opacity={0.8}, line width={1}, solid, forget plot]
        table[row sep={\\}]
        {
            \\
            1.5412601702997408  1.306242080125303  \\
            1.5389442089407321  1.3734229705441723  \\
        }
        ;
    \addplot[color={rgb,1:red,0.0;green,1.0;blue,0.498}, name path={285cc138-eec4-4d90-a209-683155680019}, draw opacity={0.8}, line width={1}, solid, forget plot]
        table[row sep={\\}]
        {
            \\
            1.5389442089407321  1.3734229705441723  \\
            1.6348828068615442  1.5322431195203052  \\
        }
        ;
    \addplot[color={rgb,1:red,0.0;green,1.0;blue,0.498}, name path={285cc138-eec4-4d90-a209-683155680019}, draw opacity={0.8}, line width={1}, solid, forget plot]
        table[row sep={\\}]
        {
            \\
            1.5389442089407321  1.3734229705441723  \\
            1.434708982548336  1.4940758047564862  \\
        }
        ;
    \addplot[color={rgb,1:red,0.0;green,1.0;blue,0.498}, name path={285cc138-eec4-4d90-a209-683155680019}, draw opacity={0.8}, line width={1}, solid, forget plot]
        table[row sep={\\}]
        {
            \\
            1.6813651102443967  1.9894126760596103  \\
            1.6892290828527665  1.9631648905951962  \\
        }
        ;
    \addplot[color={rgb,1:red,0.0;green,1.0;blue,0.498}, name path={285cc138-eec4-4d90-a209-683155680019}, draw opacity={0.8}, line width={1}, solid, forget plot]
        table[row sep={\\}]
        {
            \\
            1.6813651102443967  1.9894126760596103  \\
            1.3653640860156322  1.8616109822110092  \\
        }
        ;
    \addplot[color={rgb,1:red,0.0;green,1.0;blue,0.498}, name path={285cc138-eec4-4d90-a209-683155680019}, draw opacity={0.8}, line width={1}, solid, forget plot]
        table[row sep={\\}]
        {
            \\
            1.6813651102443967  1.9894126760596103  \\
            1.4999999999999998  3.4161152454663077  \\
        }
        ;
    \addplot[color={rgb,1:red,0.0;green,1.0;blue,0.498}, name path={285cc138-eec4-4d90-a209-683155680019}, draw opacity={0.8}, line width={1}, solid, forget plot]
        table[row sep={\\}]
        {
            \\
            1.6892290828527665  1.9631648905951962  \\
            1.6220013889866987  1.7343406005768556  \\
        }
        ;
    \addplot[color={rgb,1:red,0.0;green,1.0;blue,0.498}, name path={285cc138-eec4-4d90-a209-683155680019}, draw opacity={0.8}, line width={1}, solid, forget plot]
        table[row sep={\\}]
        {
            \\
            1.1487653185833961  1.5833802374257493  \\
            1.2622085164571029  1.7909675059457593  \\
        }
        ;
    \addplot[color={rgb,1:red,0.0;green,1.0;blue,0.498}, name path={285cc138-eec4-4d90-a209-683155680019}, draw opacity={0.8}, line width={1}, solid, forget plot]
        table[row sep={\\}]
        {
            \\
            1.1487653185833961  1.5833802374257493  \\
            1.2628404793646477  1.5074183453813343  \\
        }
        ;
    \addplot[color={rgb,1:red,0.0;green,1.0;blue,0.498}, name path={285cc138-eec4-4d90-a209-683155680019}, draw opacity={0.8}, line width={1}, solid, forget plot]
        table[row sep={\\}]
        {
            \\
            1.1487653185833961  1.5833802374257493  \\
            0.5887246387327791  1.563552573781597  \\
        }
        ;
    \addplot[color={rgb,1:red,0.0;green,1.0;blue,0.498}, name path={285cc138-eec4-4d90-a209-683155680019}, draw opacity={0.8}, line width={1}, solid, forget plot]
        table[row sep={\\}]
        {
            \\
            1.27570865405724  1.8379202298383621  \\
            1.3653640860156322  1.8616109822110092  \\
        }
        ;
    \addplot[color={rgb,1:red,0.0;green,1.0;blue,0.498}, name path={285cc138-eec4-4d90-a209-683155680019}, draw opacity={0.8}, line width={1}, solid, forget plot]
        table[row sep={\\}]
        {
            \\
            1.27570865405724  1.8379202298383621  \\
            1.2604794002114923  1.803403112861966  \\
        }
        ;
    \addplot[color={rgb,1:red,0.0;green,1.0;blue,0.498}, name path={285cc138-eec4-4d90-a209-683155680019}, draw opacity={0.8}, line width={1}, solid, forget plot]
        table[row sep={\\}]
        {
            \\
            1.27570865405724  1.8379202298383621  \\
            1.2570826968156772  2.2928010286527063  \\
        }
        ;
    \addplot[color={rgb,1:red,0.0;green,1.0;blue,0.498}, name path={285cc138-eec4-4d90-a209-683155680019}, draw opacity={0.8}, line width={1}, solid, forget plot]
        table[row sep={\\}]
        {
            \\
            0.07398354299303522  1.5  \\
            0.4862484197523599  1.566818902423211  \\
        }
        ;
    \addplot[color={rgb,1:red,0.0;green,1.0;blue,0.498}, name path={285cc138-eec4-4d90-a209-683155680019}, draw opacity={0.8}, line width={1}, solid, forget plot]
        table[row sep={\\}]
        {
            \\
            0.07398354299303522  1.5  \\
            1.1913502614960443  1.1117574812212023  \\
        }
        ;
    \addplot[color={rgb,1:red,0.0;green,1.0;blue,0.498}, name path={285cc138-eec4-4d90-a209-683155680019}, draw opacity={0.8}, line width={1}, solid, forget plot]
        table[row sep={\\}]
        {
            \\
            1.2622085164571029  1.7909675059457593  \\
            1.2604794002114923  1.803403112861966  \\
        }
        ;
    \addplot[color={rgb,1:red,0.0;green,1.0;blue,0.498}, name path={285cc138-eec4-4d90-a209-683155680019}, draw opacity={0.8}, line width={1}, solid, forget plot]
        table[row sep={\\}]
        {
            \\
            1.2622085164571029  1.7909675059457593  \\
            1.4207721973802245  1.7092578946573438  \\
        }
        ;
    \addplot[color={rgb,1:red,0.0;green,1.0;blue,0.498}, name path={285cc138-eec4-4d90-a209-683155680019}, draw opacity={0.8}, line width={1}, solid, forget plot]
        table[row sep={\\}]
        {
            \\
            1.2628404793646477  1.5074183453813343  \\
            1.276983303491567  1.5119328295486254  \\
        }
        ;
    \addplot[color={rgb,1:red,0.0;green,1.0;blue,0.498}, name path={285cc138-eec4-4d90-a209-683155680019}, draw opacity={0.8}, line width={1}, solid, forget plot]
        table[row sep={\\}]
        {
            \\
            1.6095671713405044  1.7191615435771748  \\
            1.6220013889866987  1.7343406005768556  \\
        }
        ;
    \addplot[color={rgb,1:red,0.0;green,1.0;blue,0.498}, name path={285cc138-eec4-4d90-a209-683155680019}, draw opacity={0.8}, line width={1}, solid, forget plot]
        table[row sep={\\}]
        {
            \\
            1.6095671713405044  1.7191615435771748  \\
            1.6348828068615442  1.5322431195203052  \\
        }
        ;
    \addplot[color={rgb,1:red,0.0;green,1.0;blue,0.498}, name path={285cc138-eec4-4d90-a209-683155680019}, draw opacity={0.8}, line width={1}, solid, forget plot]
        table[row sep={\\}]
        {
            \\
            1.5209194746064107  1.2869409180911981  \\
            1.472714966518481  1.25429855722196  \\
        }
        ;
    \addplot[color={rgb,1:red,0.0;green,1.0;blue,0.498}, name path={285cc138-eec4-4d90-a209-683155680019}, draw opacity={0.8}, line width={1}, solid, forget plot]
        table[row sep={\\}]
        {
            \\
            1.7522049938096065  1.2811541224295153  \\
            1.786608982970989  1.3084506635857562  \\
        }
        ;
    \addplot[color={rgb,1:red,0.0;green,1.0;blue,0.498}, name path={285cc138-eec4-4d90-a209-683155680019}, draw opacity={0.8}, line width={1}, solid, forget plot]
        table[row sep={\\}]
        {
            \\
            1.7522049938096065  1.2811541224295153  \\
            1.7602506305045384  0.9510286866762687  \\
        }
        ;
    \addplot[color={rgb,1:red,0.0;green,1.0;blue,0.498}, name path={285cc138-eec4-4d90-a209-683155680019}, draw opacity={0.8}, line width={1}, solid, forget plot]
        table[row sep={\\}]
        {
            \\
            1.9876471535093847  1.5816108873343284  \\
            1.7835778408291005  1.3953116355771533  \\
        }
        ;
    \addplot[color={rgb,1:red,0.0;green,1.0;blue,0.498}, name path={285cc138-eec4-4d90-a209-683155680019}, draw opacity={0.8}, line width={1}, solid, forget plot]
        table[row sep={\\}]
        {
            \\
            1.9876471535093847  1.5816108873343284  \\
            1.8295057313643917  1.599346634254463  \\
        }
        ;
    \addplot[color={rgb,1:red,0.0;green,1.0;blue,0.498}, name path={285cc138-eec4-4d90-a209-683155680019}, draw opacity={0.8}, line width={1}, solid, forget plot]
        table[row sep={\\}]
        {
            \\
            1.6348828068615442  1.5322431195203052  \\
            1.7527984799071734  1.4545823716018462  \\
        }
        ;
    \addplot[color={rgb,1:red,0.0;green,1.0;blue,0.498}, name path={285cc138-eec4-4d90-a209-683155680019}, draw opacity={0.8}, line width={1}, solid, forget plot]
        table[row sep={\\}]
        {
            \\
            1.276983303491567  1.5119328295486254  \\
            1.2815350666192726  1.5172242627114598  \\
        }
        ;
    \addplot[color={rgb,1:red,0.0;green,1.0;blue,0.498}, name path={285cc138-eec4-4d90-a209-683155680019}, draw opacity={0.8}, line width={1}, solid, forget plot]
        table[row sep={\\}]
        {
            \\
            1.276983303491567  1.5119328295486254  \\
            1.434708982548336  1.4940758047564862  \\
        }
        ;
    \addplot[color={rgb,1:red,0.0;green,1.0;blue,0.498}, name path={285cc138-eec4-4d90-a209-683155680019}, draw opacity={0.8}, line width={1}, solid, forget plot]
        table[row sep={\\}]
        {
            \\
            1.434708982548336  1.4940758047564862  \\
            1.5599023713656444  1.67051177226448  \\
        }
        ;
    \addplot[color={rgb,1:red,0.0;green,1.0;blue,0.498}, name path={285cc138-eec4-4d90-a209-683155680019}, draw opacity={0.8}, line width={1}, solid, forget plot]
        table[row sep={\\}]
        {
            \\
            1.2604794002114923  1.803403112861966  \\
            1.0492624180072498  1.8802492871517968  \\
        }
        ;
    \addplot[color={rgb,1:red,0.0;green,1.0;blue,0.498}, name path={285cc138-eec4-4d90-a209-683155680019}, draw opacity={0.8}, line width={1}, solid, forget plot]
        table[row sep={\\}]
        {
            \\
            1.4999999999999998  3.4161152454663077  \\
            1.2570826968156772  2.2928010286527063  \\
        }
        ;
    \addplot[color={rgb,1:red,0.0;green,1.0;blue,0.498}, name path={285cc138-eec4-4d90-a209-683155680019}, draw opacity={0.8}, line width={1}, solid, forget plot]
        table[row sep={\\}]
        {
            \\
            1.7835778408291005  1.3953116355771533  \\
            1.786608982970989  1.3084506635857562  \\
        }
        ;
    \addplot[color={rgb,1:red,0.0;green,1.0;blue,0.498}, name path={285cc138-eec4-4d90-a209-683155680019}, draw opacity={0.8}, line width={1}, solid, forget plot]
        table[row sep={\\}]
        {
            \\
            1.7835778408291005  1.3953116355771533  \\
            1.7527984799071734  1.4545823716018462  \\
        }
        ;
    \addplot[color={rgb,1:red,0.0;green,1.0;blue,0.498}, name path={285cc138-eec4-4d90-a209-683155680019}, draw opacity={0.8}, line width={1}, solid, forget plot]
        table[row sep={\\}]
        {
            \\
            1.6220013889866987  1.7343406005768556  \\
            1.8295057313643917  1.599346634254463  \\
        }
        ;
    \addplot[color={rgb,1:red,0.0;green,1.0;blue,0.498}, name path={285cc138-eec4-4d90-a209-683155680019}, draw opacity={0.8}, line width={1}, solid, forget plot]
        table[row sep={\\}]
        {
            \\
            1.4207721973802245  1.7092578946573438  \\
            1.2815350666192726  1.5172242627114598  \\
        }
        ;
    \addplot[color={rgb,1:red,0.0;green,1.0;blue,0.498}, name path={285cc138-eec4-4d90-a209-683155680019}, draw opacity={0.8}, line width={1}, solid, forget plot]
        table[row sep={\\}]
        {
            \\
            1.1913502614960443  1.1117574812212023  \\
            1.2098306655723479  1.3208922816783164  \\
        }
        ;
    \addplot[color={rgb,1:red,0.0;green,1.0;blue,0.498}, name path={285cc138-eec4-4d90-a209-683155680019}, draw opacity={0.8}, line width={1}, solid, forget plot]
        table[row sep={\\}]
        {
            \\
            0.4862484197523599  1.566818902423211  \\
            0.5887246387327791  1.563552573781597  \\
        }
        ;
    \addplot[color={rgb,1:red,0.0;green,1.0;blue,0.498}, name path={285cc138-eec4-4d90-a209-683155680019}, draw opacity={0.8}, line width={1}, solid, forget plot]
        table[row sep={\\}]
        {
            \\
            0.4862484197523599  1.566818902423211  \\
            1.0492624180072498  1.8802492871517968  \\
        }
        ;
    \addplot[color={rgb,1:red,0.0;green,1.0;blue,0.498}, name path={285cc138-eec4-4d90-a209-683155680019}, draw opacity={0.8}, line width={1}, solid, forget plot]
        table[row sep={\\}]
        {
            \\
            1.5599023713656444  1.67051177226448  \\
            1.2815350666192726  1.5172242627114598  \\
        }
        ;
    \addplot[color={rgb,1:red,0.0;green,1.0;blue,0.498}, name path={285cc138-eec4-4d90-a209-683155680019}, draw opacity={0.8}, line width={1}, solid, forget plot]
        table[row sep={\\}]
        {
            \\
            1.2098306655723479  1.3208922816783164  \\
            0.5887246387327791  1.563552573781597  \\
        }
        ;
    \addplot[color={rgb,1:red,0.0;green,1.0;blue,0.498}, name path={285cc138-eec4-4d90-a209-683155680019}, draw opacity={0.8}, line width={1}, solid, forget plot]
        table[row sep={\\}]
        {
            \\
            1.7527984799071734  1.4545823716018462  \\
            1.8295057313643917  1.599346634254463  \\
        }
        ;
    \addplot[color={rgb,1:red,0.0;green,1.0;blue,0.498}, name path={285cc138-eec4-4d90-a209-683155680019}, draw opacity={0.8}, line width={1}, solid, forget plot]
        table[row sep={\\}]
        {
            \\
            1.0492624180072498  1.8802492871517968  \\
            1.2570826968156772  2.2928010286527063  \\
        }
        ;
\end{axis}
\end{tikzpicture}
}
  \caption{\label{fig:eval.voronoi.output}
    Delaunay Triangulations (on the left) and Voronoi Diagrams (on the right)
    produced both by \texttt{CGAL.jl} and \texttt{VoronoiDelaunay.jl}.}%
\end{figure}

Surprisingly, the triangulations are not the same, explaining the divergence in
the respective Voronoi Diagrams.  Simpler point distributions illustrate this
disparity further.  Though it is not clear which of the implementations is
wrong, given \ac{CGAL} is a more mature library, it is probably safe to assume
that the implementation in \texttt{VoronoiDelaunay.jl} is the culprit.

Summarily, new implementations of complex algorithms from scratch are very
likely to produce erroneous results, paling in comparison to more mature
alternatives that may be repurposed.  Despite showing potential, new
implementations will have a hard time competing with established battle-tested
software.


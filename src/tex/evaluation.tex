% !TEX root = ../main.tex
\fancychapter{Evaluation}%
\label{chap:eval}
\cleardoublepage{}

\section{Performance \& Correctness}%
\label{sec:eval.perf}
\todo{Working title}

\subsection{ConstraintGM}%
\label{sec:eval.perf.cgm}
\todo[inline]{Introduce Fábio's work with ConstraintGM, point out the mishap of
feeling overconfident about maxima and how that prompted him, per suggestion, to
implement specific/contextual solutions for geometric constraint problems, i.e.,
what we did, instead of solely relying on a generic algebraic solver.  Recall
his benchmarks and compare them with identical benchmarks with our solutions.
Discuss the possibility that language differences might be the reason behind
performance differences alone.}

\subsection{VoronoiDelaunay.jl}%
\label{sec:eval.perf.vdjl}
\todo[inline]{Maybe leave the "how easy it is to leverage our approach to get
more, both in quantity and complexity, algorithsm from CGAL" discussion for this
part and evaluate the potential time it takes to use Voronoi Diagrams from CGAL
and either get more information on how long it took to implement
VoronoiDelaunay.jl and compare its "correctness" vs. CGAL's version of the
algorithm, assuming CGAL's results as a source of truth for correctness.
Testing revealed diagrams differed slightly when it came to some edges.  My
suspicion was the Julia algorithm "gobble" some very very very small triangles,
i.e., where the edges are very close together, but it only happens near the
outside of the diagram, and, again, a more drastic diagram can be manufactured
to showcase this.  Maybe read up more on the underlying algorithm that was
implemented in VoronoiDelaunay.jl}

\section{Case Studies}%
\label{sec:eval.studies}
\todo[inline]{Nexus article evaluation, mostly.  Maybe a bit more elaborate,
maybe a bit less.}

\subsection{Egg}%
\label{sec:eval.studies.egg}

\subsection{Rounded Trapezoid}%
\label{sec:eval.studies.rtrapezoid}

\subsection{Star with Semicircles}%
\label{sec:eval.studies.star}

\subsection{Voronoi Diagram}%
\label{sec:eval.studies.voronoi}

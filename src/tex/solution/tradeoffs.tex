% !TeX root = ../../main.tex
\section{Trade-offs}%
\label{sec:solution.tradeoffs}

Since virtually nothing comes without trade-offs and compromises, it is
paramount we address our implementation's qualities, negative and positive.

Relying on a library such as \ac{CGAL} proves to be as great as it can be
daunting.  As mentioned in \cref{sec:solution.impl.cgal}, \ac{CGAL} is a very
comprehensive and mature software library, arguably even far exceeding our
solution's needs, yet fitting it perfectly.

It is, however, an external component, and with every such component, we do not
hold as much control over it as if it was internal instead.  For example, in the
advent a bug is found within \ac{CGAL}, one cannot \emph{immediately} fix it by
altering its source code and use this fixed version.\footnote{Not unless one
wishes to maintain an entirely separate version of the software until this local
fix is applied upstream.  It can be tiresome if we wish to distribute our
solution and have to redistribute a different version of \ac{CGAL} as well
instead of the globally available releases.}  Important emphasis on bugs not
being \emph{immediately} fixable lest we forget \ac{CGAL} is still an
open-source project arguably anyone can contribute to.  Alas, said contribution
deployments are still out of our control.  In hindsight, however, it can be
considered as just an inconvenience since it is a project that is actively
maintained by very knowledgeable in the computer graphics and mathematics
fields.

\Ac{CGAL} is also a highly generic library, making heavy use of C++ templates.
Although its design makes usage an elegant experience, the same cannot be said
when trying to wrap its constructs to another language, especially a language
with a different memory management paradigm which could lead to some nasty
low-level ordeals.  Luckily, \texttt{CxxWrap.jl} helps us solve most, if not
all, of those issues.

Regarding our wrapper code,  there is one small problem we still technically did
not solve.  We are still mapping \ac{CGAL} types in an opaque fashion, fixing
the kernel on the C++ side.\footnote{There are other projects that attempt to
make \ac{CGAL} available in other language that resort to this same trick. See
\url{https://github.com/scikit-geometry/scikit-geometry} and
\url{https://github.com/CGAL/cgal-swig-bindings}.}  Alternative methods of
supplying a different kernel were explored, including deploying a different
library compiled with a different kernel.  Despite not entirely solving the
problem, it offered the user the choice of using exact constructions.  However,
this alternative became unmaintainable and virtually impossible to successfully
employ.  In part due to Julia's package pre-compilation, it is no longer viable
to attempt and use the other library.

Ideally, the geometric objects would be parametric, and the kernel (or kernels)
mapped to Julia as well, but we were met with complications that would make it
unfeasible for us to implement our solution.  As a compromise, \texttt{CGAL.jl}
fixed a kernel that provides exact predicates with inexact constructions,
favoring performance over some robustness loss.  Nonetheless, in practical
terms, it suffices for our case.

As an alternative to wrapping \ac{CGAL}, we could have explored other options in
the still growing Julia ecosystem.  Some work seems
promising,\footnote{\url{https://github.com/JuliaGeometry}} but not only are
some libraries still catching up, it is also highly unlikely they will ever meet
the quality of \ac{CGAL}.

Lastly, some of our \ac{GC} primitives employ some computation that can lead to
robustness loss as well.  Primarily the ones that involve circles, non-linear
geometric objects, typically require the computation of a square root to work
with absolute distances.  We to typically avoid those computations, postponing
them as much as possible until they are absolutely necessary.  As an example, in
the ``Tangent lines to a circle'' implementation in
\cref{lst:solution.impl.gcps.tanglines2circles}, we first compare squared
distances before requiring their absolute value.

Be that as it may, due to having fixed a kernel with inexact constructions for
\ac{CGAL}, we are bound with some robustness loss regardless.  At this point, it
can only be mitigated by also ensuring the quality of the inputs given to our
primitives, and that is reliant on the user.  If the user decides to provide
``garbage'' inputs, they will most likely be met with ``garbage'' outputs, a
concern, however, that falls out of the scope of this work.

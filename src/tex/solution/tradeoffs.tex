% !TeX root = ../../main.tex
\subsection{Trade-offs}%
\label{sec:solution.tradeoffs}

Since virtually nothing comes without trade-offs and compromises, it is
paramount we address our implementation's qualities.

\Ac{CGAL} is a highly generic library, making heavy use of C++ templates.
Although its design makes usage an elegant experience, the same cannot be said
when trying to wrap its constructs to another language. Luckily,
\texttt{CxxWrap.jl} helps us overcome this.

Regarding our wrapper code, we are still mapping \ac{CGAL} types in an opaque
fashion, fixing the kernel on the C++ side.  Ideally, objects would be
parametric, and the kernel mapped to Julia as well.  As a compromise,
\texttt{CGAL.jl} fixed a kernel that provides exact predicates with inexact
constructions, favoring performance over some robustness loss.  Nonetheless, in
practical terms, it suffices for our case.

As an alternative to wrapping \ac{CGAL}, we could have explored other options in
the still growing Julia ecosystem.  Some work looks
promising,\footnote{\url{https://github.com/JuliaGeometry}} but not only are
some libraries still catching up, it is also highly unlikely they will ever meet
the quality of \ac{CGAL}.

Lastly, some of our \ac{GC} primitives employ some computation that can lead to
robustness loss as well.  We typically avoid those computations, postponing them
as much as possible.  Be that as it may, we are bound with some robustness loss
regardless.  At this point, it can only be mitigated by also ensuring the
quality of the inputs given to our primitives, and that is reliant on the user.

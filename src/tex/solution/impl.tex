% !TEX root = ../../main.tex
\section{Implementation}%
\label{sec:solution.impl}

This section details implementation choices with regard to the chosen platforms
for realizing the initially proposed general solution architecture, previously
illustrated in \cref{fig:solution.arch}. Following a brief analysis, we expand
specifically on the concrete components corresponding to the ones within
\cref{fig:solution.arch}'s light blue rectangle.

Examining the \ac{AD} workflow portion of \cref{fig:solution.arch}, there are
depictions of \ac{CAD}, \ac{BIM}, and analysis tools, of which examples could be
Rhinoceros3D, Autodesk's Revit, and Radiance, respectively, with no particular
focus on any of them.  Digging a layer deeper, we find the \ac{AD} tool, which,
by means made available by the tools above it, produces models specific to those
tools from a description provided by the user.  The \ac{AD} tool we have chosen
was
Khepri\footnote{\url{https://github.com/aptmcl/Khepri.jl}}~\cite{Leitao:2019:GRUGEAV},
a text-based tool written in the Julia programming
language~\cite{Bezanson:2017:JAFANC}.  Khepri is the successor of another
text-based \ac{AD} tool named Rosetta~\cite{Leitao:2011:PGDCAD}, a tool written
in the Racket programming language~\cite{PLT:2010:Reference}.  It follows that
the \primitives{} were implemented in the Julia language as well, supported by
an \geomlibrary{}.  The library chosen for the effect was
\ac{CGAL}~\cite{CGAL:5.3:Project}, a highly performant and robust geometric
library written in the C++ programming language~\cite{Stroustrup:2013:CPP}.
\Ac{CGAL} is a comprehensive, mature, open-source library, offering plenty of
data structures and functionality of varying complexity, focusing to provide
easy access to efficient and reliable geometric algorithms.

The language disparity between the \primitives{} module (Julia) and the
\geomlibrary{} (C++) requires a solution for language interoperation.  In other
words, we need to make \ac{CGAL} available to the Julia language.  The Julia
language already possesses facilities that invoke functionality within libraries
written in the C~\cite{Kernighan:1988:C} or the
Fortran~\cite{Backus:1957:Fortran} programming languages, an interfacing
mechanism is commonly known as \ac{FFI}.  It allows for the repurposing of
mature software libraries in foreign languages without the need for a complete
rewrite or adaptation.\footnote{The decision to include such a mechanism at the
language's core by the language designers makes it so that the language can gain
traction early, biding time to later explore native implementations.  Arguably,
it may be one of \textit{the} fundamental features that made the language as
popular as it is and kept it afloat, unlike other similar historical examples
that might have lacked such a mechanism.}  Despite only addressing libraries
written in C and Fortran, this mechanism can also in turn be leveraged and built
upon to interface with other programming languages, e.g., Java, Python, MATLAB,
and, C++.\footnote{There is an entire GitHub organization with projects
dedicated to foreign language interoperation at
\url{https://github.com/JuliaInterop} (July 8, 2021)}

Overcoming the language interoperability hurdle, we can now start focusing on
the implementation of the \primitives{}.  These primitives build on top of the
functionality available in \ac{CGAL}, some of which is directly inherited from
it, substantially helping us in the process.  We further enriched the pool with
a few more functions, illustrating a constructive approach to \ac{GCS}, similar
and inspired by the approach of \texttt{tkz-euclide}, mentioned in
\cref{sec:related.constraints.tikz}.  By creating an abstraction over more
primitive functionality, we aimed to provide an easy to understand and utilize
set of tools so that users can avoid the error-prone process of reimplementing
it themselves.  Furthermore, we level the playing field by working at a
conceptual level which is more familiar to traditional \ac{CAD} software users
rather than falling back to the more analytic approach programming languages
naturally offer.

The following sections will elaborate further on the components emphasized in
the previous paragraphs, adopting a bottom-up-like approach.  We will discuss
\ac{CGAL} and what constructs and functionality it can provide to aid our goal,
as well as some added benefits of building on top of a very mature and
comprehensive library.  That will be followed by a section detailing how it was
possible to map said functionality to the Julia language, of which the result
was a Julia package aptly named \texttt{CGAL.jl}\footnote{Packages in the Julia
ecosystem are conventionally terminated with a \texttt{.jl} suffix, the
extension used for Julia files.  This is reminiscent of a familiar convention
followed in the Java ecosystem where libraries and tools are usually prefixed
with the letter \texttt{J}, e.g., \texttt{JUnit}, \texttt{JMeter},
\texttt{JDeveloper}, among others.}~\cite{Ventura:2021:CGAL.jl}.  Finally, we
showcase how we leveraged \texttt{CGAL.jl} to build the aforementioned
\primitives{}, a set of functionalities that implements specialized yet
comprehensible constructive approach solutions to \ac{GC} problems.

% !TeX root = ../../../main.tex
\subsection{Computational Geometry Algorithms Library}%
\label{sec:solution.impl.cgal}

\Ac{CGAL} is a software project that provides easy access to efficient and
reliable geometric algorithms in the form of a C++ library.  It is considered
the industry's \textit{de facto} standard geometric library, used in well-known
projects such as OpenSCAD.  It is also a very mature software library with
decades of Ph.D.-grade research results, still being actively maintained to this
day.  Being an open-source project, one can easily contribute to it by reporting
issues in the software as well as directly submitting patches.\footnote{The
library's source is hosted on GitHub at \url{https://github.com/CGAL/cgal}.  To
illustrate the ease with which one can contribute to the project, here is a pull
request the author submitted: \url{https://github.com/CGAL/cgal/pull/4705}.}

These factors, among others, justify our choice for our solution's
\geomlibrary{} component: we chose \ac{CGAL} because of its comprehensiveness
and decades of work invested in the production of a piece of highly mature
software, as well as the critical mass of maintainers behind it.  That is not to
say less mature software cannot be used in its stead, though it is unlikely it
can match \ac{CGAL}, be it in terms of performance, quality, or breadth.

\ac{CGAL} offers a multitude of data structures and algorithms, such as
triangulations, Voronoi diagrams, and convex hull algorithms, to name a few.
The library is broken up into three parts~\cite{CGAL:5.3:23LGK}:
\begin{enumerate}
  \item The kernel, which consists of geometric primitive objects and operations
  on these objects.  The objects are represented both as
  \begin{enumerate*}
    \item stand-alone classes parameterized by a representation class that
    specifies the underlying number types used for computation, and as
    \item members of the kernel classes, which allows for more flexibility and
    adaptability of the kernel;
  \end{enumerate*}
  \item Basic geometric data structures and algorithms, parameterized by traits
  classes that define the interface between the data structure or algorithm and
  the primitives they use;
  \item Non-geometric support facilities, such as circulators, random sources,
  and I/O support for debugging and for interfacing \ac{CGAL} with various
  visualization tools.
\end{enumerate}

Kernels in \ac{CGAL} are parametric, enabling the combination of kernels with
exact predicates and exact constructions or inexact constructions.  The latter
provide exact predicates and are faster than the former, but may produce inexact
results due to \textit{roundoff} errors.  In practice, however, inexact
construction kernels suffice for most of \ac{CGAL}'s algorithms.

\Cref{lst:solution.impl.cgal.pas} showcases an example of a very simple
\ac{CGAL} program, demonstrating the construction of points and a segment, and
performing some basic operations on them.

\begin{listing}[htbp]
  \caption[CGAL: Three points and one segment]{
    An example CGAL program illustrating how to construct some points and a line
    segment, and perform some basic operations on them.  It uses
    \mintinline{c}{double} precision floating point numbers for Cartesian
    coordinates.}\label{lst:solution.impl.cgal.pas}
  \inputminted{cpp}{cpp/points_and_segments.cpp}
\end{listing}

As mentioned, geometric primitive types are defined in the kernel.  The kernel
chosen in the example uses \texttt{double} precision floating point numbers for
the Cartesian coordinates of the point.

We can also see some predicates, such as testing the orientation of three
points, and constructions, like the distance\footnote{It is worth noting
\ac{CGAL} does \texttt{not} compute the absolute distance, offering instead to
compute the squared distance, as this avoids the additional square root
computation.  This preserves exactness and eliminates a potentially unnecessary
heavy computation.} and midpoint computation.  Predicates typically produce a
boolean logical value or one of a discrete set of possible results, whereas
constructions produce either a number or another geometric entity.

It is worth noting that floating point-based geometric computation can lead to
surprising results since we are relying on inexact constructions.  If one must
ensure that the numbers get interpreted at their full precision, all one has to
do is pick a kernel with exact constructions.  Revisiting
\cref{lst:solution.impl.cgal.pas}, it is as simple as switching the
\texttt{Simple\_cartesian} kernel with one the provides exact constructions,
e.g., \texttt{Exact\_predicates\_exact\_constructions\_kernel} or \texttt{Epeck}
for short.

Unfortunately, \ac{CGAL} is a terribly complex library under the hood,
presenting many challenges when it comes to mapping it to the Julia language.
Firstly, it is a C++ library.  Despite Julia's native capabilities for
interoperating with C, there's additional work to be done to reach C++ code.
Secondly, is problem of memory management, which differs between C/C++ and
Julia, potentially leading to memory leaks and other related issues if not
properly tended to.  Finally, \ac{CGAL} makes extensive use of C++ templates,
proving sometimes difficult to transparently map some of its constructs.

In the next section, we go over how we overcame these issues.

% !TeX root = ../../../main.tex
\subsection{From C++ to Julia}%
\label{sec:solution.impl.jlcgal}

% \todo[inline]{Reiterate the language interoperability hurdle, maybe mention
% briefly there are complications especially when memory management models differ.
% Showcase Julia's native capabilities of invoking native C++ libraries coupled w/
% an example using CGAL.  Introduce a library that helps with C++ wrapping and
% showcase a slightly more complex example, maybe involving classes and the like.
% Consider demonstrating the ease with which it is possible to wrap things
% incrementally as, on demand, requests for new features may require algorithms
% from the library that weren't wrapped.  Just like following a recipe, it's
% as simple as (1) looking at the docs, (2) mapping necessary types and functions,
% and (3) run Julia (Carefully not mapping \textit{everything}, thought that is
% what I strived to do with CGAL.jl, but that's another discussion).}

Having established \ac{CGAL} as our \textit{Exact Computation Geometry Library}
of choice, we must now overcome the quite literal language barrier between Julia
and C++.  Fortunately, the former possesses \ac{FFI} mechanisms that can aid us
in resolving this issue.  Here is an excerpt from the language's manual about
Julia's C and Fortran \ac{FFI} capabilities:\footnote{From
\url{https://docs.julialang.org/en/v1/manual/calling-c-and-fortran-code/}}

\begin{quote}
  \itshape\color{gray}
  To allow easy use of (\ldots) existing code, Julia makes it simple and
  efficient to call C and Fortran functions.  (\ldots) This is accomplished just
  by making an appropriate call with \mintinline{julia}{ccall} syntax, which
  looks like an ordinary function call.
\end{quote}

To illustrate how this \mintinline{julia}{ccall} construct can be used in the
context of our problem, we created a small wrapper around \ac{CGAL} exposing the
\texttt{squared\_distance} function.  The original function we are interested in
takes two instances of 3D points.  To facilitate the wrapping, we create the
points on the C++ side of things, instead passing primitive
\mintinline{c}{double} values representing the points' coordinates.  The C++
wrapper library is listed in \cref{lst:solution.impl.jlcgal.sqdist.cpp}.

\begin{listing}[htb]
  \inputminted{cpp}{cpp/sqdist.cpp}
  \caption[C wrapper for squared distance functionality]{
    Example C library code that wraps \ac{CGAL}'s \texttt{squared\_distance}
    global function.  The original function takes in instances of
    \texttt{Point\_3} classes, so we instantiate them from our
    \mintinline{c}{double} coordinate inputs.}%
  \label{lst:solution.impl.jlcgal.sqdist.cpp}
\end{listing}

To invoke our wrapper function, we must precede it with \mintinline{cpp}{extern
"C"} as is illustrated.  This is important because C++ compilers mangle function
names.  This technique is employed to solve a series of problems in order to
support features like function overloading. By preventing this from happening,
we can then refer to our wrapper function by its declared name, just like a C
function.

After compiling the library, we can invoke the newly wrapped function from Julia
using \mintinline{julia}{ccall}, as showcased in
\cref{lst:solution.impl.jlcgal.sqdist.jl}.

\begin{listing}[htb]
  \inputminted{julia}{jl/sqdist.jl}
  \caption[Julia squared distance example program]{
    Example Julia program that invokes the functionality from the library whose
    source is listed in \cref{lst:solution.impl.jlcgal.sqdist.cpp}.  Julia's
    \mintinline{julia}{ccall} construct converts the input arguments' types to
    the types specified in the native C function's parameter types.}%
  \label{lst:solution.impl.jlcgal.sqdist.jl}
\end{listing}

Though this is looks like a good start, the number passing strategy soon shows
its limitations. For example, think of the combinatorial explosion problem that
may arise when a function requires a number $M$ of $N$-dimensional points.  We
would then have to create a wrapper that takes $N\cdot M$ coordinates.  Such an
approach does not scale.  It is possible to overcome this limitation by
mirroring C \mintinline{c}{struct}s in Julia.

As an example, we consider the \texttt{circumcenter} function from \ac{CGAL}, a
function that takes three points as its parameters, returning the circumcenter
about the given points, under the assumption the points are not collinear.  We
could try and directly mirror \ac{CGAL}'s \texttt{Point\_3} type, but that would
require that we know its layout.  Even then, we would be breaking an abstraction
barrier that could prove detrimental if \ac{CGAL}'s developers decide to change
the type's internal representation.  To circumvent this completely, we can just
create a \mintinline{c}{struct} that opaquely wraps around \ac{CGAL}'s type.
The C++ wrapper code for this example is listed in
\cref{lst:solution.impl.jlcgal.circ.cpp}.

\begin{listing}[htbp]
  \inputminted{cpp}{cpp/circ.cpp}
  \caption[C wrapper for circumcenter functionality]{
    Example C shared library source code that wraps \ac{CGAL}'s circumcenter
    global function.  In this instance, we use an additional struct to wrap
    around \ac{CGAL}'s \texttt{Point\_3} class to facilitate data transfer.}%
  \label{lst:solution.impl.jlcgal.circ.cpp}
\end{listing}

The wrapper code looks very similar to the previous example in
\cref{lst:solution.impl.jlcgal.sqdist.cpp}.  This time, however, we introduced a
new \mintinline{c}{struct Point} that we will mirror in Julia so that we can
seamlessly pass instances of it to our externalized C++ function.\footnote{Note
that, unlike the function, the \mintinline{c}{struct} was not externalized.
This is not necessary because we do not need to refer it by name.  We need only
to match its field layout.}  All the wrapper function does is take in our
\texttt{Point}s and creates new \texttt{Point\_3} objects, using them to invoke
\ac{CGAL}'s \texttt{circumcenter} function.  The returned \texttt{Point\_3} is
then used to create a \texttt{Point}, which is then sent back upstream to Julia.

On the Julia side of things, the process is much the same as before with the
addition of a new \texttt{Point} type that contains three
\mintinline{julia}{Float64} fields. The previous example showed the latter Julia
type's correspondence to the C/C++ \mintinline{c}{double} type.  The Julia code
for this example is listed in \cref{lst:solution.impl.jlcgal.circ.jl}.

\begin{listing}[htb]
  \inputminted{julia}{jl/circ.jl}
  \caption[Julia circumcenter example program]{
    Example Julia program that invokes the functionality from the library listed
    in \cref{lst:solution.impl.jlcgal.circ.cpp}.  We use an additional Julia
    struct that's equivalent to the one specified in C to facilitate data
    transfer.}%
  \label{lst:solution.impl.jlcgal.circ.jl}
\end{listing}

We are once again met with a very familiar snippet of code.  Much like with its
respective wrapper library, there is a new \mintinline{julia}{struct Point} that
mirrors the one we created in C++.  From then on, the process is exactly the
same.

So far, we were able to extract relatively useful functions from \ac{CGAL}.  In
fact, the latter already solves the problem exemplified in a previous chapter in
\cref{sec:intro.examples.circumcenter}.  Although we could incrementally build
upon this approach, not only does it become cumbersome, but it proves
impractical, given \ac{CGAL}'s complexity.

Fortunately, the Julia community has explored methods of interoperating with
many other languages, one of them being C++.  That exploration resulted in
packages like 
\texttt{CxxWrap.jl}\footnote{\url{https://github.com/JuliaInterop/CxxWrap.jl}}.
\texttt{CxxWrap.jl} adopts an approach to language interoperation similar to
that of BOOST.Python~\cite{Abrahams:2003:BHSBP} or
pybind11\footnote{\url{https://github.com/pybind/pybind11}}.  The user writes
the code for the Julia wrapper in C++, and then simply issue an instruction on
the Julia side to initialize the library, making it available there.  The Julia
package is supported by a C++ component known as \texttt{libcxxwrap-julia}, or
the friendlier name JlCxx. This component is what the C++ wrapper code depends
on.  \Cref{lst:solution.impl.jlcgal.jlcxx} constitutes the code to wrap
necessary functionality to reproduce the example \ac{CGAL} program in
\cref{lst:solution.impl.cgal.pas}.

\begin{listing}[htbp]
  \caption[Wrapper CxxWrap code for Three points and one segment]{
    C++ wrapper code powered by JlCxx that maps the types and functions needed
    from \acs{CGAL} to reproduce the example shown in
    \cref{lst:solution.impl.cgal.pas} in Julia.}%
  \label{lst:solution.impl.jlcgal.jlcxx}
  \vspace{-8pt}
  \inputminted[highlightlines={24,30-34,36-38,40-43,46-49}]%
    {cpp}{cpp/cgal_julia.cpp}
\end{listing}

We direct our focus to the lines that are highlighted in the figure.  We define
a function that is later invoked by \texttt{CxxWrap.jl}.  In it, we start
registering the required types and functions that we need in a declarative
fashion\footnote{Order with which types and functions are registered matters.
This means we cannot add a function that depends on a type we did not yet
register.}, reminiscent of the builder design pattern.  Notice the
\texttt{JLCXX\_MODULE} symbol preceding the function definition.  That symbol
takes care of properly externalizing the function regardless of the system the
library is built for.\footnote{Think similar to \mintinline{c}{extern "C"}, but
slightly more robust.}

After compiling the wrapper code, we can load it on the Julia side resorting to
\texttt{CxxWrap.jl}.  This process is reminiscent of and analogous to the ones
illustrated earlier with simpler examples using \mintinline{julia}{ccall}.
\Cref{lst:solution.impl.jlcgal.cgal} shows a bare-bones CGAL Julia module.

\begin{listing}[htbp]
  \inputminted[highlightlines={4-6,8,9}]{julia}{jl/CGAL.jl} 
  \caption[Bare-bones Julia module wrapping some of CGAL]{
    An example Julia module that mimics \texttt{CGAL.jl}, wrapping the library
    produced from \cref{lst:solution.impl.jlcgal.jlcxx}.  It initializes the
    library and exports the mapped functionality.}%
  \label{lst:solution.impl.jlcgal.cgal}
\end{listing}

All that is necessary to make the functionality we wrapped on the C++ side
available is
\begin{enumerate*}[label= (\arabic*)]
  \item tell \texttt{CxxWrap.jl} where the library is,
  \item tell \texttt{CxxWrap.jl} to initialize itself when the module is loaded,
  and
  \item export the mapped functionality.
\end{enumerate*}
The rest of the non-highlighted code, both in this example and the previous
only serves the purpose of obtaining a human-readable representation of the C++
objects in Julia.

Finally, we reach a point where we are able to reproduce the example listed in
\cref{lst:solution.impl.cgal.pas} in the Julia language, having mapped all the
necessary functionality to do so.  \Cref{lst:solution.impl.jlcgal.pas} shows the
example, translated from C++.

\begin{listing}[htbp]
  \inputminted{julia}{jl/points_and_segments.jl}
  \caption[CGAL.jl: Three points and one segment]{
    The example program as seen in \cref{lst:solution.impl.cgal.pas} written in
    the Julia programming language using \texttt{CGAL.jl}.  The kernel
    instantiation is hidden away in the C++ layer of the wrapper code.}%
  \label{lst:solution.impl.jlcgal.pas}
\end{listing}

We illustrated how to repurpose some core functionality on which we can continue
to incrementally build upon following a similar approach to that shown in
\cref{lst:solution.impl.jlcgal.jlcxx}.  Doing so, with relatively low effort, we
can obtain the primitive objects and functionality on which our
\textit{Geometric Constraint Primitives} will be supported.  The following
section goes over how we effectively used the functionality from
\texttt{CGAL.jl} to implement constructive solutions for \ac{GC} problems.

% !TeX root = ../../../main.tex
\subsubsection{Geometric Constraint Primitives}%
\label{sec:solution.impl.gcps}

Having overcome the language barrier between C++ and Julia, we can build our
\ac{GC} primitives on top of \texttt{CGAL.jl}.  Our implementation follows a
constructive approach where the production of geometry can be done solely
resorting to a straightedge and a compass.  This makes programs easier to
understand and manually reproduce.

The following sections revisit of our initially formulated example problems from
\cref{sec:intro.examples}.

\subsubsection*{Parallel lines}%
\label{sec:solution.impl.gcps.parallel}

Revisiting our earlier examples, we now showcase implementations for those
problems using our solution, accompanied by the Khepri \ac{AD} tool.
\Cref{lst:solution.impl.gcps.parallel} shows a solution to the ``parallel
lines'' problem introduced in \cref{sec:intro.examples.parallel}.

\begin{listing}[htbp]
  \inputminted[highlightlines={4,6-7,16}]{julia}{jl/ex_parallel.jl}
  \caption[Parallel lines example using our solution]{
    Implementation of the parallel lines example illustrated in
    \cref{fig:intro.example.parallel} using Khepri alongside our solution.}%
  \label{lst:solution.impl.gcps.parallel}
\end{listing}

The highlighted \texttt{parallel} function takes a line segment \texttt{l} and a
point \texttt{p} and creates a new line segment starting at point \texttt{p}
with the same length as \texttt{l} , obtained using \texttt{CGAL.jl}'s
\texttt{to\_vector} function.  \Cref{fig:solution.impl.gcps.parallel}
illustrates the program's output in AutoCAD.

\begin{figure}[htbp]
  \includegraphics[width=\linewidth]{fig/autocad-parallel} 
  \caption{Parallel lines example using our solution, visualized in AutoCAD\@.}%
  \label{fig:solution.impl.gcps.parallel}
\end{figure}

\subsubsection*{Circumcenter}%
\label{sec:solution.impl.gcps.circumcenter}

We initially solved the circumcenter problem by intersecting triangle sides'
bisectors.  We can still approach the problem that way, defining a
\texttt{circumcenter} function similar to the one in
\cref{lst:solution.impl.gcps.circimpl}.

\begin{listing}[htbp]
  \begin{minted}{julia}
  circumcenter(a, b, c) = intersection(bisector(a, b), bisector(b, c)) 
  \end{minted}
  \caption[Initial circumcenter solution]{
    Initial implementation of \texttt{circumcenter}.}%
  \label{lst:solution.impl.gcps.circimpl}
\end{listing}

However, this functionality is already present in \ac{CGAL}.  This is a perfect
demonstration of our approach's benefits regarding repurposing a comprehensive
library with plenty of functionality.

\Cref{lst:solution.impl.gcps.circumcenter} illustrates a solution to the
``circumcenter'' problem using \ac{CGAL}'s \texttt{circumcenter} function.  The
program's output can be seen in \cref{fig:solution.impl.gcps.circumcenter}.

\begin{listing}[htbp]
  \inputminted[highlightlines={2,4-5,17}]{julia}{jl/ex_circumcenter.jl}
  \caption[Circumcenter example using our solution]{
    Implementation of the circumcenter example illustrated in
    \cref{fig:intro.example.circumcenter} using Khepri alongside our solution.}%
  \label{lst:solution.impl.gcps.circumcenter}
\end{listing}

\begin{figure}[htbp]
  \includegraphics[width=\linewidth]{fig/autocad-circumcenter} 
  \caption[Circumcenter example using our solution]{
    Circumcenter example using our solution, visualized in AutoCAD\@.}%
  \label{fig:solution.impl.gcps.circumcenter}
\end{figure}


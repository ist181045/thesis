% !TeX root = ../../../main.tex
\subsubsection{Computational Geometry Algorithms Library}%
\label{sec:solution.impl.cgal}

\Ac{CGAL} is a library that provides easy access to efficient and reliable
geometric algorithms as a C++ library.  It is considered the industry's
\textit{de facto} standard geometric library.  We chose \ac{CGAL} because of its
comprehensiveness and decades of work.

\ac{CGAL} offers a multitude of data structures and algorithms, such as
triangulations, Voronoi diagrams, and convex hull algorithms, to name a few.
The library is broken up into three parts~\cite{CGAL:5.3:23LGK}:
\begin{enumerate*}
  \item The kernel, which consists of geometric primitive objects and operations
  on these objects,
  \item basic geometric data structures and algorithms, and
  \item non-geometric support facilities for debugging and for interfacing
  \ac{CGAL} with various visualization tools.
\end{enumerate*}

\Cref{lst:solution.impl.cgal.pas} showcases an example of a very simple
\ac{CGAL} program, demonstrating the construction of points and a segment, and
performing some basic operations on them.

\begin{listing}[htbp]
  \inputminted{cpp}{cpp/points_and_segments.cpp}
  \caption[CGAL: Three points and one segment]{
    An example CGAL program illustrating object construction and some
    basic operations.}\label{lst:solution.impl.cgal.pas}
\end{listing}

It is worth noting that floating point-based computation can lead to surprising
results.  \Ac{CGAL} offers easily interchangeable kernels that provide
exact constructions, resolving this issue, albeit at a performance cost.

However, \ac{CGAL} is a terribly complex library under the hood, presenting many
challenges when it comes to mapping it to the Julia language.  Firstly, wrapping
C++ with Julia requires additional steps.  Secondly, both languages use differing
memory management mechanisms.  Finally, \ac{CGAL} abuses C++ templates, making
it cumbersome to transparently map functionality.

Fortunately, there are methods and libraries that can help us overcome some of
those issues.  We demonstrate how we overcame said issues, demonstrating it by
reproducing the example in \cref{lst:solution.impl.cgal.pas} in Julia.

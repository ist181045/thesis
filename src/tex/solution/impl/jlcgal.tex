% !TeX root = ../../../main.tex
\subsubsection{From C++ to Julia}%
\label{sec:solution.impl.jlcgal}

Having established \ac{CGAL} as our \geomlibrary{} of choice, we must now
overcome the language barrier between Julia and C++.  Fortunately, the former
possesses \ac{FFI} mechanisms that can aid us in resolving this issue.  Julia
provides a special \mintinline{julia}{ccall} construct that is capable of
efficiently calling C and Fortran functions.

However, we cannot wrap C++ functions directly because C++ compilers mangle
function names.  Target functions must be externalized first, signalling the
compiler with a special instruction.

Besides primitive types, it is possible to map Julia \mintinline{julia}{struct}s
to C \mintinline{cpp}{struct}s to facilitate data transfer.  This strategy,
however, does not scale.  Though we could incrementally build on this approach,
C++ is leaps and bounds more complex than C, which is enough to justify
exploring a different approach.

Fortunately, there is a Julia package destined to wrapping C++ code named
\texttt{CxxWrap.jl}\footnote{\url{https://github.com/JuliaInterop/CxxWrap.jl}}.
It adopts an approach where the user writes the code for the Julia wrapper in
C++ and initializes the library on the Julia side with very little effort.
\Cref{lst:solution.impl.jlcgal.jlcxx} shows the code that wraps the
functionality required to reproduce the program in
\cref{lst:solution.impl.cgal.pas} in Julia.

\Cref{lst:solution.impl.jlcgal.cgal} shows an example bare-bones Julia module
wrapping \ac{CGAL}.  As a result, we can devise the program in
\Cref{lst:solution.impl.jlcgal.pas}, which is a translation of C++ example in
\cref{lst:solution.impl.cgal.pas}.

\begin{listing}[htbp]
  \inputminted{julia}{jl/points_and_segments.jl}
  \caption{\label{lst:solution.impl.jlcgal.pas}
    The example program as seen in \cref{lst:solution.impl.cgal.pas} written in
    the Julia programming language using \texttt{CGAL.jl}.}%
\end{listing}

Our \wrapper{} component expands on this methodology to expose far more
functionality, leading to the creation of the package
\texttt{CGAL.jl}~\cite{Ventura:2021:CGAL.jl},  containing the objects and
functions we need to build our \primitives{}.  The following section goes over
how we effectively used \texttt{CGAL.jl} to implement constructive solutions for
\ac{GC} problems.

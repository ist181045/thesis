% !TeX root = ../../../main.tex
\subsection{From C++ to Julia}%
\label{sec:solution.impl.jlcgal}

% \todo[inline]{Reiterate the language interoperability hurdle, maybe mention
% briefly there are complications especially when memory management models differ.
% Showcase Julia's native capabilities of invoking native C++ libraries coupled w/
% an example using CGAL.  Introduce a library that helps with C++ wrapping and
% showcase a slightly more complex example, maybe involving classes and the like.
% Consider demonstrating the ease with which it is possible to wrap things
% incrementally as, on demand, requests for new features may require algorithms
% from the library that weren't wrapped.  Just like following a recipe, it's
% as simple as (1) looking at the docs, (2) mapping necessary types and functions,
% and (3) run Julia (Carefully not mapping \textit{everything}, thought that is
% what I strived to do with CGAL.jl, but that's another discussion).}

Having established \ac{CGAL} as our \textit{Exact Computation Geometry Library}
of choice, we must now overcome the quite literal language barrier between Julia
and C++.  Fortunately, the former possesses \ac{FFI} mechanisms that can aid us
in resolving this issue.  Here is an excerpt from the language's manual about
Julia's C and Fortran \ac{FFI} capabilities:\footnote{From
\url{https://docs.julialang.org/en/v1/manual/calling-c-and-fortran-code/}}

\begin{quote}
  \itshape\color{gray}
  To allow easy use of (\ldots) existing code, Julia makes it simple and
  efficient to call C and Fortran functions.  (\ldots) This is accomplished just
  by making an appropriate call with \mintinline{julia}{ccall} syntax, which
  looks like an ordinary function call.
\end{quote}

To illustrate how this \mintinline{julia}{ccall} construct can be used in the
context of our problem, we created a small wrapper around \ac{CGAL} exposing the
\texttt{squared\_distance} function.  The original function we are interested in
takes two instances of 3D points.  To facilitate the wrapping, we create the
points on the C++ side of things, instead passing primitive
\mintinline{c}{double} values representing the points' coordinates.  The C++
wrapper library is listed in \cref{lst:solution.impl.jlcgal.sqdist.cpp}.

\begin{listing}[htb]
  \inputminted{cpp}{cpp/sqdist.cpp}
  \caption[C wrapper for squared distance functionality]{
    Example C library code that wraps \ac{CGAL}'s \texttt{squared\_distance}
    global function.  The original function takes in instances of
    \texttt{Point\_3} classes, so we instantiate them from our
    \mintinline{c}{double} coordinate inputs.}%
  \label{lst:solution.impl.jlcgal.sqdist.cpp}
\end{listing}

To invoke our wrapper function, we must precede it with \mintinline{cpp}{extern
"C"} as is illustrated.  This is important because C++ compilers mangle function
names.  This technique is employed to solve a series of problems in order to
support features like function overloading. By preventing this from happening,
we can then refer to our wrapper function by its declared name, just like a C
function.

After compiling the library, we can invoke the newly wrapped function from Julia
using \mintinline{julia}{ccall}, as showcased in
\cref{lst:solution.impl.jlcgal.sqdist.jl}.

\begin{listing}[htb]
  \inputminted{julia}{jl/sqdist.jl}
  \caption[Julia squared distance example program]{
    Example Julia program that invokes the functionality from the library whose
    source is listed in \cref{lst:solution.impl.jlcgal.sqdist.cpp}.  Julia's
    \mintinline{julia}{ccall} construct converts the input arguments' types to
    the types specified in the native C function's parameter types.}%
  \label{lst:solution.impl.jlcgal.sqdist.jl}
\end{listing}

Though this is looks like a good start, the number passing strategy soon shows
its limitations. For example, think of the combinatorial explosion problem that
may arise when a function requires a number $M$ of $N$-dimensional points.  We
would then have to create a wrapper that takes $N\cdot M$ coordinates.  Such an
approach does not scale.  It is possible to overcome this limitation by
mirroring C \mintinline{c}{struct}s in Julia.

As an example, we consider the \texttt{circumcenter} function from \ac{CGAL}, a
function that takes three points as its parameters, returning the circumcenter
about the given points, under the assumption the points are not collinear.  We
could try and directly mirror \ac{CGAL}'s \texttt{Point\_3} type, but that would
require that we know its layout.  Even then, we would be breaking an abstraction
barrier that could prove detrimental if \ac{CGAL}'s developers decide to change
the type's internal representation.  To circumvent this completely, we can just
create a \mintinline{c}{struct} that opaquely wraps around \ac{CGAL}'s type.
The C++ wrapper code for this example is listed in
\cref{lst:solution.impl.jlcgal.circ.cpp}.

\begin{listing}[htbp]
  \inputminted{cpp}{cpp/circ.cpp}
  \caption[C wrapper for circumcenter functionality]{
    Example C shared library source code that wraps \ac{CGAL}'s circumcenter
    global function.  In this instance, we use an additional struct to wrap
    around \ac{CGAL}'s \texttt{Point\_3} class to facilitate data transfer.}%
  \label{lst:solution.impl.jlcgal.circ.cpp}
\end{listing}

The wrapper code looks very similar to the previous example in
\cref{lst:solution.impl.jlcgal.sqdist.cpp}.  This time, however, we introduced a
new \mintinline{c}{struct Point} that we will mirror in Julia so that we can
seamlessly pass instances of it to our externalized C++ function.\footnote{Note
that, unlike the function, the \mintinline{c}{struct} was not externalized.
This is not necessary because we do not need to refer it by name.  We need only
to match its field layout.}  All the wrapper function does is take in our
\texttt{Point}s and creates new \texttt{Point\_3} objects, using them to invoke
\ac{CGAL}'s \texttt{circumcenter} function.  The returned \texttt{Point\_3} is
then used to create a \texttt{Point}, which is then sent back upstream to Julia.

On the Julia side of things, the process is much the same as before with the
addition of a new \texttt{Point} type that contains three
\mintinline{julia}{Float64} fields. The previous example showed the latter Julia
type's correspondence to the C/C++ \mintinline{c}{double} type.  The Julia code
for this example is listed in \cref{lst:solution.impl.jlcgal.circ.jl}.

\begin{listing}[htb]
  \inputminted{julia}{jl/circ.jl}
  \caption[Julia circumcenter example program]{
    Example Julia program that invokes the functionality from the library listed
    in \cref{lst:solution.impl.jlcgal.circ.cpp}.  We use an additional Julia
    struct that's equivalent to the one specified in C to facilitate data
    transfer.}%
  \label{lst:solution.impl.jlcgal.circ.jl}
\end{listing}

We are once again met with a very familiar snippet of code.  Much like with its
respective wrapper library, there is a new \mintinline{julia}{struct Point} that
mirrors the one we created in C++.  From then on, the process is exactly the
same.

So far, we were able to extract relatively useful functions from \ac{CGAL}.  In
fact, the latter already solves the problem exemplified in a previous chapter in
\cref{sec:intro.examples.circumcenter}.  Although we could incrementally build
upon this approach, not only does it become cumbersome, but it proves
impractical, given \ac{CGAL}'s complexity.

Fortunately, the Julia community has explored methods of interoperating with
many other languages, one of them being C++.  That exploration resulted in
packages like 
\texttt{CxxWrap.jl}\footnote{\url{https://github.com/JuliaInterop/CxxWrap.jl}}.
\texttt{CxxWrap.jl} adopts an approach to language interoperation similar to
that of BOOST.Python~\cite{Abrahams:2003:BHSBP} or
pybind11\footnote{\url{https://github.com/pybind/pybind11}}.  The user writes
the code for the Julia wrapper in C++, and then simply issue an instruction on
the Julia side to initialize the library, making it available there.  The Julia
package is supported by a C++ component known as \texttt{libcxxwrap-julia}, or
the friendlier name JlCxx. This component is what the C++ wrapper code depends
on.  \Cref{lst:solution.impl.jlcgal.jlcxx} constitutes the code to wrap
necessary functionality to reproduce the example \ac{CGAL} program in
\cref{lst:solution.impl.cgal.pas}.

\begin{listing}[htbp]
  \caption[Wrapper CxxWrap code for Three points and one segment]{
    C++ wrapper code powered by JlCxx that maps the types and functions needed
    from \acs{CGAL} to reproduce the example shown in
    \cref{lst:solution.impl.cgal.pas} in Julia.}%
  \label{lst:solution.impl.jlcgal.jlcxx}
  \inputminted[fontsize=\small,highlightlines={24,30-34,36-38,40-43,46-49}]%
    {cpp}{cpp/cgal_julia.cpp}
\end{listing}

We direct our focus to the lines that are highlighted in the figure.  We define
a function that is later invoked by \texttt{CxxWrap.jl}.  In it, we start
registering the required types and functions that we need in a declarative
fashion\footnote{Order with which types and functions are registered matters.
This means we cannot add a function that depends on a type we did not yet
register.}, reminiscent of the builder design pattern.  Notice the
\texttt{JLCXX\_MODULE} symbol preceding the function definition.  That symbol
takes care of properly externalizing the function regardless of the system the
library is built for.\footnote{Think similar to \mintinline{c}{extern "C"}, but
slightly more robust.}

After compiling the wrapper code, we can load it on the Julia side resorting to
\texttt{CxxWrap.jl}.  This process is reminiscent of and analogous to the ones
illustrated earlier with simpler examples using \mintinline{julia}{ccall}.
\Cref{lst:solution.impl.jlcgal.cgal} shows a bare-bones CGAL Julia module.

\begin{listing}[htbp]
  \inputminted[highlightlines={4-6,8,9}]{julia}{jl/CGAL.jl} 
  \caption[Bare-bones Julia module wrapping some of CGAL]{
    An example Julia module that mimics \texttt{CGAL.jl}, wrapping the library
    produced from \cref{lst:solution.impl.jlcgal.jlcxx}.  It initializes the
    library and exports the mapped functionality.}%
  \label{lst:solution.impl.jlcgal.cgal}
\end{listing}

All that is necessary to make the functionality we wrapped on the C++ side
available is
\begin{enumerate*}[label= (\arabic*)]
  \item tell \texttt{CxxWrap.jl} where the library is,
  \item tell \texttt{CxxWrap.jl} to initialize itself when the module is loaded,
  and
  \item export the mapped functionality.
\end{enumerate*}
The rest of the non-highlighted code, both in this example and the previous
only serves the purpose of obtaining a human-readable representation of the C++
objects in Julia.

Finally, we reach a point where we are able to reproduce the example listed in
\cref{lst:solution.impl.cgal.pas} in the Julia language, having mapped all the
necessary functionality to do so.  \Cref{lst:solution.impl.jlcgal.pas} shows the
example, translated from C++.

\begin{listing}[htbp]
  \inputminted{julia}{jl/points_and_segments.jl}
  \caption[CGAL.jl: Three points and one segment]{
    The example program as seen in \cref{lst:solution.impl.cgal.pas} written in
    the Julia programming language using \texttt{CGAL.jl}.  The kernel
    instantiation is hidden away in the C++ layer of the wrapper code.}%
  \label{lst:solution.impl.jlcgal.pas}
\end{listing}

We illustrated how to repurpose some core functionality on which we can continue
to incrementally build upon following a similar approach to that shown in
\cref{lst:solution.impl.jlcgal.jlcxx}.  Doing so, with relatively low effort, we
can obtain the primitive objects and functionality on which our
\textit{Geometric Constraint Primitives} will be supported.  The following
section goes over how we effectively used the functionality from
\texttt{CGAL.jl} to implement constructive solutions for \ac{GC} problems.

% !TEX root = ../../main.tex
\subsection{ConstraintGM}%
\label{sec:eval.cgm}

ConstraintGM is a domain-specific language developed with the goal of tackling
\ac{GC} problems using the Racket \ac{TPL}.  This solution blindly relied on
Maxima~\cite{Maxima:2021:Maxima}.

This approach came at a grave performance cost for two reasons:
\begin{enumerate*}[label= (\arabic*)]
  \item the communication between ConstraintGM and Maxima was slow, and
  \item Maxima is a \emph{generic} solver.
\end{enumerate*}
The considerable performance penalty of this approach is hard to justify in the
case of simple geometric problems. This lead to the implementation of some
\ac{GC} problem solutions, creating the \ac{GFL}.

The project's benchmark involved three different \ac{GC} problems focused around
object intersection, namely
\begin{enumerate*}[label= (\arabic*)]
  \item line-line intersection,
  \item circle-line intersection, and
  \item circle-circle intersection.
\end{enumerate*}

We measured real execution time instead of \ac{CPU} time, both for ConstraintGM
and for our solution, and plotted the results in \cref{fig:eval.cgm.perf}.
Observing the results, we can see the disparity between the approach reliant on
Maxima when compared to both the \ac{GFL} and our solution, which was to be
expected.

\begin{figure}[htb]
  \begin{tikzpicture}
  \begin{semilogyaxis}[ybar=0pt,
    title={Maxima vs.\ GFL vs.\ Our solution},
    xlabel={Scenarios},
    ylabel={Average Time (ns, log)},
    width={\linewidth},
    height=5.5cm,
    bar width=1/3,
    xmin=1,xmax=13,
    xtick distance=1,
    ytick distance=1e1,
    enlarge y limits={.26,upper},
    enlarge x limits={abs=.5},
    legend columns=-1,
    table/col sep=comma
  ]
    \addplot+ table {data/cgm-maxima.csv};
    \addplot+ table {data/cgm-gfl.csv};
    \addplot+ [color=green,draw=green!60!black] table {data/jlcgal.csv};
    \legend{Maxima,GFL,Our solution}
  \end{semilogyaxis}
  \end{tikzpicture}
  \caption{\label{fig:eval.cgm.perf}%
    ConstraintGM benchmark results in a Y-bar plot.}
\end{figure}

Furthermore, we can see our solution outdoes ConstraintGM's \ac{GFL}.  This is
most likely due to the fact that we are relying on \ac{CGAL}, a library
implemented in C++.  The latter is notoriously known for being a
high-performance language, considerably outperforming Racket in a series of
benchmarks.  Nevertheless, despite some overhead in the case of Julia, the
results are still positive.

In conclusion, our solution proves capable and performant, having surpassed
ConstraintGM's \ac{GFL} by an entire order of magnitude on average.

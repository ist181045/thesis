% !TeX root = ../../main.tex
\section{Voronoi Diagrams Extended}%
\label{sec:eval.voronoi}

In the previous section, we left the problem of Voronoi Diagrams partially
unresolved, only tackling a small sub-problem required for computing the
diagram's vertices.  This section expands on it by showing how we can repurpose
\ac{CGAL}'s version of the Voronoi Diagram algorithm~\cite{CGAL:5.3:VDA2} as a
beneficial side effect of integrating with the library.  Additionally, we
compare said version with a native Julia implementation of the algorithm
described in~\cite{Springel:2010:GCHSMM} available in the
\texttt{VoronoiDelaunay.jl}\footnote{\url{https://github.com/JuliaGeometry/VoronoiDelaunay.jl}}
package.  Measurements involved obtaining an estimate of the effort required to
obtain either implementation, measuring Delaunay Triangulation construction
performance, and subsequent output triangulation and diagram comparison.

Both implementations adopt a similar incremental approach and are both robust.
The version from \texttt{VoronoiDelaunay.jl}, however, uses floating point
filtering~\cite{Springel:2010:GCHSMM}, requiring all point coordinates to be in
the interval $\left(1, 2\right) \times \left(1, 2\right) \subset \mathbb{R}^2$;
a restriction \ac{CGAL}'s implementation does not have since it uses a dynamic
floating point precision approach\footnote{Briefed at
\url{https://www.cs.cmu.edu/~quake/robust.html}}~\cite{Shewchuk:1997:APFPAFRGP}.
The \textit{limitation} in the former arguably constitutes a minor inconvenience
since the diagram may be produced in that limited range and then scaled
afterwards to meet the use case's needs.  But it is worth noting the issue is
not present in \ac{CGAL}'s implementation.  This is relevant if the input
coordinates cannot be altered to meet those needs.

To illustrate how we obtained the 2D Delaunay Triangulation and Voronoi Diagram
algorithms from \ac{CGAL}, \cref{lst:appendix.voronoi.jlcxx} shows a minimal
working example of the wrapper library code supported by JlCxx required to make
the necessary constructs and functionality available in Julia.  It follows a
familiar approach, the same used to obtain the core constructs supporting our
\primitives{} solution.  \Cref{lst:eval.voronoi.jl} shows a bare-bones Julia
module encapsulating and exposing the mapped functionality to the Julia
language.
\begin{listing}[htb]
  \inputminted{julia}{jl/Voronoi.jl}
  \caption[Bare-bones Julia module wrapping CGAL's Delaunay algorithms]{
    Bare-bones Julia module wrapping \ac{CGAL}'s 2D Delaunay Triangulation and
    Voronoi Diagrams, supported by the JlCxx wrapper in
    \cref{lst:appendix.voronoi.jlcxx}.}%
  \label{lst:eval.voronoi.jl}
\end{listing}
\texttt{CGAL.jl} contains a version of these mappings that are, however,
parameterized, and adds methods for querying, insertion, and further
manipulation.  For the purposes of testing, we used the slimmer \texttt{Voronoi}
module (\cref{lst:eval.voronoi.jl}).

The process for obtaining these code bindings is a relatively simple one that
requires bare minimal C++ knowledge and following reference documentation as if
it were a recipe book, looking up the necessary ingredients and adding them to
the mix.  At most, accounting for trial and error in case a minute detail was
overlooked, to make this functionality available, it would take no more than a
full day.  Having that, the algorithms are ready and available for use.

The algorithm present in \texttt{VoronoiDelaunay.jl} is the result of an immense
body of research, detailed in the \textsc{Arepo}
paper~\cite{Springel:2010:GCHSMM}.  Julia is an expressive language that greatly
benefits prototyping.  Changes are easy to apply and test without the overhead
of compilation, a downside of prototyping in C++, especially when compiling
(against) a sizeable codebase.  We can only infer that the development of the
algorithm did not take long to draft.  However, we can certainly say that it
took more than a full day to obtain a robust implementation, far exceeding the
effort taken by just repurposing an existing robust implementation in \ac{CGAL}.
It required interpretation and understanding of the approach described in the
article since there is no explicit algorithm listed.  This then requires the
creation of a conceptual mapping of the algorithm's entities and procedures,
later translating them into corresponding, preferably efficient, data structures
and functions in the Julia language, which, in fairness, is an easy platform to
prototype in.  These are among other trials and tribulations that we did not
have to meet in the slightest.  Our estimate is that developing the algorithm in
\texttt{VoronoiDelaunay.jl} took far longer than obtaining one via our approach,
the latter concisely, generously, fitting into the span of a full day at most.

Regarding both algorithms' performance, \cref{tab:eval.voronoi.bench} shows a
series of results obtained by batch inserting various powers-of-ten sets of
points into Delaunay Triangulation objects, effectively building the meshes.
The results are plotted in \cref{fig:eval.voronoi.bench} for a better
understanding of the tabulated results.  The source code used to perform these
tests is listed in \cref{lst:appendix.voronoi.benchjl}.  More detailed data can
be seen in \cref{lst:appendix.voronoi.data}.

\begin{table}[htb]
  \caption[Delaunay Triangulation benchmarks]{
    Execution times for the construction of Delaunay Triangulations, comparing
    implementations of the algorithm: one from \ac{CGAL} and one from
    \texttt{VoronoiDelaunay.jl}.}%
  \label{tab:eval.voronoi.bench}
  \centering
  \begin{tabular}{r*{2}{l}}
    \toprule
    \multirow{2}{*}{\textbf{\# Points}}
    & \multicolumn{2}{c}{\textbf{Execution time (mean ± σ)}} \\
    & \multicolumn{1}{c}{\texttt{Voronoi.jl}}
    & \multicolumn{1}{c}{\texttt{VoronoiDelaunay.jl}} \\
    \midrule
    10\textsuperscript{2} & \verb| 42.145 μs ±  5.638 μs|
                          & \verb| 53.799 μs ±  39.811 μs| \\
    10\textsuperscript{3} & \verb|509.757 μs ±  1.258 ms|
                          & \verb|510.851 μs ± 113.523 μs| \\
    10\textsuperscript{4} & \verb|  5.879 ms ±  3.874 ms|
                          & \verb|  5.606 ms ± 382.603 μs| \\
    10\textsuperscript{5} & \verb| 62.344 ms ± 11.980 ms|
                          & \verb| 65.103 ms ±   2.287 ms| \\
    10\textsuperscript{6} & \verb|668.331 ms ± 44.458 ms|
                          & \verb|841.925 ms ±  51.394 ms| \\
    \bottomrule
  \end{tabular}
\end{table}

\begin{figure}[htb]
  \centering
  \begin{tikzpicture}
  \begin{loglogaxis}[ybar=0pt,
    title={\texttt{Voronoi.jl} vs.\ \texttt{VoronoiDelaunay.jl}},
    xlabel={Number of Points (log)},
    ylabel={Average Time (ns, log)},
    width={\linewidth},
    height=6cm,
    bar width=3.16227766,% 10^(1/n) where n = 2 (bars)
    enlarge y limits={.2,upper},
    enlarge x limits={abs=3.16227766},
    legend columns=-1,
    table/col sep=comma,
    error bars/y explicit,
    error bars/y dir=both
  ]
    \addplot+ table [y error index=2] {data/voronoi-vdjl.csv};
    \addplot+ table [y error index=2] {data/voronoi-cgal.csv};
    \legend{VoronoiDelaunay.jl,Voronoi.jl}
  \end{loglogaxis}
  \end{tikzpicture}
  \caption[Delaunay Triangulation benchmarks]{
    Delaunay Triangulation benchmark comparison results collected form
    \cref{tab:eval.voronoi.bench} in a Y-bar plot.  Both axes follow a
    logarithmic scale.}%
  \label{fig:eval.voronoi.bench}
\end{figure}

The sets of points were randomly generated using an instance of the Mersenne
Twister pseudo-random number generator, MT19937, with the seed
\texttt{0xdeadbeef}.  Tests for 10\textsuperscript{n} points used the same
10\textsuperscript{n} random points for every sample evaluated.  Results are
pretty identical, both variants trading blows.  However, more often than not, we
see \ac{CGAL}'s variant of the algorithm beating \texttt{VoronoiDelaunay.jl}'s
by a relatively small margin, except the last result's difference is more than
marginal.  We see some more variation from \ac{CGAL}'s algorithm, but looking at
more detailed data (\cref{lst:appendix.voronoi.data}), such cases seem to be
punctual at best, proving negligible in practice.

One other interesting detail we can see represented in additional data is that
of estimated memory usage.  The algorithm in \texttt{VoronoiDelaunay.jl} appears
to use far more memory than \ac{CGAL}'s does, differing by two entire orders of
magnitude.  We did not investigate any further, though additional profiling
could help bring light to some of these results.

Finally, we take a look at the output Delaunay Triangulations and their
respective dual Voronoi Diagrams, produced by both implementations.
\Cref{fig:eval.voronoi.output} illustrates two plots of overlapping meshes: on
the left, the Delaunay Triangulations, and on the right, the respective dual
Voronoi Diagrams.  \Cref{lst:appendix.voronoi.plotsjl} shows the Julia code used
to produce the plots in the figures.

\begin{figure}[!htb]
  \resizebox{\linewidth}{!}{\begin{tikzpicture}[/tikz/background rectangle/.style={fill={rgb,1:red,1.0;green,1.0;blue,1.0}, draw opacity={1.0}}, show background rectangle]
\begin{axis}[point meta max={nan}, point meta min={nan}, title={Delaunay Triangulation}, title style={at={{(0.5,1)}}, anchor={south}, font={{\fontsize{10 pt}{13.0 pt}\selectfont}}, color={rgb,1:red,0.0;green,0.0;blue,0.0}, draw opacity={1.0}, rotate={0.0}}, legend style={color={rgb,1:red,0.0;green,0.0;blue,0.0}, draw opacity={1.0}, line width={1}, solid, fill={rgb,1:red,1.0;green,1.0;blue,1.0}, fill opacity={1.0}, text opacity={1.0}, font={{\fontsize{8 pt}{10.4 pt}\selectfont}}, text={rgb,1:red,0.0;green,0.0;blue,0.0}, cells={anchor={west}}, at={(0.98, 0.98)}, anchor={north east}}, axis background/.style={fill={rgb,1:red,1.0;green,1.0;blue,1.0}, opacity={1.0}}, anchor={north west}, xshift={1.0mm}, yshift={-1.0mm}, width={86.9mm}, height={86.9mm}, scaled x ticks={false}, xlabel={$x$}, x tick style={color={rgb,1:red,0.0;green,0.0;blue,0.0}, opacity={1.0}}, x tick label style={color={rgb,1:red,0.0;green,0.0;blue,0.0}, opacity={1.0}, rotate={0}}, xlabel style={at={(ticklabel cs:0.5)}, anchor=near ticklabel, font={{\fontsize{11 pt}{14.3 pt}\selectfont}}, color={rgb,1:red,0.0;green,0.0;blue,0.0}, draw opacity={1.0}, rotate={0.0}}, xmajorgrids={true}, xmin={0.9}, xmax={2.1}, xtick={{1.0,1.25,1.5,1.75,2.0}}, xticklabels={{$1.00$,$1.25$,$1.50$,$1.75$,$2.00$}}, xtick align={inside}, xticklabel style={font={{\fontsize{8 pt}{10.4 pt}\selectfont}}, color={rgb,1:red,0.0;green,0.0;blue,0.0}, draw opacity={1.0}, rotate={0.0}}, x grid style={color={rgb,1:red,0.0;green,0.0;blue,0.0}, draw opacity={0.1}, line width={0.5}, solid}, axis x line*={left}, x axis line style={color={rgb,1:red,0.0;green,0.0;blue,0.0}, draw opacity={1.0}, line width={1}, solid}, scaled y ticks={false}, ylabel={$y$}, y tick style={color={rgb,1:red,0.0;green,0.0;blue,0.0}, opacity={1.0}}, y tick label style={color={rgb,1:red,0.0;green,0.0;blue,0.0}, opacity={1.0}, rotate={0}}, ylabel style={at={(ticklabel cs:0.5)}, anchor=near ticklabel, font={{\fontsize{11 pt}{14.3 pt}\selectfont}}, color={rgb,1:red,0.0;green,0.0;blue,0.0}, draw opacity={1.0}, rotate={0.0}}, ymajorgrids={true}, ymin={0.9}, ymax={2.1}, ytick={{1.0,1.25,1.5,1.75,2.0}}, yticklabels={{$1.00$,$1.25$,$1.50$,$1.75$,$2.00$}}, ytick align={inside}, yticklabel style={font={{\fontsize{8 pt}{10.4 pt}\selectfont}}, color={rgb,1:red,0.0;green,0.0;blue,0.0}, draw opacity={1.0}, rotate={0.0}}, y grid style={color={rgb,1:red,0.0;green,0.0;blue,0.0}, draw opacity={0.1}, line width={0.5}, solid}, axis y line*={left}, y axis line style={color={rgb,1:red,0.0;green,0.0;blue,0.0}, draw opacity={1.0}, line width={1}, solid}, colorbar={false}, legend columns={-1}]
    \addplot[color={rgb,1:red,1.0;green,0.0;blue,0.0}, name path={feac98d9-325b-42e4-8488-6ffcf40c5241}, draw opacity={1.0}, line width={1}, solid]
        table[row sep={\\}]
        {
            \\
            1.1748122130520642  1.9119394752196963  \\
            1.3983968361280859  1.8132919843774284  \\
        }
        ;
    \addlegendentry {Voronoi.jl}
    \addplot[color={rgb,1:red,1.0;green,0.0;blue,0.0}, name path={feac98d9-325b-42e4-8488-6ffcf40c5241}, draw opacity={1.0}, line width={1}, solid, forget plot]
        table[row sep={\\}]
        {
            \\
            1.3983968361280859  1.8132919843774284  \\
            1.3702157551888376  1.9199406439438458  \\
        }
        ;
    \addplot[color={rgb,1:red,1.0;green,0.0;blue,0.0}, name path={feac98d9-325b-42e4-8488-6ffcf40c5241}, draw opacity={1.0}, line width={1}, solid, forget plot]
        table[row sep={\\}]
        {
            \\
            1.3702157551888376  1.9199406439438458  \\
            1.1748122130520642  1.9119394752196963  \\
        }
        ;
    \addplot[color={rgb,1:red,1.0;green,0.0;blue,0.0}, name path={feac98d9-325b-42e4-8488-6ffcf40c5241}, draw opacity={1.0}, line width={1}, solid, forget plot]
        table[row sep={\\}]
        {
            \\
            1.3983968361280859  1.8132919843774284  \\
            1.411079611396417  1.603287132689729  \\
        }
        ;
    \addplot[color={rgb,1:red,1.0;green,0.0;blue,0.0}, name path={feac98d9-325b-42e4-8488-6ffcf40c5241}, draw opacity={1.0}, line width={1}, solid, forget plot]
        table[row sep={\\}]
        {
            \\
            1.411079611396417  1.603287132689729  \\
            1.4094003408681601  1.8230532801244403  \\
        }
        ;
    \addplot[color={rgb,1:red,1.0;green,0.0;blue,0.0}, name path={feac98d9-325b-42e4-8488-6ffcf40c5241}, draw opacity={1.0}, line width={1}, solid, forget plot]
        table[row sep={\\}]
        {
            \\
            1.4094003408681601  1.8230532801244403  \\
            1.3983968361280859  1.8132919843774284  \\
        }
        ;
    \addplot[color={rgb,1:red,1.0;green,0.0;blue,0.0}, name path={feac98d9-325b-42e4-8488-6ffcf40c5241}, draw opacity={1.0}, line width={1}, solid, forget plot]
        table[row sep={\\}]
        {
            \\
            1.8946103360503908  1.4291503357235342  \\
            1.8488018650095905  1.693958807038143  \\
        }
        ;
    \addplot[color={rgb,1:red,1.0;green,0.0;blue,0.0}, name path={feac98d9-325b-42e4-8488-6ffcf40c5241}, draw opacity={1.0}, line width={1}, solid, forget plot]
        table[row sep={\\}]
        {
            \\
            1.8488018650095905  1.693958807038143  \\
            1.8273643609136339  1.502810534555465  \\
        }
        ;
    \addplot[color={rgb,1:red,1.0;green,0.0;blue,0.0}, name path={feac98d9-325b-42e4-8488-6ffcf40c5241}, draw opacity={1.0}, line width={1}, solid, forget plot]
        table[row sep={\\}]
        {
            \\
            1.8273643609136339  1.502810534555465  \\
            1.8946103360503908  1.4291503357235342  \\
        }
        ;
    \addplot[color={rgb,1:red,1.0;green,0.0;blue,0.0}, name path={feac98d9-325b-42e4-8488-6ffcf40c5241}, draw opacity={1.0}, line width={1}, solid, forget plot]
        table[row sep={\\}]
        {
            \\
            1.670456608990207  1.4213281597476453  \\
            1.8273643609136339  1.502810534555465  \\
        }
        ;
    \addplot[color={rgb,1:red,1.0;green,0.0;blue,0.0}, name path={feac98d9-325b-42e4-8488-6ffcf40c5241}, draw opacity={1.0}, line width={1}, solid, forget plot]
        table[row sep={\\}]
        {
            \\
            1.8273643609136339  1.502810534555465  \\
            1.7508308666292574  1.5433638719841838  \\
        }
        ;
    \addplot[color={rgb,1:red,1.0;green,0.0;blue,0.0}, name path={feac98d9-325b-42e4-8488-6ffcf40c5241}, draw opacity={1.0}, line width={1}, solid, forget plot]
        table[row sep={\\}]
        {
            \\
            1.7508308666292574  1.5433638719841838  \\
            1.670456608990207  1.4213281597476453  \\
        }
        ;
    \addplot[color={rgb,1:red,1.0;green,0.0;blue,0.0}, name path={feac98d9-325b-42e4-8488-6ffcf40c5241}, draw opacity={1.0}, line width={1}, solid, forget plot]
        table[row sep={\\}]
        {
            \\
            1.4094003408681601  1.8230532801244403  \\
            1.8488018650095905  1.693958807038143  \\
        }
        ;
    \addplot[color={rgb,1:red,1.0;green,0.0;blue,0.0}, name path={feac98d9-325b-42e4-8488-6ffcf40c5241}, draw opacity={1.0}, line width={1}, solid, forget plot]
        table[row sep={\\}]
        {
            \\
            1.8488018650095905  1.693958807038143  \\
            1.3702157551888376  1.9199406439438458  \\
        }
        ;
    \addplot[color={rgb,1:red,1.0;green,0.0;blue,0.0}, name path={feac98d9-325b-42e4-8488-6ffcf40c5241}, draw opacity={1.0}, line width={1}, solid, forget plot]
        table[row sep={\\}]
        {
            \\
            1.3702157551888376  1.9199406439438458  \\
            1.4094003408681601  1.8230532801244403  \\
        }
        ;
    \addplot[color={rgb,1:red,1.0;green,0.0;blue,0.0}, name path={feac98d9-325b-42e4-8488-6ffcf40c5241}, draw opacity={1.0}, line width={1}, solid, forget plot]
        table[row sep={\\}]
        {
            \\
            1.7508308666292574  1.5433638719841838  \\
            1.8488018650095905  1.693958807038143  \\
        }
        ;
    \addplot[color={rgb,1:red,1.0;green,0.0;blue,0.0}, name path={feac98d9-325b-42e4-8488-6ffcf40c5241}, draw opacity={1.0}, line width={1}, solid, forget plot]
        table[row sep={\\}]
        {
            \\
            1.4094003408681601  1.8230532801244403  \\
            1.7508308666292574  1.5433638719841838  \\
        }
        ;
    \addplot[color={rgb,1:red,1.0;green,0.0;blue,0.0}, name path={feac98d9-325b-42e4-8488-6ffcf40c5241}, draw opacity={1.0}, line width={1}, solid, forget plot]
        table[row sep={\\}]
        {
            \\
            1.6398598169907925  1.1640638990793377  \\
            1.8701207209378476  1.1696756919845939  \\
        }
        ;
    \addplot[color={rgb,1:red,1.0;green,0.0;blue,0.0}, name path={feac98d9-325b-42e4-8488-6ffcf40c5241}, draw opacity={1.0}, line width={1}, solid, forget plot]
        table[row sep={\\}]
        {
            \\
            1.8701207209378476  1.1696756919845939  \\
            1.670456608990207  1.4213281597476453  \\
        }
        ;
    \addplot[color={rgb,1:red,1.0;green,0.0;blue,0.0}, name path={feac98d9-325b-42e4-8488-6ffcf40c5241}, draw opacity={1.0}, line width={1}, solid, forget plot]
        table[row sep={\\}]
        {
            \\
            1.670456608990207  1.4213281597476453  \\
            1.6398598169907925  1.1640638990793377  \\
        }
        ;
    \addplot[color={rgb,1:red,1.0;green,0.0;blue,0.0}, name path={feac98d9-325b-42e4-8488-6ffcf40c5241}, draw opacity={1.0}, line width={1}, solid, forget plot]
        table[row sep={\\}]
        {
            \\
            1.1105229298118504  1.3180863440502435  \\
            1.5939507135190067  1.132305847713724  \\
        }
        ;
    \addplot[color={rgb,1:red,1.0;green,0.0;blue,0.0}, name path={feac98d9-325b-42e4-8488-6ffcf40c5241}, draw opacity={1.0}, line width={1}, solid, forget plot]
        table[row sep={\\}]
        {
            \\
            1.5939507135190067  1.132305847713724  \\
            1.3071074625477195  1.3007149568293244  \\
        }
        ;
    \addplot[color={rgb,1:red,1.0;green,0.0;blue,0.0}, name path={feac98d9-325b-42e4-8488-6ffcf40c5241}, draw opacity={1.0}, line width={1}, solid, forget plot]
        table[row sep={\\}]
        {
            \\
            1.3071074625477195  1.3007149568293244  \\
            1.1105229298118504  1.3180863440502435  \\
        }
        ;
    \addplot[color={rgb,1:red,1.0;green,0.0;blue,0.0}, name path={feac98d9-325b-42e4-8488-6ffcf40c5241}, draw opacity={1.0}, line width={1}, solid, forget plot]
        table[row sep={\\}]
        {
            \\
            1.3230645160656422  1.6671039194334296  \\
            1.1387269033584746  1.3902763912919909  \\
        }
        ;
    \addplot[color={rgb,1:red,1.0;green,0.0;blue,0.0}, name path={feac98d9-325b-42e4-8488-6ffcf40c5241}, draw opacity={1.0}, line width={1}, solid, forget plot]
        table[row sep={\\}]
        {
            \\
            1.1387269033584746  1.3902763912919909  \\
            1.4044450228102505  1.4121578056365252  \\
        }
        ;
    \addplot[color={rgb,1:red,1.0;green,0.0;blue,0.0}, name path={feac98d9-325b-42e4-8488-6ffcf40c5241}, draw opacity={1.0}, line width={1}, solid, forget plot]
        table[row sep={\\}]
        {
            \\
            1.4044450228102505  1.4121578056365252  \\
            1.3230645160656422  1.6671039194334296  \\
        }
        ;
    \addplot[color={rgb,1:red,1.0;green,0.0;blue,0.0}, name path={feac98d9-325b-42e4-8488-6ffcf40c5241}, draw opacity={1.0}, line width={1}, solid, forget plot]
        table[row sep={\\}]
        {
            \\
            1.5201537685934454  1.5121217011474073  \\
            1.4094003408681601  1.8230532801244403  \\
        }
        ;
    \addplot[color={rgb,1:red,1.0;green,0.0;blue,0.0}, name path={feac98d9-325b-42e4-8488-6ffcf40c5241}, draw opacity={1.0}, line width={1}, solid, forget plot]
        table[row sep={\\}]
        {
            \\
            1.411079611396417  1.603287132689729  \\
            1.5201537685934454  1.5121217011474073  \\
        }
        ;
    \addplot[color={rgb,1:red,1.0;green,0.0;blue,0.0}, name path={feac98d9-325b-42e4-8488-6ffcf40c5241}, draw opacity={1.0}, line width={1}, solid, forget plot]
        table[row sep={\\}]
        {
            \\
            1.5939507135190067  1.132305847713724  \\
            1.4044450228102505  1.4121578056365252  \\
        }
        ;
    \addplot[color={rgb,1:red,1.0;green,0.0;blue,0.0}, name path={feac98d9-325b-42e4-8488-6ffcf40c5241}, draw opacity={1.0}, line width={1}, solid, forget plot]
        table[row sep={\\}]
        {
            \\
            1.4044450228102505  1.4121578056365252  \\
            1.3071074625477195  1.3007149568293244  \\
        }
        ;
    \addplot[color={rgb,1:red,1.0;green,0.0;blue,0.0}, name path={feac98d9-325b-42e4-8488-6ffcf40c5241}, draw opacity={1.0}, line width={1}, solid, forget plot]
        table[row sep={\\}]
        {
            \\
            1.1387269033584746  1.3902763912919909  \\
            1.3071074625477195  1.3007149568293244  \\
        }
        ;
    \addplot[color={rgb,1:red,1.0;green,0.0;blue,0.0}, name path={feac98d9-325b-42e4-8488-6ffcf40c5241}, draw opacity={1.0}, line width={1}, solid, forget plot]
        table[row sep={\\}]
        {
            \\
            1.1387269033584746  1.3902763912919909  \\
            1.125095868483186  1.7752906912937756  \\
        }
        ;
    \addplot[color={rgb,1:red,1.0;green,0.0;blue,0.0}, name path={feac98d9-325b-42e4-8488-6ffcf40c5241}, draw opacity={1.0}, line width={1}, solid, forget plot]
        table[row sep={\\}]
        {
            \\
            1.125095868483186  1.7752906912937756  \\
            1.1105229298118504  1.3180863440502435  \\
        }
        ;
    \addplot[color={rgb,1:red,1.0;green,0.0;blue,0.0}, name path={feac98d9-325b-42e4-8488-6ffcf40c5241}, draw opacity={1.0}, line width={1}, solid, forget plot]
        table[row sep={\\}]
        {
            \\
            1.1105229298118504  1.3180863440502435  \\
            1.1387269033584746  1.3902763912919909  \\
        }
        ;
    \addplot[color={rgb,1:red,1.0;green,0.0;blue,0.0}, name path={feac98d9-325b-42e4-8488-6ffcf40c5241}, draw opacity={1.0}, line width={1}, solid, forget plot]
        table[row sep={\\}]
        {
            \\
            1.3230645160656422  1.6671039194334296  \\
            1.125095868483186  1.7752906912937756  \\
        }
        ;
    \addplot[color={rgb,1:red,1.0;green,0.0;blue,0.0}, name path={feac98d9-325b-42e4-8488-6ffcf40c5241}, draw opacity={1.0}, line width={1}, solid, forget plot]
        table[row sep={\\}]
        {
            \\
            1.3983968361280859  1.8132919843774284  \\
            1.125095868483186  1.7752906912937756  \\
        }
        ;
    \addplot[color={rgb,1:red,1.0;green,0.0;blue,0.0}, name path={feac98d9-325b-42e4-8488-6ffcf40c5241}, draw opacity={1.0}, line width={1}, solid, forget plot]
        table[row sep={\\}]
        {
            \\
            1.3230645160656422  1.6671039194334296  \\
            1.3983968361280859  1.8132919843774284  \\
        }
        ;
    \addplot[color={rgb,1:red,1.0;green,0.0;blue,0.0}, name path={feac98d9-325b-42e4-8488-6ffcf40c5241}, draw opacity={1.0}, line width={1}, solid, forget plot]
        table[row sep={\\}]
        {
            \\
            1.4235251017380506  1.5806863505858928  \\
            1.5201537685934454  1.5121217011474073  \\
        }
        ;
    \addplot[color={rgb,1:red,1.0;green,0.0;blue,0.0}, name path={feac98d9-325b-42e4-8488-6ffcf40c5241}, draw opacity={1.0}, line width={1}, solid, forget plot]
        table[row sep={\\}]
        {
            \\
            1.411079611396417  1.603287132689729  \\
            1.4235251017380506  1.5806863505858928  \\
        }
        ;
    \addplot[color={rgb,1:red,1.0;green,0.0;blue,0.0}, name path={feac98d9-325b-42e4-8488-6ffcf40c5241}, draw opacity={1.0}, line width={1}, solid, forget plot]
        table[row sep={\\}]
        {
            \\
            1.125095868483186  1.7752906912937756  \\
            1.1748122130520642  1.9119394752196963  \\
        }
        ;
    \addplot[color={rgb,1:red,1.0;green,0.0;blue,0.0}, name path={feac98d9-325b-42e4-8488-6ffcf40c5241}, draw opacity={1.0}, line width={1}, solid, forget plot]
        table[row sep={\\}]
        {
            \\
            1.4235251017380506  1.5806863505858928  \\
            1.3230645160656422  1.6671039194334296  \\
        }
        ;
    \addplot[color={rgb,1:red,1.0;green,0.0;blue,0.0}, name path={feac98d9-325b-42e4-8488-6ffcf40c5241}, draw opacity={1.0}, line width={1}, solid, forget plot]
        table[row sep={\\}]
        {
            \\
            1.4044450228102505  1.4121578056365252  \\
            1.4235251017380506  1.5806863505858928  \\
        }
        ;
    \addplot[color={rgb,1:red,1.0;green,0.0;blue,0.0}, name path={feac98d9-325b-42e4-8488-6ffcf40c5241}, draw opacity={1.0}, line width={1}, solid, forget plot]
        table[row sep={\\}]
        {
            \\
            1.5201537685934454  1.5121217011474073  \\
            1.7508308666292574  1.5433638719841838  \\
        }
        ;
    \addplot[color={rgb,1:red,1.0;green,0.0;blue,0.0}, name path={feac98d9-325b-42e4-8488-6ffcf40c5241}, draw opacity={1.0}, line width={1}, solid, forget plot]
        table[row sep={\\}]
        {
            \\
            1.4044450228102505  1.4121578056365252  \\
            1.5201537685934454  1.5121217011474073  \\
        }
        ;
    \addplot[color={rgb,1:red,1.0;green,0.0;blue,0.0}, name path={feac98d9-325b-42e4-8488-6ffcf40c5241}, draw opacity={1.0}, line width={1}, solid, forget plot]
        table[row sep={\\}]
        {
            \\
            1.670456608990207  1.4213281597476453  \\
            1.4044450228102505  1.4121578056365252  \\
        }
        ;
    \addplot[color={rgb,1:red,1.0;green,0.0;blue,0.0}, name path={feac98d9-325b-42e4-8488-6ffcf40c5241}, draw opacity={1.0}, line width={1}, solid, forget plot]
        table[row sep={\\}]
        {
            \\
            1.4044450228102505  1.4121578056365252  \\
            1.6398598169907925  1.1640638990793377  \\
        }
        ;
    \addplot[color={rgb,1:red,1.0;green,0.0;blue,0.0}, name path={feac98d9-325b-42e4-8488-6ffcf40c5241}, draw opacity={1.0}, line width={1}, solid, forget plot]
        table[row sep={\\}]
        {
            \\
            1.3230645160656422  1.6671039194334296  \\
            1.411079611396417  1.603287132689729  \\
        }
        ;
    \addplot[color={rgb,1:red,1.0;green,0.0;blue,0.0}, name path={feac98d9-325b-42e4-8488-6ffcf40c5241}, draw opacity={1.0}, line width={1}, solid, forget plot]
        table[row sep={\\}]
        {
            \\
            1.670456608990207  1.4213281597476453  \\
            1.8946103360503908  1.4291503357235342  \\
        }
        ;
    \addplot[color={rgb,1:red,1.0;green,0.0;blue,0.0}, name path={feac98d9-325b-42e4-8488-6ffcf40c5241}, draw opacity={1.0}, line width={1}, solid, forget plot]
        table[row sep={\\}]
        {
            \\
            1.8946103360503908  1.4291503357235342  \\
            1.8701207209378476  1.1696756919845939  \\
        }
        ;
    \addplot[color={rgb,1:red,1.0;green,0.0;blue,0.0}, name path={feac98d9-325b-42e4-8488-6ffcf40c5241}, draw opacity={1.0}, line width={1}, solid, forget plot]
        table[row sep={\\}]
        {
            \\
            1.5201537685934454  1.5121217011474073  \\
            1.670456608990207  1.4213281597476453  \\
        }
        ;
    \addplot[color={rgb,1:red,1.0;green,0.0;blue,0.0}, name path={feac98d9-325b-42e4-8488-6ffcf40c5241}, draw opacity={1.0}, line width={1}, solid, forget plot]
        table[row sep={\\}]
        {
            \\
            1.5939507135190067  1.132305847713724  \\
            1.8701207209378476  1.1696756919845939  \\
        }
        ;
    \addplot[color={rgb,1:red,1.0;green,0.0;blue,0.0}, name path={feac98d9-325b-42e4-8488-6ffcf40c5241}, draw opacity={1.0}, line width={1}, solid, forget plot]
        table[row sep={\\}]
        {
            \\
            1.6398598169907925  1.1640638990793377  \\
            1.5939507135190067  1.132305847713724  \\
        }
        ;
    \addplot[color={rgb,1:red,0.0;green,1.0;blue,0.498}, name path={36c44fbf-fb68-45d6-8a13-8daff49c720f}, draw opacity={0.8}, line width={1}, solid]
        table[row sep={\\}]
        {
            \\
            1.670456608990207  1.4213281597476453  \\
            1.7508308666292574  1.5433638719841838  \\
        }
        ;
    \addlegendentry {VoronoiDelaunay.jl}
    \addplot[color={rgb,1:red,0.0;green,1.0;blue,0.498}, name path={36c44fbf-fb68-45d6-8a13-8daff49c720f}, draw opacity={0.8}, line width={1}, solid, forget plot]
        table[row sep={\\}]
        {
            \\
            1.8273643609136339  1.502810534555465  \\
            1.7508308666292574  1.5433638719841838  \\
        }
        ;
    \addplot[color={rgb,1:red,0.0;green,1.0;blue,0.498}, name path={36c44fbf-fb68-45d6-8a13-8daff49c720f}, draw opacity={0.8}, line width={1}, solid, forget plot]
        table[row sep={\\}]
        {
            \\
            1.8273643609136339  1.502810534555465  \\
            1.670456608990207  1.4213281597476453  \\
        }
        ;
    \addplot[color={rgb,1:red,0.0;green,1.0;blue,0.498}, name path={36c44fbf-fb68-45d6-8a13-8daff49c720f}, draw opacity={0.8}, line width={1}, solid, forget plot]
        table[row sep={\\}]
        {
            \\
            1.3702157551888376  1.9199406439438458  \\
            1.4094003408681601  1.8230532801244403  \\
        }
        ;
    \addplot[color={rgb,1:red,0.0;green,1.0;blue,0.498}, name path={36c44fbf-fb68-45d6-8a13-8daff49c720f}, draw opacity={0.8}, line width={1}, solid, forget plot]
        table[row sep={\\}]
        {
            \\
            1.3983968361280859  1.8132919843774284  \\
            1.4094003408681601  1.8230532801244403  \\
        }
        ;
    \addplot[color={rgb,1:red,0.0;green,1.0;blue,0.498}, name path={36c44fbf-fb68-45d6-8a13-8daff49c720f}, draw opacity={0.8}, line width={1}, solid, forget plot]
        table[row sep={\\}]
        {
            \\
            1.3983968361280859  1.8132919843774284  \\
            1.3702157551888376  1.9199406439438458  \\
        }
        ;
    \addplot[color={rgb,1:red,0.0;green,1.0;blue,0.498}, name path={36c44fbf-fb68-45d6-8a13-8daff49c720f}, draw opacity={0.8}, line width={1}, solid, forget plot]
        table[row sep={\\}]
        {
            \\
            1.7508308666292574  1.5433638719841838  \\
            1.5201537685934454  1.5121217011474073  \\
        }
        ;
    \addplot[color={rgb,1:red,0.0;green,1.0;blue,0.498}, name path={36c44fbf-fb68-45d6-8a13-8daff49c720f}, draw opacity={0.8}, line width={1}, solid, forget plot]
        table[row sep={\\}]
        {
            \\
            1.4094003408681601  1.8230532801244403  \\
            1.5201537685934454  1.5121217011474073  \\
        }
        ;
    \addplot[color={rgb,1:red,0.0;green,1.0;blue,0.498}, name path={36c44fbf-fb68-45d6-8a13-8daff49c720f}, draw opacity={0.8}, line width={1}, solid, forget plot]
        table[row sep={\\}]
        {
            \\
            1.4094003408681601  1.8230532801244403  \\
            1.7508308666292574  1.5433638719841838  \\
        }
        ;
    \addplot[color={rgb,1:red,0.0;green,1.0;blue,0.498}, name path={36c44fbf-fb68-45d6-8a13-8daff49c720f}, draw opacity={0.8}, line width={1}, solid, forget plot]
        table[row sep={\\}]
        {
            \\
            1.670456608990207  1.4213281597476453  \\
            1.5201537685934454  1.5121217011474073  \\
        }
        ;
    \addplot[color={rgb,1:red,0.0;green,1.0;blue,0.498}, name path={36c44fbf-fb68-45d6-8a13-8daff49c720f}, draw opacity={0.8}, line width={1}, solid, forget plot]
        table[row sep={\\}]
        {
            \\
            1.411079611396417  1.603287132689729  \\
            1.4094003408681601  1.8230532801244403  \\
        }
        ;
    \addplot[color={rgb,1:red,0.0;green,1.0;blue,0.498}, name path={36c44fbf-fb68-45d6-8a13-8daff49c720f}, draw opacity={0.8}, line width={1}, solid, forget plot]
        table[row sep={\\}]
        {
            \\
            1.5201537685934454  1.5121217011474073  \\
            1.411079611396417  1.603287132689729  \\
        }
        ;
    \addplot[color={rgb,1:red,0.0;green,1.0;blue,0.498}, name path={36c44fbf-fb68-45d6-8a13-8daff49c720f}, draw opacity={0.8}, line width={1}, solid, forget plot]
        table[row sep={\\}]
        {
            \\
            1.4235251017380506  1.5806863505858928  \\
            1.3230645160656422  1.6671039194334296  \\
        }
        ;
    \addplot[color={rgb,1:red,0.0;green,1.0;blue,0.498}, name path={36c44fbf-fb68-45d6-8a13-8daff49c720f}, draw opacity={0.8}, line width={1}, solid, forget plot]
        table[row sep={\\}]
        {
            \\
            1.411079611396417  1.603287132689729  \\
            1.3230645160656422  1.6671039194334296  \\
        }
        ;
    \addplot[color={rgb,1:red,0.0;green,1.0;blue,0.498}, name path={36c44fbf-fb68-45d6-8a13-8daff49c720f}, draw opacity={0.8}, line width={1}, solid, forget plot]
        table[row sep={\\}]
        {
            \\
            1.411079611396417  1.603287132689729  \\
            1.4235251017380506  1.5806863505858928  \\
        }
        ;
    \addplot[color={rgb,1:red,0.0;green,1.0;blue,0.498}, name path={36c44fbf-fb68-45d6-8a13-8daff49c720f}, draw opacity={0.8}, line width={1}, solid, forget plot]
        table[row sep={\\}]
        {
            \\
            1.4044450228102505  1.4121578056365252  \\
            1.670456608990207  1.4213281597476453  \\
        }
        ;
    \addplot[color={rgb,1:red,0.0;green,1.0;blue,0.498}, name path={36c44fbf-fb68-45d6-8a13-8daff49c720f}, draw opacity={0.8}, line width={1}, solid, forget plot]
        table[row sep={\\}]
        {
            \\
            1.6398598169907925  1.1640638990793377  \\
            1.670456608990207  1.4213281597476453  \\
        }
        ;
    \addplot[color={rgb,1:red,0.0;green,1.0;blue,0.498}, name path={36c44fbf-fb68-45d6-8a13-8daff49c720f}, draw opacity={0.8}, line width={1}, solid, forget plot]
        table[row sep={\\}]
        {
            \\
            1.6398598169907925  1.1640638990793377  \\
            1.4044450228102505  1.4121578056365252  \\
        }
        ;
    \addplot[color={rgb,1:red,0.0;green,1.0;blue,0.498}, name path={36c44fbf-fb68-45d6-8a13-8daff49c720f}, draw opacity={0.8}, line width={1}, solid, forget plot]
        table[row sep={\\}]
        {
            \\
            1.5939507135190067  1.132305847713724  \\
            1.4044450228102505  1.4121578056365252  \\
        }
        ;
    \addplot[color={rgb,1:red,0.0;green,1.0;blue,0.498}, name path={36c44fbf-fb68-45d6-8a13-8daff49c720f}, draw opacity={0.8}, line width={1}, solid, forget plot]
        table[row sep={\\}]
        {
            \\
            1.6398598169907925  1.1640638990793377  \\
            1.5939507135190067  1.132305847713724  \\
        }
        ;
    \addplot[color={rgb,1:red,0.0;green,1.0;blue,0.498}, name path={36c44fbf-fb68-45d6-8a13-8daff49c720f}, draw opacity={0.8}, line width={1}, solid, forget plot]
        table[row sep={\\}]
        {
            \\
            1.8273643609136339  1.502810534555465  \\
            1.8488018650095905  1.693958807038143  \\
        }
        ;
    \addplot[color={rgb,1:red,0.0;green,1.0;blue,0.498}, name path={36c44fbf-fb68-45d6-8a13-8daff49c720f}, draw opacity={0.8}, line width={1}, solid, forget plot]
        table[row sep={\\}]
        {
            \\
            1.8946103360503908  1.4291503357235342  \\
            1.8488018650095905  1.693958807038143  \\
        }
        ;
    \addplot[color={rgb,1:red,0.0;green,1.0;blue,0.498}, name path={36c44fbf-fb68-45d6-8a13-8daff49c720f}, draw opacity={0.8}, line width={1}, solid, forget plot]
        table[row sep={\\}]
        {
            \\
            1.8946103360503908  1.4291503357235342  \\
            1.8273643609136339  1.502810534555465  \\
        }
        ;
    \addplot[color={rgb,1:red,0.0;green,1.0;blue,0.498}, name path={36c44fbf-fb68-45d6-8a13-8daff49c720f}, draw opacity={0.8}, line width={1}, solid, forget plot]
        table[row sep={\\}]
        {
            \\
            1.125095868483186  1.7752906912937756  \\
            1.1748122130520642  1.9119394752196963  \\
        }
        ;
    \addplot[color={rgb,1:red,0.0;green,1.0;blue,0.498}, name path={36c44fbf-fb68-45d6-8a13-8daff49c720f}, draw opacity={0.8}, line width={1}, solid, forget plot]
        table[row sep={\\}]
        {
            \\
            1.3983968361280859  1.8132919843774284  \\
            1.1748122130520642  1.9119394752196963  \\
        }
        ;
    \addplot[color={rgb,1:red,0.0;green,1.0;blue,0.498}, name path={36c44fbf-fb68-45d6-8a13-8daff49c720f}, draw opacity={0.8}, line width={1}, solid, forget plot]
        table[row sep={\\}]
        {
            \\
            1.3983968361280859  1.8132919843774284  \\
            1.125095868483186  1.7752906912937756  \\
        }
        ;
    \addplot[color={rgb,1:red,0.0;green,1.0;blue,0.498}, name path={36c44fbf-fb68-45d6-8a13-8daff49c720f}, draw opacity={0.8}, line width={1}, solid, forget plot]
        table[row sep={\\}]
        {
            \\
            1.8488018650095905  1.693958807038143  \\
            1.4094003408681601  1.8230532801244403  \\
        }
        ;
    \addplot[color={rgb,1:red,0.0;green,1.0;blue,0.498}, name path={36c44fbf-fb68-45d6-8a13-8daff49c720f}, draw opacity={0.8}, line width={1}, solid, forget plot]
        table[row sep={\\}]
        {
            \\
            1.8488018650095905  1.693958807038143  \\
            1.7508308666292574  1.5433638719841838  \\
        }
        ;
    \addplot[color={rgb,1:red,0.0;green,1.0;blue,0.498}, name path={36c44fbf-fb68-45d6-8a13-8daff49c720f}, draw opacity={0.8}, line width={1}, solid, forget plot]
        table[row sep={\\}]
        {
            \\
            1.4044450228102505  1.4121578056365252  \\
            1.3230645160656422  1.6671039194334296  \\
        }
        ;
    \addplot[color={rgb,1:red,0.0;green,1.0;blue,0.498}, name path={36c44fbf-fb68-45d6-8a13-8daff49c720f}, draw opacity={0.8}, line width={1}, solid, forget plot]
        table[row sep={\\}]
        {
            \\
            1.4235251017380506  1.5806863505858928  \\
            1.4044450228102505  1.4121578056365252  \\
        }
        ;
    \addplot[color={rgb,1:red,0.0;green,1.0;blue,0.498}, name path={36c44fbf-fb68-45d6-8a13-8daff49c720f}, draw opacity={0.8}, line width={1}, solid, forget plot]
        table[row sep={\\}]
        {
            \\
            1.4235251017380506  1.5806863505858928  \\
            1.5201537685934454  1.5121217011474073  \\
        }
        ;
    \addplot[color={rgb,1:red,0.0;green,1.0;blue,0.498}, name path={36c44fbf-fb68-45d6-8a13-8daff49c720f}, draw opacity={0.8}, line width={1}, solid, forget plot]
        table[row sep={\\}]
        {
            \\
            1.1387269033584746  1.3902763912919909  \\
            1.3230645160656422  1.6671039194334296  \\
        }
        ;
    \addplot[color={rgb,1:red,0.0;green,1.0;blue,0.498}, name path={36c44fbf-fb68-45d6-8a13-8daff49c720f}, draw opacity={0.8}, line width={1}, solid, forget plot]
        table[row sep={\\}]
        {
            \\
            1.4044450228102505  1.4121578056365252  \\
            1.1387269033584746  1.3902763912919909  \\
        }
        ;
    \addplot[color={rgb,1:red,0.0;green,1.0;blue,0.498}, name path={36c44fbf-fb68-45d6-8a13-8daff49c720f}, draw opacity={0.8}, line width={1}, solid, forget plot]
        table[row sep={\\}]
        {
            \\
            1.3071074625477195  1.3007149568293244  \\
            1.4044450228102505  1.4121578056365252  \\
        }
        ;
    \addplot[color={rgb,1:red,0.0;green,1.0;blue,0.498}, name path={36c44fbf-fb68-45d6-8a13-8daff49c720f}, draw opacity={0.8}, line width={1}, solid, forget plot]
        table[row sep={\\}]
        {
            \\
            1.3071074625477195  1.3007149568293244  \\
            1.1387269033584746  1.3902763912919909  \\
        }
        ;
    \addplot[color={rgb,1:red,0.0;green,1.0;blue,0.498}, name path={36c44fbf-fb68-45d6-8a13-8daff49c720f}, draw opacity={0.8}, line width={1}, solid, forget plot]
        table[row sep={\\}]
        {
            \\
            1.5201537685934454  1.5121217011474073  \\
            1.4044450228102505  1.4121578056365252  \\
        }
        ;
    \addplot[color={rgb,1:red,0.0;green,1.0;blue,0.498}, name path={36c44fbf-fb68-45d6-8a13-8daff49c720f}, draw opacity={0.8}, line width={1}, solid, forget plot]
        table[row sep={\\}]
        {
            \\
            1.5939507135190067  1.132305847713724  \\
            1.3071074625477195  1.3007149568293244  \\
        }
        ;
    \addplot[color={rgb,1:red,0.0;green,1.0;blue,0.498}, name path={36c44fbf-fb68-45d6-8a13-8daff49c720f}, draw opacity={0.8}, line width={1}, solid, forget plot]
        table[row sep={\\}]
        {
            \\
            1.1748122130520642  1.9119394752196963  \\
            1.3702157551888376  1.9199406439438458  \\
        }
        ;
    \addplot[color={rgb,1:red,0.0;green,1.0;blue,0.498}, name path={36c44fbf-fb68-45d6-8a13-8daff49c720f}, draw opacity={0.8}, line width={1}, solid, forget plot]
        table[row sep={\\}]
        {
            \\
            1.411079611396417  1.603287132689729  \\
            1.3983968361280859  1.8132919843774284  \\
        }
        ;
    \addplot[color={rgb,1:red,0.0;green,1.0;blue,0.498}, name path={36c44fbf-fb68-45d6-8a13-8daff49c720f}, draw opacity={0.8}, line width={1}, solid, forget plot]
        table[row sep={\\}]
        {
            \\
            1.125095868483186  1.7752906912937756  \\
            1.3230645160656422  1.6671039194334296  \\
        }
        ;
    \addplot[color={rgb,1:red,0.0;green,1.0;blue,0.498}, name path={36c44fbf-fb68-45d6-8a13-8daff49c720f}, draw opacity={0.8}, line width={1}, solid, forget plot]
        table[row sep={\\}]
        {
            \\
            1.1387269033584746  1.3902763912919909  \\
            1.125095868483186  1.7752906912937756  \\
        }
        ;
    \addplot[color={rgb,1:red,0.0;green,1.0;blue,0.498}, name path={36c44fbf-fb68-45d6-8a13-8daff49c720f}, draw opacity={0.8}, line width={1}, solid, forget plot]
        table[row sep={\\}]
        {
            \\
            1.8946103360503908  1.4291503357235342  \\
            1.670456608990207  1.4213281597476453  \\
        }
        ;
    \addplot[color={rgb,1:red,0.0;green,1.0;blue,0.498}, name path={36c44fbf-fb68-45d6-8a13-8daff49c720f}, draw opacity={0.8}, line width={1}, solid, forget plot]
        table[row sep={\\}]
        {
            \\
            1.8701207209378476  1.1696756919845939  \\
            1.8946103360503908  1.4291503357235342  \\
        }
        ;
    \addplot[color={rgb,1:red,0.0;green,1.0;blue,0.498}, name path={36c44fbf-fb68-45d6-8a13-8daff49c720f}, draw opacity={0.8}, line width={1}, solid, forget plot]
        table[row sep={\\}]
        {
            \\
            1.8701207209378476  1.1696756919845939  \\
            1.670456608990207  1.4213281597476453  \\
        }
        ;
    \addplot[color={rgb,1:red,0.0;green,1.0;blue,0.498}, name path={36c44fbf-fb68-45d6-8a13-8daff49c720f}, draw opacity={0.8}, line width={1}, solid, forget plot]
        table[row sep={\\}]
        {
            \\
            1.8701207209378476  1.1696756919845939  \\
            1.6398598169907925  1.1640638990793377  \\
        }
        ;
    \addplot[color={rgb,1:red,0.0;green,1.0;blue,0.498}, name path={36c44fbf-fb68-45d6-8a13-8daff49c720f}, draw opacity={0.8}, line width={1}, solid, forget plot]
        table[row sep={\\}]
        {
            \\
            1.3230645160656422  1.6671039194334296  \\
            1.3983968361280859  1.8132919843774284  \\
        }
        ;
    \addplot[color={rgb,1:red,0.0;green,1.0;blue,0.498}, name path={36c44fbf-fb68-45d6-8a13-8daff49c720f}, draw opacity={0.8}, line width={1}, solid, forget plot]
        table[row sep={\\}]
        {
            \\
            1.1105229298118504  1.3180863440502435  \\
            1.3071074625477195  1.3007149568293244  \\
        }
        ;
    \addplot[color={rgb,1:red,0.0;green,1.0;blue,0.498}, name path={36c44fbf-fb68-45d6-8a13-8daff49c720f}, draw opacity={0.8}, line width={1}, solid, forget plot]
        table[row sep={\\}]
        {
            \\
            1.1105229298118504  1.3180863440502435  \\
            1.1387269033584746  1.3902763912919909  \\
        }
        ;
    \addplot[color={rgb,1:red,0.0;green,1.0;blue,0.498}, name path={36c44fbf-fb68-45d6-8a13-8daff49c720f}, draw opacity={0.8}, line width={1}, solid, forget plot]
        table[row sep={\\}]
        {
            \\
            1.1105229298118504  1.3180863440502435  \\
            1.125095868483186  1.7752906912937756  \\
        }
        ;
\end{axis}
\begin{axis}[point meta max={nan}, point meta min={nan}, title={Voronoi Diagram}, title style={at={{(0.5,1)}}, anchor={south}, font={{\fontsize{10 pt}{13.0 pt}\selectfont}}, color={rgb,1:red,0.0;green,0.0;blue,0.0}, draw opacity={1.0}, rotate={0.0}}, legend style={color={rgb,1:red,0.0;green,0.0;blue,0.0}, draw opacity={1.0}, line width={1}, solid, fill={rgb,1:red,1.0;green,1.0;blue,1.0}, fill opacity={1.0}, text opacity={1.0}, font={{\fontsize{8 pt}{10.4 pt}\selectfont}}, text={rgb,1:red,0.0;green,0.0;blue,0.0}, cells={anchor={west}}, at={(0.98, 0.98)}, anchor={north east}}, axis background/.style={fill={rgb,1:red,1.0;green,1.0;blue,1.0}, opacity={1.0}}, anchor={north west}, xshift={89.9mm}, yshift={-1.0mm}, width={86.9mm}, height={86.9mm}, scaled x ticks={false}, xlabel={$x$}, x tick style={color={rgb,1:red,0.0;green,0.0;blue,0.0}, opacity={1.0}}, x tick label style={color={rgb,1:red,0.0;green,0.0;blue,0.0}, opacity={1.0}, rotate={0}}, xlabel style={at={(ticklabel cs:0.5)}, anchor=near ticklabel, font={{\fontsize{11 pt}{14.3 pt}\selectfont}}, color={rgb,1:red,0.0;green,0.0;blue,0.0}, draw opacity={1.0}, rotate={0.0}}, xmajorgrids={true}, xmin={-0.017083897706063264}, xmax={3.200632340328746}, xtick={{0.0,1.0,2.0,3.0}}, xticklabels={{$0$,$1$,$2$,$3$}}, xtick align={inside}, xticklabel style={font={{\fontsize{8 pt}{10.4 pt}\selectfont}}, color={rgb,1:red,0.0;green,0.0;blue,0.0}, draw opacity={1.0}, rotate={0.0}}, x grid style={color={rgb,1:red,0.0;green,0.0;blue,0.0}, draw opacity={0.1}, line width={0.5}, solid}, axis x line*={left}, x axis line style={color={rgb,1:red,0.0;green,0.0;blue,0.0}, draw opacity={1.0}, line width={1}, solid}, scaled y ticks={false}, ylabel={$y$}, y tick style={color={rgb,1:red,0.0;green,0.0;blue,0.0}, opacity={1.0}}, y tick label style={color={rgb,1:red,0.0;green,0.0;blue,0.0}, opacity={1.0}, rotate={0}}, ylabel style={at={(ticklabel cs:0.5)}, anchor=near ticklabel, font={{\fontsize{11 pt}{14.3 pt}\selectfont}}, color={rgb,1:red,0.0;green,0.0;blue,0.0}, draw opacity={1.0}, rotate={0.0}}, ymajorgrids={true}, ymin={0.05688806001896497}, ymax={3.513956813974483}, ytick={{0.5,1.0,1.5,2.0,2.5,3.0,3.5}}, yticklabels={{$0.5$,$1.0$,$1.5$,$2.0$,$2.5$,$3.0$,$3.5$}}, ytick align={inside}, yticklabel style={font={{\fontsize{8 pt}{10.4 pt}\selectfont}}, color={rgb,1:red,0.0;green,0.0;blue,0.0}, draw opacity={1.0}, rotate={0.0}}, y grid style={color={rgb,1:red,0.0;green,0.0;blue,0.0}, draw opacity={0.1}, line width={0.5}, solid}, axis y line*={left}, y axis line style={color={rgb,1:red,0.0;green,0.0;blue,0.0}, draw opacity={1.0}, line width={1}, solid}, colorbar={false}, legend columns={-1}]
    \addplot[color={rgb,1:red,1.0;green,0.0;blue,0.0}, name path={08cc5d67-9fb1-42a7-aaa0-3b67b6b2428c}, draw opacity={1.0}, line width={1}, solid]
        table[row sep={\\}]
        {
            \\
            1.2604794002114923  1.803403112861966  \\
            1.27570865405724  1.8379202298383621  \\
        }
        ;
    \addlegendentry {Voronoi.jl}
    \addplot[color={rgb,1:red,1.0;green,0.0;blue,0.0}, name path={08cc5d67-9fb1-42a7-aaa0-3b67b6b2428c}, draw opacity={1.0}, line width={1}, solid, forget plot]
        table[row sep={\\}]
        {
            \\
            1.3653640860156322  1.8616109822110092  \\
            1.27570865405724  1.8379202298383621  \\
        }
        ;
    \addplot[color={rgb,1:red,1.0;green,0.0;blue,0.0}, name path={08cc5d67-9fb1-42a7-aaa0-3b67b6b2428c}, draw opacity={1.0}, line width={1}, solid, forget plot]
        table[row sep={\\}]
        {
            \\
            1.4207721973802245  1.7092578946573438  \\
            1.496462604009042  1.7138290481784513  \\
        }
        ;
    \addplot[color={rgb,1:red,1.0;green,0.0;blue,0.0}, name path={08cc5d67-9fb1-42a7-aaa0-3b67b6b2428c}, draw opacity={1.0}, line width={1}, solid, forget plot]
        table[row sep={\\}]
        {
            \\
            1.5967479273801457  1.7145953453990177  \\
            1.496462604009042  1.7138290481784513  \\
        }
        ;
    \addplot[color={rgb,1:red,1.0;green,0.0;blue,0.0}, name path={08cc5d67-9fb1-42a7-aaa0-3b67b6b2428c}, draw opacity={1.0}, line width={1}, solid, forget plot]
        table[row sep={\\}]
        {
            \\
            1.3653640860156322  1.8616109822110092  \\
            1.496462604009042  1.7138290481784513  \\
        }
        ;
    \addplot[color={rgb,1:red,1.0;green,0.0;blue,0.0}, name path={08cc5d67-9fb1-42a7-aaa0-3b67b6b2428c}, draw opacity={1.0}, line width={1}, solid, forget plot]
        table[row sep={\\}]
        {
            \\
            1.8295057313643917  1.599346634254463  \\
            1.9876471535093847  1.5816108873343284  \\
        }
        ;
    \addplot[color={rgb,1:red,1.0;green,0.0;blue,0.0}, name path={08cc5d67-9fb1-42a7-aaa0-3b67b6b2428c}, draw opacity={1.0}, line width={1}, solid, forget plot]
        table[row sep={\\}]
        {
            \\
            1.7835778408291005  1.3953116355771533  \\
            1.9876471535093847  1.5816108873343284  \\
        }
        ;
    \addplot[color={rgb,1:red,1.0;green,0.0;blue,0.0}, name path={08cc5d67-9fb1-42a7-aaa0-3b67b6b2428c}, draw opacity={1.0}, line width={1}, solid, forget plot]
        table[row sep={\\}]
        {
            \\
            1.7835778408291005  1.3953116355771533  \\
            1.7527984799071734  1.4545823716018462  \\
        }
        ;
    \addplot[color={rgb,1:red,1.0;green,0.0;blue,0.0}, name path={08cc5d67-9fb1-42a7-aaa0-3b67b6b2428c}, draw opacity={1.0}, line width={1}, solid, forget plot]
        table[row sep={\\}]
        {
            \\
            1.8295057313643917  1.599346634254463  \\
            1.7527984799071734  1.4545823716018462  \\
        }
        ;
    \addplot[color={rgb,1:red,1.0;green,0.0;blue,0.0}, name path={08cc5d67-9fb1-42a7-aaa0-3b67b6b2428c}, draw opacity={1.0}, line width={1}, solid, forget plot]
        table[row sep={\\}]
        {
            \\
            1.6348828068615442  1.5322431195203052  \\
            1.7527984799071734  1.4545823716018462  \\
        }
        ;
    \addplot[color={rgb,1:red,1.0;green,0.0;blue,0.0}, name path={08cc5d67-9fb1-42a7-aaa0-3b67b6b2428c}, draw opacity={1.0}, line width={1}, solid, forget plot]
        table[row sep={\\}]
        {
            \\
            1.6220013889866987  1.7343406005768556  \\
            1.6990408328985667  1.996561346861277  \\
        }
        ;
    \addplot[color={rgb,1:red,1.0;green,0.0;blue,0.0}, name path={08cc5d67-9fb1-42a7-aaa0-3b67b6b2428c}, draw opacity={1.0}, line width={1}, solid, forget plot]
        table[row sep={\\}]
        {
            \\
            1.3653640860156322  1.8616109822110092  \\
            1.6990408328985667  1.996561346861277  \\
        }
        ;
    \addplot[color={rgb,1:red,1.0;green,0.0;blue,0.0}, name path={08cc5d67-9fb1-42a7-aaa0-3b67b6b2428c}, draw opacity={1.0}, line width={1}, solid, forget plot]
        table[row sep={\\}]
        {
            \\
            1.8295057313643917  1.599346634254463  \\
            1.6220013889866987  1.7343406005768556  \\
        }
        ;
    \addplot[color={rgb,1:red,1.0;green,0.0;blue,0.0}, name path={08cc5d67-9fb1-42a7-aaa0-3b67b6b2428c}, draw opacity={1.0}, line width={1}, solid, forget plot]
        table[row sep={\\}]
        {
            \\
            1.6095671713405044  1.7191615435771748  \\
            1.6220013889866987  1.7343406005768556  \\
        }
        ;
    \addplot[color={rgb,1:red,1.0;green,0.0;blue,0.0}, name path={08cc5d67-9fb1-42a7-aaa0-3b67b6b2428c}, draw opacity={1.0}, line width={1}, solid, forget plot]
        table[row sep={\\}]
        {
            \\
            1.7605048201946907  0.9405988741695328  \\
            1.7522049938096065  1.2811541224295153  \\
        }
        ;
    \addplot[color={rgb,1:red,1.0;green,0.0;blue,0.0}, name path={08cc5d67-9fb1-42a7-aaa0-3b67b6b2428c}, draw opacity={1.0}, line width={1}, solid, forget plot]
        table[row sep={\\}]
        {
            \\
            1.786608982970989  1.3084506635857562  \\
            1.7522049938096065  1.2811541224295153  \\
        }
        ;
    \addplot[color={rgb,1:red,1.0;green,0.0;blue,0.0}, name path={08cc5d67-9fb1-42a7-aaa0-3b67b6b2428c}, draw opacity={1.0}, line width={1}, solid, forget plot]
        table[row sep={\\}]
        {
            \\
            1.5412601702997408  1.306242080125303  \\
            1.7522049938096065  1.2811541224295153  \\
        }
        ;
    \addplot[color={rgb,1:red,1.0;green,0.0;blue,0.0}, name path={08cc5d67-9fb1-42a7-aaa0-3b67b6b2428c}, draw opacity={1.0}, line width={1}, solid, forget plot]
        table[row sep={\\}]
        {
            \\
            1.472714966518481  1.25429855722196  \\
            1.156326549607483  0.7154091672536667  \\
        }
        ;
    \addplot[color={rgb,1:red,1.0;green,0.0;blue,0.0}, name path={08cc5d67-9fb1-42a7-aaa0-3b67b6b2428c}, draw opacity={1.0}, line width={1}, solid, forget plot]
        table[row sep={\\}]
        {
            \\
            1.2098306655723479  1.3208922816783164  \\
            1.156326549607483  0.7154091672536667  \\
        }
        ;
    \addplot[color={rgb,1:red,1.0;green,0.0;blue,0.0}, name path={08cc5d67-9fb1-42a7-aaa0-3b67b6b2428c}, draw opacity={1.0}, line width={1}, solid, forget plot]
        table[row sep={\\}]
        {
            \\
            1.1487653185833961  1.5833802374257493  \\
            1.2628404793646477  1.5074183453813343  \\
        }
        ;
    \addplot[color={rgb,1:red,1.0;green,0.0;blue,0.0}, name path={08cc5d67-9fb1-42a7-aaa0-3b67b6b2428c}, draw opacity={1.0}, line width={1}, solid, forget plot]
        table[row sep={\\}]
        {
            \\
            1.2690346489872688  1.4321991169634372  \\
            1.2628404793646477  1.5074183453813343  \\
        }
        ;
    \addplot[color={rgb,1:red,1.0;green,0.0;blue,0.0}, name path={08cc5d67-9fb1-42a7-aaa0-3b67b6b2428c}, draw opacity={1.0}, line width={1}, solid, forget plot]
        table[row sep={\\}]
        {
            \\
            1.276983303491567  1.5119328295486254  \\
            1.2628404793646477  1.5074183453813343  \\
        }
        ;
    \addplot[color={rgb,1:red,1.0;green,0.0;blue,0.0}, name path={08cc5d67-9fb1-42a7-aaa0-3b67b6b2428c}, draw opacity={1.0}, line width={1}, solid, forget plot]
        table[row sep={\\}]
        {
            \\
            1.6095671713405044  1.7191615435771748  \\
            1.5967479273801457  1.7145953453990177  \\
        }
        ;
    \addplot[color={rgb,1:red,1.0;green,0.0;blue,0.0}, name path={08cc5d67-9fb1-42a7-aaa0-3b67b6b2428c}, draw opacity={1.0}, line width={1}, solid, forget plot]
        table[row sep={\\}]
        {
            \\
            1.5599023713656444  1.67051177226448  \\
            1.5967479273801457  1.7145953453990177  \\
        }
        ;
    \addplot[color={rgb,1:red,1.0;green,0.0;blue,0.0}, name path={08cc5d67-9fb1-42a7-aaa0-3b67b6b2428c}, draw opacity={1.0}, line width={1}, solid, forget plot]
        table[row sep={\\}]
        {
            \\
            1.5209194746064107  1.2869409180911984  \\
            1.472714966518481  1.25429855722196  \\
        }
        ;
    \addplot[color={rgb,1:red,1.0;green,0.0;blue,0.0}, name path={08cc5d67-9fb1-42a7-aaa0-3b67b6b2428c}, draw opacity={1.0}, line width={1}, solid, forget plot]
        table[row sep={\\}]
        {
            \\
            1.2690346489872688  1.4321991169634372  \\
            1.472714966518481  1.25429855722196  \\
        }
        ;
    \addplot[color={rgb,1:red,1.0;green,0.0;blue,0.0}, name path={08cc5d67-9fb1-42a7-aaa0-3b67b6b2428c}, draw opacity={1.0}, line width={1}, solid, forget plot]
        table[row sep={\\}]
        {
            \\
            1.2098306655723479  1.3208922816783164  \\
            1.2690346489872688  1.4321991169634372  \\
        }
        ;
    \addplot[color={rgb,1:red,1.0;green,0.0;blue,0.0}, name path={08cc5d67-9fb1-42a7-aaa0-3b67b6b2428c}, draw opacity={1.0}, line width={1}, solid, forget plot]
        table[row sep={\\}]
        {
            \\
            1.1487653185833961  1.5833802374257493  \\
            0.5887246387327791  1.563552573781597  \\
        }
        ;
    \addplot[color={rgb,1:red,1.0;green,0.0;blue,0.0}, name path={08cc5d67-9fb1-42a7-aaa0-3b67b6b2428c}, draw opacity={1.0}, line width={1}, solid, forget plot]
        table[row sep={\\}]
        {
            \\
            1.2098306655723479  1.3208922816783164  \\
            0.5887246387327791  1.563552573781597  \\
        }
        ;
    \addplot[color={rgb,1:red,1.0;green,0.0;blue,0.0}, name path={08cc5d67-9fb1-42a7-aaa0-3b67b6b2428c}, draw opacity={1.0}, line width={1}, solid, forget plot]
        table[row sep={\\}]
        {
            \\
            1.2622085164571029  1.7909675059457593  \\
            1.1487653185833961  1.5833802374257493  \\
        }
        ;
    \addplot[color={rgb,1:red,1.0;green,0.0;blue,0.0}, name path={08cc5d67-9fb1-42a7-aaa0-3b67b6b2428c}, draw opacity={1.0}, line width={1}, solid, forget plot]
        table[row sep={\\}]
        {
            \\
            1.2604794002114923  1.803403112861966  \\
            1.2622085164571029  1.7909675059457593  \\
        }
        ;
    \addplot[color={rgb,1:red,1.0;green,0.0;blue,0.0}, name path={08cc5d67-9fb1-42a7-aaa0-3b67b6b2428c}, draw opacity={1.0}, line width={1}, solid, forget plot]
        table[row sep={\\}]
        {
            \\
            1.4207721973802245  1.7092578946573438  \\
            1.2622085164571029  1.7909675059457593  \\
        }
        ;
    \addplot[color={rgb,1:red,1.0;green,0.0;blue,0.0}, name path={08cc5d67-9fb1-42a7-aaa0-3b67b6b2428c}, draw opacity={1.0}, line width={1}, solid, forget plot]
        table[row sep={\\}]
        {
            \\
            1.434708982548336  1.4940758047564862  \\
            1.5599023713656444  1.67051177226448  \\
        }
        ;
    \addplot[color={rgb,1:red,1.0;green,0.0;blue,0.0}, name path={08cc5d67-9fb1-42a7-aaa0-3b67b6b2428c}, draw opacity={1.0}, line width={1}, solid, forget plot]
        table[row sep={\\}]
        {
            \\
            1.2815350666192726  1.5172242627114598  \\
            1.5599023713656444  1.67051177226448  \\
        }
        ;
    \addplot[color={rgb,1:red,1.0;green,0.0;blue,0.0}, name path={08cc5d67-9fb1-42a7-aaa0-3b67b6b2428c}, draw opacity={1.0}, line width={1}, solid, forget plot]
        table[row sep={\\}]
        {
            \\
            1.2815350666192726  1.5172242627114598  \\
            1.276983303491567  1.5119328295486254  \\
        }
        ;
    \addplot[color={rgb,1:red,1.0;green,0.0;blue,0.0}, name path={08cc5d67-9fb1-42a7-aaa0-3b67b6b2428c}, draw opacity={1.0}, line width={1}, solid, forget plot]
        table[row sep={\\}]
        {
            \\
            1.434708982548336  1.4940758047564862  \\
            1.276983303491567  1.5119328295486254  \\
        }
        ;
    \addplot[color={rgb,1:red,1.0;green,0.0;blue,0.0}, name path={08cc5d67-9fb1-42a7-aaa0-3b67b6b2428c}, draw opacity={1.0}, line width={1}, solid, forget plot]
        table[row sep={\\}]
        {
            \\
            1.6348828068615442  1.5322431195203052  \\
            1.6095671713405044  1.7191615435771748  \\
        }
        ;
    \addplot[color={rgb,1:red,1.0;green,0.0;blue,0.0}, name path={08cc5d67-9fb1-42a7-aaa0-3b67b6b2428c}, draw opacity={1.0}, line width={1}, solid, forget plot]
        table[row sep={\\}]
        {
            \\
            1.5389442089407321  1.3734229705441723  \\
            1.434708982548336  1.4940758047564862  \\
        }
        ;
    \addplot[color={rgb,1:red,1.0;green,0.0;blue,0.0}, name path={08cc5d67-9fb1-42a7-aaa0-3b67b6b2428c}, draw opacity={1.0}, line width={1}, solid, forget plot]
        table[row sep={\\}]
        {
            \\
            1.5389442089407321  1.3734229705441723  \\
            1.5412601702997408  1.306242080125303  \\
        }
        ;
    \addplot[color={rgb,1:red,1.0;green,0.0;blue,0.0}, name path={08cc5d67-9fb1-42a7-aaa0-3b67b6b2428c}, draw opacity={1.0}, line width={1}, solid, forget plot]
        table[row sep={\\}]
        {
            \\
            1.5209194746064107  1.2869409180911984  \\
            1.5412601702997408  1.306242080125303  \\
        }
        ;
    \addplot[color={rgb,1:red,1.0;green,0.0;blue,0.0}, name path={08cc5d67-9fb1-42a7-aaa0-3b67b6b2428c}, draw opacity={1.0}, line width={1}, solid, forget plot]
        table[row sep={\\}]
        {
            \\
            1.2815350666192726  1.5172242627114598  \\
            1.4207721973802245  1.7092578946573438  \\
        }
        ;
    \addplot[color={rgb,1:red,1.0;green,0.0;blue,0.0}, name path={08cc5d67-9fb1-42a7-aaa0-3b67b6b2428c}, draw opacity={1.0}, line width={1}, solid, forget plot]
        table[row sep={\\}]
        {
            \\
            1.786608982970989  1.3084506635857562  \\
            1.7835778408291005  1.3953116355771533  \\
        }
        ;
    \addplot[color={rgb,1:red,1.0;green,0.0;blue,0.0}, name path={08cc5d67-9fb1-42a7-aaa0-3b67b6b2428c}, draw opacity={1.0}, line width={1}, solid, forget plot]
        table[row sep={\\}]
        {
            \\
            1.5389442089407321  1.3734229705441723  \\
            1.6348828068615442  1.5322431195203052  \\
        }
        ;
    \addplot[color={rgb,1:red,1.0;green,0.0;blue,0.0}, name path={08cc5d67-9fb1-42a7-aaa0-3b67b6b2428c}, draw opacity={1.0}, line width={1}, solid, forget plot]
        table[row sep={\\}]
        {
            \\
            1.5209194746064107  1.2869409180911984  \\
            1.7605048201946907  0.9405988741695328  \\
        }
        ;
    \addplot[color={rgb,1:red,0.0;green,1.0;blue,0.498}, name path={be70f346-6d26-4ad9-af32-44b3e8da848f}, draw opacity={0.8}, line width={1}, solid]
        table[row sep={\\}]
        {
            \\
            1.7527984799071734  1.4545823716018462  \\
            1.6348828068615442  1.5322431195203052  \\
        }
        ;
    \addlegendentry {VoronoiDelaunay.jl}
    \addplot[color={rgb,1:red,0.0;green,1.0;blue,0.498}, name path={be70f346-6d26-4ad9-af32-44b3e8da848f}, draw opacity={0.8}, line width={1}, solid, forget plot]
        table[row sep={\\}]
        {
            \\
            1.7527984799071734  1.4545823716018462  \\
            1.8295057313643917  1.599346634254463  \\
        }
        ;
    \addplot[color={rgb,1:red,0.0;green,1.0;blue,0.498}, name path={be70f346-6d26-4ad9-af32-44b3e8da848f}, draw opacity={0.8}, line width={1}, solid, forget plot]
        table[row sep={\\}]
        {
            \\
            1.7527984799071734  1.4545823716018462  \\
            1.7835778408291005  1.3953116355771533  \\
        }
        ;
    \addplot[color={rgb,1:red,0.0;green,1.0;blue,0.498}, name path={be70f346-6d26-4ad9-af32-44b3e8da848f}, draw opacity={0.8}, line width={1}, solid, forget plot]
        table[row sep={\\}]
        {
            \\
            0.4862484197523599  1.566818902423211  \\
            1.0492624180072498  1.880249287151797  \\
        }
        ;
    \addplot[color={rgb,1:red,0.0;green,1.0;blue,0.498}, name path={be70f346-6d26-4ad9-af32-44b3e8da848f}, draw opacity={0.8}, line width={1}, solid, forget plot]
        table[row sep={\\}]
        {
            \\
            0.4862484197523599  1.566818902423211  \\
            0.5887246387327791  1.563552573781597  \\
        }
        ;
    \addplot[color={rgb,1:red,0.0;green,1.0;blue,0.498}, name path={be70f346-6d26-4ad9-af32-44b3e8da848f}, draw opacity={0.8}, line width={1}, solid, forget plot]
        table[row sep={\\}]
        {
            \\
            0.4862484197523599  1.566818902423211  \\
            0.07398354299303511  1.5  \\
        }
        ;
    \addplot[color={rgb,1:red,0.0;green,1.0;blue,0.498}, name path={be70f346-6d26-4ad9-af32-44b3e8da848f}, draw opacity={0.8}, line width={1}, solid, forget plot]
        table[row sep={\\}]
        {
            \\
            1.2570826968156772  2.2928010286527063  \\
            1.4999999999999998  3.4161152454663077  \\
        }
        ;
    \addplot[color={rgb,1:red,0.0;green,1.0;blue,0.498}, name path={be70f346-6d26-4ad9-af32-44b3e8da848f}, draw opacity={0.8}, line width={1}, solid, forget plot]
        table[row sep={\\}]
        {
            \\
            1.2570826968156772  2.2928010286527063  \\
            1.27570865405724  1.8379202298383621  \\
        }
        ;
    \addplot[color={rgb,1:red,0.0;green,1.0;blue,0.498}, name path={be70f346-6d26-4ad9-af32-44b3e8da848f}, draw opacity={0.8}, line width={1}, solid, forget plot]
        table[row sep={\\}]
        {
            \\
            1.2570826968156772  2.2928010286527063  \\
            1.0492624180072498  1.880249287151797  \\
        }
        ;
    \addplot[color={rgb,1:red,0.0;green,1.0;blue,0.498}, name path={be70f346-6d26-4ad9-af32-44b3e8da848f}, draw opacity={0.8}, line width={1}, solid, forget plot]
        table[row sep={\\}]
        {
            \\
            1.3653640860156322  1.8616109822110092  \\
            1.6813651102443967  1.9894126760596103  \\
        }
        ;
    \addplot[color={rgb,1:red,0.0;green,1.0;blue,0.498}, name path={be70f346-6d26-4ad9-af32-44b3e8da848f}, draw opacity={0.8}, line width={1}, solid, forget plot]
        table[row sep={\\}]
        {
            \\
            1.3653640860156322  1.8616109822110092  \\
            1.4964626040090419  1.7138290481784513  \\
        }
        ;
    \addplot[color={rgb,1:red,0.0;green,1.0;blue,0.498}, name path={be70f346-6d26-4ad9-af32-44b3e8da848f}, draw opacity={0.8}, line width={1}, solid, forget plot]
        table[row sep={\\}]
        {
            \\
            1.3653640860156322  1.8616109822110092  \\
            1.27570865405724  1.8379202298383621  \\
        }
        ;
    \addplot[color={rgb,1:red,0.0;green,1.0;blue,0.498}, name path={be70f346-6d26-4ad9-af32-44b3e8da848f}, draw opacity={0.8}, line width={1}, solid, forget plot]
        table[row sep={\\}]
        {
            \\
            1.6095671713405044  1.7191615435771748  \\
            1.6348828068615442  1.5322431195203052  \\
        }
        ;
    \addplot[color={rgb,1:red,0.0;green,1.0;blue,0.498}, name path={be70f346-6d26-4ad9-af32-44b3e8da848f}, draw opacity={0.8}, line width={1}, solid, forget plot]
        table[row sep={\\}]
        {
            \\
            1.6095671713405044  1.7191615435771748  \\
            1.5967479273801457  1.7145953453990177  \\
        }
        ;
    \addplot[color={rgb,1:red,0.0;green,1.0;blue,0.498}, name path={be70f346-6d26-4ad9-af32-44b3e8da848f}, draw opacity={0.8}, line width={1}, solid, forget plot]
        table[row sep={\\}]
        {
            \\
            1.6095671713405044  1.7191615435771748  \\
            1.6220013889866987  1.7343406005768556  \\
        }
        ;
    \addplot[color={rgb,1:red,0.0;green,1.0;blue,0.498}, name path={be70f346-6d26-4ad9-af32-44b3e8da848f}, draw opacity={0.8}, line width={1}, solid, forget plot]
        table[row sep={\\}]
        {
            \\
            1.6348828068615442  1.5322431195203052  \\
            1.5389442089407321  1.3734229705441723  \\
        }
        ;
    \addplot[color={rgb,1:red,0.0;green,1.0;blue,0.498}, name path={be70f346-6d26-4ad9-af32-44b3e8da848f}, draw opacity={0.8}, line width={1}, solid, forget plot]
        table[row sep={\\}]
        {
            \\
            2.3386377425544103  1.6423277600009905  \\
            1.9876471535093847  1.5816108873343284  \\
        }
        ;
    \addplot[color={rgb,1:red,0.0;green,1.0;blue,0.498}, name path={be70f346-6d26-4ad9-af32-44b3e8da848f}, draw opacity={0.8}, line width={1}, solid, forget plot]
        table[row sep={\\}]
        {
            \\
            2.3386377425544103  1.6423277600009905  \\
            1.6892290828527665  1.9631648905951964  \\
        }
        ;
    \addplot[color={rgb,1:red,0.0;green,1.0;blue,0.498}, name path={be70f346-6d26-4ad9-af32-44b3e8da848f}, draw opacity={0.8}, line width={1}, solid, forget plot]
        table[row sep={\\}]
        {
            \\
            2.3386377425544103  1.6423277600009905  \\
            3.109564899629648  1.4999999999999998  \\
        }
        ;
    \addplot[color={rgb,1:red,0.0;green,1.0;blue,0.498}, name path={be70f346-6d26-4ad9-af32-44b3e8da848f}, draw opacity={0.8}, line width={1}, solid, forget plot]
        table[row sep={\\}]
        {
            \\
            3.109564899629648  1.4999999999999998  \\
            2.1788293709650297  1.2714323013949198  \\
        }
        ;
    \addplot[color={rgb,1:red,0.0;green,1.0;blue,0.498}, name path={be70f346-6d26-4ad9-af32-44b3e8da848f}, draw opacity={0.8}, line width={1}, solid, forget plot]
        table[row sep={\\}]
        {
            \\
            1.6813651102443967  1.9894126760596103  \\
            1.6892290828527665  1.9631648905951964  \\
        }
        ;
    \addplot[color={rgb,1:red,0.0;green,1.0;blue,0.498}, name path={be70f346-6d26-4ad9-af32-44b3e8da848f}, draw opacity={0.8}, line width={1}, solid, forget plot]
        table[row sep={\\}]
        {
            \\
            1.6813651102443967  1.9894126760596103  \\
            1.4999999999999998  3.4161152454663077  \\
        }
        ;
    \addplot[color={rgb,1:red,0.0;green,1.0;blue,0.498}, name path={be70f346-6d26-4ad9-af32-44b3e8da848f}, draw opacity={0.8}, line width={1}, solid, forget plot]
        table[row sep={\\}]
        {
            \\
            1.3149299332578317  0.9855507909474527  \\
            1.1913502614960443  1.1117574812212023  \\
        }
        ;
    \addplot[color={rgb,1:red,0.0;green,1.0;blue,0.498}, name path={be70f346-6d26-4ad9-af32-44b3e8da848f}, draw opacity={0.8}, line width={1}, solid, forget plot]
        table[row sep={\\}]
        {
            \\
            1.3149299332578317  0.9855507909474527  \\
            1.472714966518481  1.25429855722196  \\
        }
        ;
    \addplot[color={rgb,1:red,0.0;green,1.0;blue,0.498}, name path={be70f346-6d26-4ad9-af32-44b3e8da848f}, draw opacity={0.8}, line width={1}, solid, forget plot]
        table[row sep={\\}]
        {
            \\
            1.3149299332578317  0.9855507909474527  \\
            1.5  0.15472962852714  \\
        }
        ;
    \addplot[color={rgb,1:red,0.0;green,1.0;blue,0.498}, name path={be70f346-6d26-4ad9-af32-44b3e8da848f}, draw opacity={0.8}, line width={1}, solid, forget plot]
        table[row sep={\\}]
        {
            \\
            1.5967479273801457  1.7145953453990177  \\
            1.4964626040090419  1.7138290481784513  \\
        }
        ;
    \addplot[color={rgb,1:red,0.0;green,1.0;blue,0.498}, name path={be70f346-6d26-4ad9-af32-44b3e8da848f}, draw opacity={0.8}, line width={1}, solid, forget plot]
        table[row sep={\\}]
        {
            \\
            1.5967479273801457  1.7145953453990177  \\
            1.5599023713656444  1.67051177226448  \\
        }
        ;
    \addplot[color={rgb,1:red,0.0;green,1.0;blue,0.498}, name path={be70f346-6d26-4ad9-af32-44b3e8da848f}, draw opacity={0.8}, line width={1}, solid, forget plot]
        table[row sep={\\}]
        {
            \\
            1.2815350666192726  1.5172242627114598  \\
            1.276983303491567  1.5119328295486254  \\
        }
        ;
    \addplot[color={rgb,1:red,0.0;green,1.0;blue,0.498}, name path={be70f346-6d26-4ad9-af32-44b3e8da848f}, draw opacity={0.8}, line width={1}, solid, forget plot]
        table[row sep={\\}]
        {
            \\
            1.2815350666192726  1.5172242627114598  \\
            1.4207721973802245  1.7092578946573438  \\
        }
        ;
    \addplot[color={rgb,1:red,0.0;green,1.0;blue,0.498}, name path={be70f346-6d26-4ad9-af32-44b3e8da848f}, draw opacity={0.8}, line width={1}, solid, forget plot]
        table[row sep={\\}]
        {
            \\
            1.2815350666192726  1.5172242627114598  \\
            1.5599023713656444  1.67051177226448  \\
        }
        ;
    \addplot[color={rgb,1:red,0.0;green,1.0;blue,0.498}, name path={be70f346-6d26-4ad9-af32-44b3e8da848f}, draw opacity={0.8}, line width={1}, solid, forget plot]
        table[row sep={\\}]
        {
            \\
            1.5412601702997408  1.306242080125303  \\
            1.5389442089407321  1.3734229705441723  \\
        }
        ;
    \addplot[color={rgb,1:red,0.0;green,1.0;blue,0.498}, name path={be70f346-6d26-4ad9-af32-44b3e8da848f}, draw opacity={0.8}, line width={1}, solid, forget plot]
        table[row sep={\\}]
        {
            \\
            1.5412601702997408  1.306242080125303  \\
            1.7522049938096065  1.2811541224295153  \\
        }
        ;
    \addplot[color={rgb,1:red,0.0;green,1.0;blue,0.498}, name path={be70f346-6d26-4ad9-af32-44b3e8da848f}, draw opacity={0.8}, line width={1}, solid, forget plot]
        table[row sep={\\}]
        {
            \\
            1.5412601702997408  1.306242080125303  \\
            1.5209194746064107  1.2869409180911984  \\
        }
        ;
    \addplot[color={rgb,1:red,0.0;green,1.0;blue,0.498}, name path={be70f346-6d26-4ad9-af32-44b3e8da848f}, draw opacity={0.8}, line width={1}, solid, forget plot]
        table[row sep={\\}]
        {
            \\
            1.5209194746064107  1.2869409180911984  \\
            1.472714966518481  1.25429855722196  \\
        }
        ;
    \addplot[color={rgb,1:red,0.0;green,1.0;blue,0.498}, name path={be70f346-6d26-4ad9-af32-44b3e8da848f}, draw opacity={0.8}, line width={1}, solid, forget plot]
        table[row sep={\\}]
        {
            \\
            1.5209194746064107  1.2869409180911984  \\
            1.7574867860475272  0.9449617124415793  \\
        }
        ;
    \addplot[color={rgb,1:red,0.0;green,1.0;blue,0.498}, name path={be70f346-6d26-4ad9-af32-44b3e8da848f}, draw opacity={0.8}, line width={1}, solid, forget plot]
        table[row sep={\\}]
        {
            \\
            1.6892290828527665  1.9631648905951964  \\
            1.6220013889866987  1.7343406005768556  \\
        }
        ;
    \addplot[color={rgb,1:red,0.0;green,1.0;blue,0.498}, name path={be70f346-6d26-4ad9-af32-44b3e8da848f}, draw opacity={0.8}, line width={1}, solid, forget plot]
        table[row sep={\\}]
        {
            \\
            1.9876471535093847  1.5816108873343284  \\
            1.8295057313643917  1.599346634254463  \\
        }
        ;
    \addplot[color={rgb,1:red,0.0;green,1.0;blue,0.498}, name path={be70f346-6d26-4ad9-af32-44b3e8da848f}, draw opacity={0.8}, line width={1}, solid, forget plot]
        table[row sep={\\}]
        {
            \\
            1.9876471535093847  1.5816108873343284  \\
            1.7835778408291005  1.3953116355771533  \\
        }
        ;
    \addplot[color={rgb,1:red,0.0;green,1.0;blue,0.498}, name path={be70f346-6d26-4ad9-af32-44b3e8da848f}, draw opacity={0.8}, line width={1}, solid, forget plot]
        table[row sep={\\}]
        {
            \\
            1.2604794002114923  1.803403112861966  \\
            1.0492624180072498  1.880249287151797  \\
        }
        ;
    \addplot[color={rgb,1:red,0.0;green,1.0;blue,0.498}, name path={be70f346-6d26-4ad9-af32-44b3e8da848f}, draw opacity={0.8}, line width={1}, solid, forget plot]
        table[row sep={\\}]
        {
            \\
            1.2604794002114923  1.803403112861966  \\
            1.27570865405724  1.8379202298383621  \\
        }
        ;
    \addplot[color={rgb,1:red,0.0;green,1.0;blue,0.498}, name path={be70f346-6d26-4ad9-af32-44b3e8da848f}, draw opacity={0.8}, line width={1}, solid, forget plot]
        table[row sep={\\}]
        {
            \\
            1.2604794002114923  1.803403112861966  \\
            1.2622085164571029  1.7909675059457593  \\
        }
        ;
    \addplot[color={rgb,1:red,0.0;green,1.0;blue,0.498}, name path={be70f346-6d26-4ad9-af32-44b3e8da848f}, draw opacity={0.8}, line width={1}, solid, forget plot]
        table[row sep={\\}]
        {
            \\
            1.6220013889866987  1.7343406005768556  \\
            1.8295057313643917  1.599346634254463  \\
        }
        ;
    \addplot[color={rgb,1:red,0.0;green,1.0;blue,0.498}, name path={be70f346-6d26-4ad9-af32-44b3e8da848f}, draw opacity={0.8}, line width={1}, solid, forget plot]
        table[row sep={\\}]
        {
            \\
            1.276983303491567  1.5119328295486254  \\
            1.2628404793646477  1.5074183453813343  \\
        }
        ;
    \addplot[color={rgb,1:red,0.0;green,1.0;blue,0.498}, name path={be70f346-6d26-4ad9-af32-44b3e8da848f}, draw opacity={0.8}, line width={1}, solid, forget plot]
        table[row sep={\\}]
        {
            \\
            1.276983303491567  1.5119328295486254  \\
            1.434708982548336  1.4940758047564862  \\
        }
        ;
    \addplot[color={rgb,1:red,0.0;green,1.0;blue,0.498}, name path={be70f346-6d26-4ad9-af32-44b3e8da848f}, draw opacity={0.8}, line width={1}, solid, forget plot]
        table[row sep={\\}]
        {
            \\
            0.07398354299303511  1.5  \\
            1.1913502614960443  1.1117574812212023  \\
        }
        ;
    \addplot[color={rgb,1:red,0.0;green,1.0;blue,0.498}, name path={be70f346-6d26-4ad9-af32-44b3e8da848f}, draw opacity={0.8}, line width={1}, solid, forget plot]
        table[row sep={\\}]
        {
            \\
            1.5599023713656444  1.67051177226448  \\
            1.434708982548336  1.4940758047564862  \\
        }
        ;
    \addplot[color={rgb,1:red,0.0;green,1.0;blue,0.498}, name path={be70f346-6d26-4ad9-af32-44b3e8da848f}, draw opacity={0.8}, line width={1}, solid, forget plot]
        table[row sep={\\}]
        {
            \\
            1.2628404793646477  1.5074183453813343  \\
            1.1487653185833961  1.5833802374257493  \\
        }
        ;
    \addplot[color={rgb,1:red,0.0;green,1.0;blue,0.498}, name path={be70f346-6d26-4ad9-af32-44b3e8da848f}, draw opacity={0.8}, line width={1}, solid, forget plot]
        table[row sep={\\}]
        {
            \\
            1.2628404793646477  1.5074183453813343  \\
            1.2690346489872688  1.4321991169634372  \\
        }
        ;
    \addplot[color={rgb,1:red,0.0;green,1.0;blue,0.498}, name path={be70f346-6d26-4ad9-af32-44b3e8da848f}, draw opacity={0.8}, line width={1}, solid, forget plot]
        table[row sep={\\}]
        {
            \\
            1.2690346489872688  1.4321991169634372  \\
            1.472714966518481  1.25429855722196  \\
        }
        ;
    \addplot[color={rgb,1:red,0.0;green,1.0;blue,0.498}, name path={be70f346-6d26-4ad9-af32-44b3e8da848f}, draw opacity={0.8}, line width={1}, solid, forget plot]
        table[row sep={\\}]
        {
            \\
            1.2690346489872688  1.4321991169634372  \\
            1.2098306655723479  1.3208922816783164  \\
        }
        ;
    \addplot[color={rgb,1:red,0.0;green,1.0;blue,0.498}, name path={be70f346-6d26-4ad9-af32-44b3e8da848f}, draw opacity={0.8}, line width={1}, solid, forget plot]
        table[row sep={\\}]
        {
            \\
            1.1913502614960443  1.1117574812212023  \\
            1.2098306655723479  1.3208922816783164  \\
        }
        ;
    \addplot[color={rgb,1:red,0.0;green,1.0;blue,0.498}, name path={be70f346-6d26-4ad9-af32-44b3e8da848f}, draw opacity={0.8}, line width={1}, solid, forget plot]
        table[row sep={\\}]
        {
            \\
            1.434708982548336  1.4940758047564862  \\
            1.5389442089407321  1.3734229705441723  \\
        }
        ;
    \addplot[color={rgb,1:red,0.0;green,1.0;blue,0.498}, name path={be70f346-6d26-4ad9-af32-44b3e8da848f}, draw opacity={0.8}, line width={1}, solid, forget plot]
        table[row sep={\\}]
        {
            \\
            1.4964626040090419  1.7138290481784513  \\
            1.4207721973802245  1.7092578946573438  \\
        }
        ;
    \addplot[color={rgb,1:red,0.0;green,1.0;blue,0.498}, name path={be70f346-6d26-4ad9-af32-44b3e8da848f}, draw opacity={0.8}, line width={1}, solid, forget plot]
        table[row sep={\\}]
        {
            \\
            1.1487653185833961  1.5833802374257493  \\
            1.2622085164571029  1.7909675059457593  \\
        }
        ;
    \addplot[color={rgb,1:red,0.0;green,1.0;blue,0.498}, name path={be70f346-6d26-4ad9-af32-44b3e8da848f}, draw opacity={0.8}, line width={1}, solid, forget plot]
        table[row sep={\\}]
        {
            \\
            1.1487653185833961  1.5833802374257493  \\
            0.5887246387327791  1.563552573781597  \\
        }
        ;
    \addplot[color={rgb,1:red,0.0;green,1.0;blue,0.498}, name path={be70f346-6d26-4ad9-af32-44b3e8da848f}, draw opacity={0.8}, line width={1}, solid, forget plot]
        table[row sep={\\}]
        {
            \\
            1.7835778408291005  1.3953116355771533  \\
            1.786608982970989  1.3084506635857562  \\
        }
        ;
    \addplot[color={rgb,1:red,0.0;green,1.0;blue,0.498}, name path={be70f346-6d26-4ad9-af32-44b3e8da848f}, draw opacity={0.8}, line width={1}, solid, forget plot]
        table[row sep={\\}]
        {
            \\
            1.5  0.15472962852714  \\
            1.7574867860475272  0.9449617124415793  \\
        }
        ;
    \addplot[color={rgb,1:red,0.0;green,1.0;blue,0.498}, name path={be70f346-6d26-4ad9-af32-44b3e8da848f}, draw opacity={0.8}, line width={1}, solid, forget plot]
        table[row sep={\\}]
        {
            \\
            1.786608982970989  1.3084506635857562  \\
            2.1788293709650297  1.2714323013949198  \\
        }
        ;
    \addplot[color={rgb,1:red,0.0;green,1.0;blue,0.498}, name path={be70f346-6d26-4ad9-af32-44b3e8da848f}, draw opacity={0.8}, line width={1}, solid, forget plot]
        table[row sep={\\}]
        {
            \\
            1.786608982970989  1.3084506635857562  \\
            1.7522049938096065  1.2811541224295153  \\
        }
        ;
    \addplot[color={rgb,1:red,0.0;green,1.0;blue,0.498}, name path={be70f346-6d26-4ad9-af32-44b3e8da848f}, draw opacity={0.8}, line width={1}, solid, forget plot]
        table[row sep={\\}]
        {
            \\
            1.7574867860475272  0.9449617124415793  \\
            1.7602506305045384  0.9510286866762687  \\
        }
        ;
    \addplot[color={rgb,1:red,0.0;green,1.0;blue,0.498}, name path={be70f346-6d26-4ad9-af32-44b3e8da848f}, draw opacity={0.8}, line width={1}, solid, forget plot]
        table[row sep={\\}]
        {
            \\
            1.7522049938096065  1.2811541224295153  \\
            1.7602506305045384  0.9510286866762687  \\
        }
        ;
    \addplot[color={rgb,1:red,0.0;green,1.0;blue,0.498}, name path={be70f346-6d26-4ad9-af32-44b3e8da848f}, draw opacity={0.8}, line width={1}, solid, forget plot]
        table[row sep={\\}]
        {
            \\
            2.1788293709650297  1.2714323013949198  \\
            1.7602506305045384  0.9510286866762687  \\
        }
        ;
    \addplot[color={rgb,1:red,0.0;green,1.0;blue,0.498}, name path={be70f346-6d26-4ad9-af32-44b3e8da848f}, draw opacity={0.8}, line width={1}, solid, forget plot]
        table[row sep={\\}]
        {
            \\
            1.4207721973802245  1.7092578946573438  \\
            1.2622085164571029  1.7909675059457593  \\
        }
        ;
    \addplot[color={rgb,1:red,0.0;green,1.0;blue,0.498}, name path={be70f346-6d26-4ad9-af32-44b3e8da848f}, draw opacity={0.8}, line width={1}, solid, forget plot]
        table[row sep={\\}]
        {
            \\
            1.2098306655723479  1.3208922816783164  \\
            0.5887246387327791  1.563552573781597  \\
        }
        ;
\end{axis}
\end{tikzpicture}
}
  \caption[Voronoi Delaunay output comparison]{
    Delaunay Triangulations (on the left) and Voronoi Diagrams (on the right)
    produced both by \texttt{Voronoi.jl} and \texttt{VoronoiDelaunay.jl}.  The
    mesh was produced from a 21-point cloud.}%
  \label{fig:eval.voronoi.output}
\end{figure}

Surprisingly, the produced triangulations are not the same, which explains the
divergence in the dependent Voronoi Diagrams.  \Ac{CGAL}'s implementation
produces a few more edges, or perhaps it should be said the other way around:
\texttt{VoronoiDelaunay.jl} does not produce every possible edge.  Simpler point
clouds illustrate this disparity further.  If tested with a 3-point cloud, i.e.,
a triangle, \ac{CGAL} will properly produce said triangle while
\texttt{VoronoiDelaunay.jl} will not.  \Cref{fig:eval.voronoi.surprising} shows
this scenario blatantly.  This example uses the same code in
\cref{lst:appendix.voronoi.plotsjl} with a minor difference in the number of
points changed from the default of 21 down to 3.

\begin{figure}[htb]
  \resizebox{\linewidth}{!}{\begin{tikzpicture}[/tikz/background rectangle/.style={fill={rgb,1:red,1.0;green,1.0;blue,1.0}, draw opacity={1.0}}, show background rectangle]
\begin{axis}[point meta max={nan}, point meta min={nan}, title={Delaunay Triangulation}, title style={at={{(0.5,1)}}, anchor={south}, font={{\fontsize{10 pt}{13.0 pt}\selectfont}}, color={rgb,1:red,0.0;green,0.0;blue,0.0}, draw opacity={1.0}, rotate={0.0}}, legend style={color={rgb,1:red,0.0;green,0.0;blue,0.0}, draw opacity={1.0}, line width={1}, solid, fill={rgb,1:red,1.0;green,1.0;blue,1.0}, fill opacity={1.0}, text opacity={1.0}, font={{\fontsize{8 pt}{10.4 pt}\selectfont}}, text={rgb,1:red,0.0;green,0.0;blue,0.0}, cells={anchor={west}}, at={(0.98, 0.98)}, anchor={north east}}, axis background/.style={fill={rgb,1:red,1.0;green,1.0;blue,1.0}, opacity={1.0}}, anchor={north west}, xshift={1.0mm}, yshift={-1.0mm}, width={86.9mm}, height={86.9mm}, scaled x ticks={false}, xlabel={$x$}, x tick style={color={rgb,1:red,0.0;green,0.0;blue,0.0}, opacity={1.0}}, x tick label style={color={rgb,1:red,0.0;green,0.0;blue,0.0}, opacity={1.0}, rotate={0}}, xlabel style={at={(ticklabel cs:0.5)}, anchor=near ticklabel, font={{\fontsize{11 pt}{14.3 pt}\selectfont}}, color={rgb,1:red,0.0;green,0.0;blue,0.0}, draw opacity={1.0}, rotate={0.0}}, xmajorgrids={true}, xmin={0.9}, xmax={2.1}, xtick={{1.0,1.25,1.5,1.75,2.0}}, xticklabels={{$1.00$,$1.25$,$1.50$,$1.75$,$2.00$}}, xtick align={inside}, xticklabel style={font={{\fontsize{8 pt}{10.4 pt}\selectfont}}, color={rgb,1:red,0.0;green,0.0;blue,0.0}, draw opacity={1.0}, rotate={0.0}}, x grid style={color={rgb,1:red,0.0;green,0.0;blue,0.0}, draw opacity={0.1}, line width={0.5}, solid}, axis x line*={left}, x axis line style={color={rgb,1:red,0.0;green,0.0;blue,0.0}, draw opacity={1.0}, line width={1}, solid}, scaled y ticks={false}, ylabel={$y$}, y tick style={color={rgb,1:red,0.0;green,0.0;blue,0.0}, opacity={1.0}}, y tick label style={color={rgb,1:red,0.0;green,0.0;blue,0.0}, opacity={1.0}, rotate={0}}, ylabel style={at={(ticklabel cs:0.5)}, anchor=near ticklabel, font={{\fontsize{11 pt}{14.3 pt}\selectfont}}, color={rgb,1:red,0.0;green,0.0;blue,0.0}, draw opacity={1.0}, rotate={0.0}}, ymajorgrids={true}, ymin={0.9}, ymax={2.1}, ytick={{1.0,1.25,1.5,1.75,2.0}}, yticklabels={{$1.00$,$1.25$,$1.50$,$1.75$,$2.00$}}, ytick align={inside}, yticklabel style={font={{\fontsize{8 pt}{10.4 pt}\selectfont}}, color={rgb,1:red,0.0;green,0.0;blue,0.0}, draw opacity={1.0}, rotate={0.0}}, y grid style={color={rgb,1:red,0.0;green,0.0;blue,0.0}, draw opacity={0.1}, line width={0.5}, solid}, axis y line*={left}, y axis line style={color={rgb,1:red,0.0;green,0.0;blue,0.0}, draw opacity={1.0}, line width={1}, solid}, colorbar={false}, legend columns={-1}]
    \addplot[color={rgb,1:red,1.0;green,0.0;blue,0.0}, name path={8223002b-7729-4b6e-a2c9-1769cec69fa1}, draw opacity={1.0}, line width={1}, solid]
        table[row sep={\\}]
        {
            \\
            1.4235251017380506  1.5806863505858928  \\
            1.8946103360503908  1.4291503357235342  \\
        }
        ;
    \addlegendentry {Voronoi.jl}
    \addplot[color={rgb,1:red,1.0;green,0.0;blue,0.0}, name path={8223002b-7729-4b6e-a2c9-1769cec69fa1}, draw opacity={1.0}, line width={1}, solid, forget plot]
        table[row sep={\\}]
        {
            \\
            1.1387269033584746  1.3902763912919909  \\
            1.8946103360503908  1.4291503357235342  \\
        }
        ;
    \addplot[color={rgb,1:red,1.0;green,0.0;blue,0.0}, name path={8223002b-7729-4b6e-a2c9-1769cec69fa1}, draw opacity={1.0}, line width={1}, solid, forget plot]
        table[row sep={\\}]
        {
            \\
            1.4235251017380506  1.5806863505858928  \\
            1.1387269033584746  1.3902763912919909  \\
        }
        ;
\end{axis}
\begin{axis}[point meta max={nan}, point meta min={nan}, title={Voronoi Diagram}, title style={at={{(0.5,1)}}, anchor={south}, font={{\fontsize{10 pt}{13.0 pt}\selectfont}}, color={rgb,1:red,0.0;green,0.0;blue,0.0}, draw opacity={1.0}, rotate={0.0}}, legend style={color={rgb,1:red,0.0;green,0.0;blue,0.0}, draw opacity={1.0}, line width={1}, solid, fill={rgb,1:red,1.0;green,1.0;blue,1.0}, fill opacity={1.0}, text opacity={1.0}, font={{\fontsize{8 pt}{10.4 pt}\selectfont}}, text={rgb,1:red,0.0;green,0.0;blue,0.0}, cells={anchor={west}}, at={(0.98, 0.98)}, anchor={north east}}, axis background/.style={fill={rgb,1:red,1.0;green,1.0;blue,1.0}, opacity={1.0}}, anchor={north west}, xshift={89.9mm}, yshift={-1.0mm}, width={86.9mm}, height={86.9mm}, scaled x ticks={false}, xlabel={$x$}, x tick style={color={rgb,1:red,0.0;green,0.0;blue,0.0}, opacity={1.0}}, x tick label style={color={rgb,1:red,0.0;green,0.0;blue,0.0}, opacity={1.0}, rotate={0}}, xlabel style={at={(ticklabel cs:0.5)}, anchor=near ticklabel, font={{\fontsize{11 pt}{14.3 pt}\selectfont}}, color={rgb,1:red,0.0;green,0.0;blue,0.0}, draw opacity={1.0}, rotate={0.0}}, xmajorgrids={true}, xmin={0.12476873002050426}, xmax={3.1965007103949628}, xtick={{0.5,1.0,1.5,2.0,2.5,3.0}}, xticklabels={{$0.5$,$1.0$,$1.5$,$2.0$,$2.5$,$3.0$}}, xtick align={inside}, xticklabel style={font={{\fontsize{8 pt}{10.4 pt}\selectfont}}, color={rgb,1:red,0.0;green,0.0;blue,0.0}, draw opacity={1.0}, rotate={0.0}}, x grid style={color={rgb,1:red,0.0;green,0.0;blue,0.0}, draw opacity={0.1}, line width={0.5}, solid}, axis x line*={left}, x axis line style={color={rgb,1:red,0.0;green,0.0;blue,0.0}, draw opacity={1.0}, line width={1}, solid}, scaled y ticks={false}, ylabel={$y$}, y tick style={color={rgb,1:red,0.0;green,0.0;blue,0.0}, opacity={1.0}}, y tick label style={color={rgb,1:red,0.0;green,0.0;blue,0.0}, opacity={1.0}, rotate={0}}, ylabel style={at={(ticklabel cs:0.5)}, anchor=near ticklabel, font={{\fontsize{11 pt}{14.3 pt}\selectfont}}, color={rgb,1:red,0.0;green,0.0;blue,0.0}, draw opacity={1.0}, rotate={0.0}}, ymajorgrids={true}, ymin={1.010882628426966}, ymax={2.112657908518973}, ytick={{1.2000000000000002,1.4000000000000001,1.6,1.8,2.0}}, yticklabels={{$1.2$,$1.4$,$1.6$,$1.8$,$2.0$}}, ytick align={inside}, yticklabel style={font={{\fontsize{8 pt}{10.4 pt}\selectfont}}, color={rgb,1:red,0.0;green,0.0;blue,0.0}, draw opacity={1.0}, rotate={0.0}}, y grid style={color={rgb,1:red,0.0;green,0.0;blue,0.0}, draw opacity={0.1}, line width={0.5}, solid}, axis y line*={left}, y axis line style={color={rgb,1:red,0.0;green,0.0;blue,0.0}, draw opacity={1.0}, line width={1}, solid}, colorbar={false}, legend columns={-1}]
    \addplot[color={rgb,1:red,0.0;green,1.0;blue,0.498}, name path={4a9a37d1-bb44-42f3-82a7-1a36a3143ab8}, draw opacity={0.8}, line width={1}, solid]
        table[row sep={\\}]
        {
            \\
            3.109564899629648  1.4999999999999998  \\
            1.532938482009341  1.1128160355579375  \\
        }
        ;
    \addlegendentry {VoronoiDelaunay.jl}
    \addplot[color={rgb,1:red,0.0;green,1.0;blue,0.498}, name path={4a9a37d1-bb44-42f3-82a7-1a36a3143ab8}, draw opacity={0.8}, line width={1}, solid, forget plot]
        table[row sep={\\}]
        {
            \\
            3.109564899629648  1.4999999999999998  \\
            1.738886336883075  1.753053233015188  \\
        }
        ;
    \addplot[color={rgb,1:red,0.0;green,1.0;blue,0.498}, name path={4a9a37d1-bb44-42f3-82a7-1a36a3143ab8}, draw opacity={0.8}, line width={1}, solid, forget plot]
        table[row sep={\\}]
        {
            \\
            1.1315016278762886  1.7092761372330472  \\
            1.5  2.081475589271086  \\
        }
        ;
    \addplot[color={rgb,1:red,0.0;green,1.0;blue,0.498}, name path={4a9a37d1-bb44-42f3-82a7-1a36a3143ab8}, draw opacity={0.8}, line width={1}, solid, forget plot]
        table[row sep={\\}]
        {
            \\
            1.1315016278762886  1.7092761372330472  \\
            1.531341804702409  1.1112309382207224  \\
        }
        ;
    \addplot[color={rgb,1:red,0.0;green,1.0;blue,0.498}, name path={4a9a37d1-bb44-42f3-82a7-1a36a3143ab8}, draw opacity={0.8}, line width={1}, solid, forget plot]
        table[row sep={\\}]
        {
            \\
            1.1315016278762886  1.7092761372330472  \\
            0.21170454078581913  1.5  \\
        }
        ;
    \addplot[color={rgb,1:red,0.0;green,1.0;blue,0.498}, name path={4a9a37d1-bb44-42f3-82a7-1a36a3143ab8}, draw opacity={0.8}, line width={1}, solid, forget plot]
        table[row sep={\\}]
        {
            \\
            1.5  1.0420649476748531  \\
            1.531341804702409  1.1112309382207224  \\
        }
        ;
    \addplot[color={rgb,1:red,0.0;green,1.0;blue,0.498}, name path={4a9a37d1-bb44-42f3-82a7-1a36a3143ab8}, draw opacity={0.8}, line width={1}, solid, forget plot]
        table[row sep={\\}]
        {
            \\
            1.5  1.0420649476748531  \\
            0.21170454078581913  1.5  \\
        }
        ;
    \addplot[color={rgb,1:red,0.0;green,1.0;blue,0.498}, name path={4a9a37d1-bb44-42f3-82a7-1a36a3143ab8}, draw opacity={0.8}, line width={1}, solid, forget plot]
        table[row sep={\\}]
        {
            \\
            1.531341804702409  1.1112309382207224  \\
            1.532938482009341  1.1128160355579375  \\
        }
        ;
    \addplot[color={rgb,1:red,0.0;green,1.0;blue,0.498}, name path={4a9a37d1-bb44-42f3-82a7-1a36a3143ab8}, draw opacity={0.8}, line width={1}, solid, forget plot]
        table[row sep={\\}]
        {
            \\
            1.5  2.081475589271086  \\
            1.738886336883075  1.753053233015188  \\
        }
        ;
    \addplot[color={rgb,1:red,0.0;green,1.0;blue,0.498}, name path={4a9a37d1-bb44-42f3-82a7-1a36a3143ab8}, draw opacity={0.8}, line width={1}, solid, forget plot]
        table[row sep={\\}]
        {
            \\
            1.532938482009341  1.1128160355579375  \\
            1.738886336883075  1.753053233015188  \\
        }
        ;
\end{axis}
\end{tikzpicture}
}
  \caption[Surprising Voronoi Delaunay output]{
    \ac{CGAL}'s algorithm produced a triangle (on the left) while
    \texttt{VoronoiDelaunay.jl} did not.  Consequently, \ac{CGAL} did not
    produce a (finite) Voronoi Diagram while \texttt{VoronoiDelaunay.jl} did (on
    the right).}%
  \label{fig:eval.voronoi.surprising}
\end{figure}

It is not clear which of the implementations is wrong.  Though, given that
\ac{CGAL} is considered the \textit{de facto} industry standard library for
geometric computation, it is probably safe to assume there is something wrong
with the implementation of the algorithm in \texttt{VoronoiDelaunay.jl}.

Summarily, though we do not believe it to be entirely wrong to implement
(especially complex) algorithms and functionality from scratch, we have seen how
it can rapidly prove problematic, making it sometimes less preferable.  From our
benchmarks, there was no performance gain.  As a matter of fact, the
re-implementation was slower.  The effort put into repurposing an already
existing algorithm from what is considered to be a rather complex library is
still far lower than the effort required to implement a solid, robust version of
the algorithm from scratch.  Differences between the outputs validate our
argument here that new implementations will inevitably lead to an emergence of
erroneous behavior, whereas more established and battle-tested software will
probably have taken care of many of the issues that are still to become
apparent.

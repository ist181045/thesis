% !TEX root = ../../main.tex
\begin{abstract}
Modern Computer-Aided Design applications need to employ, to a lesser or greater
extent, geometric constraints that condition the geometric models being
produced.  As an example, consider that of a line that goes through a point and
is parallel to another line, or constructing a circle from three points.  The
specification (and solving) of geometric constraints allows the creation of
geometric shapes that, otherwise, would require substantially more complex
descriptions.  The inclusion, in a programming language, of primitive operations
for creating geometric constraints augments the expressiveness of the language
and frees the programmer from the specification of otherwise apparently
irrelevant details.  Said primitive operations could include the definition of
incidence relations, parallelism or perpendicularity, distances, angles, etc.,
to which the various parts in a geometric model must obey.  The focus of this
work is the creation and implementation, of geometric constraint primitives that
facilitate the specification of geometric forms.
\end{abstract}

\begin{keywords}
\noindent
Algorithmic Design;
Geometric Constraints;
Geometric Constraint Solving;
Parametric CAD;
Programming Language.
\end{keywords}

\clearpage
\thispagestyle{empty}

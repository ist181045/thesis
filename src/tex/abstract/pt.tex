% !TEX root = ../../main.tex
\begin{pt}
\begin{resumo}

Aplicações modernas de projecto assisto por computador precisam de empregar, em
menor ou maior grau, restrições geométricas que condicionam os modelos
produzidos.  Todavia, estas aplicações provam ser insuficientes para a produção
de modelos complexos e sofisticados.  Como resposta a esta limitação, surgiu um
novo paradigma nomeado Algorithmic Design (AD).  O paradigma consiste na criação
de modelos provenientes de descrições algorítmicas, permitindo a automatização
de tarefas repetitivas.  Infelizmente, não é um paradigma tão difundido como
métodos mais tradicionais, parcialmente devido ao tempo, esforço, e conhecimento
adicional necessário para especificar relações entre objectos.  Isto pode ser
mitigado através da incorporação de funcionalidade de restrição geométrica em
ferramentas de AD para ajudar preencher a lacuna entre AD e paradigmas mais
tradicionais.  O foco deste trabalho assenta sobre a criação, e implementação,
de funcionalidade primitiva de restrição geométrica, suportada por uma
biblioteca de computação geométrica exacta, que facilita a especificação de
formas geométricas.  Produzimos testes de referência de desempenho da nossa
solução, também têmo-la testado com quatro formas geométricas dominadas por
restrições inspiradas por projectos existentes, salientado duas abordagens
diferentes: uma abordagem analítica, naturalmente usada em programação, e uma
abordagem constructiva, em que se baseia a nossa solução.  Adicionalmente,
exploramos efeitos secundários benéficos da nossa implementação relativamente ao
reaproveitamento de funcionalidade mais complexa com muito pequeno esforço
extra.  Concluímos que a abordagem empregue pela nossa solução prova ser mais
compreensível e intuitiva para praticantes.

\end{resumo}

\begin{palavraschave}
\noindent
CAD Paramétrico;
Restrições Geométricas;
Algorithmic Design;
Computação Exacta;
Geometria Construtiva.
\end{palavraschave}
\end{pt}

\clearpage
\thispagestyle{empty}

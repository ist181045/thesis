% !TEX root = ../../main.tex
\begin{resumo}
\noindent
Aplicações de \acl{CAD} modernas precisam de empregar, a um menor ou maior grau,
restrições geométricas que condicionam os modelos geométricos produzidos.  A
título de exemplo, considere-se uma linha que atravessa um ponto e é paralela a
uma outra linha, ou a construção de um círculo com recurso a três pontos.  A
especificação (e resolução) de restrições geométricas permite a criação de
formas geométricas que, de outro modo, exigiriam descrições substancialmente
mais complexas.  A inclusão, numa linguagem de programação, de operações
primitivas capazes de criar restrições geométricas aumenta a expressividade da
linguagem e liberta o programador de especificar detalhes aparentemente
irrelevantes.  Ditas operações podem incluir a definição de relações de
incidência, paralelismo ou perpendicularidade, distâncias, e ângulos; às quais o
modelo geométrico deve obedecer.  O foco deste trabalho consiste na criação e
implementação de primitivas de restrições geométricas que facilitam a
especificação de formas geométricas.

\end{resumo}

\begin{palavraschave}
\noindent
Design Algorítmico;
Restrições Geométricas;
Resolução de Restrições Geométricas;
\acs{CAD} Paramétrico;
Linguagem de Programação
\end{palavraschave}

\clearpage
\thispagestyle{empty}

% !TEX root = ../../main.tex
\begin{pt}
\begin{resumo}
Aplicações modernas de projecto assistido por computador precisam de empregar,
em menor ou maior grau, restrições geométricas que condicionam os modelos
produzidos.  Como exemplos, considerem-se o de uma linha que atravessa um ponto
e é paralela a outra linha, ou o da construção de um círculo recorrendo a três
pontos.  A especificação (e resolução) destas restrições permite a criação de
formas geométricas que, de outro modo, exigiriam descrições substancialmente
mais complexas.  A inclusão, numa linguagem de programação, de operações
primitivas capazes de criar restrições geométricas aumenta a expressividade da
linguagem e liberta o programador de especificar detalhes desnecessários.  Essas
operações podem incluir a definição de relações de incidência, paralelismo ou
perpendicularidade, distâncias, e ângulos; às quais o modelo geométrico deve
obedecer.  O foco deste trabalho consiste na criação e implementação de
primitivas de restrições geométricas que facilitam a especificação de formas
geométricas.
\end{resumo}

\begin{palavraschave}
\noindent
Design Algorítmico;
Restrições Geométricas;
Resolução de Restrições Geométricas;
CAD Paramétrico;
Linguagem de Programação.
\end{palavraschave}
\end{pt}

\clearpage
\thispagestyle{empty}

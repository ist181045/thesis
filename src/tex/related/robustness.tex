% !TEX root = ../../main.tex
\subsection{Robustness}%
\label{sec:related.robustness}

The correctness proofs of most all geometric algorithms presented in theoretical
papers assumes exact computation with real numbers~\cite{CGAL:4.13:23LGK}.
However, floating-point numbers are represented with fixed precision in
computers, which leads to inaccurate representations of real number.  Typically,
comparisons must be performed relying on tolerances, i.e., two floating-point
numbers $a$ and $b$ are considered \textit{the same} if $|a - b| \le \epsilon$
for a given $\epsilon$.

When used without care, fixed-precision arithmetic almost always leads to
unwanted results due to marginal error accumulation caused by rounding
(\textit{roundoff}).  To help solve this problem, more robust numerical
constructs and concepts can be used.  In particular, exact numbers, such as
rational numbers or arbitrary precision numbers. The latter allow
arbitrary-precision arithmetic with the drawback that operations are slower,
however mitigating precision issues.

Several libraries already strive to implement robust geometric computation.  One
such example is \ac{CGAL}~\cite{CGAL:5.3:Project}.  Moreover, other libraries,
such as \ac{LEDA}~\cite{Mehlhorn:1989:LEDA}, and
CORE~\cite{Karamcheti:1999:CLRNGC} and its successor~\cite{Yu:2010:CORE2}, can
also be leveraged to deal with robustness problems in geometric computation.

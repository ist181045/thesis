% !TEX root = ../main.tex
\section{Introduction}%
\label{sec:intro}

Modern \ac{CAD} tools include substantial support for parametric operations and
\ac{GCS}.  These mechanisms have been developed over the past few
decades~\cite{Bettig:2011:GCSPC} and are heavily and ubiquitously used across
the Architecture, Engineering, and Construction industry.

Parametric modeling is used to design with constraints.  Users express a set of
parameters and operations, establishing restrictions between geometric entities.
The resulting geometry can be controlled from input parameters using two
computational mechanisms:
\begin{enumerate*}[label= (\arabic*)]
  \item parametric operations, and
  \item \ac{GCS}.
\end{enumerate*}

However, traditional interactive methods for parametric modeling suffer from the
disadvantage that they do not scale properly when designing more complex ideas.
In recent years, a novel approach to design named \ac{AD} has emerged, allowing
the specification of sketches and models through
algorithms~\cite{McCormack:2004:GDPDR} using either a \ac{TPL}, a \acp{VPL}, or
even a mixture of both.

Dealing with \acp{GC} can still prove to be an arduous task.  Take as an example
the sketch of a chair seat's outer frame, as seen in \cref{fig:intro.chair},
from a multi-purpose chair generation tool~\cite{Garcia:2012:ChairDNA} where the
chair's overall shape is controllable by specifying the values for a set of
input parameters.

\begin{figure}[htb]
  \includegraphics[width=.7\linewidth]{fig/chair-seat-outer-frame}
  \begin{minipage}{\linewidth}
  \scriptsize Source: Project source code, publicly unavailable.
  \end{minipage}
  \caption[Sketch of a chair seat's outer frame]{\label{fig:intro.chair}
    Sketch of a chair seat's outer frame, defined by several input parameters.}
\end{figure}

The seat's corners are defined by circles whose respective front and rear radii
are obtained by computing distances from which the circles' centers can be
obtained.  The circles are then connected through outer tangent lines, forming
the outer frame of the chair's seat.  Some of these operations, such as point
distance and tangency, must be handled carefully due to numerical robustness
issues that may arise when performing fixed-precision arithmetic computations.
This means that, on top of the design process itself, the user must spend
additional time identifying and carefully researching how to robustly implement
solutions to \acp{GC}.

To overcome this problem, this report proposes the implementation of \ac{GC}
primitives with specialized efficient solutions for different combinations of
input objects.  We additionally focus our work around \acp{TPL}, further making
them more attractive, and easier to both adopt and use.

% !TEX root = ../../main.tex
\subsection{Parametric Operations in CAD}%
\label{sec:intro.parametric}

Ivan Sutherland introduced the world to
Sketchpad~\cite{Sutherland:1964:Sketchpad} in 1963, an interactive 2D \ac{CAD}
program.  Sutherland's Sketchpad was capable of establishing atomic constraints
between objects, being the first of its kind.  The earliest 3D
system~\cite{Requicha:1980:RRS:356827.356833} dates from the 1970s.  This
system's parametric nature rested on a construction history tree.  The user
could modify an operation's parameters' values, reapply the modified history,
and regenerate the model.  Nearly a decade later,
Pro/ENGINEER\footnote{\url{https://www.ptc.com/en/products/creo/pro-engineer}}~\cite{Jabi:2013:PDA}
surfaced, enabling the creation of relations between the objects' sizes and
positions such that a change in a dimension between objects would automatically
change affected objects accordingly.  \Ac{GCS} soon became standard in drawings
by the early 1990s~\cite{Owen:1991:ASGDC,Bouma:1995:GCS}.  Efforts to expand the
benefits of \ac{GCS} beyond simple sketches were made, some systems having
implemented constraint solving in 3D.  Improvements from then on focused mostly
on robustness and operation variety.

In recent decades, emphasis shifted towards making parametric \ac{CAD} software
more interactive and user-friendly.  The intent was to make it as simple as
dragging a face of an object to where it should be instead of locating and
modifying operations buried in a construction history.  This is a tedious and
error-prone process that can lead to undesired side effects.  Nonetheless,
parametric operations will still see continued usage for the foreseeable future.

% !TEX root = ../../main.tex
\subsection{Constraints in CAD}%
\label{sec:intro.constraints}

Parametric operations allow the user to create geometric objects that satisfy
certain constraints \emph{implicitly} imposed on the objects when the user
selects the operation they want.  \Acp{GC}, on the other hand, allow the
repositioning and scaling of objects so that they satisfy constraints the user
\emph{explicitly} imposed on them.

The abstract problem of \ac{GCS} consists of assigning coordinates to
constrained geometric objects such that the constraints they are subject to are
satisfied.  Otherwise, the solver should report no such assignment can be found.

One of the important features of a solver is its \emph{competence}, which is
related to the capability of reporting unsolvability: if no solution for the
problem exists and the solver is capable of reporting unsolvability, the solver
is deemed fully competent.  Since constraint solving is mostly an exponentially
complex problem~\cite{Rossi:2006:Handbook}, partial competence suffices as long
as decent solutions can be found in affordable time and space.

In the context of \ac{GCS}, it is also important that the \ac{GC} system does
not have too few or too many constraints.  Summarily, a system can either be 
\begin{enumerate*}[label= (\arabic*)]
  \item under-constrained, if the number of solutions is unbound due to lack of
  constraint coverage;
  \item over-constrained, if there are no solutions because of contradictions;
  or
  \item well-constrained, if the number of solutions is finite.
\end{enumerate*}

Some of the subjects approached here are briefed in~\cite{Hoffmann:2005:BCS}.
The following sections present and briefly discuss the most relevant approaches
to constraint solving.

\subsubsection{Graph-Based Approaches}%
\label{sec:intro.constraints.graph}

The problem is translated into a labeled \textit{constraint graph}, where
vertices are constrained geometric objects, and edges the constraints
themselves.  These became the dominant \ac{GCS} approaches.

\subsubsection{Logic-Based Approaches}%
\label{sec:intro.constraints.logic}

The constraint problem is translated into a set of geometric assertions and
axioms which is then transformed in such a way that specific solution steps are
made explicit by applying geometric reasoning.  The solver then takes a set of
construction steps and assigns coordinate values to the geometric entities.

\subsubsection{Algebraic Approaches}%
\label{sec:intro.constraints.algebraic}

The problem is translated into a system of equations, which is generally
nonlinear.  This approach's main advantage is its completeness and dimension
independence.  However, it is difficult to decompose the equation system into
subproblems, and a general, complete solution of algebraic equations is
inefficient.  Nonetheless, small algebraic systems tend to appear in the other
approaches and are routinely solved.

\subsubsection{Symbolic Methods}%
\label{sec:intro.constraints.symbolic}

Symbolic methods rely on general equation solvers that employ techniques to
triangularize equation systems~\cite{Chou:1988:IWMMTPG,Buchberger:1995:Grobner}
that emerge from employing an algebraic approach.  These methods can produce
generic solutions, but solvers are very slow and computation demands a lot of
space, usually requiring exponential running time~\cite{Durand:1998:SNTCS}.

\subsubsection{Numerical Methods}%
\label{sec:intro.constraints.numerical}

Among the oldest approaches to constraint solving, numerical methods solve large
systems of equations iteratively.  Methods like Newton iteration work properly
if a good approximation of the intended solution can be supplied and the system
is not ill-conditioned.  Alas, such methods may find only one solution, even in
cases where there are many, and may not allow the user to select the one they
are interested in.

\subsubsection{Theorem Proving}%
\label{sec:intro.constraints.proving}

\ac{GCS} can be seen as a subproblem of geometric theorem proving, but the
latter requires general techniques, therefore requiring much more complex
methods than those required by the former.

% !TEX root = ../../main.tex
\section{\acl{GC} Problem Examples}%
\label{sec:intro.examples}

This section presents two simple examples of geometric models that are defined
through the specification of \acp{GC}, and the respective solutions using
intuitive algebraic formulation, accompanied by programmatic solutions using
\acs{TikZ}~\cite{Tantau:2021:TikZ} and Eukleides~\cite{Obrecht:2010:EM}.
Depictions of the aforementioned models can be seen in \cref{fig:intro.example}.
The examples are limited to the two-dimensional Euclidean plane over real
numbers, $\mathbb{R}^2$.  Solutions for analogous problems in three-dimensional
Euclidean space, $\mathbb{R}^3$, exist as well.

\begin{figure}[htpb]
  \subcaptionbox{\label{fig:intro.example.parallel}%
    Line that goes through $C$, strictly parallel to $\overleftrightarrow{AB}$.}
    [.45\linewidth]{\resizebox{!}{.2\textheight}{%
      \begin{tikzpicture}[rotate=20]
  \tkzDefPoints{0/0/A,3/0/B,1/1/C}
  \tkzDefLine[parallel=through C](A,B) \tkzGetPoint{D}
  \tkzDrawLines[add=.1 and .1](A,B C,D)
  \tkzDrawPoints(A,B,C)
  \tkzLabelPoints(A,B,C)
\end{tikzpicture}
}}
  \hspace{\fill}
  \subcaptionbox{\label{fig:intro.example.circumcenter}%
    $\odot O_r$ circumscribed about $\triangle ABC$.}
    [.45\linewidth]{\resizebox{!}{.2\textheight}{%
      \begin{tikzpicture}
  \tkzDefPoints{0/0/A,4/0/B,1/3/C}
  \tkzCircumCenter(A,B,C) \tkzGetPoint{O}
  \tkzDefMidPoint(A,B) \tkzGetPoint{AB}
  \tkzDefMidPoint(A,C) \tkzGetPoint{AC}
  \tkzDefMidPoint(B,C) \tkzGetPoint{BC}
  \tkzDrawSegments[style=dashed](AB,O AC,O BC,O)
  \tkzMarkRightAngles(A,AB,O B,BC,O C,AC,O)
  \tkzDrawPolygon(A,B,C)
  \tkzDrawCircle(O,A)
  \tkzDrawSegment(O,A)
  \tkzDrawPoints(A,B,C,O)
  \tkzLabelLine[above](O,A){$r$}
  \tkzLabelPoints[below left](A)
  \tkzLabelPoints[below right](B)
  \tkzLabelPoints[above left](C)
  \tkzLabelPoints(O)
\end{tikzpicture}
}}
  \caption[Geometric models defined using GCs]{
    Geometric models defined using \ac{GC} relations:
    \subref{fig:intro.example.parallel} showcases line parallelism, and
    \subref{fig:intro.example.circumcenter} showcases a circle circumscription
    about a triangle.}\label{fig:intro.example}
\end{figure}

The first problem is that of a parallelism constraint: specifying a line that
goes through a given point while also being strictly parallel to another already
defined line.  The second problem is a circumscription constraint: defining a
circle that tightly wraps around a triangle, i.e., the circle's circumference
goes through three given non-collinear points.

\subsection{Parallel lines}%
\label{sec:intro.examples.parallel}

Let $A,\,B,\,C \in \mathbb{R}^2$ such that $C$ is a point in the line which is
strictly parallel to the line $\overleftrightarrow{AB}$ (see
\cref{fig:intro.example.parallel}).

A line in $\mathbb{R}^2$ can be described by the parametric equation
\begin{equation}\label{eq:line.parametric.2}
  P_Q = Q + \lambda\vec{u} \Rightarrow
  \begin{cases}
    x = x_Q + \lambda u_x \\
    y = y_Q + \lambda u_y
  \end{cases},\,\lambda \in \mathbb{R}
\end{equation}\equations{Parametric equation of a line in $\mathbb{R}^2$}
where $Q = (x_Q, y_Q)$ is a point on the line that goes through $P_Q = (x, y)$,
and $\vec{u} = (u_x, u_y)$ is the vector that drives the line.  To then describe
the line that goes through $C$ and is parallel to $\overleftrightarrow{AB}$, one
must compute the base point $Q$, trivially $C$, and the directional vector
$\vec{u}$, which can be obtained from $\overleftrightarrow{AB}$.  Let $Q = C$,
and $\vec{u} = B - A$, such that
\[ P_C = C + \lambda \vec{u},\,\lambda \in \mathbb{R}. \]

\Cref{lst:intro.example.parallel.tikz} shows the code used to produce the
example shown in \cref{fig:intro.example.parallel} using \acs{TikZ} with the
\texttt{tkz-euclide}\footnote{\url{https://ctan.org/pkg/tkz-euclide}} \LaTeX{}
package, using \texttt{tkzDefLine}, which takes two points, $A$ and $B$, with
the \texttt{parallel} transformation option.  This option takes the point $C$
the resulting line goes through.  The result is a point $D = C + \vec{u}$, which
can be obtained using \texttt{tkzGetPoint} to later draw the line. 

\begin{listing}[htb]
  \inputminted[highlightlines=3]{latex}{tikz/ex-parallel.tikz}
  \caption[Parallel lines example using \texttt{tkz-euclide}]{
    Parallel lines example from \cref{fig:intro.example.parallel} using
    \texttt{tkz-euclide}.  The highlighted line shows how to define the line
    $L_C$ parallel to $\overleftrightarrow{AB}$.}%
  \label{lst:intro.example.parallel.tikz}
\end{listing}

\Cref{lst:intro.example.parallel.euk} shows the code used to produce an
identical figure using Eukleides.  In Eukleides, the parallel line $L_C$ can be
obtained through the \texttt{parallel} function, which takes the line
$\overleftrightarrow{AB}$ it is parallel to and the point $C$ it goes through.

\begin{listing}[htb]
  \inputminted[highlightlines=3]{text}{euk/ex-parallel.euk}
  \caption[Parallel lines example using Eukleides]{
    Parallel lines example from \cref{fig:intro.example.parallel} using
    Eukleides.  The highlighted line shows how to define the line $L_C$
    parallel to $\overleftrightarrow{AB}$.}%
  \label{lst:intro.example.parallel.euk}
\end{listing}

\begin{comment}
We can determine if two lines are parallel by determining the angle $\theta$
between them, and verifying it is equal to $0$.
%
\begin{equation}\label{eq:angle.vectors.2}
  \theta = \arccos \frac{\vec{u} \cdot \vec{v}}%
                        {||\vec{u}|| \cdot ||\vec{v}||},~%
  \theta \in \mathbb{R}.
\end{equation}
%
Having $\vec{v} = \lambda \vec{u}, \lambda \in \mathbb{R}$, and knowing
%
\begin{equation}\label{eq:dot.vector.2.same}
  \vec{u} \cdot \vec{u} = ||\vec{u}||^2,
\end{equation}
%
then \eqref{eq:angle.vectors.2} becomes
%
\[
  \begin{split}
    \theta & = \arccos \frac{\vec{u} \cdot \lambda\vec{u}}%
                            {||\vec{u}|| \cdot ||\lambda\vec{u}||}\\%
           & = \arccos \frac{\lambda\vec{u} \cdot \vec{u}}%
                            {\lambda||\vec{u}||^2}\\%
           & = 0.
  \end{split}
\]
%
This means that we can compute a directional vector for $CS$ from the line $AB$,
where $\vec{u} = B - A$.  Finally, we can obtain the equation for the parallel
line $CS$
\end{comment}

\subsection{Circumcenter}%
\label{sec:intro.examples.circumcenter}

Let $A,\,B,\,C,\,O \in \mathbb{R}^2$ be points such that $O$ is the center point
of a circle of radius $r$, $\odot O_r$, that is circumscribed about the triangle
$\triangle ABC$ (see \cref{fig:intro.example.circumcenter}).

A precondition for this computation is that $\triangle ABC$ is not degenerate,
i.e., its vertices are non-collinear.  That can be verified by computing the
cross product of any two distinct vectors that drive $\triangle ABC$'s edges and
verifying it does not equate to zero.

\begin{comment}
The Laplace expansion for the determinant of a generic matrix $A \in
\mathbb{R}^{n \times n}$ is given by
%
\begin{align}
  \det(A) = \begin{vmatrix}
              a_{11} & \cdots & a_{1n} \\
              \vdots & \ddots & \vdots \\
              a_{n1} & \cdots & a_{nn}
            \end{vmatrix}
          &= a_{1j} C_{1j} + \cdots + a_{nj} C_{nj}%
            = a_{i1} C_{i1} + \cdots + a_{in} C_{in} \nonumber \\
          &= \sum_{i'=1}^n a_{i'j}C_{i'j}%
            \label{eq:matrix.det.laplace.nxn.col} \\
          &= \sum_{j'=1}^n a_{ij'}C_{ij'}%
            \label{eq:matrix.det.laplace.nxn.row},
\end{align}
%
where $i,j \in [1,n] \subset \mathbb{N}$,
%
\begin{equation}\label{eq:matrix.cofactor}
  C_{ij} = (-1)^{i+j} \det(A_{ij})
\end{equation}
%
is the $i,j$ cofactor of $A$, and
%
\begin{equation}\label{eq:matrix.minor}
  A_{ij} = \begin{bmatrix}
    a_{11}     & \cdots & a_{1j-1}     & a_{1j+1}     & \cdots & a_{1n}     \\%
    \vdots     & \ddots & \vdots       & \vdots       & \ddots & \vdots     \\%
    a_{i-1\,1} & \cdots & a_{i-1\,j-1} & a_{i-1\,j+1} & \cdots & a_{i-1\,1} \\%
    a_{i+1\,1} & \cdots & a_{i+1\,j-1} & a_{i+1\,j+1} & \cdots & a_{i+1\,1} \\%
    \vdots     & \ddots & \vdots       & \vdots       & \ddots & \vdots     \\%
    a_{n1}     & \cdots & a_{nj-1}     & a_{nj+1}     & \cdots & a_{nn}     \\
  \end{bmatrix}_{(n - 1) \times (n - 1)}
\end{equation}
%
is the minor of matrix $A$ without the $i$-th row and $j$-th column.  This can
be further simplified for a $2\times 2$ matrix.  Let $B\in\mathbb{R}^{2\times
2}$ be the matrix whose columns are the vectors $\vec{AB} = (a, b)$ and
$\vec{AC} = (c, d)$, and $i' = 1$, for instance, such that,
from~\eqref{eq:matrix.det.laplace.nxn.col},
%
\begin{equation}\label{eq:matrix.det.laplace.2x2.col}
  \det(B) = \begin{vmatrix}
              a & c\\
              b & d
            \end{vmatrix}
          = \sum_{j=1}^2 a_{1j} C_{1j}
          = (-1)^{1+1}a \cdot d + (-1)^{1+2}c \cdot b
          = ad - cb
\end{equation}
\end{comment}

\begin{comment}
Let $A \in \mathbb{R}^{2 \times 2}$ be the matrix whose columns are the vectors
$\vec{AB} = (a, b)$ and $\vec{AC} = (c, d)$, for instance, such that
%
\begin{equation}\label{eq:matrix.det.2x2}
  \det(A) = \begin{vmatrix}
              a & c\\
              b & d
            \end{vmatrix}%
          = ad - cb
\end{equation}
% 
If the determinant is found to be $0$, then there is no possible solution.
Otherwise, one can proceed to draw $\odot O_r$.
\end{comment}

To draw $\odot O_r$, we must compute both its center and radius.  Its radius $r$
can be trivially defined as the distance of the center $O$ to any of the
$\triangle ABC$'s vertices, i.e., $r = \left\|O - A\right\| = \left\|O -
B\right\| = \left\|O - C\right\|$.  To determine $O$, one must compute the
intersection of the perpendicular bisectors of the triangle's edges.  Said
bisectors are the mediators between an edge's vertices, which can be described
by~\cref{eq:line.parametric.2}, where $P$ is the midpoint between the vertices,
and $\vec{u}$ is a vector normal to the edge.  The midpoint $M_{P_1P_2}$ of two
points $P_1,\,P_2 \in \mathbb{R}^2$ is given by
\begin{equation}\label{eq:midpoint.2}
  M_{P_1P_2} = \frac{P_1 + P_2}{2}%
             = \left(\frac{x_1 + x_2}{2}, \frac{y_1 + y_2}{2}\right).
\end{equation}\equations{Midpoint between two points in $\mathbb{R}^2$}
Further, the scalar product of two vectors $\vec{u}, \vec{v} \in \mathbb{R}^2$
is given by
\begin{equation}\label{eq:vector.scalar.2}
  \vec{u} \cdot \vec{v} = (u_x, u_y) \cdot (v_x, v_y) = u_x v_x + u_y v_y.
\end{equation}\equations{Scalar product of vectors in $\mathbb{R}^2$}
The normal vector $\vec{n}$ is such that, for some vector $\vec{u}$,
\[ \vec{u} \cdot \vec{n} = 0. \]
A vector $\vec{n} \in \mathbb{R}^2$ normal to another vector $\vec{u}$ can be
easily obtained by swapping the components of $\vec{u}$ while negating one of
them, a property easily verified by applying \cref{eq:vector.scalar.2}.

\begin{comment}
This comes as a direct result from applying a rotation transformation of 90
degrees, or $\pi/2$ radians, to $\vec{u}$, like so
%
\[
  \vec{n} = R(\pi/2)\vec{u}%
  = \begin{bmatrix}
      \cos(\pi/2) & -\sin(\pi/2) \\
      \sin(\pi/2) & \cos(\pi/2)
    \end{bmatrix}
    \begin{bmatrix}
      u_1 \\ u_2
    \end{bmatrix}
  = \begin{bmatrix}
      0 & -1 \\
      1 & 0
    \end{bmatrix}
    \begin{bmatrix}
      u_1 \\ u_2
    \end{bmatrix}
  = \begin{bmatrix}
      -u_2 \\ u_1
    \end{bmatrix}.
\]
%
Let $\vec{u},\,\vec{n} \in \mathbb{R}^2$, such that $\vec{u} = (u_x,
u_y),\,\vec{n} = (-u_y, u_x)$.  From~\eqref{eq:vector.dot.2}, we have
\[ \vec{u} \cdot \vec{n} = u_x u_y - u_y u_x = 0.  \]
\end{comment}

Computing the edges' midpoints and respective normal vectors, we can then
describe the mediators.  Let $M_{AB},\,M_{AC},\,M_{BC} \in \mathbb{R}^2$ be the
midpoints, by \cref{eq:midpoint.2}, of the respective edges, and
$\vec{u}_1,\,\vec{u}_2,\,\vec{u}_3 \in \mathbb{R}^2$ the edges' normal vectors,
such that
\[
  \begin{split}
    P_{M_{AB}} = M_{AB} + \lambda_1 \vec{u}_1 \\
    P_{M_{AC}} = M_{AC} + \lambda_2 \vec{u}_2 \\
    P_{M_{BC}} = M_{BC} + \lambda_3 \vec{u}_3 \\
  \end{split},\,\lambda_i \in \mathbb{R}.
\]
This problem can be further simplified by eliminating one of the redundant
bisectors.  Since the intersection of two lines already yields a single point,
we can eliminate one of the equations.  Say we discard the mediator of line
$\overleftrightarrow{BC}$.  We then require that
\[
  P_{M_{AB}} = P_{M_{AC}} \stackrel{\eqref{eq:line.parametric.2}}{\Rightarrow}
  \begin{cases}
    x_{M_{AB}} + \lambda_1 u_{1x} = x_{M_{AC}} + \lambda_2 u_{2x} \\
    y_{M_{AB}} + \lambda_1 u_{1y} = y_{M_{AC}} + \lambda_2 u_{2y} \\
  \end{cases}.
\]
Every variable is known except for $\lambda_1$ and $\lambda_2$, but the equation
system can be solved in order to assign values to both of them since we have
exactly two equations that relate them.  Finally, we can define $O$ using one of
the equations with the respectively found $\lambda$, i.e., using $L_{M_{AB}}$,
for instance, we have \[ O = M_{AB} + \lambda_1 \vec{u}_1. \]

\Cref{lst:intro.example.circumcenter.tikz} shows the code used to produce the
example in \Cref{fig:intro.example.circumcenter} using \acs{TikZ} with the
\texttt{tkz-euclide} \LaTeX{} package.  To compute the center point of $\odot
O_r$, one can use \texttt{tkzCircumCenter}, which takes three points, $A$, $B$,
and $C$, and generates the result $O$, obtainable using \texttt{tkzGetPoint}.

\begin{listing}[htb]
  \inputminted[highlightlines=3]{latex}{tikz/ex-circumcenter.tikz}
  \caption[Circumcenter example using TikZ]{
    Circumcenter example from \cref{fig:intro.example.circumcenter} using
    \acs{TikZ} alongside \texttt{tkz-euclide}.  The highlighted line shows how
    to obtain the center of $\odot O_r$ via the non-degenerate triangle
    $\triangle ABC$.}%
  \label{lst:intro.example.circumcenter.tikz}
\end{listing}

\Cref{lst:intro.example.circumcenter.euk} shows the code that produces an
identical figure using Eukleides.  In Eukleides, one can use the \texttt{circle}
function, which similarly takes three points, $A$, $B$, and $C$, and generates
the circle $\odot O_r$ circumscribed about $\triangle ABC$, while $O$ can be
obtained using the \texttt{center} function.

\begin{listing}[htb]
  \inputminted[highlightlines=2]{text}{euk/ex-circumcenter.euk}
  \caption[Circumcenter example using Eukleides]{
    Circumcenter example from \Cref{fig:intro.example.circumcenter} using
    Eukleides.  The highlighted line shows how to obtain the center of $\odot
    O_r$ via the non-degenerate triangle $\triangle ABC$.}%
  \label{lst:intro.example.circumcenter.euk}
\end{listing}

Both languages used to produce the examples' solutions provide a sensible set of
constraint primitives.  However, in the particular case of \texttt{tkz-euclide},
the syntax required for describing the models is outdated, rigid, and may cause
confusion.  For example, in
\cref{lst:intro.example.parallel.tikz,lst:intro.example.circumcenter.tikz},
command results can not be used directly as inputs to other commands and must
instead be obtained using another command to create a permanent symbol
associated with the resulting value.  By contrast, functions and expressions'
results in modern languages can be used directly as well as stored by using a
far friendlier assignment syntax.  Nonetheless, the underlying ideas can be
repurposed and adapted, implementing them in a modern and more expressive
language.

% !TEX root = ../../main.tex
\section{\acl{AD} Tools}
\label{sec:related.ad}

As discussed in \cref{sec:intro.ad}, \ac{AD} tools have been integrated into
several modern \ac{CAD} and \ac{BIM} applications; tools that use \acp{TPL},
\acp{VPL}, or even a mixture of both approaches.

Other tools, like OpenJSCAD and ImplicitCAD, are standalone \ac{CAD} software
hosted on the web.  Being cloud-based is advantageous in many fronts: it is
inherently portable, removes the additional typical installation steps required
for desktop applications.  Alas, being relatively new, they are lacking features
in comparison to the immense feature-set of applications such as AutoCAD.

\Cref{tab:related.ad.summary} succinctly summarizes a list of \ac{CAD} software
that includes the capability of designing resorting to the usage of a
programming language, as well as other \ac{AD} software and tools that live
detached from existing software.  From there, Dynamo and Grasshopper are further
comparatively discussed, being relatively similar tools, however integrated
within \ac{CAD}/\ac{BIM} software designed for performing different specific
tasks.  Moreover, both include \ac{TPL} and \ac{VPL} support in different forms.

\begin{table}[htbp]
  \begin{tabularx}{\textwidth}{|*{4}{c|}X|}
    \hline
    \textbf{Application} & \textbf{Tool} & \textbf{\acs{TPL}}
      & \textbf{\acs{VPL}} & \textbf{Note}\\
    \hline
    \hline
    \multirow{5}{*}{AutoCAD \cite{Autodesk:1982:AutoCAD}}
      & \multirow{2}{*}{.NET \acs{API}\label{acro:API}}
      & \multirow{2}{*}{\checkmark} & \multirow{2}{*}{\xmark}
      & \multirow{2}{*}{\parbox{\linewidth}{
        Powerful, but very verbose; C\# \& VB.NET}}\\
      &&&& \\ \cline{2-5}
      & \multirow{2}{*}{\parbox{7em}{\centering ActiveX Automation}}
        & \multirow{2}{*}{\checkmark} & \multirow{2}{*}{\xmark}
        & \multirow{2}{*}{\parbox{\linewidth}{
          Deprecated, bundled separately; \acs{VBA}\label{acro:VBA}}}\\
      &&&& \\ \cline{2-5}
      & Visual LISP & \checkmark & \xmark & \acs{IDE}\label{acro:IDE};
        AutoLISP extension\\
    \hline
    Dynamo Studio
      & \multirow{2}{*}{Dynamo \cite{Keough:2012:Dynamo}}
      & \multirow{2}{*}{\checkmark} & \multirow{2}{*}{\checkmark}
      & \multirow{2}{*}{\parbox{\linewidth}{%
        Data flow paradigm; Associative programming support through
        DesignScript}}\\\cline{1-1}
    Revit \cite{RevitTechCorp:2002:Revit} &&&&\\
    \hline
    ArchiCAD \cite{Graphisoft:2018:ArchiCAD}
      & \multirow{2}{*}{Grasshopper \cite{Rutten:2018:Grasshopper}}
      & \multirow{2}{*}{\checkmark} & \multirow{2}{*}{\checkmark}
      & \multirow{2}{*}{\parbox{\linewidth}{%
        Data flow paradigm; Rhino \acs{SDK} access, C\# \& VB.NET}}\\\cline{1-1}
    \multirow{4}{*}{Rhinoceros3D \cite{McNeel:2018:Rhinoceros3D}}
      &&&& \\ \cline{2-5}
      & \multirow{2}{*}{Python Scripting} & \multirow{2}{*}{\checkmark}
        & \multirow{2}{*}{\xmark}
        & \multirow{2}{*}{\parbox{\linewidth}{%
          Simple language; Create custom Grasshopper components}}\\
      &&&&\\\cline{2-5}
      & RhinoScript & \checkmark & \xmark & VBScript based\\
    \hline
    \multirow{5}{*}{\texttt{Standalone$^\dag$}}
      & ImplicitCAD \cite{Longtin:2018:ImplicitCAD}
        & \checkmark & \xmark & Web hosted; OpenSCAD inspired\\\cline{2-5}
      & OpenJSCAD \cite{Mueller:2019:OpenJSCAD}
        & \checkmark & \xmark & Web hosted; JavaScript\\\cline{2-5}
      & OpenSCAD \cite{Kintel:2019:OpenSCAD}
        & \checkmark & \xmark & Solid 3D models; Simple \acs{DSL}\label{acro:DSL}\\\cline{2-5}
      & \multirow{2}{*}{Rosetta \cite{Leitao:2011:PGDCAD}}
        & \multirow{2}{*}{\checkmark} & \multirow{2}{*}{\xmark}
        & \multirow{2}{*}{\parbox{\linewidth}{%
          Portable tool; Multiple front- and back-end support}}\\
      &&&&\\
    \hline
  \end{tabularx}
  \scriptsize
  $^\dag$These tools are standalone software, i.e., not directly integrated into
  any specific \ac{CAD} application.
  \caption[Table of programmatic \acs{CAD}/\acs{BIM} and \acs{AD} software]{%
    \ac{CAD}/\ac{BIM} software with programmatic capabilities and \ac{AD}
    software/tools.  Added notes per tool shortly outline deemed significant
    characteristics.}
  \label{tab:related.ad.summary}
\end{table}

\subsection{Dynamo}
\label{sec:related.ad.dynamo}

An open source \ac{AD} tool available as a plug-in for Revit or by itself within
Dynamo Studio, Dynamo extends \ac{BIM} with the data and logic environment of a
graphical algorithm editor \cite{Keough:2012:Dynamo}.  Dynamo can be used
through both a \ac{VPL} and a \ac{TPL}, showcased in
\cref{fig:related.ad.dynamo.node2code}.

\begin{figure}[htbp]
  \includegraphics[width=\textwidth]{fig/dynamo-node-to-code}
  {\scriptsize
  Source: \url{http://primer.dynamobim.org/en/07_Code-Block/7-2_Design-Script-syntax.html}
  (Jan 2019)}
  \caption[Dynamo's visual interface with node to code translation]{%
    Showcase of Dynamo's visual interface containing a workflow that produces
    the model on the top left.  The figure also shows Dynamo's capability of
    converting a the workflow to a single DesignScript code block.}
  \label{fig:related.ad.dynamo.node2code}
\end{figure}

In its visual form, Dynamo offers a wide variety of functions, called nodes,
most of them capable of generating an even wider variety of geometry through
node combination, wiring one's outputs to another's inputs, and resorting to
pre-defined mutable parameters which can serve as some of the nodes' initial
inputs.  The workflow itself is the final product: a visual program, usually
designed to execute a specific task.  Dynamo further allows extension through the
creation of custom nodes which can be shared as packages.

One of the nodes in Dynamo, aptly named code block, allows the usage of a
\ac{TPL}; a language called DesignScript.  Originally developed my Robert Aish
\cite{Aish:2011:DesignScript}, DesignScript is a multi-paradigm domain-specific
language and is the programming language at the core of Dynamo itself.  So much
so that entire workflows can be reduced to one code block (see
\Cref{fig:related.ad.dynamo.node2code}).

DesignScript is an associative language, which maintains a graph of dependencies
with variables.  Executing a script will effectively propagate the variables'
values accordingly.  By default, code blocks in Dynamo follow an associative
paradigm.  The user can, however, switch to an imperative paradigm approach
instead effortlessly if needed.

This \textit{change-propagation} mechanism in DesignScript, consequently present
in Dynamo, makes Dynamo a great tool for dealing with constraints.  However,
most users might not fully exercise DesignScript's associative capabilities and
instead approach the problem with the mindset of an imperative programming
paradigm given its overwhelming presence in and adoption by major well-known
\acp{TPL}.

\subsection{Grasshopper}
\label{sec:related.ad.grasshopper}

Grasshopper is a graphical algorithm editor tightly integrated with
Rhinoceros3D, destined for designers who are exploring generative algorithms
\cite{Rutten:2018:Grasshopper}.  In spite of tight integration with Rhino, a
\ac{CAD} application, it is possible to use Grasshopper along with ArchiCAD
\cite{Graphisoft:2018:ArchiCAD,Graphisoft:2018:RGACAD}, a \ac{BIM} tool.
\Cref{fig:related.ad.grasshopper.islamic-pattern} shows a simple example of a
Grasshopper workflow.

\todo[inline]{Replace example with Rythmic Gynmastics Center, Moscow, Russia}

\begin{figure}[htbp]
  \includegraphics[width=\textwidth]{fig/grasshopper-islamic-pattern}
  {\scriptsize
  Source: \url{https://www.grasshopper3d.com/photo/islamic-pattern-parakeet}
  (Jan 2019)}
  \caption[Islamic Pattern in Grasshopper using Parakeet]{
    Islamic Pattern, by Esmaeil Mottaghi.  On top is the Grasshopper workflow to
    produce the pattern below it, aided by Parakeet
    \cite{Esmaeil:2018:Parakeet}.}
  \label{fig:related.ad.grasshopper.islamic-pattern}
\end{figure}

It is a closed-source product, designed by David Rutten and developed by McNeel
and Associates, Rhino's developers.  Its \ac{VPL} is as simple to use as
Dynamo's, which is crucial for users who are not familiar with programming using
a \ac{TPL}.  Nonetheless, it offers a \ac{TPL} alternative by way of custom
programmatic components.  Using C\# or VB.NET, the user can create custom code
components with access to Rhino's \ac{SDK} and OpenNURBS
\cite{Lear:2018:openNURBS} within Rhino.  Alternatively, through GhPython
\cite{Giulio:2017:GhPython}, the user can also write Python code.  Unlike
DesignScript, Python and the .NET languages don't support an associative
programming model.

Functions in Grasshopper are called components and work just like Dynamo's
nodes; a wide variety of them exist, most of them capable of producing geometry,
and they are composable, generating a workflow destined to accomplish a specific
task.

Both Dynamo and Grasshopper's visual approach suffer from the unproportionate
scalability between the code and the respective model's complexity
\cite{Leitao:2014:PESLGD}.  Sophisticated modelling workflows tend to become
difficult to properly represent, and harder for a human to efficiently interpret
when compared to a textual approach.  This disadvantage, however, is mitigated
with their respective \ac{TPL} alternatives.

